\documentclass[parskip=half]{scrbook}

\usepackage{mypackage}
\newcommand{\mydate}{}
\newcommand{\newdate}[1]{\renewcommand{\mydate}{#1}}

\usepackage{fancyhdr}

\pagestyle{fancy}
\fancyfoot[OC, EC]{}
\fancyfoot[EL, OR]{\textbf\thepage}
\fancyhead[ORH]{\textbf{\nouppercase\rightmark}}
\fancyhead[ELH]{\textbf{\nouppercase\leftmark}}
\fancyhead[ER, OL]{}

\usepackage{titlesec}
\titlespacing*{\chapter}{0pt}{-20pt}{20pt}
\titleformat{\chapter}[display]
{\bfseries\Large}
{\filleft\scshape{\chaptertitlename} \Huge\thechapter}
{4ex}
{\titlerule
	\vspace{2ex}%
	\filright}
[\vspace{2ex}%
\titlerule]

\makeatletter
\renewcommand{\maketitle}{
\begin{titlepage}
\vspace*{2cm}

\begin{center}
\rule{12cm}{1.5pt}
\vspace{0.5cm}

{\Huge\bfseries \@title}

\vspace{.6cm}

{\LARGE \@date

\@author}
\vspace{1cm}
\rule{12cm}{1.5pt}

\vspace{6cm}
{\LARGE Vorlesungsmitschrieb von \\[0.1\baselineskip] Patrick Arras \\[0.05\baselineskip] Jonas Müller \\[\baselineskip]
Heidelberg, den \today}

\end{center}
\end{titlepage}
}
\makeatother

\allowdisplaybreaks

\usepackage[ocgcolorlinks=true, pdfusetitle=true]{hyperref} % linking inside of the document, should be last package

\title{Modulformen I}
\author{Prof. Dr. Winfried Kohnen \\[0.2\baselineskip] und Johann Franke}
\date{Sommersemester 2018}

\begin{document}

\pagenumbering{roman}
\pagestyle{plain}

\maketitle

\chapter*{Vorwort}

Dies ist ein nicht offizielles Skript der Vorlesung Modulformen 1 aus dem Sommersemester 2018 gehalten von Professor Winfried Kohnen und Johann Franke an der Universität Heidelberg.
Das Skript wurde in der Vorlesung mitgetext und mit pdflatex kompiliert.
Deshalb kann es Fehler enthalten und wir übernehmen keine Garantie für die Richtigkeit.

Bei Fehlern kann ich unter folgender Mailadresse erreicht werden:

\begin{center}
\mail{jj@mathphys.stura.uni-heidelberg.de}
\end{center}

Die aktuellste Version des Skriptes befindet sich immer unter
\begin{center}
\texttt{\url{https://github.com/jenuk/modulformen/blob/master/script.pdf}}
\end{center}

Die \LaTeX-Source Dateien findet man hier, auf Fehler kann hier alternativ über neues Issue aufmerksam gemacht werden:
\begin{center}
\texttt{\url{https://github.com/jenuk/modulformen/tree/master}}
\end{center}

Der Comic auf der Titelseite ist eine Überarbeitung von Patrick vom Comic \url{xkcd.com/179} von \url{xkcd.com}. 
Die Schrift kommt von \url{github.com/ipython/xkcd-font}.

\clearpage
\phantomsection
\addcontentsline{toc}{chapter}{Inhaltsverzeichnis}
\tableofcontents

\clearpage
\pagenumbering{arabic}
\setcounter{page}{0}
\pagestyle{fancy}

%chapter 1
\chapter{Grundlegende Tatsachen}

\section{Ergebnisse aus Funktionentheorie 2 (Erinnerung)}
\subsection{Fundamentalbereich}
Wie üblich sei
\[
	\HH = \Set{z \in \CC \mid \Im z > 0}
\]
die obere Halbebene und
\[
	\SL_2(\ZZ) = \Set{M \in M_2(\ZZ) \mid \det M = 1}
	\,.
\]

\label{DefSL2Z}
Dann operiert $\SL_2(\ZZ)$ auf $\HH$ durch
\[
	\abcd \circ z = \frac{az+b}{cz+d}
	\,,
\]
das heißt $E \circ z = z$ und $(M_1M_2) \circ z = M_1 \circ (M_2 \circ z)$.
Hierbei beachte man, dass
\[
	\Im \Bigl(\frac{az+b}{cz+d}\Bigr) = \frac{\Im z}{\abs{cz+d}^2}
	\,.
\]

$\Gamma(1) = \SL_2(\ZZ) \subset \SL_2(\RR)$ ist eine diskrete Untergruppe, spezielle Matrizen in $\Gamma(1)$ sind
\[
	T = \begin{pmatrix}
			1 & 1\\
			0 & 1
		\end{pmatrix}
	\qquad \text{und} \qquad
	S = \begin{pmatrix}
			0 & -1\\
			1 & 0
		\end{pmatrix}
\]
die Translation $T \circ z = z + 1$ und Stürzung $S \circ z = - \frac{1}{z}$.

Man interessiert sich für die Operation von diskreten Untergruppen $\Gamma \subset \SL_2(\ZZ)$ insbesondere $\Gamma = \Gamma(1)$.

\begin{defi}
	Eine Teilmenge $\F \subset \HH$ heißt \myemph{Fundamentalbereich} für die Operationen von $\Gamma \subset \SL_2(\RR)$ auf $\HH$, falls:
	\begin{enumerate}
		\item $\F$ ist offen,
		\item zu jedem $z \in \HH$ existiert ein $M \in \Gamma$ mit $M \circ z \in \closure\F$,
		\item Sind $z_1, z_2 \in \F$ und $z_2 = M \circ z_1$ mit $M \in \Gamma$, dann gilt $M = \pm E$ und somit $z_1 = z_2$.
	\end{enumerate}
\end{defi}

\begin{bsp}
	Die Menge $\F_1 := \Set{z=x+iy \mid \abs x < \frac{1}{2},\ \abs z > 1}$ ist ein Fundamentalbereich für die Operation von $\Gamma(1)$ auf $\HH$, dieser wird auch \myemph{Modulfigur} genannt.
	Siehe \autoref{fig:fundamentalbreichVolle}.
	
	\begin{figure}
	\begin{center}
		\includestandalone{images/chapter1/fundamentalbereich}
		\caption{Der Fundamentalbereich $\F_1$ der vollen Modulgruppe.}
		\label{fig:fundamentalbreichVolle}
	\end{center}
	\end{figure}
\end{bsp}

\begin{beme}
	Identifikationen in $\closure{\F_1}$ finden nur auf dem Rand statt. (Die Geraden $x = \pm \frac{1}{2}$ werden miteinander identifiziert unter $T$ bzw $T^{-1}$, Punkte auf den Kreisbögen rechts oder links von $i$ werden unter $S$ identifiziert.
\end{beme}

\begin{satz}
	Die Gruppe $\Gamma(1)$ wird erzeugt von $S$ und $T$.
\end{satz}

\subsection{Modulform}

\begin{defi}
	Eine Abbildung $f\colon \HH \ra \closure\CC = \CC \cup \{\infty\}$ heißt \myemph{Modulfunktion} vom Gewicht $k \in \ZZ$ für $\Gamma(1)$, falls gilt:
	\begin{enumerate}
		\item $f$ ist auf $\HH$ meromorph,
		\item $f(\frac{az+b}{cz+d}) = (cz+d)^k f(z)$ für alle $\abcd \in \Gamma(1)$,
		\item $f$ ist meromorph in $\infty$.
	\end{enumerate}
\end{defi}

\emph{Bedeutung von} (iii):
Wendet man (ii) an mit $M = T$, so erhält man $f(z+1) = f(z)$.
Sei $\mathcal R = \Set{q \in \CC \mid 0 < \abs q < 1}$.
Die Abbildung $z \mapsto q = e^{2\pi iz}$ bildet $\HH$ auf $\mathcal R$ ab und $F(q) := f(z)$ ist wohldefiniert und holomorph bis auf mögliche Polstellen, die sich prinzipiell gegen $q=0$ häufen könnten.
Bedingung (iii) fordert nun, dass $q=0$ eine unwesentliche isolierte Singularität\footnote{Das heißt, es handelt sich um eine hebbare Singularität oder eine Polstelle.} von $F$ ist.
Nach Funktionentheorie 1 hat dann $F$ eine Laurententwicklung
\[
	F(q) = \sum_{n \geq n_0} a_n q^n
	\qquad \text{für }
	0 < \abs q < \abs{q_0}
	\,,
\]
wobei $n_0 \in \ZZ$ fest.
Damit erhalten wir also
\[
	f(z) = \sum_{n \geq n_0} a_n e^{2\pi i nz}
	\qquad \text{für }
	 0 < y_0 < y
	 \,.
\]

\begin{defi}
	Ein solches $f$ heißt \myemph{Modulform}, falls $f$ auf $\HH$ und in $\infty$ holomorph ist (letzteres bedeutet, dass $F$ in $q=0$ hebbar ist, also $f(z) = \sum_{n \geq 0} a_n e^{2\pi inz}$ für alle $z\in\HH$).
	Eine Modulform heißt \myemph{Spitzenform}, falls $a_0 = 0$.
\end{defi}

\begin{beme}
	Die Fourierkoeffizienten $a_n$ sind im Allgemeinen wichtige und interessante Größen (z.\,B. Darstellungsanzahlen von natürlichen Zahlen durch quadratische Formen, etwa $r_4(n) = \#\Set{(x,y,z,w) \in \ZZ^4 \mid n = x^2+y^2+z^2+w^2}$ oder die Anzahl von Punkten auf elliptischen Kurven über $\FF_p$).
\end{beme}

\begin{defi}
	Sei $f\colon \HH \ra \CC$, $k\in\ZZ$, $M = \abcd \in \SL_2(\RR)$.
	Man setzt
	\[
		(f|_kM)(z) := (cz+d)^{-k} f\Bigl(\frac{az+b}{cz+d}\Bigr)
	\]
	für $z\in\HH$, dies ist der \myemph{Peterssonscher Strichoperator}.
\end{defi}

Dann gilt $f|_kE = f$ und $f|_k(M_1M_2) = (f|_kM_1)|_kM_2$ für alle $M_1$, $M_2 \in \SL_2(\RR)$.
Es folgt:
\begin{enumerate}
	\item Es gilt $(f|_kM)(z) = (cz+d)^{-k} f(\frac{az+b}{cz+d}) = f(z)$ für alle $M = \abcd \in \Gamma(1)$ genau dann, wenn dies für $S$ und $T$ gilt, d.\,h. $f(-\frac{1}{z}) = z^k f(z)$ und $f(z+1) = f(z)$, da $S$ und $T$ $\SL_2(\ZZ)$ erzeugen.
	\item Eine Funktion $f\colon \HH \ra \CC$ ist genau dann eine Modulform vom Gewicht $k$, wenn $f$ eine Fourierentwicklung
	\[
		f(z) = \sum_{n \geq 0} a_n e^{2\pi inz}
		\qquad \text{für }
		z \in \HH
	\]
	hat und zusätzlich gilt
	\[
		f\Bigl(-\frac{1}{z}\Bigr) = z^k f(z)
		\,.
	\]
\end{enumerate}

\subsection{Beispiele für Modulformen}

\subsubsection{Thetareihen}

\begin{defi}
	Sei $A \in M_m(\RR)$ symmetrisch und positiv definit.
	Dann heißt
	\[
		\theta_A(z) = \sum_{g \in \ZZ^m} e^{\pi i A[g]z}
		\qquad \text{für }
		z \in \HH
	\]
	eine \myemph{Thetareihe}, wobei $A[g] := g^t A g$ für $g \in \ZZ^m \cong M_{m,1}(\ZZ)$.
\end{defi}

\begin{satz-list}
	\item $\theta_A(z)$ ist gleichmäßig absolut konvergent auf $y \geq y_0 > 0$.
	Insbesondere ist $\theta_A(z)$ auf $\HH$ holomorph.
	\item Es gilt die Theta-Transformationsformel: $\theta_{A^{-1}} = \sqrt{\det A} \cdot (\frac{z}{i})^{\frac{m}{2}} \theta_A(z)$.
\end{satz-list}

\begin{satz}
	Sei $A \in M_m(\ZZ)$ symmetrisch, positiv definit, gerade\footnote{Das heißt für alle $\mu \in \Set{1, \ldots, m}$ gilt $a_{\mu\mu}$ ist gerade} und $\det A = 1$.
	Dann gilt $8|m$ und $\theta_A(z)$ ist eine Modulform vom Gewicht $\frac{m}{2}$ für $\Gamma(1)$.
	
	\emph{Beachte} $\theta_A(z) = 1 + \sum_{n \geq 1} r_A(n) q^n$ wobei $r_A(n)$ die Anzahl der Darstellungen von $n$ durch die ganzzahlige, positive definite quadratische Form $x \mapsto \frac{1}{2} x^t A x$ auf $\ZZ^m$ ist.
\end{satz}

\subsubsection{Eisensteinreihen}

\begin{defi}
	Sei $k \in \ZZ$, $k$ gerade und $k \geq 4$.
	Dann heißt
	\[
		G_k(z) = \sumprime_{m,n} \frac{1}{(mz+n)^k}
		\qquad \text{für }
		z \in \HH
	\]
	\myemph{Eisensteinreihe} vom Gewicht $k$.\footnote{$\displaystyle \sumprime_{m,n} := \sum_{\substack{(m,n) \in \ZZ^2 \\ (m,n) \not= (0,0)}}$}
\end{defi}

\begin{satz-list}
	\item $G_k(z)$ ist gleichmäßig absolut konvergent auf $D_\epsilon = \Set{ z=x+iy \mid y \geq \epsilon,\ x^2 \leq \frac{1}{\epsilon}}$, insbesondere also holomorph auf $\HH$.
	\item $G_k$ ist Modulform vom Gewicht $k$ für $\Gamma(1)$.
	\item Es gilt
	\[
		G_k(z) = 2\zeta(k) + \frac{2(2\pi i)^k}{(k-1)!} \sum_{n \geq 1} \sigma_{k-1}(n)q^n
	\]
	wobei $\zeta(k) = \sum_{n=1}^\infty \frac{1}{n^k}$ und $\sigma_{k-1}(n) = \sum_{d|n} d^{k-1}$.
	
	Setze $E_k := \frac{1}{2\zeta(k)} G_k$ die \myemph{normalisierte Eisensteinreihe}.
	Benutze nun
	\[
		\zeta(k) = \frac{(-1)^{\frac{k}{2}-1}2^{k-1} B_k}{k!} \pi^k
	\]
	für $k$ gerade und $k \geq 2$.
	Damit folgt
	\[
		E_k = 1 - \frac{2k}{B_k} \sum_{n \geq 1} \sigma_{k-1}(n)q^n
		\,,
	\]
	wobei alle $B_k$ rationale Zahlen sind. Speziell gilt 
	\begin{align*}
		B_4 &= -\frac 1{30} &&\Ra& E_4 &= 1 + 240 \sum_{n \geq 1} \sigma_3 (n) q^n
		\,, \\
		B_6 &= \frac 1{42} &&\Ra& E_6 &= 1 - 504 \sum_{n \geq 1} \sigma_5 (n) q^n
		\,.
	\end{align*}
\end{satz-list}

\subsection{Valenzformel und Anwendungen}

\begin{satz}[Valenzformel]
	Sei $f$ eine Modulfunktion vom Gewicht $k \in \ZZ$, $f \not \equiv 0$. Dann gilt
	\[
	\ord_\infty f + \frac 12 \ord_i f + \frac 13 \ord_\rho f + \smashoperator{\sum_{\substack{z \in \linksmodulo{\Gamma(1)}{\HH} \\ z \not \sim i, \rho}}} \ord_z f = \frac {k}{12}
	\,.
	\]
	Hierbei ist $\rho = e^{\frac{2 \pi i}{3}}$ und
	\[
	\ord_\infty f := \ord_{q = 0} F(q)
	\]
	mit $F(q) = f(z)$ für $q = e^{2\pi i z}$.
\end{satz}

\begin{bewe}
	Zum Nachweis reduziert man auf den Fall, dass $f$ außer in $z = \rho, - \conj{\rho}, i$ keine Null- oder Polstellen auf $\partial \closure{F_1}$ hat und berechnet
	\[
	\frac{1}{2\pi i} \int_{\mathcal C} \frac{f'(z)}{f(z)} \opd z
	\,,
	\]
	wobei die Kurve $\mathcal C$ wie in \autoref{fig:valenzformelweg} gewählt ist.
	
	\begin{figure}
		\begin{center}
			\includestandalone[scale=.7]{images/chapter1/valenzformelweg}
			\caption{Die Kurve $\mathcal C$, wobei $A$ und $E$ so gewählt sind, dass $\mathcal C$ alle Null- und Polstellen (außer eventuell $i, \rho$) einschließt.}
			\label{fig:valenzformelweg}
		\end{center}
	\end{figure}
\end{bewe}

\begin{defi}
	Sei
	\[
	\Delta (z) = \frac{1}{1728} \left( E_4^3(z) - E_6^2(z) \right)
	\]
	die \myemph{Diskriminantenfunktion}. Dann ist $\Delta$ eine Spitzenform vom Gewicht $k = 12$ mit $\Delta(z) \neq 0 \, \forall z \in \HH$ und $\ord_\infty \Delta = 1$, d.\,h. $\Delta = q + \ldots$.
\end{defi}

\begin{beme}
	$\Delta$ ist in gewisser Weise die \myquote{erste} von 0 verschiedene Spitzenform und wurde von vielen Mathematikern studiert.
	
	\begin{bsp-list}
		\item Schreibe $\Delta(z) = \sum_{n \geq 1} \tau (n) q^n$, dann heißt $n \mapsto \tau (n)$ \myemph{Ramanujan-Funktion}. Es gilt: $\tau (n) \in \ZZ$ für alle $n \geq 1$. Ferner lässt sich zeigen, dass $\tau (n) \equiv \sigma_{11}(n) \mod 691$, mithilfe von $B_{12} = - \frac{691}{2730}$.
		\item Vermutung: $\tau (n) \neq 0$ für alle $n \geq 1$ (Lehmer).
	\end{bsp-list}
	
\end{beme}


Sei $M_k$ der $\CC$-Vektorraum der Modulformen vom Gewicht $k \in \ZZ$ und $S_k \subset M_k$ der Unterraum der Spitzenformen.

\begin{beme}
	$M_k = \Set {0}$ für $k$ ungerade, da $f((-E) \circ z) = f(z) = (-1)^k f(z)$.
\end{beme}

\begin{satz}\label{M_k1}
	Sei $k \in \ZZ$ gerade. Dann gilt:
	\begin{enumerate}
		\item $M_k = \Set {0}$ für $k < 0$ und $M_2 = \Set {0}$.
		\item $M_0 = \CC$.
		\item $M_k = \CC E_k \oplus S_k$, falls $k \geq 4$.
		\item Die Abbildung $f \mapsto f \cdot \Delta$ gibt einen Isomorphismus von $M_{k-12}$ auf $S_k$.
		\item $\dim M_k < \infty$.
	\end{enumerate}
\end{satz}

\begin{satz}
	Sei $k \geq 0$ gerade. Dann gilt:
	\[
	\dim M_k = \begin{cases}
	\floor{\frac k{12}} & \text{ falls } k \equiv 2 \mod 12\\
	1 + \floor{\frac k{12}} & \text{ falls } k \not \equiv 2 \mod 12
	\end{cases}
	\,.
	\]
\end{satz}

\begin{bsp-list}\label{M_k2}
	\item $M_4 = \CC E_4$.
	\item $M_6 = \CC E_6$.
	\item $M_8 = \CC E_8 = \CC E_4^2$.
	\item $M_{10} = \CC E_{10} = \CC E_4 E_6$.
	\item $M_{12} = \CC E_{12} \oplus \CC \Delta$.
	\item $M_{14} = \CC E_{14}$.
\end{bsp-list}

% Ab hier neu, aber gehört thematisch noch zu oben!

\begin{satz}\label{satz:basis_modulformen}
	Sei $k \geq 0$ gerade. Dann bilden $E_4^\alpha E_6^\beta$ mit $4\alpha + 6\beta = k$ eine Basis von $M_k$, insbesondere gilt also
	\[
	M_k = \smashoperator{\bigoplus_{\substack{\alpha, \beta \geq 0\\ 4\alpha + 6\beta = k}}} \CC E_4^\alpha E_6^\beta
	\,.
	\]
\end{satz}

\begin{bewe}
	Wir zeigen zunächst induktiv, dass die Monome $M_k$ erzeugen. Für $k \leq 10$ ist dies nach Beispiel \ref{M_k2} klar. Sei also $k \geq 12$. Man bestimme eine beliebige Kombination $\alpha, \beta \geq 0$ mit $4 \alpha + 6 \beta = k$ und setze $g := E_4^\alpha E_6^\beta \in M_k$ mit konstantem Term gleich 1.
	
	Sei nun $f \in M_k$ beliebig mit konstantem Term $a_0$. Dann ist $f - a_0 \cdot g \in S_k$. Nach Satz \ref{M_k1}, iv) gilt daher $f - a_0 \cdot g = \Delta \cdot h$ mit $h \in M_{k-12}$. Nach Induktionsvoraussetzung ist $h$ eine Linearkombination von Monomen $E_4^\gamma E_6^\delta$ mit $4 \gamma + 6 \delta = k - 12$. Aber $\Delta = \frac{1}{1728} (E_4^3 - E_6^2)$ und daher ist $f - a_0 \cdot g$ Linearkombination von Monomen $E_4^{\gamma + 3}E_6^{\delta}$ und $E_4^{\gamma}E_6^{\delta + 2}$. Wegen
	\[
	4(\gamma + 3) + 6 \delta = k - 12 + 12 = k
	\,,
	\]
	\[
	4 \gamma + 6 (\delta + 2) = k - 12 + 12 = k
	\,,
	\]
	ist also auch $f$ als Linearkombination von Monomen der behaupteten Form schreibbar. Somit erzeugen die Monome tatsächlich $M_k$.
	
	Noch zu zeigen ist, dass die Monome über $\CC$ linear unabhängig sind. Beweis durch Widerspruch: \emph{Angenommen}, es existiere eine nicht-triviale lineare Relation
	\[
	\smashoperator{\sum_{\substack{\alpha, \beta \geq 0\\ 4 \alpha + 6 \beta = k}}} \lambda_{\alpha, \beta} E_4^\alpha E_6^\beta = 0
	\,.
	\]
	\emph{Fall 1:} Sei $k \equiv 0 \mod 4$. Dann sind alle $\beta$ gerade, also schreibe jeweils $\beta = 2 \beta'$ mit $\beta' \geq 0$. Es folgt $\alpha = \frac k4 - 3 \beta'$ und somit
	\[
	E_4^\alpha E_6^\beta = E_4^{\frac k4 - 3\beta'}E_6^{2\beta'} = E_4^{\frac k4} \left( \frac {E_6^2}{E_4^3} \right)^{\beta'}
	\,.
	\]
	Da $E_4^{\frac k4}$ nicht die Nullfunktion ist, ergibt sich eine nicht-triviale Polynom-Relation für $\frac{E_6^2}{E_4^3}$, d.\,h. die meromorphe Funktion $\frac{E_6^2}{E_4^3}$ ist Nullstelle eines nicht-trivialen Polynoms über $\CC$. Da $\CC$ algebraisch abgeschlossen ist (jedes nicht-konstante Polynom über $\CC$ zerfällt vollständig über $\CC$ in Linearfaktoren), ist $\frac{E_6^2}{E_4^3}$ somit konstant.
	
	Wir zeigen $\frac{E_6^2}{E_4^3} \equiv 0$ mit einem \emph{Trick}: Es gilt $E_6 (- \frac 1z) = z^6 E_6(z)$, denn $E_6 \in M_6$. Auswerten in $z = i = - \frac 1i$ liefert $E_6 (i) = 0$. Ferner gilt
	\[
	E_4(z) = 1 + 240 \sum_{n \geq 1} \sigma_3 (n) e^{2\pi i n z} \quad \Ra \quad E_4 (i) = 1 + 240 \sum_{n \geq 1} \sigma_3 (n) e^{-2 \pi n}
	\,.
	\] 
	Da alle Summanden positiv sind, folgt $E_4(i) \neq 0$ und somit $\frac{E_6^2(i)}{E_4^3(i)} = 0$. Dies impliziert jedoch, da $\frac{E_6^2}{E_4^3}$ konstant ist, bereits $E_6 \equiv 0$. \blitz
	
	\emph{Fall 2:} Sei $k \equiv 2 \mod 4$, dann sind alle $\beta$ ungerade. Analoges Vorgehen zum ersten Fall liefert ebenfalls einen Widerspruch.
	
	Somit sind die Monome über $\CC$ linear unabhängig.
\end{bewe}

\begin{beme}
	Der Satz impliziert additive Faltungsformeln für die multiplikativen Funktionen $\sigma_{k-1} (n)$ (weiterhin $k \in \ZZ$, $k \geq 4$ gerade). \myquote{Multiplikativ} bedeutet hier
	\[
	\ggt (m,n) = 1 \Ra \sigma_{k-1}(m \cdot n) = \sigma_{k-1}(m)\cdot \sigma_{k-1}(n)
	\,.
	\]
\end{beme}

\begin{bsp}
	$E_8 = E_4^2$, ferner $E_4 = 1 + 240 \sum_{n \geq 1} \sigma_3(n) q^n$, also $\sigma_7 (n) = \sigma_3 (n) + 120 \sum_{m=1}^{n-1} \sigma_3 (n-m) \sigma_3 (m)$.
	
	Allgemeiner kann man $E_k$ ausdrücken als Linearkombination von Monomen der Form $E_4^\alpha E_6^\beta$ und erhält hieraus Formeln für $\sigma_{k-1}(n)$.
\end{bsp}
\section[Die Modulinvariante \texorpdfstring{$j$}{j}]{Die Modulinvariante {\boldmath $j$}}

\begin{defi}
Sei $j := \frac{E_4^3}{\Delta}$.
\end{defi}

\begin{satz-list}\label{satz:j_eigenschaften}
	\item $j$ ist holomorph auf $\HH$ und hat einen einfachen Pol in $\infty$.
	\item $j$ ist eine Modulfunktion vom Gewicht $0$.
	\item $j$ liefert eine Bijektion $\linksmodulo{\Gamma(1)}{\HH} \cong \CC$.
\end{satz-list}

\begin{bewe-list}
	\item  Da $\Delta(z) \not= 0$ für alle $z\in\HH$, ist $j(z)$ holomorph auf $\HH$.
	Ferner gilt
	\[
	\ord_\infty j = \ord_\infty E_4^3 - \ord_\infty \Delta = 0 - 1 = -1
	\,.
	\]
	\item Da $E_4^3$, $\Delta \in M_{12}$ folgt die Aussage.
	\item Sei $\lambda \in \CC$. Dann ist zu zeigen, dass die Modulfunktion $j_\lambda := j - \lambda$ vom Gewicht Null eine modulo $\SL_2(\ZZ)$ eindeutig bestimmte Nullstelle hat.
	Man wendet auf $j_\lambda$ die Valenzformel an!
	Es gilt $\ord_z j_\lambda \geq 0$ für alle $z\in\HH$ und $\ord_\infty j_\lambda = -1$.
	Da $k = 0$ folgt mit der Valenzformel
	\[
	-1 + n + \frac{n'}{2} + \frac{n''}{3} = 0
	\]
	mit $n$, $n'$, $n'' \in \NN_0$.
	Also
	\begin{equation}\label{eq:basicj_valenzformel}
	n + \frac{n'}{2} + \frac{n''}{3} = 1
	\end{equation}
	Man prüft nach: die einzigen Lösungen $(n,n',n'') \in \NN_0^3$ von \eqref{eq:basicj_valenzformel} sind $(1,0,0)$, $(0,2,0)$ und $(0,0,3)$.
	Dies impliziert die Behauptung.
\end{bewe-list}

\begin{satz}\label{satz:charakterisierung_modulfunktion_0}
	Sei $f\colon \HH \to \closure{\CC}$ eine meromorphe Funktion. Dann sind folgende Aussagen äquivalent:
	\begin{enumerate}
		\item $f$ ist eine Modulfunktion vom Gewicht 0.
		\item $f$ ist Quotient zweier Modulformen gleichen Gewichts.
		\item $f$ ist eine rationale Funktion in $j$.
	\end{enumerate}
\end{satz}

\begin{bewe-list}
	\item[(iii) $\Rightarrow$ (ii)] Sei $f = \frac{P(j)}{Q(j)}$ wobei $P(X) = a_0 + a_1X + \ldots + a_mX^m$ mit $a_\nu \in \CC$, $a_m \not= 0$ und $Q(X) = b_0 + b_1X + \ldots + b_nX^n$ mit $b_\nu \in \CC$, $b_n \not= 0$ mit $Q \not\equiv 0$, insbesondere also auch $Q(j) \not\equiv 0$.
	Wegen $j = \frac{E_4^3}{\Delta}$ folgt
	\begin{align*}
	f
	&= \frac{a_0 + a_1\frac{E_4^3}{\Delta} + \ldots + a_m\bigl(\frac{E_4^3}{\Delta}\bigr)^m}{b_0 + b_1\frac{E_4^3}{\Delta} + \ldots + b_n\bigl(\frac{E_4^3}{\Delta}\bigr)^n} \\
	&= \frac{(a_0\Delta^m + a_1E_4^3\Delta^{m-1} + \ldots + a_m(E_4^3)^m)\cdot \Delta^n}{(b_0\Delta^n + b_1E_4^3\Delta^{n-1} + \ldots + b_n(E_4^3)^n) \cdot \Delta^m}
	\,.
	\end{align*}
	Hier sind Zähler und Nenner Modulformen vom Gewicht $12(m+n)$.
	Also folgt die Behauptung.
	
	\item[(ii) $\Rightarrow$ (i)] klar
	
	\item[(i) $\Rightarrow$ (iii)] Sei $f$ eine Modulfunktion vom Gewicht Null und $f \not\equiv 0$.
	Seien $z_1, \ldots z_r$ die modulo $\Gamma(1)$ verschiedenen Polstellen von $f$ und $m_1, \ldots m_r$ deren Ordnungen.
	Sei
	\[
	P(z)
	:= \prod_{\nu = 1}^r \bigl(j(z) - j(z_\nu)\bigr)^{m_\nu}
	\,.
	\]
	Dann gilt
	\[
	\ord_{z_\nu} P
	= \ord_{z_\nu} \bigl(j(z) - j(z_\nu)\bigr)^{m_\nu}
	= m_\nu \ord_{z_\nu} \bigl(j(z) - j(z_\nu)\bigr)
	\geq m_\nu
	\,.
	\]
	Dann ist $P(z)f(z)$ eine Modulfunktion vom Gewicht Null und holomorph auf $\HH$.
	Da $P(z)$ ein Polynom in $j$ ist, genügt es die Behauptung für $P(z)f(z)$ zu zeigen.
	Insbesondere kann man voraussetzen, dass $f$ holomorph auf $\HH$ ist.
	Da $\ord_\infty \Delta = 1$, gibt es $n\in\NN_0$ so dass $g := \Delta^nf$ in unendlich holomorph ist.
	Dann ist $f = \frac{g}{\Delta^n}$ und $g$ ist eine Modulform vom Gewicht $12n$.
	Nach \autoref{satz:basis_modulformen} ist $g$ eine Linearkombination von Monomen $E_4^\alpha E_6^\beta$ mit $4\alpha + 6\beta = 12n$.
	Es genügt somit die Behauptung für $\frac{E_4^\alpha E_6^\beta}{\Delta^n}$ zu zeigen.
	Insbesondere gilt $3|\alpha$ und $2|\beta$, schreibe $\alpha = 3p$ und $\beta = 2q$.
	Dann gilt
	\[
	\frac{E_4^\alpha E_6^\beta}{\Delta^n}
	= \frac{(E_4^3)^p (E_6^2)^q}{\Delta^{p+q}}
	= j^p (j-1728)^q
	\,,
	\]
	denn $j-1728 = j - \frac{E_4^3 - E_6^2}{\Delta} = \frac{E_4^3}{\Delta} - \frac{E_4^3-E_6^2}{\Delta} = \frac{E_6^2}{\Delta}$.
\end{bewe-list}

\begin{beme-list}
	\item Der Quotient $\linksmodulo{\Gamma(1)}{\HH}$ besitzt in natürlicher Weise die Struktur einer Riemannschen Fläche isomorph zu $S^2 \setminus\Set{\text{Punkt}}$ indem man die Ränder in $\closure{\F_1}$ identifiziert.
	Fügt man den Punkt $\infty$ hinzu, so erhält an $\closure{\linksmodulo{\Gamma(1)}{\HH}} := \linksmodulo{\Gamma(1)}{\HH} \cup \Set{\infty} \cong S^2$ (die Sphäre in $\RR^3$).
	\autoref{satz:j_eigenschaften} (iii) besagt dann, dass $j$ ein Isomorphismus von $\closure{\linksmodulo{\Gamma(1)}{\HH}} \cong S^2 \cong \mathds P^1(\CC) = \CC \cup \infty$ ist.
	\autoref{satz:charakterisierung_modulfunktion_0} entspricht dann der Tatsache, dass die einzigen meromorphen Funktionen auf $S^2$ die rationalen Funktionen sind.
	
	\item Man kann zeigen (schwer!)
	\[
	\Delta(z) = q \prod_{n\geq1} (1-q^n)^{24}
	\,.
	\]
	Damit folgt
	\begin{align*}
	j
	&= \frac{E_4^3}{\Delta}
	= \frac{1}{q} \biggl(1 + 240\sum_{n\geq1} \sigma_3(n)q^n\biggr)^3 \frac{1}{\prod_{n\geq1} (1 - q^n)^{24}} \\
	&= \frac{1}{q} \biggl(1 + 240\sum_{n\geq1} \sigma_3(n)q^n\biggr)^3 \prod_{n\geq1} \Bigl(\sum_{m\geq0} q^{mn}\Bigr)^{24} \\
	&= \frac{1}{q} + 744 + \sum_{n \geq 1} c(n)q^n \qquad \text{mit } c(n) \in \NN
	\,.
	\end{align*}
	Also hat die $j$-Funktion eine Fourierentwicklung in $q$, wobei die Koeffizienten positive ganzen Zahlen sind.
	
	\item Man zeigt leicht: $\frac{1}{\prod_{n\geq1} (1-q^n)} = 1 + \sum_{n\geq1} p(n)q^n$ wobei $p(n)$ die Anzahl der Partionen von $n$ ist, d.\,h. die Anzahl der Zerlegungen von $n$ als Summe positiver, ganzer Zahlen (Beispielsweise $p(4) = 5$, denn $4 = 3 + 1 = 2 + 2 = 2 + 1 + 1 = 1 + 1 + 1 + 1$).
	Man sagt: die erzeugende Reihe von $p(n)$ wird durch $\frac{1}{\prod_{m\geq1} (1-q^n)}$ gegeben.
	
	\emph{Beachte} $1 + \sum_{n\geq1} p(n)q^n = \frac{e^{\pi i\frac{z}{12}}}{\eta(z)}$ wobei $\eta(z) = e^{\pi i\frac{z}{12}} \prod_{n\geq1} (1-q^n)$ die sogenannte \myemph{Dedekindische $\eta$-Funktion} ist.
	Beachte $\eta^{24} = \Delta$.
	$\eta$ sollte also eine Modulform vom Gewicht $\frac{1}{2}$ sein.
	Mit Hilfe der Theorie der Modulformen kann man zeigen $p(n) \sim \frac{1}{4\sqrt 3 n} \cdot e^{\pi \sqrt{\frac{3}{2}n}}$ für $n\to\infty$ (hier $a(n) \sim b(n)$ genau dann, wenn $\lim_{n\to\infty} \frac{a(n)}{b(n)} = 1$).
\end{beme-list}

%chapter 2
\chapter{Heckeoperatoren}

\section{Vorbemerkung, Motivation}
\label{VorbemerkungHecke}

\begin{defi}
Definiere die Gruppe
\[
\GL_2^+(\RR) = \Set{ \abcd \in M_2(\RR) \Bigm| ad - bc > 0 }
\,,
\]
welche $\SL_2(\RR)$ als Untergruppe enthält.
\end{defi}

\begin{defi-list}
\item Seien $z \in \HH$ und 
\[
M = \abcd \in \GL_2^+(\RR)
\,,
\]
dann setze
\[
M \circ z := \frac{az+b}{cz+d}
\,.
\]
\item Für $k \in \ZZ$, $M \in \GL_2^+(\RR)$ und $f \colon \HH \to \CC$ setze
\[
(f |_k M)(z) := (ad - bc)^{\frac{k}{2}} (cz + d)^{-k} f \left( M \circ z \right)
\,.
\]
\end{defi-list}

Diese Definitionen verallgemeinern die früheren Definitionen für $\SL_2(\RR)$ (siehe \ref{DefSL2Z}). Beachte, dass weiterhin für alle $\lambda \in \RR_+$ gilt:
\[
f |_k 
\begin{pmatrix}
\lambda & 0\\
0 & \lambda
\end{pmatrix}
= f
\,.
\]

\begin{lemm-list} \label{LemmaGL2+R}
\item Die Abbildung $(M, z) \mapsto M \circ z$ definiert eine Operation von $\GL_2^+(\RR)$ auf $\HH$.
\item Man hat $f |_k M_1 M_2 = (f |_k M_1) |_k M_2$.
\end{lemm-list}

\begin{bewe-list}
\item Rechne nach und beachte hierbei, dass $\Im \Bigl( \frac{az+b}{cz+d} \Bigr) = (ad-bc) \frac{\Im z}{\abs{cz+d}^2}$.
\item Für $M = \abcd \in \GL_2^+(\RR)$ setze $j(M, z) := cz+d$. Dann gilt für beliebige Matrizen $M_1, M_2 \in \GL_2^+(\RR)$, dass
\[
j(M_1 M_2, z) = j(M_1, M_2 \circ z) \cdot j(M_2, z)
\,,
\]
woraus wegen $(cz+d)^{-k} = j(M, z)^{-k}$ die Behauptung folgt.
\end{bewe-list}

\emph{Ziel:} Definition gewisser linearer Operatoren $T \colon M_k \to M_k$ auf den Vektorräumen $M_k$ (Modulformen vom Gewicht $k \in \ZZ$) durch geeignete Mittelbildung.

\emph{Idee:} Sei $\mathcal{M} \subset \GL_2^+(\RR)$ eine Teilmenge mit folgenden Eigenschaften (mit $\cdot$ die gewöhnliche Matrizenmultiplikation): \begin{enumerate}
\item $\Gamma (1) \cdot \mathcal M \subset \mathcal M$
\item $\mathcal M \cdot \Gamma(1) \subset \mathcal M$
\item $\mathcal M$ zerfällt in endlich viele disjunkte Rechtsnebenklassen, d.h.
\[
\mathcal M = \dot{\bigcup_{M \in \linksmodulo{\Gamma(1)}{\mathcal M}}} \Gamma(1) \cdot M
\,,
\]
wobei die Vereinigung disjunkt und endlich ist. 
\end{enumerate}
Für eine Modulform $f \in M_k$ setze dann
\[
f | T_{\mathcal M} := \sum_{M \in \linksmodulo{\Gamma(1)}{\mathcal M}} f|_k M
\,.
\]
Dann ist $f | T_{\mathcal M}$ wohldefiniert, denn jede Rechtsnebenklasse $\Gamma (1) \cdot M \in \linksmodulo{\Gamma(1)}{\mathcal M}$ besteht aus Vertretern der Form $N M$ mit $N \in \Gamma (1)$ und es gilt
\[
f |_k N M = (f |_k N) |_k M = f |_k M
\,
\]
wegen Lemma \ref{LemmaGL2+R}, ii) und $f |_k N = f$ für beliebiges $N \in \Gamma (1)$, da $f \in M_k$.

\emph{Ferner:} Sei eine Matrix $N \in \Gamma(1)$ gegeben. Dann ist 
\[
(f | T_{\mathcal M}) |_k N = \sum_{M \in \linksmodulo{\Gamma(1)}{\mathcal M}} f |_k M N = \sum_{M \in \linksmodulo{\Gamma(1)}{\mathcal M}} f |_k M = f | T_{\mathcal M}
\,,
\]
denn mit $M$ durchläuft auch $MN$ ein Vertretersystem der Rechtsnebenklassen. (Begründung: Sind zwei Matrizen $M_1, M_2 \in \mathcal M$ nicht äquivalent unter Linksmultiplikation mit $\Gamma(1)$, so gilt dies trivialerweise auch für $M_1N, M_2N$. Auch ist
\[
\mathcal M N = \Bigl( \dot{\bigcup_{M \in \linksmodulo{\Gamma(1)}{\mathcal M}}} \Gamma(1) \cdot M \Bigr) N = \dot{\bigcup_{M \in \linksmodulo{\Gamma(1)}{\mathcal M}}} \Gamma(1) \cdot M N = \mathcal M
\,,
\]
denn nach Voraussetzung gilt sowohl $\mathcal M N \subset \mathcal M$ als auch $\mathcal M = \mathcal M \inv{N}N \subset \mathcal M N$.)

\emph{Folgerung:} $f | T_{\mathcal M}$ hat das Transformationsverhalten einer Modulform vom Gewicht $k$.

\section{Die Heckeoperatoren $T(n)$}

\begin{defi}\label{defiM(n)}
Sei $n \in \NN$. Setze 
\[
\mathcal M(n) := \Set{ \abcd \in M_2(\ZZ) \Bigm| ad - bc = n }
\,.
\]
\end{defi}

\emph{Beobachtung:} $\mathcal M(n)$ ist invariant unter Links- und Rechtsmultiplikation von $\Gamma (1)$. 

\begin{lemm}\label{lemma:Mn_schoen}
\[
\mathcal M (n) = \dot{\bigcup_{\substack{ad = n\\ d > 0\\
b (\operatorname{mod} d)}}} \Gamma(1) \cdot 
\begin{pmatrix}
a & b\\
0 & d
\end{pmatrix}
\,,
\]
wobei die Vereinigung über alle Matrizen $\begin{pmatrix}a&b\\0&d\end{pmatrix}$ geht, derart dass $a,b,d \in \ZZ$, $ad = n$, $d > 0$, und $b$ ein volles Restsystem modulo $d$ durchläuft (also z.B. $b \in \Set{1, 2, \ldots, d}$).
\end{lemm}

\begin{bewe}
Die Inklusion $\supseteq$ ist klar, zeige also noch $\subset$. Sei dazu $M = \abcd \in \mathcal M (n)$. Da $ad-bc = n > 0$, können $a$ und $c$ nicht gleichzeitig Null sein. Deswegen existiert $t := \ggt (a,c) \in \NN$. Also sind $-\frac ct$ und $\frac at$ teilerfremd und es existieren $\alpha, \beta \in \ZZ$ mit
\[
\begin{pmatrix}
\alpha & \beta\\
-\frac ct & \frac at
\end{pmatrix}
\in \Gamma (1)
\,.
\]
Dann ist
\[
\begin{pmatrix}
\alpha & \beta\\
-\frac ct & \frac at
\end{pmatrix} \abcd
= \begin{pmatrix}
* & *\\ 0 & *
\end{pmatrix}
\,.
\]
Man kann also voraussetzen, dass $c = 0$. Wegen $\det M = n$ gilt dann $ad = n$. Multipliziert man gegebenenfalls mit $- E$, so kann man annehmen, dass $d > 0$. Schließlich multipliziere für $\nu \in \ZZ$ mit 
\[
\begin{pmatrix}
1 & \nu\\
0 & 1
\end{pmatrix} 
\in \Gamma (1) \Ra 
\begin{pmatrix}
1 & \nu\\
0 & 1
\end{pmatrix}
\begin{pmatrix}
a & b\\
0 & d
\end{pmatrix}
= 
\begin{pmatrix}
a & b + \nu d\\
0 & d
\end{pmatrix}
\,.
\]
Durch geeignete Wahl von $\nu \in \ZZ$ kann man erreichen, dass $b + \nu d$ in einem vorgegebenen Restsystem modulo $d$ liegt. Damit ist die Inklusion $\subset$ gezeigt.

Noch zu zeigen ist, dass die Vereinigung disjunkt ist (die Endlichkeit ist nach Konstruktion klar). Angenommen, für zwei Matrizen
\[
\begin{pmatrix}
a & b\\
0 & d
\end{pmatrix},
\begin{pmatrix}
a' & b'\\
0 & d'
\end{pmatrix}
\]
(mit $ad = n = a'd'$, $d > 0$, $d' > 0$ und $b, b'$ Vertreter zweier Restklassen modulo $d$ bzw.\ $d'$) existiere ein $N \in \Gamma (1)$, sodass
\[
N 
\begin{pmatrix}
a & b\\
0 & d
\end{pmatrix}
= 
\begin{pmatrix}
a' & b'\\
0 & d'
\end{pmatrix}
\,.
\]
Dann folgt, dass die untere linke Komponente von $N$ Null ist, $N \in \SL_2(\ZZ)$ also die Gestalt
\[
N = 
\begin{pmatrix}
\pm 1 & \nu\\
0 & \pm 1
\end{pmatrix}
\]
mit $\nu \in \ZZ$ hat. Damit ist 
\[
N
\begin{pmatrix}
a & b\\
0 & d
\end{pmatrix}
=
\begin{pmatrix}
\pm 1 & \nu\\
0 & \pm 1
\end{pmatrix}
\begin{pmatrix}
a & b\\
0 & d
\end{pmatrix}
= 
\begin{pmatrix}
\pm a & \pm b + \nu d\\ 
0 & \pm d
\end{pmatrix}
\overset !=
\begin{pmatrix}
a' & b'\\ 
0 & d'
\end{pmatrix}
\,.
\]
Es folgt $d' = \pm d$ und da $d, d' > 0$ nach Voraussetzung bereits $d = d'$. Die Diagonalelemente von $N$ sind also beide $+1$ und es folgt $b' = b + \nu d$. Wegen $d = d'$ stammen $b, b'$ beide aus dem gleichen Restsystem modulo $d$. Da sie sich nur um ein Vielfaches von $d$ unterscheiden, folgt
\[
\begin{pmatrix}
a' & b' \\ 0 & d'
\end{pmatrix} = \begin{pmatrix}
a & b \\ 0 & d
\end{pmatrix}
\,.
\]
\end{bewe}

\begin{defi}
Sei $n \in \NN$. Man setze dann für $f \in M_k$
\[
f | T(n) := n^{\frac k2 - 1} \sum_{M \in \linksmodulo{\Gamma(1)}{\mathcal M (n)}} f |_k M
\,.
\]
\end{defi}

\begin{satz-list}\label{TnEndoMk}
\item Durch $T(n)$ wird eine lineare Abbildung $M_k \to M_k$ definiert. Diese lässt $S_k$ invariant (gemeint ist: Spitzenformen werden auf Spitzenformen geschickt). Man nennt $T(n)$ den $n$-ten \myemph{Hecke-Operator}.
\item Ist $f = \sum_{m \geq 0} a(m) q^m \in M_k$, so gilt
\[
f | T(n) = \sum_{m \geq 0} \left( \sum_{d | (m,n)} d^{k-1} a\left(\frac{mn}{d^2}\right) \right) q^m
\,.
\]

\emph{Beachte:} Der konstante Term von $f | T(n)$ ist gleich
\[
\sum_{d|n} d^{k-1} a(0) = \sigma_{k-1}(n) a(0)
\]
\end{satz-list}

\begin{bsp}
Sei $n = p$ prim. Dann ist
\begin{align*}
f | T(p) &= p^{\frac k2 - 1} \sum_{m \geq 0} \left( \sum_{d | (m,p)} d^{k-1} a\left(\frac{mp}{d^2}\right) \right) q^m\\
&= p^{\frac k2 - 1} \sum_{m \geq 0} \Bigl( a(mp) + p^{k-1} a\Bigl( \frac mp\Bigr) \Bigr) q^m
\,,
\end{align*}
wobei $a\left(\frac mp\right) := 0$ für $p \!\! \not | \, m$, denn
\[
\sum_{d | (m,p)} d^{k-1} a \left( \frac {mn}{d^2} \right) = a(mp) + 
\begin{cases}
0 & \text{ falls } p \!\! \not | \, m\\ 
p^{k-1} a \left(\frac mp \right) & \text{ falls } p | \, m
\end{cases}
\]
\end{bsp}

\begin{bewe-list}
\item Nach den Überlegungen in \autoref{VorbemerkungHecke} wissen wir, dass $f | T(n)$ das Transformationsverhalten einer Modulform vom Gewicht $k$ besitzt. Auch ist $f | T(n)$ als Summe holomorpher Funktionen selbst holomorph auf $\HH$. Zu zeigen verbleibt noch, dass $f | T(n)$ holomorph in $\infty$ ist und den Raum $S_k$ invariant lässt. Beides folgt direkt aus Teil ii) des Satzes.

\item Benutze \autoref{lemma:Mn_schoen}, damit folgt
\begin{align*}
f|T(n)
&= n^{\frac{k}{2}-1} \sum_{\substack{ad=n\\ d>0 \\ b\bmod d}} f|_k \mymat ab0d \\
&= n^{\frac{k}{2}-1} \sum_{\substack{ad=n\\ d>0 \\ b\bmod d}} n^{\frac{k}{2}} d^{-k} f\Bigl(\frac{az+b}{d}\Bigr) \\
&= n^{k-1} \sum_{\substack{m>0\\ ad=n,\ d>0\\ b\bmod d}} d^{-k} a(m) e^{2\pi im \frac{az+b}{d}} \\
&= n^{k-1} \sum_{\substack{m>0\\ d|n,\ d>0}} d^{-k} a(m) e^{2\pi im \frac{n}{d^2}z} \biggl( \sum_{b\bmod d} e^{2\pi im \frac{b}{d}}\biggr)
\,.
\end{align*}

% einschub
Es gilt
\[
\sum_{b \bmod d} e^{2\pi im \frac{b}{d}}
= \begin{cases}
0 & \text{falls } d\nmid m \\
d & \text{falls } d| m
\end{cases}
\]

Allgemein $1+q+\ldots q^{N-1} = \frac{q^n-1}{q-1} = 0$, falls $q\not=1$ und $q^N = 1$, wende dies an mit $q=e^{2\pi i \frac{m}{d}}$, $N=d$.
% einschub ende
Damit erhalten wir, wobei zu beachten ist, dass die Vertauschung wegen absoluter Konvergenz gerechtfertigt sind
\begin{align*}
f|T(n)
&= n^{k-1} \sum_{\substack{m \geq 0 \\ m \equiv 0 \bmod d \\ d|n,\ d > 0}} d^{-k+1} a(m) e^{2\pi i\frac{mn}{d^2}z} &&(m\mapsto md) \\
&= \sum_{\substack{m \geq 0\\ d|n,\ d>0}} \Bigl(\frac{n}{d}\Bigr)^{k-1} a(md) e^{2\pi i \frac{mn}{d}z} &&\Bigl(d\mapsto\frac{n}{d}\Bigr) \\
&= \sum_{\substack{m \geq 0 \\ d|n,\ d>0}} d^{k-1} a\Bigl(\frac{mn}{d}\Bigr) e^{2\pi imdz} && (md\mapsto m) \\
&= \sum_{\substack{m\geq 0 \\ m\equiv 0 \bmod d \\ d|n,\ d>0}} d^{k-1} a\Bigl(\frac{mn}{d^2}\Bigr) e^{2\pi imz} \\
&= \sum_{m\geq 0} \Biggl(\sum_{d|m,\ d|n} d^{k-1} a\biggl(\frac{mn}{d^2}\biggr)\Biggr) q^m
\,.
\end{align*}
\end{bewe-list}

\begin{satz}\label{TmTn}
	Für alle $m$, $n \in \NN$ gilt
	\[
	T(m) T(n) = \sum_{d | (m,n)} d^{k-1} T \left( \frac {mn}{d^2} \right)
	\]
	Speziell gilt (vergleiche mit Ramanujan-$\tau$-Funktion):
	\begin{enumerate}
		\item $T(n)T(m) = T(mn)$ falls $\ggt(m,n) = 1$
		\item $T(p) T(p^\nu) = T(p^{\nu+1}) + p^{k-1} T(p^{\nu-1})$ für $p$ prim und $\nu \geq 1$.
	\end{enumerate}
	
	Beachte dass (ii) äquivalent ist zur Identität
	\[
	\frac{1}{1-T(p)X+p^{k-1}X^2} = \sum_{\nu \geq 0} T(p^\nu) X^\nu
	\]
\end{satz}

\begin{bewe}
	in mehreren Schritten:
	1. Schritt: Beweis von (i):
	Seien $m$, $n$ teilerfremd.
	Benutze \autoref{lemma:Mn_schoen}, dann gilt
	\begin{align*}
	f|T(m)T(n)
	&= (mn)^{\frac{k}{2}-1} \sum_{\substack{ad=m \vphantom{d'} \\ d>0,\ \vphantom{b'}b\bmod d}} \Biggl(\sum_{\substack{a'd' = n\\ d'>0,\ b' \bmod d'}} f|_k \mymat ab0d \mymat{a'}{b'}{0}{d'} \Biggr) \\
	&= (mn)^{\frac{k}{2}-1} \sum_{\substack{ad=m \vphantom{d'} \\ d>0,\ \vphantom{b'}b\bmod d}} \Biggl(\sum_{\substack{a'd' = n\\ d'>0,\ b' \bmod d'}} f|_k \mymat{aa'}{ab'+bd'}{0}{dd'} \Biggr)
	\,.
	\end{align*}
	Durchläuft $d$ alle positiven Teiler von $m$ und $d'$ alle positiven Teiler von $n$, so durchläuft $D := dd'$ alle positiven Teiler von $mn$, denn $\ggt(m, n) = 1$.
	Setzt man $A := aa'$, so gilt dann $AD = mn$.
	Ferner gilt: Durchläuft $b$ ein volles Restsystem $\bmod d$ und $b'$ ein solches $\mod d'$, so durchläuft $B = ab+bd'$ ein volles Restsystem $\bmod dd'$, denn in der Tat genügt es zu zeigen, dass diese Zahlen inkongruent $\bmod dd'$ sind, denn dann sind dies genau $dd'$ paarweise inkongruente Zahlen.
	Angenommen
	\[
	ab_1' + b_1d' \equiv ab_2' +b_2d' \mod dd'
	\,,
	\]
	dann gilt
	\[
	a(b_1' - b_2') \equiv d'(b_2 - b_1) \mod dd'
	\,.
	\]
	Dies impliziert $a(b_1'- b_2') \equiv 0 \mod d'$. Aber $\ggt(a, d') = 1$, denn $a|m$ und $d'|n$ und $\ggt(m, n) = 1$ nach Voraussetzung.
	Also folgt $b_1' \equiv b_2' \mod d'$, also $b_1' = b_2'$.
	Es folgt jetzt $b_2 \equiv b_1 \mod d$, also $b_2 = b_1$.
	Also folgt die Behauptung.
	Und damit
	\[
	f|T(m)T(n)
	= (mn)^{\frac{k}{2}-1} \sum_{\substack{AD = mn\\ D > 0,\ B \bmod D}} f|_k \mymat AB0D
	= f|_kT(mn)
	\,.
	\]
	
	
	2. Schritt: Beweis von (ii):
	Es gilt nach \autoref{lemma:Mn_schoen}:
	\[
	f|T(p) = p^{\frac{k}{2}-1} \Bigl(f|_k \mymat p001 + \sum_{\mu \bmod p} f|_k \mymat 1\mu0p \Bigr)
	\]
	und
	\[
	f|T(p^\nu) = (p^\nu)^{\frac{k}{2}-1} \sum_{\substack{0 \leq \beta \leq \nu \\ b \bmod p^\beta}} f|_k \mymat{p^{\nu-\beta}}b0{p^\beta}
	\,.
	\]
	
	Dann
	\begin{align}\label{eq:tptpnu}
	f|T(p)T(p^\nu)
	&= (p^{\nu+1})^{\frac{k}{2}-1} \biggl( \sum_{\substack{0 \leq \beta \leq \nu \\ b \bmod p^\beta}} f|_k \mymat p001 \mymat {p^{\nu-\beta}}b0{p^\beta} + \sum_{\substack{0 \leq \beta \leq \nu \\ b \bmod p^\beta \\ \mu\bmod p}} f|_k \mymat 1\mu0p \mymat {p^{\nu-\beta}}b0{p^\beta} \biggr) \nonumber \\
	&= (p^{\nu+1})^{\frac{k}{2}-1} \biggl( \sum_{\substack{0 \leq \beta \leq \nu\\b \bmod p^\beta}} f|_k \mymat {p^{\nu +1-\beta}}{pb}{0}{p^\beta} + \sum_{\substack{0 \leq \beta \leq \nu \\ b \bmod p^\beta \\ \mu \bmod p}} f|_k \mymat{p^{\nu-\beta}}{b+\mu p^\beta}0{p^{\beta+1}} \biggr)
	\end{align}
	Betrachte 2. Summe in \eqref{eq:tptpnu}:
	Durchläuft $b$ ein Restsystem modulo $p^\beta$ und $\mu$ ein Restsystem modulo $p$, so durchläuft $b+\mu p^\beta$ ein solches modulo $p^{\beta+1}$ (denn insgesamt $p^{\beta+1}$ Zahlen, paarweise inkongruent modulo $p^{\beta+1}$).
	Man sieht daher, dass die 2. Summe gleich
	\[
	f|T(p^{\nu+1}) - (p^{\nu+1})^{\frac{k}{2}} f|_k \mymat {p^{\nu+1}}001
	\]
	ist.
	
	Betrachte 1. Summe in \eqref{eq:tptpnu}. Diese ist gleich
	\[
	(p^{\nu + 1})^{\frac{k}{2}-1} \biggl(f|_k \mymat{p^{\nu+1}}001 + \sum_{\substack{1 \leq \beta \leq \nu \\ b \bmod p^\beta}} f|_k \mymat{p^{\nu+1-\beta}}{pb}0{p^\beta} \biggr)
	\,.
	\]
	Man erhält also
	\[
	f|_k T(p)T(p^\nu)
	= f|T(p^{\nu+1}) + (p^{\nu+1})^{\frac{k}{2}-1} \underbrace{\sum_{\substack{1 \leq \beta \leq \nu\\ b \bmod p^\beta}} f|_k \mymat p00p |_k \mymat {p^{\nu-\beta}}b0{p^{\beta-1}}
	}_{=: R}
	\]
	In $R$ ersetze $\beta$ durch $\beta + 1$, erhalte
	\[
	R = \sum_{\substack{0 \leq \beta \leq \nu-1\\ b \bmod p^{\beta+1} }} f|_k \mymat {p^{\nu-1-\beta}}b0{p^{\beta}}
	\,,
	\]
	Man setze $b = \tilde b + \mu p^\beta$ wobei $\mu$ modulo $p$ und $\tilde b$ modulo $p^\beta$ läuft
	\[
	R = \sum_{\substack{ 0 \leq \beta \leq \nu-1\\ \tilde b \bmod p^\beta \\ \mu \bmod p}} f|_k \mymat 1\mu01 |_k \mymat {p^{\nu-1-\beta}}{\tilde b}0{p^\beta}
	\]
	da $f$ Periode 1 hat, erhält man
	\[
	(p^{\nu+1})^{\frac{k}{2}-1}R
	= p^{k-1}(p^{\nu-1})^{\frac{k}{2}-1} \sum_{\substack{0 \leq \beta \leq \nu-1 \\ \tilde b \bmod p^\beta}} f|_k \mymat {p^{\nu-1-\beta}}{\tilde b}0{p^\beta}
	= p^{k-1} f|_kT(p^{\nu-1})
	\]
	
	3. Schritt: zeige durch Induktion nach $s \in \NN$ (Übungsaufgabe), dass 
	\[
	T(p^\nu) T(p^s) = \sum_{\alpha = 0}^{\min \{\nu,\,s\}} (p^\alpha)^{k-1} T(p^{\nu + s - 2\alpha})
	\,,
	\]
	was sich mit Teilern der Form $d = p^\alpha$ umschreiben lässt zu
	\[
	T(p^\nu) T(p^s) = \sum_{d \vert (p^\nu, p^s)} d^{k-1} T\Bigl(\frac{p^{\nu + s}}{d^2}\Bigr)
	\,.
	\]
	
	4. Schritt: der allgemeine Fall! Induktion über die verschiedenen Primteiler von $m$. Sei $m = p^r m'$, $n = p^s n'$ mit $p \nmid m', p \nmid n'$. Dann folgt mit i), dass
	\begin{align*}
	T(m)T(n)
	&= T(m'p^r) T(n'p^s)
	= T(m')T(p^r)T(n')T(p^s) \\
	&= T(m')T(n') T(p^r) T(p^s)
	\,.
	\end{align*}
	Wendet man dieses Argument nun induktiv auf $T(m')T(n')$ und weitere gemeinsame Primteiler an, so kann man davon ausgehen, dass $m', n'$ nach endlich vielen Iterationen teilerfremd sind. Dann kann man mit i) und Schritt 3 schreiben
	\[
	T(m)T(n) = \biggl(\sum_{d \vert (m',n')} d^{k-1} T\Bigl(\frac {m'n'}{d^2}\Bigr)\biggr) \biggl(\sum_{t \vert (p^r, p^s)} t^{k-1} T\Bigl(\frac{p^rp^s}{t^2}\Bigr) \biggr)
	\,,
	\]
	was sich nach erneuter Anwendung von i) vereinfacht zu
	\[
	T(m)T(n) = \sum_{\substack{d \vert (m',n')\\t \vert (p^r, p^s)}} (dt)^{k-1} T\Bigl(\frac {p^rm'p^sn'}{(dt)^2}\Bigr)
	\]
	und mit $D = dt$ schließlich zu
	\[
	T(m)T(n) = \sum_{D \vert (m,n)} D^{k-1} T\Bigl(\frac {mn}{D^2}\Bigr)
	\,.
	\]
\end{bewe}



%chapter 3
\chapter{Das Petersson'sche Skalarprodukt}

\section{Invariantes Maß und Skalarprodukt}

\emph{Ziel:} Definition eines \glqq{}natürlichen\grqq{} Skalarprodukts auf $S_k$. Hierzu benötigt man zunächst ein $\Gamma(1)$-invariantes Maß auf $\RR^2$.

\begin{defi}
Für $z = x + iy \in \HH$ setze man
\[
\opd \omega(z) := \frac {\opd x \opd y}{y^2}
\,.
\]
\end{defi}

\begin{satz}
	Die Differentialform $\opd w = \frac{\opd y \opd y}{y^2}$ für $z=x+iy\in\HH$ ist $\SL_2(\RR)$-invariant, d.\,h. $\opd w(M\circ z) = \opd w(z)$ für alle $M\in\SL_2(\RR)$.
\end{satz}

\begin{bewe}
Es gilt $\opd \omega(z) = \frac {i}{2y^2} \opd z \conj{\opd z}$, denn
\begin{align*}
\opd z \conj{\opd z} &= (\opd x + i \opd y)(\opd x - i \opd y)\\
&= \opd x \opd x - i \opd x \opd y + i \opd y \opd x + \opd y \opd y\\
&= 0 - i \opd x \opd y - i \opd x \opd y + 0\\
&= -2i \opd x \opd y\,.
\end{align*}
Sei nun $M = \abcd \in \SL_2(\RR)$, dann folgt unter Verwendung von
\[
\frac {\opd (M \circ z)}{\opd z} = \frac {\opd \frac{az + b}{cz + d}}{\opd z} = \frac {a(cz+d) - (az+b)c}{(cz+d)^2} = \frac{1}{(cz+d)^2}
\]
die Behauptung nach
\begin{align*}
\opd \omega(M \circ z) &= \frac {i}{2 (\Im \left(M \circ z\right))^2} \opd (M \circ z) \conj{\opd (M \circ z)}\\
&= \frac {i}{2 \frac{y^2}{\abs{cz+d}^4}} \frac{\opd z}{(cz+d)^2} \conj{\frac {\opd z} {(cz+d)^2}}\\
&= \frac {i}{2 \frac{y^2}{\abs{cz+d}^4}} \frac{\opd z}{(cz+d)^2} \frac {\conj{\opd z}} {\conj{(cz+d)^2}}\\
&= \frac{i \abs{cz+d}^4}{2 y^2} \cdot \frac {1}{\abs{cz+d}^4} \cdot \opd z \conj{\opd z}\\
&= \frac {i}{2y^2} \opd z \conj{\opd z}\\
&= \opd \omega(z)\,.
\end{align*}
\end{bewe}

\emph{Ansatz:} Seien $f, g \in S_k$, dann setze
\[
	\scalarprd fg := \int_{\closure{\mathcal F}} y^k f(z) \conj{g(z)} \opd \omega
	\,,
\]
wobei $\mathcal F$ ein Fundamentalbereich ist.

\begin{beme}\label{beme:d_eta}
	Sei $\eta = \frac{\opd z}{y}$. Dann gilt $\opd \eta = \frac{\opd x \opd y}{y^2}$, denn
	\[
	\opd \eta
	= \opd \Bigl(\frac{\opd x}{y} + i\frac{\opd y}{y}\Bigr)
	= -\frac{1}{y^2} \opd y \opd x + i\Bigl(-\frac{1}{y^2}\Bigr) \opd y \opd y
	= \frac{\opd x \opd y}{y^2}
	\,.
	\]
\end{beme}

\begin{erin}
	Eine Teilmenge $\F \subset \HH$ heißt Fundamentalbereich (für $\Gamma(1)$), falls gilt:
	\begin{enumerate}
		\item $\F$ ist offen,
		\item für alle $z\in\HH$ existiert $M \in \Gamma(1)$ mit $M \circ z \in \closure \F$,
		\item sind $z_1$, $z_2 \in \F$ und $z_2 = M \circ z_1$ mit $M \in \Gamma(1)$, dann gilt $M = \pm E$ und $z_1 = z_2$.
	\end{enumerate}
\end{erin}

\emph{Beobachtung}: Für jede Teilmenge $A \subset \CC \cong \RR^2$ ist der Rand $\partial A$ abgeschlossen, daher meßbar. Wir werden oft fordern, dass $\partial F$ eine Nullmenge ist.

\begin{bsp}\label{bsp:fundamentalbereich}
	Der Rand des Standardfundamentalbereich \[\F_1 = \Set{z = x+iy\in\HH \mid \abs x < \frac{1}{2}, \abs z > 1}\] ist eine Nullmenge.
\end{bsp}

\begin{satz}\label{satz:int_fundamentalbereich_invariant}
	Seien $\F_1$ und $\F_2$ Fundamentalbereiche derart, dass $\partial F_1$ und $\partial F_2$ Nullmengen sind.
	Sei $f\colon \HH \ra \CC$ meßbar und $\Gamma(1)$-invariant, d.\,h. $f(M \circ z) = f(z)$ für alle $M \in \Gamma(1)$.
	Ferner gelte 
\[
	\int_{\closure{\F_1}} \abs f \opd w < \infty
	\,,
\]
also dass $\abs f$ über $\closure{\F_1}$ integrierbar ist (dies impliziert, dass $f$ über $\closure{\F_1}$ integrierbar ist).
	
	Dann ist $f$ auch über $\closure{F_2}$ integrierbar und
	\[
	\int_{\closure{\F_1}} f \opd w
	= \int_{\closure{\F_2}} f \opd w
	\,.
	\]
\end{satz}

\begin{bewe}
	Nach Eigenschaft (ii) eines Fundamentalbereichs gilt (mit $\Gamma(1)' = \modulo{\Gamma(1)}{\pm E}$)
	\[
	\HH = \bigcup_{M \in \Gamma(1)'} M^{-1} \circ \closure{\F_1} = \bigcup_{M \in \Gamma(1)'} M \circ \closure{\F_2}
	\,.
	\]
	
	Nach Eigenschaft (iii) gilt $M \circ \F_1 \cap N \circ \F_1 = \emptyset$ für $M \not= \pm N$.
	Da $\closure{\F_1} = \F_1 \cup \partial\F_1$ und $\partial F_1$ eine Nullmenge, folgt
	\[
	M \circ \closure{\F_1} \cap N\circ\closure{\F_1} \text{ eine Nullmenge für } M\not=\pm N
	\,,
	\]
	denn
	\begin{align*}
	M \circ \closure{\F_1} \cap N \circ \closure{\F_1}
	&= M \circ (\F_1 \cup \partial \F_1) \cap N \circ (\F_1 \cup \partial \F_1) \\
	&= (M \circ \F_1 \cup M \circ (\partial \F_1)) \cap (N \circ \F_1 \cup N \circ (\partial F_1)) \\
	&= (M \circ \F_1 \cap N \circ \F_1) \cup (M \circ F_1 \cap N \circ (\partial \F_1)) \cup \ldots
	\,.
	\end{align*}
	
	Es gilt
	\[
	\int\limits_{\closure{\F_1}} f \opd w
	= \int\limits_{\HH \cap \closure{\F_1}} f \opd w
	= \int\limits_{\bigcup\limits_{M \in \Gamma(1)'} \!\!\!\!\!\! M \circ \closure{\F_2} \cap \closure{\F_1}} f \opd w
	\,,
	\]
	wobei zu beachten ist, dass es sich um eine abzählbare Vereinigung von meßbaren Mengen handelt. Da die paarweisen Durchschnitte jeweils Maß Null haben, gilt die abzählbare Additivität des Integrals:
	\begin{align*}
	\int\limits_{\closure{\F_1}} f \opd w
	&= \sum_{M \in \Gamma(1)'} \int\limits_{M\circ \closure{\F_2} \cap \closure{\F_1}} f \opd w
	= \sum_{M \in \Gamma(1)'} \ \int\limits_{\closure{\F_2} \cap M^{-1} \circ \closure{F_1}} f(M \circ z) \opd w(M \circ z) \\
	&= \sum_{M \in \Gamma(1)'}\ \int\limits_{\closure{F_2} \cap M^{-1} \circ \closure{F_1}} f \opd w
	= \ldots = \int_{\closure{F_2}} f\opd w
	\,.
	\end{align*}
\end{bewe}

\begin{bsp}
	Für jeden Fundamentalbereich $\F$, so dass $\partial \F$ eine Nullmenge ist, gilt
	\[
	\vol\bigl(\linksmodulo{\Gamma(1)}\HH\bigr) = \int\limits_{\closure \F} \opd w = \frac{\pi}{3} < \infty
	\,.
	\]
\end{bsp}

\begin{bewe}
	Es genügt nach \autoref{satz:int_fundamentalbereich_invariant} den Fall von $\F = \F_1$ zu betrachten, wobei $\F_1$ der Standard-Fundamentalbereich ist (siehe \autoref{bsp:fundamentalbereich}).
	
	Es gilt für $\F_c := \F_1 \cap \Set{z \in \HH \mid y \leq c}$
	\[
	\int\limits_{\closure{\F_1}} \frac{\opd x \opd y}{y^2}
	= \lim_{c \to \infty} \int\limits_{\closure{\F_c}} \frac{\opd x \opd y}{y^2}
	= \lim_{c \to \infty} \int\limits_{\closure{\F_c}} \opd \eta
	= \lim_{c \to \infty} \int\limits_{\partial \closure{\F_c}} \frac{\opd z}{y}
	\,,
	\]
	wobei die letzte Gleichheit wegen dem Satz von Stokes und \autoref{beme:d_eta} folgt.
	
	Das Integral über die Gerade $z(t) = ic + t$ für $t \in [-\frac{1}{2}, \frac{1}{2}]$ ergibt
	\[
	\int_{-\frac{1}{2}}^{\frac{1}{2}} \frac{1}{c} \opd t = - \frac{1}{c} \xto{c \to \infty} 0
	\,.
	\]
	
	Die Integrale über die beiden Geradenstücke heben sich auf, wegen entgegengesetzer Orientierung und da $y$ invariant unter $z \mapsto z +1$.
	Damit bleibt das Integral über den Kreisbogen, dieser wird parametrisiert durch
	\[
	z(t) = e^{it} = \cos t + i\sin t \qquad \text{für}\quad \frac{2\pi}{3} \geq t \geq \frac{\pi}{3}
	\,.
	\]
	Das Integral muss hierbei reellwertig sein und hat damit den Wert
	\[
	\int_{\frac{2\pi}{3}}^{\frac{\pi}{3}} \frac{i(\cos t + i\sin t)}{\sin t} \opd t
	= \int_{\frac{2\pi}{3}}^{\frac{\pi}{3}} \Re\Bigl(\frac{i\cos t - \sin t}{\sin t}\Bigr) \opd t
	= \int_{\frac{2\pi}{3}}^{\frac{\pi}{3}} (-1)\opd t
	= \frac{2\pi}{3} - \frac{\pi}{3} = \frac{\pi}{3}
	\,.
	\]
\end{bewe}

\begin{satz}\label{satz:spitzenformen_beschraenkt}
	Für $f\in M_k$ setze man $g(z) := y^{\frac{k}{2}} \abs{f(z)}$ für $z\in\HH$.
	Dann gilt
	\begin{enumerate}
		\item $g$ ist invariant unter $\Gamma(1)$.
		\item Ist $f \in S_k$, dann ist $g$ auf $\HH$ beschränkt.
	\end{enumerate}
\end{satz}

\begin{bewe-list}
	\item Es gilt
	\[
	g\Bigl(\frac{az+b}{cz+d}\Bigr)
	= \biggl(\Im \Bigl(\frac{az+b}{cz+d}\Bigr)\biggl)^{\frac{k}{2}} \abs{f\Bigl(\frac{az+b}{cz+d}\Bigr)}
	= \Bigl(\frac{y}{\abs{cz+d}^2}\Bigr)^{\frac{k}{2}} \abs{cz+d}^k \abs{f(z)} = g(z)
	\,.
	\]
	
	\item Es ist $\HH = \bigcup_{M\in\Gamma(1)} M \circ \closure{\F_1}$. Da nach (i) $g$ invariant unter $\Gamma(1)$ ist, genügt es zu zeigen, dass $g$ auf $\closure{\F_1}$ beschränkt ist.
	Aber $\closure{\F_c} := \closure{\F_1} \cap \Set{z \in \HH \mid y \leq c} \subset \closure{\F_1}$ ist kompakt für $c > 0$ beliebig.
	Wegen Stetigkeit genügt es also zu zeigen, dass $g$ für $y \to \infty$ beschränkt ist.
	Sei $f(z) = \sum_{n\geq 1} a(n) e^{2\pi inz}$ (beachte $n\geq1$ wegen $f \in S_k$), dann gilt für $y \geq c$
	\begin{align*}
	\abs{g(z)} = y^{\frac{k}{2}} \abs{f(z)}
	&= y^{\frac{k}{2}} \abs{e^{2\pi iz} \sum_{n\geq 1} a(n) e^{2\pi i(n-1)z} } \\
	&\leq y^{\frac{k}{2}} e^{-2\pi y} \left( \sum_{n\geq 1} \abs{a(n)} e^{-2\pi (n-1)y} \right) \\
	&\leq y^{\frac{k}{2}} e^{-2\pi y} e^{2\pi c} \left( \sum_{n\geq1} \abs{a(n)} e^{-2\pi nc} \right) \\
	&= \frac{y^{\frac{k}{2}}}{e^{2\pi y}} \cdot K
	\xto{y\to\infty} 0
	\,.
	\end{align*}
\end{bewe-list}

\begin{defi}
	Für $f$, $g \in M_k$ derart, dass $f \cdot g \in S_{2k}$, definieren wir das \myemph{Petersson-Skalarprodukt} durch
	\begin{equation}\label{eq:skalarprodukt}
	\scalarprd fg := \int_{\closure\F} f(z) \conj{g(z)} y^k \opd w
	\,,
	\end{equation}
	wobei $\F$ ein Fundamentalbereich wie oben ist.
\end{defi}

\begin{satz-list}
	\item \eqref{eq:skalarprodukt} ist absolut konvergent und hängt nicht von der Auswahl von $\F$ ab.
	\item $S_k \times S_k \ra \CC$, $(f,\,g) \mapsto \scalarprd fg$ ist ein Skalarprodukt auf $S_k$.
\end{satz-list}

\begin{bewe-list}
	\item Beachte $fg \in S_{2k}$ und $\abs{f(z)\conj{g(z)}} y^k = y^k \abs{f(z)g(z)}$, wende \autoref{satz:spitzenformen_beschraenkt} (ii) an und bemerke $\int_{\closure \F} \opd w < \infty$.
	Die Unabhängigkeit von $\F$ folgt aus \autoref{satz:int_fundamentalbereich_invariant}.
	\item Klar.
\end{bewe-list}


%chapter 4
\chapter{Poincaré-Reihen}

\emph{Motivation}: Die Abbildung $S_k \ra \CC$, $f \mapsto a_f(n) = \text{$n$-ter Fourierkoeffizient von $f$}$ ist ein lineares Funktional.
Nach dem Darstellungssatz von Fréchet-Riesz existiert ein eindeutig bestimmtes $\tilde P_n$ für $n\in\NN$ mit
\[
	a_f(n) = \scalarprd f{\tilde P_n} \qquad \text{für alle } f\in S_k\,.
\]

\emph{Frage}: Kann man $\tilde P_n$ explizit angeben? Antwort: ja!

\begin{defi}
	Sei $k \in 2\ZZ$, $k \geq 4$, $n\in\NN$. Dann heißt die formale Reihe
	\[
		P_n(z)
		= \frac{1}{2} \sum_{\substack{(c,d)\in\ZZ^2 \\ \ggt(c,d) = 1 \\ ad-bc = 1}} (cz+d)^{-k} e^{2\pi in \frac{az+b}{cz+d}}
		\qquad \text{für } z \in \HH
	\]
	die $n$-te Poincaré Reihe vom Gewicht $k$ für $\Gamma(1)$.
	Summiert wird über alle $(c,d) \in \ZZ^2$ mit $\ggt(c,d) = 1$ und zu jedem solchen Paar ist $(a,b) \in \ZZ^2$ zu bestimmen, so dass $ad-bc = 1$, d.\,h. $\abcd \in \Gamma(1)$).
	Dies ist unabhängig von der Auswahl von $a$, $b$, denn ist auch $a'$, $b'$ ein solches Paar, so gilt $a' = a+mc$, $b' = b + md$ für ein $m \in \ZZ$ und somit
	\[
		\frac{a'z+b'}{cz+d} = \frac{az+b}{cz + d} + m
	\]
	mit $m \in \ZZ$ und $e^{2\pi inm} = 1$.
\end{defi}
\section{Anwendungen}

\begin{beme}
Es gilt $P_0 = E_k$, wie man durch Vergleich mit \autoref{Ek_per_Gamma(1)infty} leicht einsieht.
\end{beme}

\begin{satz-list}\label{<f,Pn>}
\item Die Reihe $P_n$ konvergiert auf Kompakta in $\HH$ gleichmäßig absolut, stellt also dort eine holomorphe Funktion dar. Es gilt $P_n \in S_k$ für $n \geq 1$.
\item Es gilt 
\[
	\scalarprd {f}{P_n} = \frac{(k-2)!}{(4\pi n)^{k-1}} a_f(n)
\]
für alle $f \in S_k$ mit $f = \sum_{m\geq 1} a_f(m) q^m$.
\end{satz-list}

\begin{bewe-list}
\item Wegen $\frac{az+b}{cz+d} \in \HH$ ist
\[
	\abs{e^{2\pi i n \frac{az+b}{cz+d}}} \leq 1
\]
und daher 
\[
	\sum_{\substack{(c,d)\in\ZZ^2 \\ \ggt(c,d) = 1 \\ ad-bc = 1}} \abs{cz+d}^{-k} \cdot \abs{e^{2\pi i n \frac{az+b}{cz+d}}} \leq \sum_{\substack{(c,d)\in\ZZ^2 \\ \ggt(c,d) = 1 \\ ad-bc = 1}} \abs{cz+d}^{-k}
	\,,
\]
sodass die Reihe der Absolutbeträge nach \autoref{Ek_per_Gamma(1)infty} durch die Eisensteinreihe von Gewicht $k$ majorisiert wird. Letztere konvergiert nach FT~2 auf Kompakta in $\HH$ gleichmäßig absolut.

Zeige noch $P_n \in S_k$ für $n \geq 1$. Schreibe zunächst
\[
	P_n(z) = \frac 12 \sum_{M \in \linksmodulo{\Gamma(1)_\infty}{\Gamma(1)}} (e^n |_k M)(z)
\]
mit $e^n(z) := e^{2 \pi inz}$ und beachte, dass $e^n |_k M = e^n$ für $M \in \Gamma(1)_\infty$. Hierbei ist wie in \autoref{Ek_per_Gamma(1)infty} 
\[
	\Gamma(1)_\infty := \Set{M \in \SL_2(\ZZ) \mid M = \mymat ab0d}
	\,.
\]	
Für $P_n \in S_k$ müssen wir zeigen, dass $P_n |_k N = P_n$ für alle $N \in \SL_2(\ZZ)$ (klar, da auch $MN$ ein Vertretersystem für $\linksmodulo{\Gamma(1)_\infty}{\Gamma(1)}$ bildet) und zudem in $z = i \infty$ verschwindet. Wie im Fall der Eisensteinreihen ist hierfür zu zeigen, dass 
\[
	\lim_{z \to i\infty} P_n(z) = 0
	\text{ also }
	\lim_{\nu \to \infty} P_n(z_\nu) = 0
\]
für jede Folge von $z_\nu \in \HH$ mit $z_\nu \to i\infty$. Wegen gleichmäßiger Konvergenz gilt
\[
	\lim_{\nu \to \infty} P_n(z_\nu) = \frac 12 \sum_{\substack{(c,d)\in\ZZ^2 \\ \ggt(c,d) = 1 \\ ad-bc = 1}} \lim_{\nu \to \infty} (cz_\nu + d)^{-k} e^{2\pi in \frac{az_\nu + b}{cz_\nu + d}}
\]
und alle Grenzwerte unter der Summe sind 0. In der Tat ist der Exponentialterm wegen $\frac{az_\nu + b}{cz_\nu + d} \in \HH$ beschränkt und für $c \neq 0$ strebt $(cz_\nu + d)^{-k}$ gegen 0. Andererseits ist für $c = 0$ der vordere Term gleich $d^{-k}$ und somit beschränkt, während
\[
	\tfrac{az_\nu + b}{d} \to i\infty \quad \Ra \quad e^{2\pi in \frac{az_\nu + b}{d}} \to 0
	\,.
\]
Damit ist alles gezeigt.

\item Unter Benutzung der Darstellung
\[
	P_n(z) = \frac 12 \sum_{M \in \linksmodulo{\Gamma(1)_\infty}{\Gamma(1)}} (e^n |_k M)(z)
\]
zeigt man mit dem gleichen \glqq{}Konvolutionstrick\grqq{} wie im Beweis von \autoref{CharEk}, dass
\[
	\scalarprd {f}{P_n} = \int_0^\infty \int_{-\frac 12}^{\frac 12} f(z) e^{\conj{2\pi inz}} y^{k-2} \opd x \opd y
	\,.
\]
Man stelle sich hierzu vor, dass $\HH$ als disjunkte Vereinigung von Bildern des exakten Fundamentalbereichs unter Linksmultiplikation mit $M \in \Gamma(1)$ entsteht. Teilt man nun $\Gamma(1)_\infty$ heraus, also alle Translationen, so verbleibt noch der Streifen $\abs{x} < \frac 12$, $0 < y$. 

Es gilt weiter für beliebiges $f \in S_k$ mit Darstellung $f(z) = \sum_{m \geq 1} a(m)q^m$, wie üblich $q = \exp(2\pi iz)$ und $z = x + iy$, dass
\begin{align*}
	\scalarprd {f}{P_n} &= \int_0^\infty \int_{-\frac 12}^{\frac 12} \sum_{m \geq 1} a(m) e^{2 \pi imx}e^{-2\pi my}e^{-2\pi inx}e^{-2\pi ny}y^{k-2} \opd x \opd y\\
	&= \int_0^\infty \int_{-\frac 12}^{\frac 12} \sum_{m \geq 1} a(m) e^{2 \pi i(m-n)x}e^{-2\pi (m+n)y}y^{k-2} \opd x \opd y\,.
\end{align*}
Wegen 
\[
	\int_{- \frac 12}^{\frac 12} e^{2 \pi irx} \opd x = \delta_{r,0} := \begin{cases}1, & r = 0\\0, & r \neq 0\end{cases} \quad \text{(Kronecker-Delta)}
\]
für beliebiges $r \in \ZZ$ folgt
\begin{align*}
	\scalarprd {f}{P_n} &= \int_0^\infty \sum_{m \geq 1} a(m) \delta_{m,n} e^{-2\pi (m+n)y}y^{k-2} \opd y \\
	&= a(n) \int_0^\infty e^{-4\pi ny} y^{k-2} \opd y \\
	&= a(n) \frac{1}{(4\pi n)^{k-1}} \underbrace{\int_0^\infty e^{-y} y^{k-2} \opd y}_{= \Gamma(k-1)} \\
	&= a(n) \frac{(k-2)!}{(4\pi n)^{k-1}}
	\,.
\end{align*}
\end{bewe-list}

\begin{koro}
Die Poincaré-Reihen $\Set{ P_n \mid n \in \NN}$ zu einem festen Gewicht $k \geq 4$ mit $k$~gerade, erzeugen den Raum $S_k$.
\end{koro}

\begin{bewe}
Angenommen die $P_n$ erzeugen nicht ganz $S_k$, dann existiert ein $f \in S_k$ mit $\scalarprd {f}{P_n} = 0$ für alle $n \in \NN$. Mit \autoref{<f,Pn>}, ii) folgt hieraus aber $a(n) = 0$ für alle $n \in \NN$ und damit $f \equiv 0$.
\end{bewe}

\begin{satz}\label{satz:Pn_Fourier}
Die Reihe $P_n$ für $n \geq 1$ hat die Fourier-Entwicklung
\[
	P_n(z) = \sum_{m \geq 1} g_n(m) q^m
\]
mit
\[
	g_n(m) := \delta_{m,n} + 2\pi \cdot (-1)^{\frac k2} \cdot \left(\frac mn\right)^{\frac {k-1}2} \cdot \sum_{c \geq 1} \left[ \frac 1c \cdot K(m,n,c) \cdot J_{k-1}\left(\frac{4\pi \sqrt{mn}}c\right) \right]
	\,.
\]
Hierbei ist die Kloosterman-Summe $K$ definiert als
\[
	K(m,n,c) := \sum_{\substack{d (\operatorname{mod} c) \\ (c,d)=1}} e^{2\pi i \frac{md + n\bar{d}}{c}}
	\,,
\]
wobei $\bar d \in \ZZ$ mit $\bar d d \equiv 1 \mod c$ ist, und die Besselfunktion $J_{k-1}$ definiert als
\[
	J_{k-1}(x) := \left(\frac x2\right)^{k-1} \sum_{\ell \geq 0} \frac{(-\frac 14 x^2)^\ell}{\ell! (k-1+\ell)!}
	\,.
\]
\end{satz}

\begin{bewe}
Nach Definition ist 
\[
	P_n(z) = \frac 12 \sum_{\substack{(c,d)\in\ZZ^2 \\ \ggt(c,d) = 1 \\ ad-bc = 1}} (cz+d)^{-k} e^{2\pi in \frac{az+b}{cz+d}}
	\,.
\]
Ist $c = 0$, so folgt aus $\ggt(c,d) = 1$ bereits $d = a = \pm 1$ und unabhängig von $b \in \ZZ$ ergibt sich zweimal der Term
\[
	\frac 12 (\pm 1)^{-k} e^{2\pi in \frac{\pm z + b}{\pm 1}} = \frac 12 e^{2\pi in z} e^{\pm 2\pi in b} = \frac 12 e^{2 \pi inz}
	\,,
\]
zusammengenommen also $e^{2 \pi inz}$. Die übrigen Terme ergeben den Beitrag 
\[
	\sum_{\substack{c \geq 1, d \in \ZZ \\ \ggt(c,d) = 1 \\ ad-bc = 1}} (cz+d)^{-k} e^{2 \pi i n \frac{az+b}{cz+d}} = \sum_{c \geq 1} \sum_{\substack{d' (\operatorname{mod} c) \\ \ggt(c,d') = 1 \\ ad'-b'c = 1}} \sum_{\nu \in \ZZ} (c(z + \nu) +d')^{-k} e^{2\pi in \frac{a(z+\nu)+b'}{c(z+\nu)+d'}}
	\,.
\]
Die rechte Seite entsteht aus der linken, indem man für festes $c \geq 1$ und ein festes Vertretersystem $d' (\operatorname{mod} c)$ jedes $d \in \ZZ$ mit $\ggt(c,d) = 1$ in der Form $d = d' + c\nu$ mit $v \in \ZZ$ und $d'$ im vorgegebenen Vertretersystem schreibt. Schreibt man zudem mit geeignetem $b' \in \ZZ$ auch $b = b' + a\nu$, so wird die Bedingung $ad - bc = 1$ zu
\[
	1 = ad - bc = a(d' + c\nu) - (b' + a\nu)c = ad' - b'c
\]
und die obige Darstellung folgt durch Ausklammern von $c$ und $a$. Im Folgenden schreiben wir wieder $d$ und $b$ statt $d'$ und $b'$.

\begin{lemm}
Sei $A = \mymat abcd \in \SL_2(\RR)$ mit $c > 0$. Sei $\gamma > 0$ beliebig (insbesondere nicht unbedingt ganzzahlig). Dann gilt für alle $z \in \HH$
\[
	\sum_{\nu \in \ZZ} (c(z + \nu) +d)^{-k} e^{2\pi i\gamma \frac{a(z+\nu)+b}{c(z+\nu)+d}} 
	= 
	\tfrac{2\pi (-1)^{\frac k2}}c \sum_{m \geq 1} \left(\!\frac m\gamma\!\right)^{\!\!\frac{k-1}2} \!\! J_{k-1}\left(\!\tfrac{4\pi\sqrt{m\gamma}}c\right) e^{\frac{2\pi i}c (\gamma a + md)} e^{2\pi imz}
	\,.
\]
\end{lemm}

\begin{bewe}
Es genügt, diese Aussage nur für den Fall $A = \mymat 0{-1}10$ zu zeigen, d.h.
\[
	\sum_{\nu \in \ZZ} (z + \nu)^{-k} e^{-2\pi i\gamma \frac{1}{z+\nu}} = 2\pi (-1)^{\frac k2} \sum_{m \geq 1} \left(\!\frac m\gamma\!\right)^{\!\!\frac{k-1}2} \!\! J_{k-1}(4\pi\sqrt{m\gamma}) e^{2\pi imz}
	\,.
\]
In der Tat: Ersetzt man in dieser Gleichung $z$ durch $z + \frac dc$ und $\gamma$ durch $\frac{\gamma}{c^2}$ und multipliziert dann mit $c^{-k} e^{2\pi i \gamma \frac ac}$, so wird die linke Seite zu
\begin{align*}
	c^{-k} e^{2\pi i\gamma \frac ac} \sum_{\nu \in \ZZ} (z + \tfrac dc + \nu)^{-k} e^{-2\pi i \frac{\gamma}{c^2} \frac{1}{z + \frac dc + \nu}} 
	&= \sum_{\nu \in \ZZ} (cz + d + c\nu)^{-k} e^{2\pi i\gamma \frac ac - 2\pi i \frac{\gamma}{c} \frac{1}{cz + d + c\nu}} \\
	&= \sum_{\nu \in \ZZ} (c(z + \nu) + d)^{-k} e^{\frac{2\pi i\gamma}c \left( a - \frac{1}{c(z + \nu) + d} \right)} \\[-12pt]
	&= \sum_{\nu \in \ZZ} (c(z + \nu) + d)^{-k} e^{\frac{2\pi i\gamma}c \frac{ac(z + \nu) + \overbrace{\scriptscriptstyle ad - 1}^{= bc}}{c(z + \nu) + d}} \\
	&= \sum_{\nu \in \ZZ} (c(z + \nu) + d)^{-k} e^{2\pi i\gamma \frac{a(z + \nu) + b}{c(z + \nu) + d}}
\end{align*}
sowie die rechte Seite zu

\begin{align}\label{eq:fourierreihePn}
&c^{-k} e^{2\pi i \gamma \frac ac} 2\pi (-1)^{\frac k2} \sum_{m\geq 1} \left(\!\frac {mc^2}{\gamma}\right)^{\!\!\frac {k-1}2} \!\! J_{k-1} \left(\! 4\pi \sqrt{\frac{m\gamma}{c^2}}\right) e^{2\pi i m(z + \frac dc)} \\
&\qquad = c^{-k+2\frac{k-1}2} 2\pi (-1)^{\frac k2} \sum_{m\geq 1} \left(\!\frac {m}{\gamma}\!\right)^{\!\!\frac {k-1}2} \!\! J_{k-1} \left(\! \tfrac{4\pi \sqrt{m\gamma}}c\right) e^{2\pi i \gamma \frac ac + 2\pi i mz + 2\pi im\frac dc} \nonumber \\
&\qquad = \frac{2\pi (-1)^{\frac k2}}c \sum_{m\geq 1} \left(\!\frac {m}{\gamma}\!\right)^{\!\!\frac {k-1}2} \!\! J_{k-1} \left(\! \tfrac{4\pi \sqrt{m\gamma}}c\right) e^{\frac{2\pi i}c \left(\gamma a + md\right)} e^{2\pi i mz} \nonumber
\,.
\end{align}

Die linke Seite von \eqref{eq:fourierreihePn} konvergiert gleichmäßig absolut auf kompakten Mengen in $\HH$ und hat den Limes 0 für $q\to \infty$ (gleicher Beweis wie in \autoref{<f,Pn>}).
Sie hat daher eine Fourierentwicklung $\sum_{m\geq 1} c(m) q^m$ mit
\[
c(m) = \int_{ic}^{ic+1} \biggl( \sum_{\nu \in \ZZ} (z+\nu)^{-k} e^{-2\pi i \gamma \frac{1}{z+\nu}} \biggr) e^{-2\pi imz} \opd z
\stackrel{z \mapsto is}{=} -i^{-k+1} \int_{c-i \infty}^{c + i\infty} s^{-k} e^{-2\pi \frac{\gamma}{s}} e^{2\pi ms} \opd s
\,.
\]
Es gilt nach \myquote{Abramowitz-Stegun}, Seite 1026, Formel 29.3.80:
\[
\frac{1}{2\pi i} \int_{c-i\infty}^{c+i\infty} s^{-k} e^{-\frac{\alpha}{s}} e^{ts} \opd s = \Bigl(\frac{t}{\alpha}\Bigr)^{\frac{k-1}{2}} J_{k-1}(2\sqrt{\alpha t})
\,.
\]

Setzt man $\alpha = 2\pi \gamma$, $t = 2\pi m$, so folgt
\[
c(m) = -i^{-k+1} \cdot 2\pi i \Bigl( \frac{2\pi m}{2\pi \gamma}\Bigr)^{\frac{k-1}{2}} J_{k-1} (2\sqrt{2\pi \gamma \cdot 2 \pi m}) = (-1)^{\frac{k}{2}} \cdot 2\pi \Bigl(\frac{m}{\gamma}\Bigr)^{\frac{k-1}{2}} J_{k-1} (4\pi \sqrt{m\gamma})
\]
wie behauptet.


\end{bewe}

Nach dem Lemma folgt nun mit $\gamma = $, dass
\[
P_n(z) = e^{2\pi nz} + \sum_{c\geq1} \sum_{\substack{d \bmod c\\ \ggt(d,c) = 1}} \frac{2\pi (-1)^{\frac{k}{2}}}{c} \sum_{m\geq 1} \Bigl(\frac{m}{n}\Bigr)^{\frac{k-1}{2}} J_{k-1} \Bigl(\frac{4\pi \sqrt{mn}}{c}\Bigr) \cdot e^{\frac{2\pi i}{c}(na+md)} e^{2\pi imz}
\,.
\]
Die Behauptung folgt hieraus nach Vertauschung der Summationen über $c$ und $m$ (absolute Konvergenz) unter Beachtung von $ad = 1 + bc \equiv 1 \mod c$.

\end{bewe}

\subsection[Die Ramanujan \texorpdfstring{$\tau$}{tau}-Funktion]{Die Ramanujan {\boldmath $\tau$}-Funktion}

\begin{satz}
Sei $\Delta(z) = \sum_{n \geq 1} \tau(n) q^n \in S_{12}$ ($\tau(1) = 1$, $\tau(2) = -24, \ldots $).
Dann gilt
\begin{align*}
\tau(n) \not= 0 &\Rla P_{n,12} \not= 0 \Rla &\Rla g_n(n) \not= 0
\,,
\end{align*}
wobei $g_n(n)$ der $n$-te Fourier-Koeffizient von $P_{n,12}$ ist (siehe \autoref{satz:Pn_Fourier}).
\end{satz}
\begin{bewe}
Es gilt $P_n = c_n \cdot \Delta$ mit $c_n \in \CC \setminus \Set {0}$.
(Man kann zudem zeigen, dass $c_n \in \RR$, dies benötigen wir im Folgenden jedoch nicht.)
Aus $\scalarprd{\Delta}{P_n} \sim \tau(n)$\footnote{Hier meint $\sim$ die Proportionalität: $x \sim y \Rla x = ky$ für $k$ konstant.} (siehe \autoref{<f,Pn>} (ii)) folgt $c_n \scalarprd \Delta \Delta \sim \tau(n)$, also gilt $\tau(n) = 0$ genau dann, wenn $c_n = 0$. Aber $c_n = 0$ genau dann, wenn $P_n \equiv 0$, und dies gilt genau dann, wenn $g_n(n) \sim \scalarprd{P_n}{P_n} = 0$.
Hieraus folgt die Behauptung.
\end{bewe}

\begin{beme}
Es wird vermutet, dass $\tau(n) \not= 0$ für alle $n\in\NN$ (Lehmer).
\end{beme}

\subsection[Die Peterssonschen Formeln]{Die Peterssonschen Formeln und Abschätzungen für Fourier-Koeffizienten}

Sei $\Set{f_1, f_2, \ldots f_g}$ irgendeine orthogonale Basis von $S_k$ (nach dem Gram-Schmidt-Verfahren kann man z.\,B. jedes $f\in S_k\setminus\{0\}$ zu irgendeiner orthogonalen Basis $\Set{f, \ldots, f_g}$ ergänzen).
Dann gilt nach \autoref{<f,Pn>} (ii) für jedes $n\in\NN$
\[
P_n = \frac{(k-2)!}{(4\pi n)^{k-1}} \sum_{\nu=1}^g \frac{\conj{a_\nu(n)}}{\scalarprd{f_\nu}{f_\nu}} f_\nu
\,,
\]
wenn $f_\nu = \sum_{m\geq 1} a_\nu(m) q^m$.
Nimmt man auf beiden Seiten den $m$-ten Fourier-Koeffizienten, so erhält man
\[
g_n(m) = \frac{(k-2)!}{(4\pi n)^{k-1}} \sum_{\nu=1}^g \frac{a_\nu(m) \conj{a_\nu(n)}}{\scalarprd{f_\nu}{f_\nu}}
\,.
\]
Damit folgt
\[
g_n(n) = \frac{(k-2)!}{(4\pi n)^{k-1}} \sum_{\nu = 1}^g \frac{\abs{a_\nu(n)}^2}{\scalarprd{f_\nu}{f_\nu}}
\,.
\]
Speziell ist
\[
\abs{a_\nu(n)}^2 \leq \norm{f_\nu}^2 \frac{(4\pi n)^{k-1}}{(k-2)!} g_n(n)
\,.
\]
Für $g_n(n)$ substituiert man aus \autoref{satz:Pn_Fourier} explizite Formeln. Benutzt man $J_n(x) = \mathcal{O}(\min \{x^{-\frac{1}{2}}, x^n\})$ (einfach) und $K(n,n,c) = \mathcal{O}_\epsilon( (n,c)^{\frac{1}{2}}c^{\frac{1}{2}+\epsilon})$ (Weilsche Abschätzung, tiefliegend), so erhält man nach einigen Rechnungen
\[
g_n(n) = \mathcal{O}_\epsilon (n^{\frac{1}{2} + \epsilon})
\,,
\]
also folgt
\[
a_\nu(n) = \mathcal{O}_\epsilon (n^{\frac{k}{2}-\frac{1}{4} + \epsilon} )
\,.
\]

\begin{satz}
Sei $f \in S_k$. Dann gilt $a(n) = \mathcal{O}_\epsilon (n^{\frac{k}{2}-\frac{1}{4} + \epsilon})$, für $\epsilon > 0$.
\end{satz}

\begin{beme-list}
\item Man kann leicht zeigen, dass $a(n) \ll_f n^{\frac{k}{2}}$ (siehe \autoref{Fourierkoeff-Abschätzungen}).
\item Mit der Theorie der L-Reihen zu Modulformen kann man $a(n) \ll_{f,\epsilon} n^{\frac{k}{2} - \frac{1}{4} + \epsilon}$ für alle $\epsilon > 0$ zeigen.
\item Nach Deligne (sehr tiefliegend) gilt sogar $a(n) \ll_{f,\epsilon} n^{\frac{k}{2}-\frac{1}{2}+\epsilon}$ für alle $\epsilon > 0$ (Ramanujan-Petersson-Vermutung). 
Diese Abschätzung ist bereits bestmöglich, denn nach Rankin gilt
\[
\limsup_{n\to\infty} \frac{\abs{a_f(n)}^2}{n^{k-1}} = \infty
\,.
\]
%	\[
%		y^2 = x^3 + ax + b \mod p
%	\]

\end{beme-list}

\subsection{Hecke-Operatoren sind hermitesch}

\begin{satz}\label{Hecke(Pm)}
Sei $P_m$ für $m \in \NN$ die $m$-te Poincaré-Reihe in $S_k$.
Dann ist
\[
	P_m | T(n) = \sum_{d | (m,n)} \Bigl( \frac{n}{d} \Bigr)^{k-1} P_{\frac{mn}{d^2}}
\,.
\]
\end{satz}
\begin{bewe}
Nach Definition ist
\[
P_m = \frac{1}{2} \sum_{M\in \linksmodulo{\Gamma(1)_\infty}{\Gamma(1)}} e^m |_k M
\]
unabhängig vom Vertretersystem von $\linksmodulo{\Gamma(1)_\infty}{\Gamma(1)}$.
Es gilt
\[
2 P_m | T(n) = n^{\frac{k}{2}-1} \sum_{\substack{M \in \linksmodulo{\Gamma(1)_\infty}{\Gamma(1)}\\ N \in \linksmodulo{\Gamma(1)}{\mathcal{M}(n)}}} e^m|_k MN
= n^{\frac{k}{2}-1} \sum_{R \in \linksmodulo{\Gamma(1)_\infty}{\mathcal{M}(n)}} e^m|R
\]
Wir behaupten nun, dass die Menge 
\begin{equation*}
\begin{split}
\{\,NM \mid &N = \mymat ab0d\colon ad=n, d>0,\ b \bmod d, \\ &M = \mymat \alpha\beta\gamma\delta\colon (\gamma, \delta) \in \ZZ^2\colon \ggt(\gamma, \delta) = 1 \text{ und } (\alpha, \beta) \in \ZZ^2 \text{ fixiert s.\,d. } \alpha\delta - \beta\gamma = 1\,\}
\end{split}
\end{equation*}
ein Vertretersystem für $\linksmodulo{\Gamma(1)_\infty}{\mathcal{M}(n)}$ ist.

Wir zeigen zunächst, dass die gesamten Matrizen inäquivalent modulo $\Gamma(1)_\infty$ sind.
Angenommen
\[
\mymat 1\nu01 NM = N'M'
\]
mit $\nu \in \ZZ$ und $N$, $N'$ und $M$, $M'$ wie oben.
Daraus folgt
\[
N'^{-1} \mymat 1\nu01 N = M'M^{-1}
\,,
\]
also
\[
\mymat*{\frac{d'}{d}}{\frac{d'b-b'd+\nu d'd}{n}}{0}{\frac{d}{d'}}
= M' M^{-1}
\,.
\]
Da $M'M^{-1}$ Komponenten in $\ZZ$ hat, folgt $\frac{d}{d'}$, $\frac{d'}{d} \in \ZZ$, also $d = \pm d'$, also $d = d'$ und $a' = a$ und somit 
\[
M'M^{-1} \in \Gamma(1)_\infty
\,,
\]
d.\,h. $M' = M$, da Vertretersystem modulo $\Gamma(1)_\infty$.
Dann folgt aber $\mymat 1\nu01 N = N'$, bzw. $b' = b + \nu d$, also $b = b'$ und damit $N' = N$. Die Matrizen in der oben angegebenen Menge sind also tatsächlich inäquivalent modulo $\Gamma(1)_\infty$. 

Es verbleibt noch zu zeigen, dass sich jedes $\mymat ABCD \in \mathcal M (n)$ schreiben lässt als
\[
	\mymat* ABCD = \mymat* 1\nu 01 \mymat* ab0d \mymat* \alpha \beta \gamma \delta
	\,,
\]
d.h. 
\[
	\mymat* ABCD \mymat* \delta{-\beta}{-\gamma}\alpha = \mymat* 1\nu 01 \mymat* ab0d
\]
mit $\nu \in \ZZ$ und $ad = n, d > 0, b (\operatorname{mod} d)$ und $(\gamma, \delta) \in \ZZ^2, \ggt(\gamma, \delta) = 1, \alpha\delta - \beta\gamma = 1$. Man bestimmt zunächst $(\gamma, \delta) \in \ZZ^2$ mit $\ggt(\gamma, \delta) = 1$, sodass $C\delta - D\gamma = 0$, dann ist
\[
	\mymat* ABCD \mymat* \delta{-\beta}{-\gamma}\alpha = \mymat* {*}{*}0{*} \in \mathcal M(n)
	\,,
\]
also
\[
	\mymat* ABCD \mymat* \delta{-\beta}{-\gamma}\alpha = \mymat* a{\tilde b}0d 
\]
mit $\tilde b \in \ZZ, ad = n$. Indem man gegebenenfalls mit $-E$ multipliziert, d.h. $(\gamma, \delta)$ durch $(-\gamma, -\delta)$ ersetzt, kann man auch $d > 0$ erreichen. Wähle nun $\nu \in \ZZ$, sodass $\tilde b = b + \nu d$. Dies zeigt die Behauptung, dass die oben angegebene Menge ein Vertretersystem für $\linksmodulo{\Gamma(1)_\infty}{\mathcal{M}(n)}$ ist.

Es gilt nun
\[
	2 P_m | T(n) = n^{\frac k2 - 1} \sum_{M \in \linksmodulo {\Gamma(1)_\infty}{\Gamma(1)}} \left( \sum_{\substack{ad = n, d > 0 \\ b (\operatorname{mod} d)}} e^m |_k \mymat* ab0d \right) |_k M
	\,.
\]
Die innere Summe ist gleich 
\[
	\sum_{\substack{d|n \\ b (\operatorname{mod} d)}} n^{\frac k2} d^{-k} e^{2\pi im \left( \frac {n}{d^2} z + \frac bd \right)} = n^{\frac k2} \sum_{d | (m,n)} d^{1-k} e^{2\pi i \frac {mn}{d^2} z}
\]
wegen 
\[
	\sum_{b (\operatorname{mod} d)} e^{2\pi i \frac bd m} = \begin{cases} d, &d|m \\ 0, &\text{sonst}\end{cases}
	\,.
\]
Hieraus folgt die Behauptung.
\end{bewe}

\begin{satz}\label{T(n)herm}
Die Operatoren $T(n), n \in \NN$ eingeschränkt auf $S_k$ sind hermitesch bezüglich des Petersson-Skalarproduktes, d.h.
\[
	\scalarprd {f|T(n)}g = \scalarprd f{g|T(n)} \qquad \forall f,g \in S_k
	\,.
\]
\end{satz}

\begin{bewe}
Man zeigt dies normalerweise, indem man Modulformen zu sogenannten Kongruenzuntergruppen von $\Gamma(1)$ und deren Skalarprodukt definiert und dann gewisse Invarianzeigenschaften des Skalarproduktes (beim Übergang von einer Untergruppe zur anderen) beachtet. Wir werden hier die Behauptung unter Benutzung von \autoref{Hecke(Pm)} beweisen. Da die $P_m$ mit $m \in \NN$ den Raum $S_k$ erzeugen, genügt es zu zeigen, dass
\[
	\scalarprd {f|T(n)}{P_m} = \scalarprd f{P_m|T(n)}
	\,.
\]
Man schreibe $f = \sum_{l \geq 1} a(l) q^l$ und $f | T(n) = \sum_{l \geq 1} b(l) q^l$. Nach \autoref{<f,Pn>}, ii) ist
\begin{align*}
	\scalarprd {f|T(n)}{P_m} &= \frac {(k-2)!}{(4\pi m)^{k-1}} b(m) \\
	&= \frac {(k-2)!}{(4\pi m)^{k-1}} \sum_{d|(m,n)} d^{k-1} a\left( \frac {mn}{d^2} \right)
	\,.
\end{align*}
Andererseits ist nach \autoref{Hecke(Pm)}:
\begin{align*}
	\scalarprd f{P_m|T(n)} &= \sum_{d|(m,n)} \left( \frac nd \right)^{k-1} \scalarprd f{P_{\frac {mn}{d^2}}} \\
	&= \sum_{d|(m,n)} \left( \frac nd \right)^{k-1} \frac {(k-2)!}{(4\pi \frac {mn}{d^2})^{k-1}} a\left( \frac {mn}{d^2} \right) \\
	&= \frac {(k-2)!}{(4\pi m)^{k-1}} \sum_{d|(m,n)} d^{k-1} a\left( \frac {mn}{d^2} \right)
	\,.
\end{align*}
\end{bewe}

\begin{koro}
Die Eigenwerte von $T(n)$ sind reell.
\end{koro}
\begin{bewe}
Ist nach \autoref{T(n)herm} und LA 1 klar.
\end{bewe}

\begin{koro}\label{normEigfkt:id/orth}
Seien $f, g$ normalisierte Eigenformen in $S_k$. Dann ist entweder $f = g$ oder $\scalarprd fg = 0$.
\end{koro}

\begin{bewe}
Seien $f = \sum_{n \geq 1} a(n) q^n$ und $g = \sum_{n \geq 1} b(n) q^n$. Wegen $a(1) = b(1) = 1$ ist dann $f | T(n) = a(n) f$ und $g | T(n) = b(n) g$. Daher gilt mit \autoref{T(n)herm}
\[
	a(n) \scalarprd fg = \scalarprd {f|T(n)}g = \scalarprd f{g|T(n)} = \conj{b(n)} \scalarprd fg = b(n) \scalarprd fg
	\,.
\]
Aus $\scalarprd fg \neq 0$ folgt damit $a(n) = b(n)$ für alle $n \in \NN$, also $f = g$.
\end{bewe}

\begin{lemm}
Sei $V$ ein endlich-dimensionaler komplexer Hilbertraum mit Skalarprodukt $\scalarprd \cdot \cdot$ und sei $\Set {T_\mu}_{\mu \in I}$ eine Familie von hermiteschen, miteinander kommutierenden Endomorphismen von $V$. Dann besitzt $V$ eine orthogonale Basis bestehend aus gemeinsamen Eigenvektoren aller Operatoren $T_\mu$ mit $\mu \in I$. 
\end{lemm}

\begin{bewe}
Sei $W$ die Menge der $\CC$-linearen Endomorphismen von $V$, aufgefasst als reeller Vektorraum. Wegen $\dim_\CC V < \infty$ ist auch $\dim_\RR W < \infty$. Die $T_\mu$ erzeugen daher einen endlich-dimensionalen Unterraum von $W$, sodass es genügt, die Aussage für endlich viele Operatoren $T_1, \ldots, T_m$ zu zeigen.

Wir zeigen zunächst durch Induktion nach $m$, dass $V$ einen gemeinsamen nichttrivialen Eigenvektor von $T_1, \ldots, T_m$ enthält. Für $m = 1$ ist dies klar, da $V$ wegen $T_1$ hermitesch einen nichttrivialen Eigenvektor von $T_1$ enthält. Sei nun $m \geq 2$ und $\lambda$ ein Eigenwert von $T_1$ mit zugehörigem Eigenraum $V_\lambda := \Set {v \in V \mid T_1 v = \lambda v }$. Für alle $\mu \in \Set {2, \ldots, m}$ besteht nach Voraussetzung die Kommutativität $T_\mu T_1 = T_1 T_\mu$ und daher gilt $T_\mu V_\lambda \subset V_\lambda$. Nach Induktionsvoraussetzung besitzt nun $V_\lambda$ einen nichttrivialen gemeinsamen Eigenvektor von $T_2, \ldots, T_m$. Dieser ist nach Definition von $V_\lambda$ auch Eigenvektor von $T_1$.

Wir zeigen abschließend die Aussage des Lemmas durch Induktion nach $\dim_\CC V$. Für $\dim_\CC V = 1$ ist die Aussage klar. Sei also $m = \dim_\CC V \geq 2$. Man schreibe $V = \CC v \oplus (\CC v)^\bot$, wobei $v$ ein Eigenvektor aller $T_\mu$ mit $\mu \in \Set {1, \ldots, m}$ ist. Da die $T_\mu$ hermitesch sind und $\CC v$ invariant lassen, lassen sie auch $(\CC v)^\bot$ invariant. Nach Induktionsvoraussetzung besitzt $(\CC v)^\bot$ bereits eine orthogonale Basis von Eigenvektoren für alle $T_\mu$. Hieraus folgt die Behauptung.
\end{bewe}

\begin{koro}
Der Raum $S_k$ besitzt eine orthogonale Basis von gemeinsamen Eigenfunktionen für alle $T(n)$ mit $n \in \NN$.
\end{koro}
\begin{bewe}
Folgt direkt aus dem obigen Lemma mit $V = S_k$ und $\Set {T_\mu}_{\mu \in I} = \Set {T(n)}_{n \in \NN}$.
\end{bewe}

\begin{beme}
Nach \autoref{normEigfkt:id/orth} ist diese orthogonale Basis bis auf Permutation und Multiplikation mit Skalaren in $\CC^\times$ eindeutig bestimmt.
\end{beme}


%chapter 5
\chapter{Die Eichler-Selberg-Spurformel auf \texorpdfstring{$\SL_2(\ZZ)$}{SL_2(Z)}}

Sei von nun an stets $k \geq 4$ gerade und wie üblich $T(m)$ mit $m \geq 1$ der $m$-te Hecke-Operator auf $M_k (\Gamma(1))$. Wir wissen bereits, dass wir $T(m)$ zu einem Endomorphismus auf $S_k$ einschränken können.

\emph{Ziel:} Bestimmung einer analytischen (einfach) und arithmetischen (schwer) Formel für die Spur $\Tr T(m)$ für alle $m \in \NN$.

Sei $\HH$ wie üblich die obere Halbebene und $h$ eine Funktion $h: \HH \times \HH \to \CC, (z, z') \mapsto h(z, z')$, welche in beiden Variablen eine Spitzenform von Gewicht $k$ darstellt, d.h. 
\[
	h(\cdot, z') \in S_k \quad \forall z' \in \HH \qquad \text{und} \qquad h(z, \cdot) \in S_k \quad \forall z \in \HH
	\,.
\]
Für $f \in S_k$ definieren wir dann $f \ast h$ als die Funktion
\begin{equation}\label{eq:faltungskern}
	f \ast h \colon \HH \to \CC, z' \mapsto (f \ast h)(z') := \int_{\mathcal F} f(z) \conj{ h(z, \conj{-z'})} y^{k-2} \opd x \opd y \qquad (z = x + iy)
	\,.
\end{equation}
Dies ist im Wesentlichen das Petersson-Skalarprodukt $\scalarprd f{h(\cdot, \conj{-z'})}$. Wir wollen zunächst zeigen, dass $T(m) \colon S_k \to S_k$ als ein Integral dieses Typs geschrieben werden kann mit einem bestimmten Kern $h = h_m$ (bis auf eine Konstante). Aus diesen Überlegungen folgt dann auch sogleich eine analytische Formel für $\Tr T(m)$.

Sei $f_1$, \ldots, $f_r$ eine Basis von normierten, simultanen, orthogonalen Eigenformen für die $T(m)$, d.\,h.

\begin{align*}
f_i &= \sum_{n=1}^\infty a_n^i q^n \\
a_1^i &= 1 \\
T(m)f_i &= a_m^if_i \\
\scalarprd {f_i}{f_j} = 0 &\Rla i \not= j
\,.
\end{align*}

%\begin{erin}
%	$\scalarprd {f_i}{f_j} = 0$ genau dann, wenn $i \not= j$.
%\end{erin}

Für $m \geq 1$ definieren wir
\[
h_m(z, z') = \smashoperator{\sum_{ad-bc=m}} \, (czz' + dz' + az + b)^{-k}
\,,
\]
dabei erstreckt sich die Summe über alle ganzzahligen Matrizen $\abcd$ mit Determinante $m$.
Offenbar gilt ebenso
\[
h_m(z,z') = \smashoperator{\sum_{ad-bc=m}} \, (cz+d)^{-k} \Bigl(z' + \frac{az+b}{cz+d}\Bigr)^{-k}
= m^{- \frac k2} \smashoperator{\sum_{M \in \mathcal{M}(m)}} \, (z'+z)^{-k}|_{k,z} M
\,.
\]
Man zeigt schnell wegen $k \geq 4$, dass die Reihe auf kompakten Teilmengen $K \subset \HH \times \HH$ absolut und gleichmäßig konvergiert und dort in beiden Variablen eine holomorphe Funktion darstellt.
Da $\mathcal{M}(m) \xto{\sim} \mathcal{M}(m)$, $M \mapsto ML$ mit $L \in \SL_2(\ZZ)$,
folgt $h(z,z')|_{k,z} L = h(z, z')$.

Da weiter
\begin{align*}
\lim_{z \to i\infty} \smashoperator[r]{\sum_{M \in \M(m)}} \, (z+z')^{-k}|_{k,z} M
= \smashoperator[l]{\sum_{M \in \M(m)}} \lim_{z \to i\infty} (z'+z)^{-k}|_{k,z} M
= 0\,,
\end{align*}
folgt $h_m(\cdot, z') \in S_k$ und $h_m(z, \cdot) \in S_k$ aus Symmetriegründen.


\begin{satz}\label{analytischeSpurformel}
	Sei
	\begin{equation}%\tag{2}
	C_k = \frac{(-1)^{\frac{k}{2}} \pi}{2^{k-3}(k-1)}\,.
	\end{equation}
	Dann gilt
	\begin{enumerate}
		\item Die Funktion $C_k^{-1} m^{k-1} h_m(z,z')$ ist ein Kern für den Operator $T(m)\colon S_k \ra S_k$, das heißt
		\begin{equation}\label{eq:faltung_hm}%\tag{3}
		(f * h_m)(z') = C_km^{-k+1} (f|T(m))(z')
		\,.
		\end{equation}
		\item Es gilt die Identität 
		\begin{equation}%\tag{4}
		C_k^{-1} m^{k-1} h_m(z,z') = \sum_{i=1}^r a_m^i \frac{f_i(z) \cdot f_i(z')}{\scalarprd{f_i}{f_i}}
		\,.
		\end{equation}
		\item Die Spur $\Tr T(m)$ ist gegeben durch
		\begin{equation}%\tag{5}
		\Tr T(m) = C_k^{-1} m^{k-1} \int_\F h_m(z, -\conj z) y^{k-2} \opd x \opd y
		\,.
		\end{equation}
	\end{enumerate}
\end{satz}

\begin{bewe}
	Sei zunächst $m=1$.
	Falls $\gamma = \abcd \in \SL_2(\ZZ)$, dann gilt
	\[
	(c\conj z + d)^{-k} f(z) y^k = f(\gamma z) \cdot \Im (\gamma z)^k
	\,.
	\]
	Aus der Definition von $h_m$ erhalten wir also
	\[
	f(z)\conj{h_1(z, z')} y^k
	= \sum_{\gamma \in \Gamma(1)} \bigl(\conj{z'} + \gamma \conj z\bigr)^{-k} f(\gamma z) \Im(\gamma z)^k
	\]
	und demnach
	\begin{equation}\label{eq:faltung_h1}%\tag{6}
	\begin{split}
	(f * h_1)(z') &= \int_\F \sum_{\gamma \in \Gamma(1)} (-z' + \gamma \conj z)^{-k} f(\gamma z) \Im(\gamma z)^k \frac{\opd x \opd y}{y^2} \\
	&= \sum_{\gamma \in \Gamma(1)} \int_{\gamma \circ \F} (-z' + \conj z)^{-k} f(z) y^{k-2} \opd x \opd y \\
	&= 2 \int_0^\infty \int_{-\infty}^\infty (x-iy - z')^{-k} f(x+iy) y^{k-2} \opd x \opd y
	\,.
	\end{split}
	\end{equation}
	Nach Cauchy's Formel (und da $f$ Spitzenform) gilt
	\[
	\int_{-\infty}^\infty (x-iy - z')^{-k} f(x+iy) \opd x
	= \frac{2\pi i}{(k-1)!} f^{(k-1)} (2iy + z')
	\,.
	\]
	
	Daraus folgt, dass die rechte Seite von \eqref{eq:faltung_h1} wie folgt umgeformt werden kann:
	\begin{equation*}
	\begin{split}
	(f * h_1)(z') &= \frac{4\pi i}{(k-1)!} \int_0^\infty y^{k-2} f^{(k-1)} (2iy + z') \opd y \\
	&= \frac{4\pi i}{(k-1)!} \int_0^\infty \frac{1}{(2i)^{k-2}} \frac{\opd^{k-2}}{\opd t^{k-2}} f'(2ity + z') \Big|_{t=1} \opd y \\
	&= \frac{4\pi i}{(k-1)!} \frac{1}{(2i)^{k-2}} \frac{\opd^{k-2}}{\opd t^{k-2}} \int_0^\infty f'(2i ty + z') \opd y \Big|_{t=1} \\
	&= \frac{4\pi i}{(k-1)!} \frac{1}{(2i)^{k-2}} \frac{\opd^{k-2}}{\opd t^{k-2}} \Bigl( 0 - \frac{f(z')}{2it}\Bigr) \Big|_{t=1} \\
	&= C_k f(z')
	\,.
	\end{split}
	\end{equation*}
	Das beweist \eqref{eq:faltung_hm} im Fall $m=1$.
	Für den allgemeinen Fall beachte
	\[
	\begin{split}
	h_1(z,z')|T(m)
	&= 1^{- \frac k2} \biggl(\sum_{\gamma\in\Gamma(1)} (z'+z)^{-k} |_{k,z} \gamma\biggr)\Big|T(m) \\
	&= m^{\frac k2 - 1} \smashoperator{\sum_{\substack{M \in \linksmodulo{\Gamma(1)}{\M(m)}\\ \gamma \in \Gamma(1)}}} \, (z'+z)^{-k} |_{k,z} \gamma M \\
	&= m^{\frac k2 - 1} \smashoperator{\sum_{R \in \M(m)}} \, (z'+z)^{-k}|_{k,z} R \\
	&= m^{k-1} h_m(z,z')
	\,.
	\end{split}
	\]
	Damit folgt (i).
	
	Für (ii) beachte, dass wir $h_m$ schreiben können als
	\[
	h_m(z,z') = \sum_{i,j=1}^r c_{ij}f_i(z)f_j(z')
	\,,
	\]
	denn: Da $f_1$, \ldots, $f_r$ Basis der Spitzenformen, können wir 
	\[
	h_m(z, z') = g_1(z) f_1(z') + \ldots + g_r(z) f_r(z')
	\]
	mit Funktionen $g_j \colon \HH \ra \CC$ schreiben. 
	Diese sind auch Spitzenformen, denn für $z_1$, \ldots, $z_r$ paarweise inäquivalent unter $\Gamma(1)$ und $z_\nu \not\equiv i, \rho \mod \Gamma(1)$ ist die Matrix in
	\[
	\begin{pmatrix}
	h_m(z,z_1) \\
	\vdots \\
	h_m(z,z_r)
	\end{pmatrix}
	= \begin{pmatrix}
	f_1(z_1) & \ldots & f_r(z_1) \\
	\vdots & \ddots & \vdots \\
	f_1(z_r) & \ldots & f_r(z_r)
	\end{pmatrix}
	\cdot 
	\begin{pmatrix}
	g_1(z) \\
	\vdots \\
	g_r(z)
	\end{pmatrix}
	\]
	invertierbar, sonst würden nichttriviale $\alpha_1$, \ldots, $\alpha_r \in \CC$ existieren, sodass die Funktion $\alpha_1 f_1 + \ldots + \alpha_r f_r$ die $r$ Nullstellen $z_1, \ldots, z_r$ besäße. Da es zudem eine Spitzenform ist, könnte man über die Valenzformel bereits $\alpha_1 f_1 + \ldots + \alpha_r f_r \equiv 0$ folgern, was im Widerspruch dazu steht, dass die $f_j$ eine Basis bilden.
	Somit sind die $g_j$ als Linearkombination von Spitzenformen selbst Spitzenformen und daher als Linearkombination der $f_i$ darstellbar.
	
	Wende nun \eqref{eq:faltungskern} auf die Funktion $f = f_\mu$ mit $1 \leq \mu \leq r$, an:
	\[
	\begin{split}
	(f_\mu * h_m)(z')
	&= \int_\F f_\mu(z) \sum_{i,j=1}^r \conj{c_{ij} f_i(z) f_j(-\conj{z'})} y^{k-2} \opd x \opd y \\
	&= \sum_{i,j=1}^r \conj{c_{ij}} f_j(z') \int_\F f_\mu(z) \conj{f_i(z)} y^{k-2} \opd x \opd y \\
	&= \sum_{j=1}^r \conj{c_{\mu j}} f_j(z') \scalarprd{f_\mu}{f_\mu}
	\stackrel{\text{(i)}}{=} C_km^{1-k} a_m^\mu f_\mu(z')
	\,.
	\end{split}
	\]
	Hierbei geht unter Anderem ein, dass alle Fourierkoeffizienten von $f_j$ reell sind, da es sich um eine normalisierte Hecke-Eigenform handelt.
	Da $f_1$, \ldots $f_r$ zudem eine Basis ist, folgt
	\[
	c_{\mu j} =
	\begin{cases}
	0 & , \text{ falls } j \not= \mu \\
	C_k m^{1-k} a_m^\mu \scalarprd{f_\mu}{f_\mu}^{-1} & , \text{ falls } j = \mu
	\end{cases}
	\,.
	\]
	und damit (ii).
	
	Für (iii) beachte
	\[\begin{split}
	C_k^{-1} m^{k-1} \int_\F h_m(z, -\conj{z}) y^{k-2}\opd x \opd y
	&= \int_\F \sum_{i=1}^r a_m^i \frac{f_i(z)f_i(-\conj z)}{\scalarprd{f_i}{f_i}} y^{k-2} \opd x \opd y \\
	&= \sum_{i=1}^r a_m^i\int_\F \frac{f_i(z) \conj{f_i(z)}}{\scalarprd{f_i}{f_i}} y ^{k-2} \opd x \opd y \\
	&= \sum_{i=1}^r a_m^i = \Tr T(m)
	\,.
	\end{split}
	\]
\end{bewe}

Die zweite arithmetische Darstellung liefert eine explizite Beschreibung der Spur.
Dafür müssen wir etwas ausholen.

\begin{defi}
	Ein Polynom $q \in \ZZ[X, Y]$ mit $q(X,Y) = aX^2 + bXY + cY^2$ heißt ganze, binäre \myemph{quadratische Form}.
	Diese ist induziert von der Matrix
	\[
	Q = \mymat*{a}{\frac{b}{2}}{\frac{b}{2}}{c}
	\]
	via $q(x,y) = (x, y) \cdot Q \cdot (\begin{smallmatrix} x \\ y \end{smallmatrix})$.
	
	$q$ heißt positiv definit, falls $q(x,y) > 0$ für alle $(x,y) \in \ZZ^2\setminus\{(0,0)\}$.
	
	Wir bezeichnen $D = b^2 - 4ac$ als die \myemph[quadratische Form!Diskriminante einer quadratischen Form]{Diskriminante} von $q$.
	
	Zwei quadratische Formen $q$ und $q'$ heißen äquivalent, falls es eine Matrix $U \in \SL_2(\ZZ)$ gibt mit $Q' = U^t Q U$. Man kann zeigen, dass die Diskriminante eine Klasseninvariante ist.
	Die Rückrichtung ist im Allgemeinen falsch.
	
	Definiere eine Abbildung
	\[
	H\colon \ZZ \ra \QQ
	\]
	durch
	\begin{enumerate}
		\item $H(n) = 0$ für $n < 0$,
		\item $H(0) = - \frac{1}{12}$,
		\item $H(n)$ ist für $n > 0$ die Zahl der Äquivalenzklassen positiv definiter binärer ganzer quadratischen Formen mit Diskriminante $D = b^2-4ac = -n < 0$, wobei Klassen mit Repräsentanten der Form $d\cdot (X^2 + Y^2)$ respektive $e \cdot (X^2 + XY + Y^2)$ mit Vielfachheit $\frac{1}{2}$ beziehungsweise $\frac{1}{3}$ gezählt werden sollen.
	\end{enumerate}
	Man kann zeigen, dass $H(n)$ wohldefiniert ist.
	Definiere zudem Polynome via
	\[
	(1-tx+Nx^2)^{-1} = \sum_{k=0}^\infty P_{k+2}(t,N) x^k
	\]
\end{defi}

Mit diesen Werkzeugen gilt nun folgende arithmetische Spurformel:

\begin{theorem}[Spurformel, Eichler-Selberg]
	Sei $k \geq 4$ gerade und $m > 0$ beliebig. Dann gilt
	\begin{equation}\label{eq:arithmetischeSpurformel}
	\Tr T(m) = -\frac{1}{2} \sum_{t=-\infty}^\infty P_k(t,m) H(4m-t^2) - \frac{1}{2} \sum_{d|m} \min\Bigl(d, \frac{m}{d}\Bigr)^{k-1}\,.
	\end{equation}
\end{theorem}

\begin{bewe}
	Wir werden den Beweis im Folgenden skizzieren, jedoch stellenweise auf Serge Lang: \glqq{}Introduction to Modular Forms\grqq{} verweisen. Für eine Einführung in die Theorie der quadratischen Formen, welche im Beweis eine wichtige Rolle spielt, verweisen wir auf Don Zagier: \glqq{}Zetafunktionen und quadratische Körper\grqq{}. Natürlich ist dieses Hintergrundwissen nicht klausurrelevant.

In \autoref{analytischeSpurformel} haben wir die Identität
\[
	\Tr T(m) = C_k^{-1} m^{k-1} \int_{\F} \sum_{ad-bc = m} \frac{y^k}{\left(c \abs{z}^2 + d \conj z - az - b\right)^k} \frac{\opd x \opd y}{y^2}
\]
gezeigt. Die innere Summe ist hierbei invariant unter $\Gamma(1)$, da das Integral ansonsten nicht unabhängig von der Wahl des Fundamentalbereiches $\F$ wäre. Genauer behaupten wir: Ersetzt man $z$ durch $\gamma z$ mit $\gamma \in \Gamma(1)$ in den Summanden, so ist dies äquivalent dazu, die Matrix $M = \abcd$ in der Summation durch $\gamma^{-1} M \gamma$ zu ersetzen. Denn: 

Sei $\gamma = \mymat \alpha \beta \delta \epsilon \in \Gamma(1)$ und $M = \mymat abcd \in \M(m)$. Dann gilt
\begin{align*}
	\gamma^{-1} M \gamma &= \mymat \epsilon {-\beta} {-\delta} \alpha \mymat abcd \mymat \alpha \beta \delta \epsilon \\
	&= \mymat {\epsilon a - \beta c} {\epsilon b - \beta d} {-\delta a + \alpha c} {- \delta b + \alpha d} \mymat \alpha \beta \delta \epsilon \\
	&= \mymat {\epsilon a \alpha - \beta c \alpha + \epsilon b \delta - \beta d \delta} {\epsilon a \beta - \beta^2 c + \epsilon^2 b - \beta d \epsilon} {-\delta a \alpha + \alpha^2 c - \delta^2 b + \alpha d \delta} {-\delta a \beta + \alpha c \beta - \delta b \epsilon + \alpha d \epsilon}
	\,.
\end{align*}
Andererseits gilt 
\begin{align*}
	& \frac {\Im(\gamma \circ z)} {c \abs {\gamma \circ z}^2 + d \cdot \left( \conj{\gamma \circ z} \right) - a \cdot \left( \gamma \circ z \right) - b} \\
	&\quad = \frac {\frac {\Im z} {\abs{\delta z + \epsilon}^2}} {c \abs {\frac{\alpha z + \beta}{\delta z + \epsilon}}^2 + d \frac{\alpha \conj z + \beta}{\delta \conj z + \epsilon} - a \frac{\alpha z + \beta}{\delta z + \epsilon} - b} \\
	&\quad = \frac {\Im z} {c(\alpha \conj z + \beta) (\alpha z + \beta) + d(\alpha \conj z + \beta)(\delta z + \epsilon) - a (\alpha z + \beta)(\delta \conj z + \epsilon) -b (\delta \conj z + \epsilon)(\delta z + \epsilon)} \\
	&\quad = \frac {\Im z} {\abs{z}^2 (c\alpha^2 + d \alpha \delta - a \alpha \delta - b \delta^2) + \conj z (c \alpha \beta + d \alpha \epsilon - a \beta \delta - b \delta \epsilon) + z (\ldots) + \ldots}
	\,.
\end{align*}
Multipliziert man, wie im letzten Schritt angedeutet, den Nenner vollständig aus, sortiert die Summanden nach dem Auftreten von $z, \conj z$ und vergleicht abschließend die Koeffizienten mit den Einträgen von $\gamma^{-1} M \gamma$, so ist die Behauptung klar.

Da die Matrizen $M$ und $\gamma^{-1} M \gamma$ nicht nur dieselbe Determinante, sondern auch dieselbe Spur haben (rechne nach und beachte $\alpha \epsilon - \beta \delta = 1$), lässt sich die Summe sogar nach der Spur der Matrizen in $\Gamma(1)$-invariante Teile der Form
\[
	I(m,t) := C_k^{-1} m^{k-1} \int_\F \sum_{\substack{ad-bc = m\\a+d = t}} \frac{y^k}{\left(c \abs{z}^2 + d \conj z - az - b\right)^k} \frac{\opd x \opd y}{y^2}
\]
zerlegen, sodass
\[
	\Tr T(m) = \sum_{t = -\infty}^\infty I(m,t)
	\,.
\]

Wir werden im Folgenden beweisen:
\begin{equation}
	\label{eq:meanI(m,t)}
	\frac 12 (I(m,t) + I(m, -t)) = \begin{cases}
		- \frac 12 P_k(t,m) H(4m - t^2) &\text{, falls } t^2 - 4m < 0 \\
	\frac{k-1}{24} m^{\frac {k-2}2} - \frac 14 m^{\frac {k-1}2} &\text{, falls } t^2 - 4m = 0 \\
	- \frac 12 \left( \frac{\abs{t} - u}2 \right)^{k-1} &\text{, falls } t^2 - 4m = u^2, u \in \NN \\
	0 &\text{, falls } t^2 - 4m > 0 \text{ kein Quadrat} \,.
\end{cases}
\end{equation}
Wie man durch eine Rechnung einsieht, impliziert dies die Aussage des Theorems: Hierbei nutzt man aus, dass $H$ für negative Argumente verschwindet und $H(0) := - \frac 1{12}$ ist, um die linke Summe in \eqref{eq:arithmetischeSpurformel} von solchen $t$ mit $t^2 - 4m < 0$ auf alle $t \in \ZZ$ auszudehnen. Die rechte Summe kommt durch die Fälle $t^2 - 4m = u^2$ mit $u \in \NN$ zustande, denn: Es gilt
\[
	m = \frac {t^2 - u^2}4 = \frac {t+u}2 \cdot \frac {t-u}2 = \abs{\frac {t+u}2} \cdot \abs{\frac {t-u}2}
	\,,
\]
wobei $t, u$ wegen $t^2 - 4m = u^2$ entweder beide gerade oder beide ungerade sind. Damit sind die beiden Faktoren rechts immer ganzzahlig und wegen des Absolutbetrages positiv. Da über alle $t \in \ZZ$ summiert wird, trifft $\abs{\frac {t+u}2}$ jeden Teiler $d$ von $m$. Gleichzeitig trifft $\abs{\frac {t-u}2}$ den Teiler $\frac md$. Und wie man sich leicht überzeugt, ist
\[
	\frac{\abs{t} - u}2 = \min \Biggl( \abs{\frac {t+u}2} ,\abs{\frac {t-u}2} \Biggr) = \min \Bigl( d, \frac md \Bigr)
	\,.
\]

Wir müssen im Folgenden also \glqq{}nur\grqq{} noch das Integral $I(m, t)$ studieren. Dazu bemerken wir zunächst folgendes Lemma:

\begin{lemm}\label{Mt(m)=QD}
Es seien $\M_t(m)$ die Menge der ganzzahligen Matrizen mit Determinante $m$ und Spur $t$ sowie $Q_D$ die Menge der binären quadratischen Formen mit Diskriminante $D = t^2 - 4m$. Dann sind die beiden Mengen gleichmächtig.
\end{lemm}

\begin{bewe}
Wir konstruieren eine konkrete Bijektion
\[
	\phi \colon \M_t (m) \to Q_D, \; \abcd \mapsto g(u, v) = cu^2 + (d-a)uv - bv^2
	\,.
\]
Diese ist wegen 
\begin{align*}
	D_g 
	&= (d-a)^2 + 4bc 
	= d^2 - 2ad + a^2 + 4ad + 4 (bc-ad) \\
	&= (d+a)^2 - 4(ad-bc) 
	= t^2 - 4m
\end{align*}
wohldefiniert. Betrachte nun einen Kandidaten für die Umkehrabbildung
\[
	\phi^{-1} \colon Q_D \to \M_t (m), \; g(u, v) = \alpha u^2 + \beta uv + \gamma v^2 \mapsto \mymat* {\frac 12 (t - \beta)} {-\gamma} \alpha {\frac 12(t + \beta)}
	\,.
\]
Für die Wohldefiniertheit dieser Abbildung rechnet man nach, dass die angegebene Matrix Determinante $\frac 14 (t^2 - \beta^2) + \alpha \gamma = \frac 14 (t^2 - D_g) = m$, Spur $t$ sowie ganzzahlige Einträge besitzt. Letzteres folgt aus $\alpha, \gamma, \beta, t$ ganzzahlig und
\[
	D_g = t^2 - 4m = \beta^2 - 4\alpha \gamma \quad \Ra \quad t^2 \equiv \beta^2 \mod 2 \quad \Ra \quad t \equiv \beta \mod 2
	\,.
\]

Wendet man nun auf $\abcd \in \M_t(m)$ die Abbildungen $\phi$ und $\inv \phi$ an, so erhält man mit $\alpha = c$, $\beta = d - a$ und $\gamma = -b$ nach Definition von $\phi$ sowie mit $t = a+d$ die Matrix
\[
	\mymat* {\frac 12 (t - \beta)} {-\gamma} \alpha {\frac 12(t + \beta)} = \mymat* {\frac 12 (a + d - (d-a))}bc{\frac 12(a + d + (d-a))} = \mymat* abcd
\]
und es folgt $\phi^{-1} \circ \phi = id_{\M_t(m)}$. Analog lässt sich auch $\phi \circ \phi^{-1} = id_{Q_D}$ zeigen. Damit ist das Lemma bewiesen.
\end{bewe}

Für jede binäre quadratische Form $g(u, v) = \alpha u^2 + \beta uv + \gamma v^2$ und $z = x + iy \in \CC, t \in \RR$ beliebig definieren wir nun
\[
	R_g(z,t) := \frac {y^k} {\left( \alpha(x^2 + y^2) + \beta x + \gamma - ity \right)^k}
	\,.
\]
Beachte, dass $g$ von der Matrix $\abcd$ herrührt via der Bijektion $\phi$ aus \autoref{Mt(m)=QD} durch $\alpha = c$, $\beta = d - a$ und $\gamma = -b$. Damit gilt
\[
	I(m, t) = C_k^{-1} m^{k-1} \int_\F \smashoperator[r]{\sum_{g, D_g = D}} R_g(z,t) \frac {\opd x \opd y}{y^2}
	\,,
\]
wobei sich die Summe über alle binären quadratischen Formen $g \in Q_D$, also mit Diskriminante $D_g = D := t^2 - 4m$, erstreckt.

Ein beliebiges Element $\gamma \in \Gamma(1)$ liefert, angewendet auf $g$ durch
\[
	\gamma g (u, v) := g \left( \gamma \begin{pmatrix}u\\v\end{pmatrix} \right)
	\,,
\]
wieder eine quadratische Form $\gamma g$. Es gilt zudem $R_{\gamma g} (z, t) = R_g (\gamma z, t)$. Daher können wir für jede Diskriminante $D = t^2 - 4m \equiv 0,1 \mod 4$ (Übung!) die Summe über alle quadratischen Formen mit dieser Diskrimante wie folgt aufteilen:
\[
	\smashoperator[r]{\sum_{g, D_g = D}} R_g (z, t) = \smashoperator[l]{\sum_{\substack{g, D_g = D \\ g (\operatorname{mod} \Gamma(1))}}} \smashoperator[r]{\sum_{\gamma \in \modulo {\Gamma(1)} {\Gamma_g}}} R_{\gamma g} (z,t) = \smashoperator[l]{\sum_{\substack{g, D_g = D \\ g (\operatorname{mod} \Gamma(1))}}} \smashoperator[r]{\sum_{\gamma \in \modulo {\Gamma(1)} {\Gamma_g}}} R_g (\gamma z,t)
	\,.
\]
Hierbei erstreckt sich die erste Summe nur noch über ein Vertretersystem aller Klassen modulo $\Gamma(1)$ von binären quadratischen Formen $g$ mit Diskriminante $D_g = D$. Die zweite Summe erstreckt sich über alle Nebenklassen $\modulo {\Gamma(1)}{\Gamma_g}$, wobei $\Gamma_g$ die Fixgruppe von $g$ ist. 

Für $D \neq 0$ gibt es nur endlich viele Klassen modulo $\Gamma(1)$, sodass die erste Summe endlich ist und wir schreiben können:
\begin{equation}\label{eq:intFsumGforDneq0}
	\int_\F \smashoperator[r]{\sum_{g, D_g = D}} R_g(z,t) \frac{\opd x \opd y}{y^2} = \smashoperator[l]{\sum_{\substack{g, D_g = D \\ g (\operatorname{mod} \Gamma(1))}}} \int_{\F_g} R_g (z,t) \frac{\opd x \opd y}{y^2}
\end{equation}
mit 
\[
	\F_g = \smashoperator{\bigcup_{\gamma \in \modulo {\Gamma(1)}{\Gamma_g}}} \, \gamma \F
\]
einem Fundamentalbereich für die Operation von $\Gamma_g$ auf $\HH$.

Für $D = 0$ ist dagegen ein (unendliches) System von Repräsentanten gegeben durch die Formen $\Set {g_r(u,v) = rv^2 \mid r \in \ZZ }$. Die Fixgruppe eines solchen $g_r$ ist gegeben durch
\[
	\Gamma_{g_r} = \begin{cases} \Gamma(1) &\text{, falls } r = 0\\ \Gamma(1)_\infty &\text{, falls } r \neq 0 \,. \end{cases}
\]
Damit finden wir
\begin{equation}\label{eq:intFsumGforD=0}
	\int_\F \smashoperator[r]{\sum_{g, D_g = 0}} R_g(z,t) \frac{\opd x \opd y}{y^2} = \int_\F R_{g_0} (z,t) \frac{\opd x \opd y}{y^2} + \int_{\F_\infty} \sum_{\substack{r \in \ZZ \\ r \neq 0}} R_{g_r}(z,t) \frac{\opd x \opd y}{y^2}
	\,,
\end{equation}
wobei $\F_\infty$ ein Fundamentalbereich für die Operation von $\Gamma(1)_\infty$ auf $\HH$ ist. Ein Beispiel hierfür ist der Vertikalstreifen $\Set {z \in \HH \mid 0 < \Re (z) < 1}$.

Wir zeigen nun die in \eqref{eq:meanI(m,t)} behaupteten Formeln für den Mittelwert der Integrale
\[
	\frac 12 \left( I(m,t) + I(m,-t) \right)
\]
abhängig von $D := t^2 - 4m$ und unterscheiden dieselben vier Fälle:

\emph{Fall 1:} $D < 0$. In diesem Fall ist $\Gamma_g$ endlich (genau genommen von Ordnung $1 \leq \abs{\Gamma_g} \leq 3$, Beweis entfällt aus Zeitgründen). Für eine quadratische Form $g(u,v) = \alpha u^2 + \beta uv + \gamma v^2$ mit Diskriminante $D_g = D = t^2 - 4m$ erhalten wir demnach:
\begin{align*}
	\int_{\F_g} R_g (z,t) \frac{\opd x \opd y}{y^2} 
	&= \frac {1}{\abs{\Gamma_g}} \int_\HH R_g (z,t) \frac{\opd x \opd y}{y^2} \\
	&= \frac {1}{\abs{\Gamma_g}} \int_\HH \frac{y^{k-2}}{\left( \alpha(x^2 + y^2) + \beta x + \gamma - ity \right)^k} \opd x \opd y
	\,.
\end{align*}
Fasse nun die obere Halbebene $\HH$ als Teilmenge des $\RR^2$ auf und nutze den Diffeomorphismus
\[
	\Phi: \HH \to \HH, (x, y) \mapsto \left( \frac {2x-\beta}{2\alpha}, \frac y\alpha \right)
\]
mit Jacobi-Matrix
\[
	D\Phi = \mymat* {\inv \alpha}00{\inv \alpha} \quad \Ra \quad \det D\Phi = \frac 1{\alpha^2}
\]
zur Substitution per Transformationssatz: 
\begin{align*}
	\int_{\F_g} R_g (z,t) \frac{\opd x \opd y}{y^2}
	&= \frac {1}{\abs{\Gamma_g}} \int_\HH \frac {1}{\alpha^2} \frac{\left(\frac {y}{\alpha} \right)^{k-2}}{\left( \alpha \left( \left( \frac {2x-\beta}{2\alpha} \right)^2 + \left( \frac y\alpha \right)^2 \right) + \beta \left( \frac {2x-\beta}{2\alpha} \right) + \gamma - it \frac y\alpha \right)^k} \opd x \opd y \\
	&= \frac {1}{\abs{\Gamma_g}} \int_\HH \frac{y^{k-2}}{\left( (x^2 + y^2) - \frac {\beta^2}4 + \gamma \alpha - ity \right)^k} \opd x \opd y \\
	&= \frac {1}{\abs{\Gamma_g}} \int_\HH \frac{y^{k-2}}{\left( \abs{z}^2 - \frac 14 D_g - ity \right)^k} \opd x \opd y
	\,.
\end{align*}
Im letzten Schritt geht hierbei ein, dass $D_g = \beta^2 - 4 \alpha \gamma$, also $- \frac {\beta^2}4 + \alpha \gamma = - \frac 14 D_g$. Das Integral hängt somit nicht von den konkreten Parametern $\alpha, \beta, \gamma$ der Form $g$ ab, sondern nur von ihrer Diskriminante $D_g$. Da wir nur Formen $g$ mit Diskriminante $D_g = D = t^2 - 4m$ betrachten, können wir mit
\[
	I(D, t) := \int_\HH \frac{y^{k-2}}{\left( \abs{z}^2 - \frac 14 D - ity \right)^k} \opd x \opd y
\]
schreiben:
\[
	\sum_{\substack{g, D_g = D \\ g (\operatorname{mod} \Gamma(1))}} \int_{\F_g} R_g(z,t) \frac{\opd x \opd y}{y^2} = \smashoperator[l]{\sum_{\substack{g, D_g = D \\ g (\operatorname{mod} \Gamma(1))}}} \frac{1}{\abs{\Gamma_g}} I(D,t) = 2 H(-D) \cdot I(D,t)
\]
Die letzte Umformung liefert einen derart einfachen Term, da die Definition von $H$ wegen
\[
	\abs{\Gamma_g} = \begin{cases}
	2 &\text{, falls } g \sim d\cdot (X^2 + Y^2) \text{ für ein } d \in \ZZ \setminus \Set {0} \\
	3 &\text{, falls } g \sim e \cdot (X^2 + XY + Y^2) \text{ für ein } e \in \ZZ \setminus \Set {0}	\\
	1 &\text{, sonst }
	\end{cases}
\]
bereits den Vorfaktor $\frac 1{\abs{\Gamma_g}}$ beinhaltet. Der Faktor $2$ rührt von der Tatsache her, dass in der Definition von $H$ nur Klassen positiv definiter quadratischer Formen gezählt werden. Die obige Summe berücksichtigt jedoch Klassen aller Formen $g$ mit geeigneter Diskriminante $D_g = D$. Da alle diese Formen wegen $D_g = D < 0$ definit (also entweder positiv oder negativ definit) sind, besteht die Summe aus genau doppelt so vielen Summanden wie durch $H$ angegeben (zu jeder positiv definiten Form erhält man eine negativ definite Form gleicher Diskriminante durch Multiplikation mit $-1$ und umgekehrt).

Nutzt man nun die für beliebiges $A \in \CC_{-}$ gültige Formel
\[
	\int_{-\infty}^\infty (x^2 + A)^{-k} \opd x = \frac \pi{(k-1)!} \cdot \frac 12 \cdot \frac 32 \cdot \frac 52 \cdots \left( k-\frac 32 \right) \cdot A^{\frac 12 -k}
	\,,
\]
erhält man mit $A = y^2 - ity - \frac 14 D$ schließlich
\begin{align*}
	I(D, t) &= \int_\HH \frac{y^{k-2}}{\left( \abs{z}^2 - \frac 14 D - ity \right)^k} \opd x \opd y \\
	&= \int_0^\infty y^{k-2} \int_{-\infty}^\infty (x^2 + y^2 - ity - \frac 14 D)^{-k} \opd x \opd y && \Big| \text{ siehe oben} \\
	&= \frac \pi{(k-1)!} \cdot \frac 12 \cdot \frac 32 \cdot \frac 52 \cdots \left( k-\frac 32 \right) \int_0^\infty (y^2 - ity - \frac 14 D)^{\frac 12 - k} y^{k-2} \opd y && \Big| \text{ Leibniz} \\
	&= \frac{\pi}{(k-1)!} \cdot \frac 12 \cdot \frac 1{i^{k-2}} \left( \! \frac {\opd}{\opd t} \right)^{\!\! k-2} \int_0^\infty (y^2 - ity - \frac 14D)^{-\frac 32} \opd y \\
	&= \frac{\pi i^{k-2}}{2(k-1)!} \left( \! \frac {\opd}{\opd t} \right)^{\!\! k-2} \left[ \frac{4}{t^2 - D} \cdot \frac{y - \frac 12 it}{\sqrt{y^2 - ity - \frac 14 D}} \right]_0^\infty && \Big| \text{ nachrechnen!} \\
	&= \frac{\pi i^{k-2}}{2(k-1)!} \left( \! \frac {\opd}{\opd t} \right)^{\!\! k-2} \left( \frac 4{\sqrt{\abs{D}}} \cdot  \frac 1{\sqrt{\abs{D}} - it} \right) \\
	&= \frac{2\pi}{k-1} \cdot \frac 1{\sqrt{\abs{D}}} \cdot \frac {i^{k-2}}{(k-2)!} \left( \! \frac {\opd}{\opd t} \right)^{\!\! k-2} \left( \frac 1{\sqrt{\abs{D}} - it} \right) \\
	&= \frac {2\pi}{k-1} \cdot \frac 1{\sqrt{\abs{D}}} \cdot \frac 1{\left( \sqrt{\abs{D}} - it \right)^{k-1}}
	\,.
\end{align*}
Zusammengefasst gilt also (unter Beachtung von $D = t^2 - 4m < 0$), dass
\begin{align*}
	I(m, t) 
	&= C_k^{-1} m^{k-1} \int_\F \smashoperator[r]{\sum_{g, D_g = D}} R_g(z,t) \frac {\opd x \opd y}{y^2} \\
	&\overset{\eqref{eq:intFsumGforDneq0}}= C_k^{-1} m^{k-1} \smashoperator[l]{\sum_{\substack{g, D_g = D \\ g (\operatorname{mod} \Gamma(1))}}} \int_{\F_g} R_g (z,t) \frac{\opd x \opd y}{y^2} \\
	&= C_k^{-1} m^{k-1} \cdot 2 H(-D) \cdot I(D,t) \\
	&= C_k^{-1} m^{k-1} \cdot 2 H (4m-t^2) \cdot \frac {2\pi}{k-1} \cdot \frac 1{\sqrt{4m-t^2}} \cdot \frac 1{\left( \sqrt{4m-t^2} - it \right)^{k-1}}
	\,,
\end{align*}
was sich mit $\rho := \frac 12 ( t + i \sqrt{4m-t^2} )$ nach kurzer Rechnung vereinfachen lässt zu:
\[
	I(m, t) = \frac{\conj{\rho}^{k-1}}{\rho - \conj{\rho}} H(4m-t^2)
	\,.
\]

Beachtet man nun, dass
\[
	\rho + \conj{\rho} = \frac 12 \left( t + i \sqrt{4m-t^2} \right) + \frac 12 \left( t - i \sqrt{4m-t^2} \right) = t
\]
und
\[
	\rho \conj{\rho} = \abs{\rho}^2 = \frac 14 \abs{t^2 + 4m - t^2} = m
	\,,
\]
so kann man für die per
\[
	(1 - tx + Nx^2)^{-1} = \sum_{k=0}^\infty P_{k+2}(t,N)x^k = P_2(t,N) + P_3(t,N)x + P_4(t,N)x^2 + \ldots
\]
definierten Polynome $P_k(t, N)$ mithilfe von Partialbruchzerlegung und der geometrischen Reihe die Beziehung
\[
	P_k(t, m) = \frac{\rho^{k-1} - \conj{\rho}^{k-1}}{\rho - \conj{\rho}}
\]
zeigen.

Daraus folgt in der Tat wie in \eqref{eq:meanI(m,t)} behauptet
\[
	\frac 12 \left( I(m,t) + I(m,-t) \right) = - \frac 12 \left( \frac{\rho^{k-1}}{\rho - \conj{\rho}} + \frac{- \conj{\rho}^{k-1}}{\rho - \conj{\rho}} \right) H(4m-t^2) = - \frac 12 P_k(t,m) H(4m-t^2)
	\,.
\]

\emph{Fall 2:} $D = 0$. Wir benutzen hierfür die oben hergeleitete Formel \eqref{eq:intFsumGforD=0}
\[
	\int_\F \smashoperator[r]{\sum_{g, D_g = 0}} R_g(z,t) \frac{\opd x \opd y}{y^2} = \int_\F R_{g_0} (z,t) \frac{\opd x \opd y}{y^2} + \int_{\F_\infty} \sum_{\substack{r \in \ZZ \\ r \neq 0}} R_{g_r}(z,t) \frac{\opd x \opd y}{y^2}
	\,.
\]

Wegen $g_0(u,v) \equiv 0$ sind alle Parameter $\alpha, \beta, \gamma$ der quadratischen Form $g_0$ gleich 0, sodass sich $R_{g_0}$ und damit der erste Summand leicht berechnen lassen:
\begin{align*}
	\int_\F R_{g_0} (z,t) \frac{\opd x \opd y}{y^2} 
	&= \int_\F \left( \frac y{-ity} \right)^{\!\! k} \frac{\opd x \opd y}{y^2} 
	= \int_\F \left( \frac it \right)^{\!\! k} \frac{\opd x \opd y}{y^2} \\
	&= \left( \frac it \right)^{\!\! k} \int_\F \frac{\opd x \opd y}{y^2}
	= \frac {(-1)^{\frac k2}}{t^k} \cdot \frac \pi 3 
	= (-1)^{\frac k2} \frac \pi{3t^k}
	\,.
\end{align*}

Für den zweiten Term finden wir mit $g_r(u,v) = rv^2$, dass
\begin{align*}
	\int_{\F_\infty} \sum_{\substack{r \in \ZZ \\ r \neq 0}} R_{g_r}(z,t) \frac{\opd x \opd y}{y^2}
	&= \int_0^\infty \int_0^1 y^{k-2} \sum_{\substack{r \in \ZZ \\ r \neq 0}} (r-ity)^{-k} \opd x \opd y && \Big| \text{ PBZ des } \cot \\
	&= \frac{i^{k-2}}{(k-2)!} \left( \! \frac {\opd}{\opd t} \right)^{\!\! k-2} \int_0^\infty \frac 1{t^2y^2} - \frac {\pi^2}{\sinh^2(\pi ty)} \opd y \\
	&= \frac{i^{k-2}}{(k-1)!} \left( \! \frac {\opd}{\opd t} \right)^{\!\! k-2} \frac \pi {\abs{t}} \\ 
	&= (-1)^{\frac {k-2}{2}} \frac{\pi}{k-1} \abs{t}^{-k+1}
	\,.
\end{align*}

Für $t = \pm 2\sqrt m$ (da $D := t^2 - 4m = 0$) bekommen wir damit nach kurzer Rechnung
\begin{align*}
	I(m,t) 
	&= C_k^{-1} m^{k-1} \int_\F \smashoperator[r]{\sum_{g, D_g = 0}} R_g(z,t) \frac {\opd x \opd y}{y^{k-2}} \\
	&= C_k^{-1} m^{k-1} \int_\F R_{g_0} (z,t) \frac{\opd x \opd y}{y^2} + C_k^{-1} m^{k-1} \int_{\F_\infty} \sum_{\substack{r \in \ZZ \\ r \neq 0}} R_{g_r}(z,t) \frac{\opd x \opd y}{y^2} \\
	&= C_k^{-1} m^{k-1} (-1)^{\frac k2} \frac \pi{3t^k} + C_k^{-1} m^{k-1} (-1)^{\frac {k-2}{2}} \frac{\pi}{k-1} \abs{t}^{-k+1} \\
	&= \frac {k-1}{24} m^{\frac {k-2}2} - \frac 14 m^{\frac {k-1}2}
	\,.
\end{align*}

\emph{Fall 3:} $D = u^2$ mit $u \in \NN$. Wie in Fall 1 ($D < 0$) gibt es hier nur eine endliche Anzahl von Klassen mit Diskriminante $D$ und $\Gamma_g$ ist eine endliche Gruppe. Es folgt damit ähnlich wie in Fall 1
\[
	\int_\F \smashoperator[r]{\sum_{g, D_g = D}} R_g(z,t) \frac {\opd x \opd y}{y^2} = H_D \cdot I(D, t)
\]
mit
\[
	H_D := \smashoperator[l]{\sum_{\substack{g, D_g = D \\ g (\operatorname{mod} \Gamma(1))}}} \frac 1{\abs{\Gamma_g}}
\]
und (vergleiche Fall 1)
\[
	I(D, t) := \int_\HH \frac{y^{k-2}}{\left( \abs{z}^2 - \frac 14 D - ity \right)^k} \opd x \opd y
	\,.
\]
Man kann nun zeigen, dass in Fall 3 alle Fixgruppen $\Gamma_g$ trivial sind (d.h. $\abs{\Gamma_g} = 1$) und es darüber hinaus genau $u$ Klassen quadratischer Formen $g$ mit Diskriminante $D_g = u^2$ gibt. Hieraus folgt $H_D = u$. 

Analog zu Fall 1 können wir zudem wieder das Integral umformen zu
\[
	I(D, t) = \frac{\pi i^{k-2}}{2(k-1)!} \left( \frac {\opd}{\opd t} \right)^{k-2} \left[ \frac{4}{t^2 - D} \frac{y - \frac 12 it}{\sqrt{y^2 - ity - \frac 14 D}} \right]_0^\infty
	\,,
\]
wobei der ausgewertete Term rechts diesmal unter Beachtung von $D = t^2 - 4m > 0$ zu 
\[
	\frac {-4}{\sqrt{D}} \frac 1{\sqrt{D} + \abs{t}}
\]
wird. Es folgt wegen $\sqrt D = \sqrt {u^2} = u$, dass
\[
	I(D, t) = (-1)^{\frac {k-2}2} \frac{2\pi}{k-1} \cdot \frac 1{\sqrt{D}} \cdot \frac 1{(\sqrt{D} + \abs{t})^{k-1}} = (-1)^{\frac {k-2}2} \frac{2\pi}{k-1} \cdot \frac 1u \cdot \frac 1{(u + \abs{t})^{k-1}}
	\,,
\]
und somit nach kurzer Rechnung
\begin{align*}
	I(m,t)
	&= C_k^{-1} m^{k-1} \int_\F \smashoperator[r]{\sum_{g, D_g = D}} R_g(z,t) \frac {\opd x \opd y}{y^2} \\
	&= C_k^{-1} m^{k-1} H_D \cdot I(D, t) \\ 
	&= C_k^{-1} m^{k-1} (-1)^{\frac {k-2}2} \frac{2\pi}{k-1} \cdot \frac 1{(u + \abs{t})^{k-1}} \\
	&= - \frac 12 \left( \frac{\abs{t} - u}2 \right)^{k-1}
	\,.
\end{align*}

\emph{Fall 4}: $D > 0$, aber keine Quadratzahl. Wir zeigen in diesem Fall wie in \eqref{eq:meanI(m,t)} behauptet, dass 
\begin{align*}
	\frac 12 \left( I(m,t) + I(m,-t) \right) 
	&= \frac 12 C_k^{-1} m^{k-1} \cdot \left( \int_\F \smashoperator[r]{\sum_{g, D_g = D}} R_g(z,t) \frac {\opd x \opd y}{y^2} + \int_\F \smashoperator[r]{\sum_{g, D_g = D}} R_g(z,-t) \frac {\opd x \opd y}{y^2} \right) \\
	& \overset{\eqref{eq:intFsumGforDneq0}}= \frac 12 C_k^{-1} m^{k-1} \cdot \smashoperator[l]{\sum_{\substack{g, D_g = D \\ g (\operatorname{mod} \Gamma(1))}}} \left( \int_{\F_g} R_g (z,t) \frac{\opd x \opd y}{y^2} + \int_{\F_g} R_g (z,-t) \frac{\opd x \opd y}{y^2} \right)
	= 0
	\,,
\end{align*}
indem wir für alle Formen $g$ mit $D_g = D$ zeigen, dass
\[
	\int_{\F_g} R_g (z, t) \frac {\opd x \opd y}{y^2} + \int_{\F_g} R_g (z, -t) \frac {\opd x \opd y}{y^2} = 0
	\,.
\]
Sei hierfür $g(u, v) = \alpha u^2 + \beta uv + \gamma v^2$ eine quadratische Form mit Diskriminante $D > 0$ kein Quadrat und seien außerdem $w > w'$ die Lösungen der Gleichung $\alpha u^2 + \beta u + \gamma = 0$ (wegen $D_g = D > 0$ gibt es zwei verschiedene reelle Lösungen). Dann transformiert die Matrix
\[
	M = (w - w')^{- \frac 12} \mymat*{w'}w11 \in \GL_2(\RR)
\]
$g$ in $Mg$ mit
\[
	M g(u, v) = \sqrt{D} uv
	\,.
\]
In der Tat: Ist $T = \sqrt{w-w'}$, so gilt
\begin{align*}
	M^t \mymat*{\alpha}{\frac \beta 2}{\frac \beta 2}{\gamma} M 
	&= \frac{1}{T^2} \mymat*{w'}1w1 \mymat*{\alpha}{\frac \beta 2}{\frac \beta 2}{\gamma} \mymat*{w'}w11 \\
	&= \frac{1}{T^2} \mymat*{\alpha w' + \frac \beta 2}{\frac \beta 2 w' + \gamma}{\alpha w + \frac \beta 2}{\frac \beta 2 w + \gamma} \mymat*{w'}w11 \\
	&= \frac{1}{T^2} \mymat*{\alpha w'^2 + \frac \beta 2 w' + \frac \beta 2 w' + \gamma}{\alpha ww' + \frac \beta 2 w + \frac \beta 2 w' + \gamma}{\alpha ww' + \frac \beta 2 w' + \frac \beta 2 w + \gamma}{\alpha w^2 + \frac \beta 2 w + \frac \beta 2 w + \gamma} \\
	&= \frac{1}{T^2} \mymat*{\alpha w'^2 + \beta w' + \gamma}{\alpha ww' + \frac \beta 2 (w + w') + \gamma}{\alpha ww' + \frac \beta 2 (w' + w) + \gamma}{\alpha w^2 + \beta w + \gamma}
	\,.
\end{align*}
Wegen $\det M = \det M^t = \pm 1$ bleibt die Determinante der Matrix (und somit auch die Diskriminante der von der Matrix induzierten quadratischen Form) trotz Transformation unverändert. Da sowohl $w$ als auch $w'$ die Gleichung $\alpha u^2 + \beta u + \gamma = 0$ lösen, verschwinden beide Diagonaleinträge und die Form $Mg$ lässt sich schreiben als $Mg(u, v) = b uv$ für ein $b \in \RR$. Hieraus erhält man nach kurzer Rechnung wie behauptet $Mg(u, v) = \sqrt D uv$.

Nach nichttrivialen Überlegungen weiß man: Die Gruppe $\inv M \Gamma_g M$ ist zyklisch und kann durch $\mymat \epsilon 0 0 {\frac 1 \epsilon}$ erzeugt werden, wobei $\epsilon > 1$ die Fundamentaleinheit des Ganzheitsrings $R$ in $\QQ(\sqrt{D})$ ist (hier geht auch ein, dass $D$ keine Quadratzahl ist, sonst wäre $\QQ(\sqrt{D}) = \QQ$). Daher können wir den Fundamentalbereich $\F_g$ so wählen, dass $\inv M \F_g$ ein Kreisring $\inv M \F_g = \Set {z \in \HH \mid r_0 \leq \abs{z} \leq \epsilon^2 r_0}$ in der oberen Halbebene ist. Es gilt somit
\begin{align*}
	I_+
	&:= \int_{\F_g} R_g (z, t) \frac {\opd x \opd y}{y^2} \\
	&= \int_{\F_g} R_{M g} (\inv M z, t) \frac {\opd x \opd y}{y^2} \\
	&= \int_{\inv M \F_g} R_{M g} (z, t) \frac {\opd x \opd y}{y^2} && \Big| Mg(u,v) = \sqrt{D} uv \\
	&= \int_{\substack{y > 0 \\ r_0 \leq \abs{z} \leq \epsilon^2 r_0}} \left( \sqrt{D}x - ity \right)^{-k} y^{k-2} \opd x \opd y
	\,.
\end{align*}

Erhalte nun mit Polarkoordinaten $z = x + iy = r e^{i \theta}$, dass
\begin{align*}
	I_+
	&= \int_0^{\pi} \int_{r_0}^{r_0 \epsilon^2} \left( \sqrt{D} \cos \theta - it \sin \theta \right)^{-k} \left( \sin \theta \right)^{k-2} \frac {\opd r \opd \theta}r \\
	&= \log (\epsilon^2) \int_0^\pi \left( \sqrt{D} \cos \theta - it \sin \theta \right)^{-k} \left( \sin \theta \right)^{k-2} \opd \theta
\end{align*}
und wegen der Symmetrieeigenschaften von $\sin$ und $\cos$ analog
\begin{align*}
	I_-
	&:= \int_{\F_g} R_g (z, -t) \frac {\opd x \opd y}{y^2} \\
	&= \log (\epsilon^2) \int_0^\pi \left( \sqrt{D} \cos \theta + it \sin \theta \right)^{-k} \left( \sin \theta \right)^{k-2} \opd \theta \\
	&= \log (\epsilon^2) \int_{-\pi}^0 \left( \sqrt{D} \cos \theta - it \sin \theta \right)^{-k} \left( \sin \theta \right)^{k-2} \opd \theta
	\,.
\end{align*}

Wir sind also fertig, wenn wir
\[
	I := \frac {I_+ + I_-}{\log (\epsilon^2)} = \int_{-\pi}^\pi \left( \sqrt{D} \cos \theta - it \sin \theta \right)^{-k} \left( \sin \theta \right)^{k-2} \opd \theta = 0
\]
zeigen können. Schreibe hierfür das Integral um zum Kurvenintegral
\begin{align*}
	I
	&= \int_{-\pi}^{\pi} \left( \sqrt D \frac {e^{i\theta} + e^{-i\theta}}2 - it \frac {e^{i\theta} - e^{-i\theta}}{2i} \right)^{-k} \left( \frac {e^{i\theta} - e^{-i\theta}}{2i} \right)^{k-2} \opd \theta \\
	&= \int_{-\pi}^{\pi} \left( \sqrt D i \frac {e^{i\theta} + e^{-i\theta}}{e^{i\theta} - e^{-i\theta}} - it \right)^{-k} \left( \frac {e^{i\theta} - e^{-i\theta}}{2i} \right)^{-2} \opd \theta \\
	&= \int_{-\pi}^{\pi} \left( \sqrt D i \frac {e^{2i\theta} + 1}{e^{2i\theta} - 1} - it \right)^{-k} \left( \frac {e^{2i\theta} - 1}{2i} \cdot e^{-i\theta} \right)^{-2} \opd \theta \\
	&= \frac 1{2i} \int_{-\pi}^{\pi} \left( \sqrt D i \frac {e^{2i\theta} + 1}{e^{2i\theta} - 1} - it \right)^{-k} \left( \frac {e^{2i\theta} - 1}{2i} \right)^{-2} 2ie^{2i\theta} \opd \theta \\
	&= \frac 1{2i} \int_{\mathcal C} \left( \sqrt D i \frac {z+1}{z-1} - it \right)^{-k} \left( \frac{z-1}{2i} \right)^{-2} \opd z
\end{align*}
mit $\mathcal C \colon \left[ -\pi, \pi \right] \to \CC, \theta \mapsto e^{2i\theta}$ geschlossen. Wegen $k \geq 4 > 2$ hebt der linke Faktor die Polstelle des rechten Faktors bei $z = 1$. Die Polstelle des linken Terms liegt bei
\[
	\sqrt D i \frac{z+1}{z-1} - it \overset != 0 \quad \Ra \quad z \overset != \frac {\frac t{\sqrt{D}} + 1}{\frac t{\sqrt{D}} - 1} > 1
	\,,
\]
sodass der Integrand für $\delta > 0$ klein genug auf der Kreisscheibe $U_{1 + \delta}(0) \supset \closure \EE$ holomorph ist. Nach dem Residuensatz ist das Integral entlang $\mathcal C = \partial \EE$ daher $0$. Damit folgt für $D = t^2 - 4m > 0$ kein Quadrat, dass tatsächlich $\frac 12 \left( I(m,t) + I(m,-t) \right) = 0$ gilt.

Hiermit haben wir \eqref{eq:meanI(m,t)} vollständig verifiziert und damit die in \eqref{eq:arithmetischeSpurformel} behauptete arithmetische Spurformel bewiesen.

\end{bewe}

%chapter 6
\chapter{L-Reihen zu Modulformen}

\section{Dirichletreihen}

\emph{Ziel}: Angabe elementarer Eigenschaften von \myquote{Dirichletreihen}, welche eine besondere Stellung in der analytischen Theorie der Zahlen einnehmen.

\begin{defi}\label{def:dirichletreihe}
	Formell ist eine \myemph{Dirichletreihe} eine Reihe der Form
	\[
		\sum_{n=1}^{\infty} a(n) e^{-\lambda_n s}
		\,,
	\]
	wobei die $\lambda_n$ reelle Zahlen sind mit $\lambda_1 < \lambda_2 < \lambda_3 < \ldots \xto{n\to\infty} \infty$ und $s= \sigma +it$ eine komplexe Zahl ist.
\end{defi}

\begin{beme-list}
	\item $\lambda_n = n$ ist naheliegend führt aber mit $z = e^{-s}$ zur Theorie der Potenzreihen, die wir schon ausgiebig in der Funktionentheorie I studiert haben.
	
	\item $\lambda_n = \log (n)$: Mit dieser Wahl lässt sich die obere Reihe \myquote{schöner} schreiben als
	\begin{equation}\label{eq:gewDirichtletreihe}
		\sum_{n=1}^\infty a(n) n^{-s}
		\,.
	\end{equation}
	Das ist der für die Zahlentheorie relevante Fall.
	Eine Reihe der Gestalt \eqref{eq:gewDirichtletreihe} heißt \myemph[Dirichletreihe!Gewöhnliche Dirichletreihe]{gewöhnliche Dirichletreihe}.
	
	\item Im Gegensatz zu Potenzreihen weisen Dirichletreihen ein anderes Konvergenzverhalten auf als das \myquote{auf Kreisscheiben} auf.
	Während also für eine Potenzreihe $f(z) = \sum_{n=0}^\infty a(n) z^n$ stets ein $0 \leq R \leq \infty$ existiert, so dass $f$ für $z \in \CC$ mit $\abs z < R$ konvergiert und für $\abs z > R$ divergiert, \myquote{konvergieren Dirichletreihen auf Halbebenen statt Kreisscheiben}.
	Dies präzisiert der nächste Satz
\end{beme-list}

\begin{satz}\label{satz:konvergenzDirichletreihen}
	Es sei $F(s) := \sum_{n=1}^\infty a(n) e^{-\lambda_n s}$ eine Dirichletreihe wie in \autoref{def:dirichletreihe} definiert.
	Ist diese für ein $s = s_0$ konvergent, so konvergiert sie auch für alle $Re(s) > \sigma_0\ (= \Re(s_0))$ und gleichmäßig auf Kompakten Mengen.
	
	Somit existiert eine reelle Zahl $\sigma_0$, so dass die Reihe für alle $s$ mit $\Re s > \sigma_0$ konvergiert und für alle $\Re(s) < \sigma_0$ divergiert (falls überall konvergent, setze $\sigma_0 = -\infty$, falls überall divergent setze $\sigma_0 = \infty$).
	
	Die in dem Gebiet $\Set{s \in \CC \mid \Re(s) > \sigma_0}$ durch $F(s)$ definierte Funktion ist dort holomorph, die Ableitungen sind gegeben durch
	\[
		F^{(k)}(s) = (-1)^k \sum_{n=1}^\infty \lambda_n^k a(n) e^{-\lambda_n s}\,,
	\]
	wobei die rechts stehende Dirichletreihe auch für $\sigma > \sigma_0$ konvergiert.
	
	Die Zahl $\sigma_0$ heißt \myemph{Konvergenzabszisse} der Dirichletreihe $F(s)$.
\end{satz}

\begin{bewe}
	
	Es genügt die gleichmäßige Konvergenz im gesamten Bereich zu zeigen, da damit die Existenz eines $\sigma_0$ folgt und die Holomorphieaussagen aus der gleichmäßigen Konvergenz über den Satz von Weierstraß ersichtlich sind.
	
	Wir zeigen, dass die Reihe in jedem Gebiet (siehe \autoref{fig:winkelbereich})
	\begin{equation}\label{eq:winkelbereich}
		\abs{\Arg (s - s_0)} \leq \frac{\pi}{2} - \epsilon < \frac{\pi}{2}
	\end{equation}
	gleichmäßig konvergiert.
	Das ist stärker als die Aussage des Satzes, da jede kompakte Menge $K \subset \Set{s \in \CC \mid \Re(s) > \sigma_0}$ in einem solchen Gebiet liegt.
	
	\begin{figure}
		\begin{center}
			\includestandalone[scale=.8]{images/chapter6/winkelbereich}
			\caption{Das Gebiet aus \eqref{eq:winkelbereich}}
			\label{fig:winkelbereich}
		\end{center}
	\end{figure}

	Wir führen die Bezeichnungen
	\[
	A(N) = \sum_{n=1}^N a(n)\,, \qquad A(M,N) = \sum_{n=M}^N a(n)\,, \qquad A(M,M-1) = 0
	\]
	ein, die in diesem Paragraphen mehrmals benutzt werden. Ohne Einschränkung können wir $s_0 = 0$ voraussetzen (indem wir $s$ durch $s+s_0$ und $a(n)$ durch $a(n)e^{-\lambda_ns_0}$ ersetzen).
	Dann ist $\sum_{n=1}^\infty a(n)$ konvergent und es gibt zu vorgegebenen $\epsilon > 0$ ein $N_0$, so dass $\abs{A(M,N)} \leq \epsilon$ für alle $N_0 \leq M < N$.
	Dann gilt für $N > M \geq N_0$:\footnote{dieses Verfahren wir auch als abelsche Summation bezeichnet}
	\[
	\sum_{n=M}^N a(n)e^{-\lambda_ns}
	&= \sum_{n=M}^N \bigl(A(M,n) - A(M,n-1)\bigr)e^{-\lambda_ns} \\
	&= A(M,M)e^{-\lambda_Ms} - A(M,M) e^{-\lambda_{M+1}s} + A(M, M+1)e^{-\lambda_{M+1}s} \\ &\qquad + \ldots + A(M,N-1) e^{-\lambda_{N-1}s} - A(M,N-1)e^{-\lambda_Ns} + A(M,N) e^{-\lambda_Ns} \\
	&= \sum_{n=M}^{N-1} A(M,n) \bigl(e^{-\lambda_ns} - e^{-\lambda_{n+1}s}\bigr) + A(M,N)e^{-\lambda_Ns}\,.
	\]
	
	Es ist
	\[
	\abs{e^{-\lambda s} - e^{-\lambda_{n+1}s}}
	&= \abs{s \int_{\lambda_n}^{\lambda_{n+1}} e^{-sn} \opd n} \\
	&\leq \abs s \int_{\lambda_n}^{\lambda_{n+1}} \abs{e^{-sn}} \opd n \\
	&= \frac{\abs s}{\sigma} \bigl(e^{-\lambda_n \sigma} - e^{-\lambda_{n+1}\sigma}\bigr) \qquad (\sigma = \Re s) \,.
	\]
	
	Die Größe $\frac {\abs s}\sigma$ ist in den Bereichen $\abs{\Arg(s)} \leq \frac{\pi}{2} - \delta$ durch eine Konstante $C_\delta > 0$ beschränkt.
	Somit ist für $\sigma > 0$:
	\[
	\abs{ \sum_{n=M}^N a(n) e^{-\lambda_ns} }
	& \leq \sum_{n=M}^{N-1} \abs{A(M,n)} \cdot \abs*{e^{-\lambda_n s} - e^{-\lambda_{n+1}s}} + \abs{A(M,N)} \cdot \abs*{e^{-\lambda_Ns}} \\
	&\leq C_\delta \epsilon \sum_{n=M}^{N-1} \bigl(e^{-\lambda_n\sigma} - e^{-\lambda_{n+1}\sigma}\bigr) + \epsilon e^{-\lambda_N \sigma} \\
	&\leq C_\delta \epsilon e^{-\lambda_M\sigma} + \epsilon e^{-\lambda_N \sigma} \\
	& \leq (C_\delta + 1)e^{-\lambda_{N_0}\sigma} \epsilon
	\,,
	\]
	womit die gleichmäßige Konvergenz in diesem Bereich folgt.

\end{bewe}

Analog zur bedingten Konvergenzabszissen kann man (bei gewöhnlichen Dirichletreihen) die absoluten Konvergenzabszisse definieren als die bedingten Abszisse von
\[
\sum_{n=1}^\infty \abs{a(n)} n^{-s}
\,.
\]
Wir bezeichnen $\sigma_a$ als die \myemph[Konvergenzsabszisse!absolute Konvergenzabsizee]{absoluten} und $\sigma_c$ als die \myemph[Konvergenzsabszisse!bedingten Konvergenzabsizee]{bedingten Konvergenzabszisse}.

\begin{satz}
	Ist $F$ eine gewöhnliche Dirichletreihe mit $\sigma_c \in \RR$, so gilt
	\[
		\sigma_c \leq \sigma_a \leq \sigma_c + 1
	\]
\end{satz}

\begin{bewe}
	Übung!
\end{bewe}

\begin{beme}
	Wie in der Theorie der Potenzreihen gibt es auch in der Theorie der Dirichletreihen eine Methode zur Berechnung der Konvergenzsabszisse.
	Ist $\sum_{n=1}^\infty a(n)$ divergent, so folgt für $F(s) = \sum_{n=1}^\infty a(n) e^{-\lambda_n s}$ die Formel
	\[
		\sigma_c = \limsup_{n\to\infty} \frac{\log{A(n)}}{\lambda_n}
		\,.
	\]
	
	Es gibt einen noch viel wichtigeren Unterschied zwischen Dirichletreihen und den uns geläufigen Potenzreihen.
	Bei den Potenzreihen kann man den Konvergenzradius nicht nur in Abhängigkeit der Koeffizienten, sondern auch durch das Verhalten der durch die Reihe dargestellte lokal analytische Funktion bestimmen, nämlich als kleinster Absolutbetrag der singulären Punkte (ohne Einschränkung ist der Entwicklungspunkt $z_0 = 0$): stellt die Reihe $\sum_{n=0}^\infty a(n)z^n$ eine Funktion dar, die sich auf eine Kreisscheibe $\abs z < r$ holomorph fortsetzen lässt, so ist sie in diesem Bereich auch absolut konvergent.
	
	Für Dirichletreihen stimmt das nicht.
	Beispielsweise hat die Funktion
	\[
		F(s) = 1 - \frac 1{2^s} + \frac 1{3^s} - \frac 1{4^s} + \ldots
	\]
	eine Forsetzung zu einer ganzen Funktion (wie wir noch sehen werden!), aber die Reihe konvergiert nur für Werte $s \in \CC$ mit $\Re s > 0$.
	
	Nur in Spezialfällen können wir auf die Existenz von singulären Punkten auf dem Rand der Konvergenzhalbebene schließen.
\end{beme}

\begin{satz}[Landau]
	Sei $\sum_{n=1}^\infty a(n)n^{-s}$ eine gewöhnliche Dirichletreihe mit nicht-negativen Koeffizienten und Konvergenzabszisse $\sigma_c$.
	Dann hat die durch $F(s) = \sum_{n=1}^\infty a(n)n^{-s}$ definierte Funktion in $s=\sigma_c$ einen singulären Punkt.
\end{satz}
\begin{bewe}
	Ohne Einschränkung sei $\sigma_c = 0$.
	Nehmen wir an, die Funktion $F(s)$ wäre in einer Umgebung $U_\epsilon(0)$ holomorph fortsetzbar. Dann würde sie um $s=1$ eine Taylorentwicklung haben mit Konvergenzradius $R > 1$, da kein Punkt in $\partial U_1(1)$ singulär ist.
	Also wäre für geeignetes $\delta > 0$ die Reihe $\sum_{k=1}^\infty \frac{(-1-\delta)^{k}}{k!} F^{(k)}(1)$ konvergent und gleich $F(-\delta)$, denn nach \autoref{satz:konvergenzDirichletreihen} ist
	\[
	\sum_{k=0}^\infty \frac{(-1-\delta)^k}{k!} F^{(k)}(1)
	&\stackrel{\phantom{a(n) \geq 0}}{=} \sum_{k=0}^\infty \frac{(1+\delta)^k}{k!} \sum_{n=1}^\infty \frac{(\log n)^k a(n)
	}{n} \\
	&\stackrel{a(n) \geq 0}{=} \sum_{n=1}^\infty \frac{a(n)}{n}  \sum_{k=0}^\infty \frac{(\log n)^k (1+\delta)^k}{k!} \\
	&\stackrel{\phantom{a(n) \geq 0}}{=} \sum_{n=1}^\infty \frac{a(n)}{n} e^{(1+\delta)\log(n)}
	= \sum_{n=1}^\infty a(n)n^\delta
	\,.
	\]
	Damit gilt $\sigma_c \leq -\delta < \sigma_c$. \blitz
\end{bewe}

\begin{satz}
	Seien $\sum_{n=1}^\infty a(n)e^{-\lambda_ns}$ und $\sum_{n=1}^\infty b(n)n^{-\lambda_ns}$ zwei Dirichletreihen, die in einem Gebiet $U \subset \CC$ konvergieren und dort die selbe analytische Funktion darstellen.
	Dann ist $a(n) = b(n)$ für alle $n\in\NN$.
\end{satz}
\begin{bewe}
	Nehmen wir an, dies sei nicht der Fall.
	Sei $m$ der kleinste Index mit $a(m) \not= b(m)$.
	Dann gilt für $\sigma$ groß genug (Identitätssatz)
	\[
	0 &= e^{\lambda_m \sigma} \biggl( \sum_{n=m}^\infty a(n) e^{-\lambda_n \sigma} - \sum_{n=m}^\infty b(n) e^{-\lambda_n \sigma} \biggl) \\
	&= a(m) - b(m) + \sum_{n=m+1}^\infty \bigl(a(n)-b(n)\bigr) e^{-(\lambda_n-\lambda_m)\sigma}
	\]
	In der Tat hat jedes Glied in der Reihe wegen $\lambda_n > \lambda_m$ den Limes 0 und die gleichmäßige Konvergenz impliziert, dass die Reihe für $\sigma\to\infty$ gegen 0 strebt, was $a(m) \not= b(m)$ widerspricht.
\end{bewe}
\section{Formale Eigenschaften von Dirichletreihen}

Ab jetzt bezeichnen wir gewöhnliche Dirichletreihen als Dirichletreihen.
Die Regeln für die Handhabung von Dirichletreihen sind anders als die bei Potenzreihen, daher wollen wir diese jetzt näher erläutern.

Es ist klar, dass die Summe zweier Dirichletreihen die Reihe ist, deren allgemeiner Koeffizient die Summe der Koeffizienten der einzelnen Reihen ist.
Aber wie bildet man das Produkt?

Seien
\[
	F(s) = \sum_{n=1}^\infty a(n)n^{-s}\,, \qquad G(s) = \sum_{n=1}^\infty b(n)n^{-s}
\]
zwei in einer offenen Menge $U \subset \CC$ durch absolut konvergente Dirichletreihen gegebene Funktionen, dann ist in $U$:
\[
	F(s) G(s)
	&= \sum_{n=1}^\infty \sum_{m=1}^\infty a(n)b(m) n^{-s}m^{-s} \\
	&= \sum_{n,m=1}^\infty a(n)b(m) (nm)^{-s} \\
	&= \sum_{k=1} \biggl(\underbrace{\sum_{d|k} a(d)b\Bigl(\frac{k}{d}\Bigr)}_{=:c(k)}\biggr) k^{-s}\,.
\]
Das heißt die additive Faltung $\sum_{n+m=k} a(n)b(m)$, die die Multiplikation von Potenzreihen beschreibt, wird durch die multiplikative Faltung $\sum_{d|k} a(d)b(\frac{k}{d})$ bei Dirichletreihen ersetzt.
Diese Tatsache ist für große Bedeutung der Dirichletreihen in der Zahlentheorie verantwortlich.
\section{Die Mellin-Transformation}

\begin{erin}
	Die Gammafunktion $\Gamma(s)$ ist definiert durch
	\[
		\Gamma(s) = \lim_{N\to\infty} \frac{(N-1)!N^s}{s(s+1) \ldots (s+N-1)}\,.
	\]
	Es gilt der folgende Satz
	\begin{satz}
		Die Funktion $\Gamma(s)$ ist in ganz $\CC\setminus-\NN_0$ holomorphe Funktion mit einfachen Polen in $s=0$, $-1$, $-2$, \ldots{}
		Es gilt die Funktionalgleichung
		\[
			s\Gamma(s) = \Gamma(s+1) \qquad \forall s \in \CC\setminus-\NN_0\,.
		\]
		Weiter gilt $\Gamma(n) = (n-1)!$ für $n \in \NN$ und
		\[
			\res_{s=-n} \Gamma(s) = \frac{(-1)^n}{n!}
			\qquad \forall n \in \NN_0\,.
		\]
	\end{satz}
	\begin{bewe}
		Siehe FT 2 oder Busam-Freitag FT 1.
	\end{bewe}

	Da $\Gamma(s) \not= 0$ für alle $s \in \CC$ ist $\frac{1}{\Gamma(s)}$ eine ganze Funktion und es gilt die Produktentwicklung
	\[
		\frac{1}{\Gamma(s)} = s e^{\gamma s} \prod_{j=1}^\infty \Bigl(1 + \frac{s}{j}\Bigr)e^{-\frac{s}{j}}
	\]
	wobei $\gamma = \lim_{N \to \infty} 1 + \frac{1}{2} + \ldots + \frac{1}{N} - \log(N)$.
	
	Eine der wichtigen Formeln für die Gammafunktion ist gegeben durch
	\begin{satz}
		Es gilt in jedem Winkelbereich $W_\delta = \Set{s \in \CC \mid -\pi + \delta < \Arg(s) < \pi - \delta}$
		\[
			\Gamma(s)
			= \sqrt{2\pi} \cdot s^{s - \frac{1}{2}} e^{-s} e^{H(s)}
		\]
		wobei $H(s)$ eine in $\CC_-$ holomorphe Funktion ist mit der Eigenschaft
		\[
			\lim_{\substack{\abs s \to \infty \\ s \in W_\delta}} H(s) = 0\,.
		\]
	\end{satz}
	Diese ist vor allem Dingen dafür geeignet den exponentiellen Abfall der Funktion $\Gamma(s)$ auf vertikalen Streifen $\sigma_1 < \Re s < \sigma_2$ beweisen.
\end{erin}

Eine für uns sehr wichtige Darstellung der Gammafunktion ist die Integraldarstellung von Euler:

\begin{satz}
Es gilt für alle $s \in \CC$ mit $\Re (s) > 0$:
\begin{equation}\label{eq:GammaEulerIntegral}
	\Gamma(s) = \int_0^\infty e^{-x} x^{s-1} \opd x
	\,.
\end{equation}
\end{satz}

\begin{bewe}
Siehe FT 2: Man verifiziert die Funktionalgleichung über partielle Integration und nutzt anschließend den Satz von Wielandt. 
\end{bewe}

Die enorme Bedeutung der Gammafunktion in der Zahlentheorie wird in ihrem Zusammenspiel mit Dirichletreihen deutlich:

\begin{satz}[Mellin-Transformation]\label{Mellin-Trafo}
Es sei $F(s) = \sum_{n=1}^\infty \frac {a(n)}{n^s}$ eine Dirichletreihe, welche irgendwo konvergiert. Dann gilt für die zugehörige Potenzreihe $P(z) = \sum_{n=1}^\infty a(n) z^n$ und für alle $s \in \CC$ mit $\Re (s) > \max \Set {0, \sigma_a(F)}$:
\[
	F(s) = \frac 1{\Gamma(s)} \int_0^\infty P(e^{-x}) \underbrace{x^{s-1}}_{\text{Mellin-Kern}} \opd x
	\,.
\]
\end{satz}

\begin{bewe}
Für beliebiges $n \in \NN$ machen wir in \eqref{eq:GammaEulerIntegral} die Substitution $x = ny$ und sehen, dass
\[
	\Gamma(s) = n^s \int_0^\infty e^{-ny} y^{s-1} \opd y
	\,.
\]
Daraus folgt für alle $s \in \CC$ mit $\Re (s) > \max \Set {0, \sigma_a(F)}$, dass
\begin{align*}
	F(s) = \sum_{n=1}^\infty \frac {a(n)}{n^s}
	&= \sum_{n=1}^\infty \frac {a(n)}{\Gamma(s)} \int_0^\infty e^{-ny} y^{s-1} \opd y \\
	&= \frac 1{\Gamma(s)} \sum_{n=1}^\infty \int_0^\infty a(n) e^{-ny} y^{s-1} \opd y \\
	&\overset {(\ast)}= \frac 1{\Gamma(s)} \int_0^\infty y^{s-1} \sum_{n=1}^\infty a(n) e^{-ny} \opd y \\
	&= \frac 1{\Gamma(s)} \int_0^\infty P(e^{-y}) y^{s-1} \opd y
	\,.
\end{align*}
Die Vertauschung von Integral und Summe bei $(\ast)$ ist nach Satz von Lebesgue gerechtfertigt wegen $\sigma := \Re (s) > \max \Set {0, \sigma_a(F)}$ nach Voraussetzung und daher
\begin{align*}
	\sum_{n=1}^\infty \int_0^\infty \abs{a(n) e^{-ny} y^{s-1} } \opd y
	&= \sum_{n=1}^\infty \abs{a(n)} \int_0^\infty e^{-ny} y^{\sigma-1} \opd y && \big| \; \sigma > 0 \\
	&= \sum_{n=1}^\infty \abs{a(n)} n^{-\sigma} \Gamma(\sigma) \\
	&= \Gamma(\sigma) \sum_{n=1}^\infty \abs{\frac {a(n)}{n^{\sigma}}} && \big| \; \sigma > \sigma_a(F) \\
	&< \infty
	\,.
\end{align*}
\end{bewe}

\begin{bsp-list}
\item Ist $F(s) = \zeta(s)$, so erhalten wir für $s \in \CC$ mit $\Re (s) > 1$:
\begin{equation}\label{eq:ZetaMellinInt}
	F(s) = \zeta(s) = \frac 1{\Gamma(s)} \int_0^\infty \frac {x^{s-1}}{e^x - 1} \opd x
	\,.
\end{equation}
\item Ist $G(s) = 1 - \frac 1{2^s} + \frac 1{3^s} - \frac 1{4^s} \pm \ldots = (1 - 2^{1-s}) \zeta(s)$, so gilt sogar für alle $s \in \CC$ mit $\Re (s) > 0$ (folgt nicht vollständig aus \autoref{Mellin-Trafo}, lässt sich aber beweisen):
\[
	G(s) = \frac 1{\Gamma(s)} \int_0^\infty \frac {x^{s-1}}{e^x + 1} \opd x
	\,.
\]
\end{bsp-list}

Zum Schluss beweisen wir noch eine nützliche Darstellung von $\Log \Gamma(s+1)$ in Form der expliziten Taylor-Entwicklung im Bereich $\abs{s} < 1$: 

\begin{satz}\label{LogGamma(s+1)}
Es gilt für alle $s \in U_1(0)$:
\[
	\Log \Gamma(s+1) = - \gamma s + \sum_{n=2}^\infty (-1)^n \frac {\zeta(n)}n s^n
	\,.
\]
\end{satz}

\begin{bewe}
Es gilt
\[
	\Gamma(s+1) = s \Gamma(s) = \lim_{N \to \infty} \frac {N^s}{\left( 1 + \frac s1 \right) \left( 1 + \frac s2 \right) \cdots \left( 1 + \frac s{N-1} \right)}
\]
und somit folgt durch Logarithmieren:
\begin{align*}
	\Log \Gamma(s+1) 
	&= \lim_{N \to \infty} \left[ s \Log N - \sum_{n=1}^{N-1} \Log \left( 1 + \frac sn \right) \right] \\
	&= \lim_{N \to \infty} \left[ s \log N - \sum_{n=1}^{N-1} \left( \frac sn - \frac {s^2}{2n^2} + \frac {s^3}{3n^3} \mp \ldots \right) \right] \\
	&= \lim_{N \to \infty} \Bigg[ s \left( \log N - \left( 1 + \frac 12 + \frac 13 + \ldots + \frac 1{N-1} \right) \right) \\
	&\qquad\qquad + \frac {s^2}2 \left( 1 + \frac 1{2^2} + \frac 1{3^2} + \ldots + \frac 1{(N-1)^2} \right) \\
	&\qquad\qquad - \frac {s^3}3 \left( 1 + \frac 1{2^3} + \frac 1{3^3} + \ldots + \frac 1{(N-1)^3} \right) \\
	&\qquad\qquad \pm \ldots \, \Bigg] \\
	&= - \gamma s + \frac {\zeta(2)}2 s^2 - \frac {\zeta(3)}3 s^3 \pm \ldots
	\,.
\end{align*}
Da die Folgen $a_r(N) = \sum_{n=1}^N \frac 1{n^r}$ für $r \geq 2$ und $N \to \infty$ gleichmäßig gegen die Grenzwerte $\zeta(r)$ konvergieren, ist die Vertauschung von Limes und Summation ($\pm \ldots$) im letzten Schritt erlaubt.
\end{bewe}

\subsection{Die Riemannsche Zetafunktion}

Die einfachste und wichtigste Dirichletreihe ist die Riemannsche Zetafunktion
\[
	\zeta(s) 
	:= \sum_{n=1}^\infty \frac 1{n^s}
	= \prod_{p \in \PP} \frac 1{1 - p^{-s}}
	= \frac 1{\Gamma(s)} \int_0^\infty \frac {x^{s-1}}{e^x - 1} \opd x
	\,,
\]
wobei alle drei Darstellungen nur für $s \in \CC$ mit $\Re(s) > 1$ gültig sind. Die wichtigsten bisher bewiesenen Eigenschaften der Zetafunktion sind im folgenden Satz zusammengefasst:

\begin{satz}\label{Zeta-Fakten}
Die auf $\Set {z \in \CC \mid \Re(s) > 1}$ durch $\zeta(s) := \sum_{n=1}^\infty \frac 1{n^s}$ definierte Funktion $\zeta$ besitzt eine meromorphe Fortsetzung in die komplexe Zahlenebene $\CC$ mit einem einfachen Pol an der Stelle $s = 1$ mit Residuum $1$. Dies ist zugleich ihre einzige Polstelle. 

Die Werte der Zetafunktion bei nichtpositiven ganzen Zahlen sind rational, genauer:
\begin{align*}
	\zeta(0) &= - \frac 12 \\
	\zeta(-2n) &= 0 && \forall \, n \in \NN \\
	\zeta(1-2n) &= - \frac {B_{2n}}{2n} && \forall \, n \in \NN
	\,,
\end{align*}
wobei die rationalen Zahlen $B_2 = \frac 16$, $B_4 = - \frac 1{30}$, \ldots die durch
\[
	\frac t{e^t - 1} = \sum_{k=0}^\infty \frac {B_k}{k!} t^k \quad \text{ für } t \in U_{2\pi}(0)
\]
definierten \myemph{Bernoulli-Zahlen} sind. 

Die Werte der Zetafunktion bei positiven geraden Zahlen sind durch
\[
	\zeta(2n) = \frac {(-1)^{n-1} 2^{2n-1} B_{2n}}{(2n)!} \pi^{2n}, \quad n \in \NN
\]
gegeben.
\end{satz}

\begin{bewe}
Wir entwickeln zunächst
\[
	\frac t{e^t - 1} = \frac t{t + \frac{t^2}{2!} + \frac {t^3}{3!} + \ldots} = 1 - \frac t2 + \frac {t^2}{12} - \frac {t^4}{720} \pm \ldots
\]
und definieren $B_n$ als das $n!$-fache des Koeffizienten von $t^n$ auf der rechten Seite. Aus
\[
	\frac t{e^t - 1} - \frac {-t}{e^{-t} - 1} = -t
\]
folgt, dass abgesehen von $B_1 = - \frac 12$ alle $B_n$ mit $n$ ungerade verschwinden. Setze nun für beliebiges $n \in \NN$
\[
	f_n(t) := \sum_{k=0}^n (-1)^k \frac {B_k}{k!} t^k = 1 + \frac t2 + \frac {B_2}{2!}t^2 + \ldots + \frac {B_n}{n!}t^n
\]
unter Beachtung von $(-1)^k B_k = B_k$ für $k > 1$ wegen $B_k = 0 = - B_k$ für ungerade $k > 1$. 

Dann gilt für alle $s \in \CC$ mit $\sigma := \Re(s) > 1$ und für $n \in \NN$ beliebig:
\begin{align*}
	\Gamma(s) \zeta(s) 
	&\overset{\eqref{eq:ZetaMellinInt}}= \int_0^\infty \frac {t^{s-1}}{e^t - 1} \opd t \\
	&= \int_0^\infty \frac {te^t}{e^t - 1} e^{-t} t^{s-2} \opd t \\
	&= \underbrace{\int_0^\infty \left( \frac {te^t}{e^t - 1} - f_n(t) \right) e^{-t} t^{s-2} \opd t}_{=: I_1(s)} + \underbrace{\int_0^\infty f_n(t) e^{-t} t^{s-2} \opd t}_{=: I_2(s)}
	\,,
\end{align*}
Die Funktion $t \mapsto \frac {te^t}{e^t - 1}$ ist lokal um $t = 0$ holomorph und hat dort die Taylorentwicklung
\[
	\frac {te^t}{e^t - 1} = \frac {-t}{e^{-t} - 1} = \sum_{k=0}^\infty \frac {B_k}{k!} (-t)^k = \sum_{k=0}^\infty (-1)^k \frac {B_k}{k!} t^k = f_\infty(t)
	\,,
\]
sodass für $t \to 0$
\[
	\frac {te^t}{e^t - 1} - f_n(t) = \mathcal O (t^{n+1})
\]
gilt. Somit ist der Integrand von $I_1(s)$ für $t \to 0$ in $\mathcal O (t^{n+\sigma-1})$ und fällt für $t \to \infty$ ohnehin exponentiell ab. Fixiere nun ein $n \in \NN$ und betrachte die Halbebene $\HH_{-n} := \Set {s \in \CC \mid \sigma := \Re(s) > -n}$. Dort stellt das Integral wegen $n+\sigma-1 > -1$ somit eine holomorphe Funktion dar.

Das zweite Integral ist zwar nur für $\sigma > 1$ konvergent, lässt sich aber elementar berechnen zu
\begin{align*}
	I_2(s) 
	&= \int_0^\infty f_n(t) e^{-t} t^{s-2} \opd t \\
	&= \int_0^\infty \left( 1 + \frac t2 + \frac {B_2}{2!}t^2 + \ldots + \frac {B_n}{n!}t^n \right) e^{-t} t^{s-2} \opd t \\
	&= \Gamma(s-1) + \frac 12 \Gamma(s) + \sum_{k=2}^n \frac {B_k}{k!} \Gamma(s+k-1)
	\,.
\end{align*}

Da dies eine auf ganz $\CC$ meromorphe Funktion ist, folgt wegen $n \in \NN$ beliebig, dass $\zeta (s)$ eine in ganz $\CC$ meromorphe Fortsetzung besitzt. Genauer erhalten wir durch Einsetzen der Integrale $I_1(s)$ und $I_2(s)$ sowie durch Ausnutzen der Funktionalgleichung von $\Gamma(s)$, dass insgesamt gilt:
\begin{align*}
	\zeta(s) 
	&= \frac 1{\Gamma(s)} \left( I_1(s) + I_2(s) \right) \\
	&= \frac {I_1(s)}{\Gamma(s)} + \frac 1{\Gamma(s)} \left( \Gamma(s-1) + \frac 12 \Gamma(s) + \sum_{k=2}^n \frac {B_k}{k!} \Gamma(s+k-1) \right) \\
	&= \frac {I_1(s)}{\Gamma(s)} + \frac 1{s-1} + \frac 12 + \sum_{k=2}^n \frac{B_k}{k!} s(s+1)(s+2)\ldots(s+k-2)
\end{align*}
Da $I_1(s)$ für $n \in \NN$ beliebig auf $\HH_{-n}$ holomorph und $\Gamma$ zudem nullstellenfrei ist, zeigt diese Formel zugleich, dass $\zeta(s) - \frac 1{s-1}$ auf ganz $\CC$ holomorph ist. Somit ist auch der zweite Teil der Behauptung bewiesen und nur die konkreten Werte der Zetafunktion verbleiben noch zu zeigen.

Sei dazu $s \in \ZZ$ mit $-n < s \leq 0$, dann ist $\frac {I_1(s)}{\Gamma(s)}$ wegen des Pols von $\Gamma(s)$ an dieser Stelle gleich Null und es folgt
\begin{align*}
	\zeta(s) 
	&= \frac 1{s-1} + \frac 12 + \sum_{k=2}^n \frac{B_k}{k!} s(s+1)(s+2)\ldots(s+k-2) \\
	&= \frac 1{s-1} + \frac 12 + \frac s{12} - \frac {s(s+1)(s+2)}{720} + \frac {s(s+1)(s+2)(s+3)(s+4)}{30240} \mp \ldots
	\,.
\end{align*}
Dies zeigt, dass 
\begin{align*}
	\zeta(0) &= \frac 1{-1} + \frac 12 = - \frac 12 \,, \\
	\zeta(-1) &= \frac 1{-2} + \frac 12 - \frac 1{12} = -\frac 1{12} \,, \\
	\zeta(-2) &= \frac 1{-3} + \frac 12 - \frac 16 = 0 \,, \\
	\zeta(-3) &= \frac 1{-4} + \frac 12 - \frac 14 + \frac 1{120} = \frac 1{120} \,.
\end{align*}

Es ist klar, dass man dieses Verfahren für beliebig große $n$ fortsetzen könnte, um $\zeta(-n)$ zu berechnen. Jedoch kann man auch eine geschlossene Form entwickeln: Aus dem erarbeiteten Ausdruck 
\[
	\zeta(s)
	= \frac 1{s-1} + \frac 12 + \sum_{r=2}^n \frac{B_r}{r!} s(s+1)(s+2)\ldots(s+r-2)
\]
erhält man für $s = -k$ und beliebiges $n > k$ (z.B. $n = k+1$) den Ausdruck
\begin{align*}
	\zeta(-k)
	&= \frac 1{-k-1} + \frac 12 + \sum_{r=2}^n \frac {B_r}{r!} (-k)(-k+1) \ldots (-k+r-2) \\
	&= - \frac 1{k+1} + \frac 12 + \sum_{r=2}^{k+1} (-1)^{r-1} \frac {B_r}{r!} \frac {k!}{(k+1-r)!} \\
	&= - \frac 1{k+1} \sum_{r=0}^{k+1} \binom {k+1}r B_r
	\,.
\end{align*}
Die Bernoulli-Zahlen erfüllen nun für beliebiges $n \in \NN$ die Beziehung
\[
	\sum_{r=0}^n \binom nr B_r = (-1)^n B_n
	\,,
\]
was per Koeffizientenvergleich und mit $k = n-r$ aus
\begin{align*}
	\sum_{n=0}^\infty \left( \sum_{r=0}^n \binom nr B_r \right) \frac {t^n}{n!}
	&= \sum_{r=0}^\infty \sum_{k=0}^\infty \frac {B_r t^{r+k}}{r!k!} \\
	&= \underbrace{\left( \sum_{r=0}^\infty \frac {B_r}{r!} t^r \right)}_{= \frac t{e^t - 1}} \underbrace{\left( \sum_{k=0}^\infty \frac {t^k}{k!} \right)}_{= e^t} \\
	&= \frac {-t}{e^{-t} - 1}
	= \sum_{k=0}^\infty \frac {B_n}{n!} (-t)^n
	= \sum_{k=0}^\infty (-1)^n B_n \frac {t^n}{n!}
\end{align*}
folgt. Damit ergibt sich wie behauptet
\[
	\zeta(-k) = - \frac 1{k+1} \sum_{r=0}^{k+1} \binom {k+1}r B_r = - \frac {B_{k+1}}{k+1}
	\,.
\]

Zuletzt verbleibt noch, die Behauptung für die Werte $\zeta(2n)$ mit $n \in \NN$ zu beweisen. Unter Benutzung des Eulerschen Ergänzungssatzes
\[
	\Gamma(1-s) \Gamma(s) = \frac \pi{\sin \pi s}
\]
erhalten wir
\begin{align*}
	\sum_{n=1}^\infty (-1)^{n-1} 2^{2n-1} \pi^{2n} \frac {B_{2n}}{(2n)!} s^{2n}
	&= - \frac 12 \sum_{n=1}^\infty \frac {B_{2n}}{(2n)!} (2\pi is)^{2n} \\
	&= - \frac 12 \left[ \sum_{n=0}^\infty \frac {B_{2n}}{(2n)!} (2\pi is)^{2n} - 1 \right] \\
	&= - \frac 12 \left[ \sum_{n=0}^\infty \frac {B_n}{n!} (2\pi is)^n - 1 - 2 B_1 \pi i s \right] \\
	&= - \frac 12 \left[ \frac {2\pi is}{e^{2\pi is} - 1} - 1 + \pi is \right] \\
	&= \frac 12 - \frac {\pi is}2 \cdot \left( \frac 2{e^{2\pi is} - 1} + 1 \right) \\
	&= \frac 12 - \frac {\pi is}2 \cdot \left( \frac {2 + e^{2\pi is} - 1}{e^{2\pi is} - 1} \right) \\
	&= \frac 12 - \frac {\pi is}2 \cdot \frac {e^{\pi is} + e^{-\pi is}}{e^{\pi is} - e^{-\pi is}} \\
	&= \frac 12 \left( 1 - \frac {\pi s}{\tan \pi s} \right) \\
	&= \frac s2 \left( \frac 1s - \frac \pi{\tan \pi s} \right) && \Big| \; \text{nachrechnen!} \\
	&= \frac s2 \frac {\opd}{\opd s} \Log \frac {\pi s}{\sin \pi s} && \Big| \; \text{Ergänzungssatz} \\
	&= \frac s2 \frac {\opd}{\opd s} \Log \big( \Gamma(1+s) \Gamma(1-s) \big) && \Big| \; \text{\autoref{LogGamma(s+1)}} \\
	&= \frac s2 \frac {\opd}{\opd s} \left[ \sum_{n=2}^\infty \left( (-1)^n \frac {\zeta(n)}n + \frac {\zeta(n)}n \right) s^n \right] \\
	&= \frac s2 \frac {\opd}{\opd s} \left[ \sum_{n=1}^\infty 2\frac {\zeta(2n)}{2n} s^{2n} \right] \\
	&= \sum_{n=1}^\infty \zeta(2n) s^{2n}
	\,.
\end{align*}
Hieraus folgt über Koeffizientenvergleich wie behauptet für $n \in \NN$ beliebig
\[
	\zeta(2n) = \frac {(-1)^{n-1} 2^{2n-1} B_{2n}}{(2n)!} \pi^{2n} 
	\,.
\]
\end{bewe}

Die Tatsache, dass die Werte von $\zeta(2n)$ und $\zeta(1-2n)$ dieselben Bernoulli-Zahlen enthalten, lässt erahnen, dass es eine Beziehung zwischen beiden Werten geben könnte. Dies in der Tat der Fall: Setzen wir 
\[
	\xi(s) := \pi^{- \frac s2} \cdot \Gamma \! \left( \frac s2 \right) \cdot \zeta(s)
	\,,
\]
so gilt für alle $s \in \CC \setminus \Set {0,1}$ die Gleichheit
\begin{equation}\label{eq:Xi-Beziehung}
	\xi(1-s) = \xi(s)
	\,.
\end{equation}
Diese Relation wurde zuerst von Euler vermutet und schließlich von Riemann bewiesen.

Für $\sigma > 1$ ist die rechte Seite der Gleichung \eqref{eq:Xi-Beziehung} von $0$ verschieden, was sich mit der Darstellung von $\zeta$ als Eulerprodukt leicht einsehen lässt. Es folgt dann aus \eqref{eq:Xi-Beziehung}, dass für $\sigma < 0$ nur die in \autoref{Zeta-Fakten} bereits ermittelten \glqq{}trivialen Nullstellen\grqq{} $s = -2n$ mit $n \in \NN$ als Nullstellen von $\zeta$ infrage kommen. Zudem kann man zeigen, dass $\zeta$ auf den Geraden $\Re(s) = 0$ und $\Re(s) = 1$ keine Nullstellen besitzt. Die einzigen \glqq{}nicht-trivialen\grqq{} Nullstellen von $\zeta$ liegen somit im sogenannten \glqq{}kritischen Streifen\grqq{} $\Set {s \in \CC \mid 0 \leq \sigma := \Re(s) \leq 1}$. Die \glqq{}ersten\grqq{} hiervon haben die Form
\begin{align*}
	\tfrac 12 &\pm 14,134725\ldots i \,, \\
	\tfrac 12 &\pm 21,022040\ldots i \,, \\
	\tfrac 12 &\pm 25,010852\ldots i \,.
\end{align*}
Es ist bekannt, dass $\zeta$ unendlich viele nicht-triviale Nullstellen besitzt. Die bis heute ungelöste \myemph{Riemann-Vermutung} besagt, dass all diese Nullstellen den Realteil $\frac 12$ haben.

\subsection{Heckesche L-Reihen}

Einer Modulform $f = \sum_{n=0}^\infty a(n) q^n \in M_k$ ordnet man die L-Reihe
\[
	L(f,s) := \sum_{n=1}^\infty a(n)n^{-s}
\]
zu. Nach Hecke impliziert das Transformationsverhalten von $f$ \glqq{}gute\grqq{} analytische Eigenschaften für $L(f,s)$, wie zum Beispiel die Existenz einer meromorphen Fortsetzung nach $\CC$ oder die Gültigkeit einer Funktionalgleichung. Der Übergang von $f$ zu $L(f,s)$ erfolgt mittels Mellin-Transformation.

Konvention: Sei im Folgenden $k \geq 4$ stets gerade. 

\begin{defi}
Sei $f = \sum_{n=0}^\infty a(n) q^n \in M_k$. Dann heißt die Reihe 
\[
	L(f,s) := \sum_{n=1}^\infty a(n)n^{-s}
\]
die \myemph{Heckesche L-Reihe} zu $f$. 
\end{defi}

\begin{satz}
Sei $f = \sum_{n=0}^\infty a(n) q^n \in M_k$. Dann gilt:
\begin{enumerate}
\item $a(n) = \mathcal O(n^{k-1})$.
\item Ist $f \in S_k$, so gilt sogar $a(n) = \mathcal O(n^{\frac k2})$.
\end{enumerate}
\end{satz}

\begin{bewe}
Wegen $M_k = \CC E_k \oplus S_k$ und
\[
	E_k \propto G_k = 2 \zeta(k) + \frac {2(2\pi i)^k}{(k-1)!} \sum_{n=1}^\infty \sigma_{k-1}(n) q^n
\]
folgt die Aussage (i) mit (ii) und
\[
	\sigma_{k-1}(n) := \sum_{d|n} d^{k-1} = n^{k-1} \sum_{d|n} \left( \frac dn \right)^{k-1} \leq n^{k-1} \underbrace{\sum_{l=1}^\infty \frac 1{l^{k-1}}}_{< \infty} = \mathcal O(n^{k-1})
	\,.
\]
Wir müssen also nur noch (ii) zeigen. Sei dazu $f \in S_k$. Nach Definition ist
\[
	a(n) = \int_{ci}^{ci + 1} f(z) e^{-2\pi inz} \opd z
\]
mit $c \in \RR_{>0}$ beliebig. Man schreibe $f(z) = y^{- \frac k2} y^{\frac k2} f(z)$. Wie schon früher gezeigt, ist $g(z) := y^{\frac k2} f(z)$ auf ganz $\HH$ beschränkt. Es folgt somit
\[
	a(n) = \int_{ci}^{ci + 1} y^{-\frac k2} g(z) e^{-2\pi inz} \opd z = \int_0^1 c^{- \frac k2} g(t+ic) e^{2\pi nc} e^{-2 int} \opd t
\]
und damit $\abs{a(n)} \leq c^{- \frac k2} e^{2\pi nc} M$, wobei $M > 0$ nicht mehr von $n$ abhängt. Man wähle $c = \frac 1n$ und folgere $\abs{a(n)} = \mathcal O(n^{\frac k2})$. Damit ist alles gezeigt.
\end{bewe}

\begin{koro}
Sei $f \in M_k$. Dann gilt $\sigma_a (L(f,s)) \leq k$. Ist zudem $f \in S_k$, so gilt sogar $\sigma_a (L(f,s)) \leq \frac k2 + 1$. Die Funktion $s \mapsto L(f,s)$ ist in der Halbebene $\Re(s) > \sigma_c (L(f,s))$ holomorph.
\end{koro}

\begin{satz}[Hecke]\label{satz:hecke}
Sei $f = \sum_{n=0}^\infty a(n) q^n \in M_k$. Setzt man für $\Re(s) > k$
\[
	L^*(f,s) := (2\pi)^{-s} \Gamma(s) L(f,s)
	\,,
\]
dann gilt: Die Funktion 
\[
	s \mapsto L^*(f,s) - \frac {a(0)}s + \frac {(-1)^{\frac k2} a(0)}{k-s}
\]
hat eine holomorphe Fortsetzung auf ganz $\CC$, ist beschränkt in jedem Vertikalstreifen $\Set {s \in \CC \mid \nu_1 \leq \Re(s) \leq \nu_2}$ und erfüllt die Funktionalgleichung
\[
	L^*(f,k-s) = (-1)^{\frac k2} L^*(f,s)
	\,.
\]
Ist $f \in S_k$, so ist $a(0) = 0$ und daher sogar $s \mapsto L^*(f,s)$ selbst bereits eine ganze Funktion.
\end{satz}

\begin{bewe}
Nach \autoref{Mellin-Trafo} (Mellin-Transformation) ist
\[
	L(f,s) 
	= \frac 1{\Gamma(s)} \int_0^\infty \sum_{n=1}^\infty a(n) (e^{-x})^n x^{s-1} \opd x
	= \frac 1{\Gamma(s)} \int_0^\infty \Bigg( \underbrace{\sum_{n=0}^\infty a(n) e^{-nx}}_{= f(ix)} - a(0) \Bigg) x^{s-1} \opd x
	\,,
\]
sodass nach Substitution $x \mapsto y := 2\pi x$ für alle $s \in \CC$ mit $\Re(s) > k$ gilt:
\begin{align*}
	L^*(f,s) 
	&= (2\pi)^{-s} \Gamma(s) L(f,s) \\
	&= \int_0^\infty \left( f(iy) - a(0) \right) y^{s-1} \opd y \\
	&= \underbrace{\int_0^1 \left( f(iy) - a(0) \right) y^{s-1} \opd y}_{=: I_1(s)} + \underbrace{\int_1^\infty \left( f(iy) - a(0) \right) y^{s-1} \opd y}_{=: I_2(s)}
	\,.
\end{align*}
Wie man schnell sieht, ist $L^*(f,s)$ für $\Re(s) > k$ holomorph. Darüber hinaus sieht man schnell ein, dass $I_2(s)$ für alle $s \in \CC$ konvergent (also eine ganze Funktion) und außerdem auf jedem Vertikalstreifen beschränkt ist. Es verbleibt somit nur noch das Studium von
\[
	I_1(s) = - a(0) \int_0^1 y^{s-1} \opd y + \int_0^1 f(iy) y^{s-1} \opd y
	\,.
\]
Nun gilt für den ersten Summanden
\[
	\int_0^1 y^{s-1} \opd y = \left[ \frac {y^s}s \right]_0^1 = \frac 1s
\]
und für den zweiten Summanden nach Substitution $y \mapsto \inv y$ mit $\opd (\inv y) = -y^{-2} \opd y$:
\begin{align*}
	\int_0^1 f(iy) y^{s-1} \opd y
	&= \int_\infty^1 f(i \inv y) y^{-s+1} \opd (\inv y) \\
	&= \int_1^\infty f \big( S \circ (iy) \big) y^{-s-1} \opd y \qquad \qquad \qquad \qquad \Big| \; f \in M_k \\
	&= \int_1^\infty (iy)^k f(iy) y^{-s-1} \opd y \\
	&= (-1)^{\frac k2} \int_1^\infty f(iy) y^{k-s-1} \opd y \\
	&= (-1)^{\frac k2} \left( \int_1^\infty \left( f(iy) - a(0) \right) y^{k-s-1} \opd y + a(0) \int_1^\infty y^{k-s-1} \opd y \right) \\
	&= (-1)^{\frac k2} \left( I_2(k-s) - \frac {a(0)}{k-s} \right)
	\,.
\end{align*}

Insgesamt erhalten wir also
\begin{align}
	\label{L*-SummeGanz}
	\notag{}
	L^*(f,s) + \frac {a(0)}s + \frac {(-1)^{\frac k2} a(0)}{k-s}
	&= I_1(s) + I_2(s) + \frac {a(0)}s + \frac {(-1)^{\frac k2} a(0)}{k-s} \\
	&= I_2(s) + (-1)^{\frac k2} I_2(k-s)
	\,.	
\end{align}
Da $I_2(s)$ ganz und auf jedem Vertikalstreifen beschränkt ist, müssen wir nur noch die Funktionalgleichung nachrechnen. Hierzu ersetzen wir in \eqref{L*-SummeGanz} das $s$ durch $k-s$ und beobachten ($k$ gerade):
\[
	L^*(f,k-s) + \frac {a(0)}{k-s} + \frac {(-1)^{\frac k2} a(0)}s
	&\overset{\eqref{L*-SummeGanz}}= I_2(k-s) + (-1)^{\frac k2} I_2(s) \\
	&= (-1)^{\frac k2} \left( (-1)^{\frac k2} I_2(k-s) + I_2(s) \right) \\
	&\overset{\eqref{L*-SummeGanz}}= (-1)^{\frac k2} \left( L^*(f,s) + \frac {a(0)}s + \frac {(-1)^{\frac k2} a(0)}{k-s} \right) \\
	&= (-1)^{\frac k2} L^*(f,s) + \frac {(-1)^{\frac k2} a(0)}s + \frac {a(0)}{k-s}
	\,,
\]
was nach Subtraktion von $\frac {a(0)}{k-s} + \frac {(-1)^{\frac k2} a(0)}s$ genau die Behauptung ergibt.
\end{bewe}


\begin{bsp-list}
	\item Sei $f = 1 - \frac{2k}{B_k} \sum_{m=1}^\infty \sigma_{k-1}(m)q^m$ die normierte Eisensteinreihe in $M_k$.
	Es ist
	\[
	\sum_{m=1}^\infty \frac{\sigma_{k-1}(m)}{m^s}
	= \zeta(s) \zeta(s-k+1)
	\,.
	\]
	Es folgt, dass $(\zeta(s)\zeta(s-k+1))^*$ die vervollsändigte Heckesche L-Reihe $L^*(E_k,s)$ ist und $(-1)^{\frac{k}{2}}$ invariant unter $s \mapsto k-s$.
	
	\item $f = \Delta \in S_{12}$.
	Dann hat $L^*(\Delta, s) = (2\pi)^{-s} \Gamma(s)L(\Delta, s)$ eine holomorphe Fortsetzung auf $\CC$ und ist invariant unter $s \mapsto 12-s$.
	
	\emph{Kuriose Anwendung}:
	Sei $f = \sum_{n \geq 1} a(n)q^n \in S_k$ und seien fast alle $a(n)$ gleich 0.
	Dann ist $f = 0$.
	\begin{bewe}
		Die Funktion $L^*(f, s) = (s\pi)^{-s} \Gamma(s) L(f, s)$ ist holomorph in $\CC$.
		Also folgt $L(f,s) = 0$ für $s = 0$, $-1$, $-2$, \ldots
		
		Nach Voraussetzung
		\[
		&\phantom{\Rla}\qquad f = \sum_{n=1}^N a(n) q^n \\
		&\Ra \qquad L(f,s) = \sum_{n=1}^N a(n) n^{-s} \\
		&\Ra \qquad L(f, -\nu) = \sum_{n=1}^N a(n) n^\nu = 0 \qquad \forall \nu = 0, 1, 2, \ldots \\
		&\Ra \qquad \begin{pmatrix}
		1 & 1 &\ldots & 1 \\
		\vdots & \vdots & \ddots & \vdots \\
		1^{N-1} & 2^{N-1} & \ldots & & N^{N-1}
		\end{pmatrix}
		\begin{pmatrix}
		a(1) \\
		\vdots \\
		a(N)
		\end{pmatrix}
		= 0 \\
		&\Ra a = 0 \\
		&\Ra f \equiv 0
		\,.
		\]
	\end{bewe}
\end{bsp-list}


%Anhang

\appendix
\chapter{Exkurs: Produktdarstellung der Diskriminantenfunktion}

\begin{satz}
Für die Diskriminantenfunktion $\Delta$ gilt die Produktentwicklung
\[
\Delta (z) := \frac{1}{1728} \bigl( E_4^3 (z) - E_6^2 (z) \bigr) = q \prod_{m=1}^\infty \bigl( 1 - q^m \bigr)^{24}
\,,
\]
wobei wie üblich $q := \exp (2\pi iz)$ ist.
\end{satz}

\begin{bewe}
Ein erster Beweis dieser Identität stammt von Jacobi; ein weiterer Beweis, der allein mit elementaren Mitteln auskommt, wird auf Übungsblatt 4 geführt werden. Im Folgenden soll ein vergleichsweise einfacher Beweis von Professor Kohnen selbst vorgestellt werden, der unter anderem auf die Hecke-Operatoren zurückgreift.

1. Schritt: Wir leiten eine zur Behauptung äquivalente Aussage her. Nehme also an, die Produktdarstellung gelte, dann können wir die logarithmische Ableitung bilden (verifiziere durch Nachrechnen unter Beachtung der Produktregel):
\begin{align*}
\frac {\Delta'}{\Delta} &= 2\pi i - 2\pi i \cdot 24 \sum_{m=1}^\infty m \frac {q^m}{1-q^m} \\
&= 2\pi i \Bigl( 1 - 24 \sum_{m=1}^\infty m \sum_{a=1}^\infty q^{ma} \Bigr) \\
&= 2\pi i \Bigl( 1 - 24 \sum_{n=1}^\infty \sigma_1(n) q^n \Bigr)
\,.
\end{align*}
Es genügt also, folgende Aussage zu zeigen:
\[
\frac {\Delta'}{\Delta} = 2\pi i E_2
\,,
\]
wobei $E_2(z) := 1 - 24 \sum_{n=1}^\infty \sigma_1(n) q^n$. Dies ist der \ldots

2.Schritt: Für $n \in \NN$ betrachte nun wie in \autoref{defiM(n)}
\[
\mathcal{M}(n) := \Set{M \in \ZZ^{2 \times 2} \mid \det (M) = n}
\,.
\]
Wohl bekannt ist aus \autoref{lemma:Mn_schoen}, dass 
\[
\mathcal{M}(n) = \smashoperator{\bigcup_{\substack{ad = n\\ d > 0\\b (\operatorname{mod} d)}}^{\boldsymbol \cdot}} \, \Gamma(1) \cdot 
\begin{pmatrix}
a & b\\
0 & d
\end{pmatrix}
\]
und damit $\# \linksmodulo{\Gamma(1)}{\mathcal{M}(n)} = \sigma_1(n)$. Definiere nun einen \glqq{}multiplikativen Hecke-Operator\grqq{} $\mathfrak{M}_n$, der eine Modulform $f$ vom Gewicht $k$ bezüglich $\Gamma(1)$ auf eine solche von Gewicht $\sigma_1 (n) \cdot k$ abbildet durch
\begin{equation}
\label{Gl 1}
\mathfrak{M}_n (f) := \smashoperator{\prod_{\gamma \in \linksmodulo{\Gamma(1)}{\mathcal M(n)}}} f |_k \gamma = \smashoperator{\prod_{\substack{ad = n\\ d > 0\\b (\operatorname{mod} d)}}} f |_k \mymat{a}{b}{0}{d}
\,.
\end{equation}
Dies ist wohldefiniert (argumentiere dazu wie bei $T(n)$ in \autoref{VorbemerkungHecke}). Wendet man dies nun auf $f = \Delta$ an, dann ist $\mathfrak{M}_n(f) = \mathfrak{M}_n(\Delta)$ eine Modulform vom Gewicht $12 \sigma_1(n)$ ohne Nullstellen in $\HH$ und mit $\ord_\infty \bigl( \mathfrak{M}_n(\Delta) \bigr) = \sigma_1(n)$.

Aus der Valenzformel folgt jetzt $\mathfrak{M}_n(\Delta) = c \cdot \Delta^{\sigma_1(n)}$ für ein $c \in \CC^\times$. Durch logarithmisches Ableiten beider Seiten von \autoref{Gl 1} erhalten wir mit $f  = \Delta$ und $\mathfrak{M}_n(\Delta) = c \cdot \Delta^{\sigma_1(n)}$, dass
\begin{equation}
\label{Gl 2}
\sigma_1(n) \frac {\Delta'}{\Delta} = \smashoperator{\sum_{\substack{ad = n\\ d > 0\\ b (\operatorname{mod} d)}}} d^{-2}n \frac{\Delta'}{\Delta} \Bigl( \frac{az+b}{d} \Bigr) = \smashoperator{\sum_{\substack{ad = n\\ d > 0\\ b (\operatorname{mod} d)}}} \frac{\Delta'}{\Delta} \Big|_2 \mymat{a}{b}{0}{d}
\,,
\end{equation}
denn die Ableitung von
\[
f |_k \mymat{a}{b}{0}{d} = n^{\frac k2} d^{-k} f\Bigl( \frac{az+b}{d} \Bigr)
\]
ist für beliebiges $f \colon \HH \to \CC$ gegeben durch
\[
\Bigl(f |_k \mymat{a}{b}{0}{d}\Bigr)' = n^{\frac k2} d^{-k} \frac ad f' \Bigl( \frac{az+b}{d} \Bigr) = n^{\frac k2} d^{-k-2} n f'\Bigl( \frac{az+b}{d} \Bigr)
\,.
\]
Setzt man $\frac{\Delta'}{\Delta} = 2\pi i \sum_{m=0}^\infty a(m) q^m$, so ergibt sich aus \autoref{Gl 2} unter formaler Anwendung der Hecke-Operatoren (siehe Beweis von \autoref{TnEndoMk}, ii)) für beliebige $m,n \in \NN$
\[
\sigma_1(n) a(m) = \sum_{d\vert(m,n)} d a\bigl( \frac{mn}{d^2} \bigr)
\,.
\]
Einsetzen von $m = 1$ liefert
\[
\sigma_1(n) a(1) = a(n)
\]
und garantiert damit, dass $\frac{\Delta'}{\Delta}$ von der Form
\[
\frac{\Delta'}{\Delta}(z) = 2\pi i \Bigl( a(0) + a(1) \sum_{n=1}^\infty \sigma_1(n) q^n \Bigr)
\]
ist. Multipliziert man nun beide Seiten mit $\Delta(z) = \sum_{m=1}^\infty \tau(m) q^m$ und beachtet dabei $\tau(1) = 1$ sowie $\tau(2) = -24$, ergibt sich durch Koeffizientenvergleich
\[
a(0) = 1 \qquad \text{und} \qquad a(1) = -24
\,,
\]
womit alles gezeigt ist.
\end{bewe}






\printindex

%\cleardoublepage
%\phantomsection
%\addcontentsline{toc}{chapter}{Liste der Sätze}
%\listoftheorems[ignoreall, onlynamed={satz,satz-list,satz-noind,satz-ind}]

\end{document}