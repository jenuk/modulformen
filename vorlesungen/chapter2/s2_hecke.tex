\section{Die Heckeoperatoren $T(n)$}

\begin{defi}
Sei $n \in \NN$. Setze 
\[
\mathcal M(n) := \Set{ \abcd \in M_2(\ZZ) \Bigm| ad - bc = n }
\,.
\]
\end{defi}

\emph{Beobachtung:} $\mathcal M(n)$ ist invariant unter Links- und Rechtsmultiplikation von $\Gamma (1)$. 

\begin{lemm}
\[
\mathcal M (n) = \dot{\bigcup_{\substack{ad = n\\ d > 0\\
b (\operatorname{mod} d)}}} \Gamma(1) \cdot 
\begin{pmatrix}
a & b\\
0 & d
\end{pmatrix}
\,,
\]
wobei die Vereinigung über alle Matrizen $\begin{pmatrix}a&b\\0&d\end{pmatrix}$ geht, derart dass $a,b,d \in \ZZ$, $ad = n$, $d > 0$, und $b$ ein volles Restsystem modulo $d$ durchläuft (also z.B. $b \in \Set{1, 2, \ldots, d}$).
\end{lemm}

\begin{bewe}
Die Inklusion $\supseteq$ ist klar, zeige also noch $\subset$. Sei dazu $M = \abcd \in \mathcal M (n)$. Da $ad-bc = n > 0$, können $a$ und $c$ nicht gleichzeitig Null sein. Deswegen existiert $t := \ggt (a,c) \in \NN$. Also sind $-\frac ct$ und $\frac at$ teilerfremd und es existieren $\alpha, \beta \in \ZZ$ mit
\[
\begin{pmatrix}
\alpha & \beta\\
-\frac ct & \frac at
\end{pmatrix}
\in \Gamma (1)
\,.
\]
Dann ist
\[
\begin{pmatrix}
\alpha & \beta\\
-\frac ct & \frac at
\end{pmatrix} \abcd
= \begin{pmatrix}
* & *\\ 0 & *
\end{pmatrix}
\,.
\]
Man kann also voraussetzen, dass $c = 0$. Wegen $\det M = n$ gilt dann $ad = n$. Multipliziert man gegebenenfalls mit $- E$, so kann man annehmen, dass $d > 0$. Schließlich multipliziere für $\nu \in \ZZ$ mit 
\[
\begin{pmatrix}
1 & \nu\\
0 & 1
\end{pmatrix} 
\in \Gamma (1) \Ra 
\begin{pmatrix}
1 & \nu\\
0 & 1
\end{pmatrix}
\begin{pmatrix}
a & b\\
0 & d
\end{pmatrix}
= 
\begin{pmatrix}
a & b + \nu d\\
0 & d
\end{pmatrix}
\,.
\]
Durch geeignete Wahl von $\nu \in \ZZ$ kann man erreichen, dass $b + \nu d$ in einem vorgegebenen Restsystem modulo $d$ liegt. Damit ist die Inklusion $\subset$ gezeigt.

Noch zu zeigen ist, dass die Vereinigung disjunkt ist (die Endlichkeit ist nach Konstruktion klar). Angenommen, für zwei Matrizen
\[
\begin{pmatrix}
a & b\\
0 & d
\end{pmatrix},
\begin{pmatrix}
a' & b'\\
0 & d'
\end{pmatrix}
\]
(mit $ad = n = a'd'$, $d > 0$, $d' > 0$ und $b, b'$ Vertreter zweier Restklassen modulo $d$ bzw.\ $d'$) existiere ein $N \in \Gamma (1)$, sodass
\[
N 
\begin{pmatrix}
a & b\\
0 & d
\end{pmatrix}
= 
\begin{pmatrix}
a' & b'\\
0 & d'
\end{pmatrix}
\,.
\]
Dann folgt, dass die untere linke Komponente von $N$ Null ist, $N \in \SL_2(\ZZ)$ also die Gestalt
\[
N = 
\begin{pmatrix}
\pm 1 & \nu\\
0 & \pm 1
\end{pmatrix}
\]
mit $\nu \in \ZZ$ hat. Damit ist 
\[
N
\begin{pmatrix}
a & b\\
0 & d
\end{pmatrix}
=
\begin{pmatrix}
\pm 1 & \nu\\
0 & \pm 1
\end{pmatrix}
\begin{pmatrix}
a & b\\
0 & d
\end{pmatrix}
= 
\begin{pmatrix}
\pm a & \pm b + \nu d\\ 
0 & \pm d
\end{pmatrix}
\overset !=
\begin{pmatrix}
a' & b'\\ 
0 & d'
\end{pmatrix}
\,.
\]
Es folgt $d' = \pm d$ und da $d, d' > 0$ nach Voraussetzung bereits $d = d'$. Die Diagonalelemente von $N$ sind also beide $+1$ und es folgt $b' = b + \nu d$. Wegen $d = d'$ stammen $b, b'$ beide aus dem gleichen Restsystem modulo $d$. Da sie sich nur um ein Vielfaches von $d$ unterscheiden, folgt
\[
\begin{pmatrix}
a' & b' \\ 0 & d'
\end{pmatrix} = \begin{pmatrix}
a & b \\ 0 & d
\end{pmatrix}
\,.
\]
\end{bewe}

\begin{defi}
Sei $n \in \NN$. Man setze dann für $f \in M_k$
\[
f | T(n) := n^{\frac k2 - 1} \sum_{M \in \linksmodulo{\Gamma(1)}{\mathcal M (n)}} f |_k M
\,.
\]
\end{defi}

\begin{satz-list}
\item Durch $T(n)$ wird eine lineare Abbildung $M_k \to M_k$ definiert. Diese lässt $S_k$ invariant (gemeint ist: Spitzenformen werden auf Spitzenformen geschickt). Man nennt $T(n)$ den $n$-ten \myemph{Hecke-Operator}.
\item Ist $f = \sum_{m \geq 0} a(m) q^m \in M_k$, so gilt
\[
f | T(n) = n^{\frac k2 - 1} \sum_{m \geq 0} \left( \sum_{d | (m,n)} d^{k-1} a\left(\frac{mn}{d^2}\right) \right) q^m
\,.
\]
\end{satz-list}

\emph{Beachte:} Der konstante Term von $f | T(n)$ ist gleich
\[
n^{\frac k2 - 1} \sum_{d|n} d^{k-1} a(0) = n^{\frac k2 - 1} \sigma_{k-1}(n) a(0)
\]

\begin{bsp}
Sei $n = p$ prim. Dann ist
\begin{align*}
f | T(p) &= p^{\frac k2 - 1} \sum_{m \geq 0} \left( \sum_{d | (m,p)} d^{k-1} a\left(\frac{mp}{d^2}\right) \right) q^m\\
&= p^{\frac k2 - 1} \sum_{m \geq 0} \Bigl( a(mp) + p^{k-1} a\Bigl( \frac mp\Bigr) \Bigr) q^m
\,,
\end{align*}
wobei $a\left(\frac mp\right) := 0$ für $p \!\! \not | \, m$, denn
\[
\sum_{d | (m,p)} d^{k-1} a \left( \frac {mn}{d^2} \right) = a(mp) + 
\begin{cases}
0 & \text{ falls } p \!\! \not | \, m\\ 
p^{k-1} a \left(\frac mp \right) & \text{ falls } p | \, m
\end{cases}
\]
\end{bsp}

\begin{bewe-list}
\item Nach den Überlegungen in \ref{VorbemerkungHecke} wissen wir, dass $f | T(n)$ das Transformationsverhalten einer Modulform vom Gewicht $k$ besitzt. Auch ist $f | T(n)$ als Summe holomorpher Funktionen selbst holomorph auf $\HH$. Zu zeigen verbleibt noch, dass $f | T(n)$ holomorph in $\infty$ ist und den Raum $S_k$ invariant lässt. Beides folgt direkt aus Teil ii) des Satzes.
\item Nächste VL ...
\end{bewe-list}

\begin{satz-list}
\item $T(m) T(n) = \sum_{d | (m,n)} d^{k-1} a \left( \frac {mn}{d^2} \right)$
\end{satz-list}

Speziell gilt (vergleiche mit Ramanujan-$\tau$-Funktion: \begin{enumerate}
\item $T(m) T(n) = T(mn)$, falls $(m,n) = 1$.
\item $T(p) T(p^n) = T(p^{n+1}) + p^{k+1} T(p^{\nu-1})$ für $p$ prim und $\nu \geq 1$.
\end{enumerate}
