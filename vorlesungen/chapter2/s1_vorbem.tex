\section{Vorbemerkung, Motivation}
\label{VorbemerkungHecke}

\begin{defi}
Definiere die Gruppe
\[
\GL_2^+(\RR) = \Set{ \abcd \in M_2(\RR) \Bigm| ad - bc > 0 }
\,,
\]
welche $\SL_2(\RR)$ als Untergruppe enthält.
\end{defi}

\begin{defi-list}
\item Seien $z \in \HH$ und 
\[
M = \abcd \in \GL_2^+(\RR)
\,,
\]
dann setze
\[
M \circ z := \frac{az+b}{cz+d}
\,.
\]
\item Für $k \in \ZZ$, $M \in \GL_2^+(\RR)$ und $f \colon \HH \to \CC$ setze
\[
(f |_k M)(z) := (ad - bc)^{\frac{k}{2}} (cz + d)^{-k} f \left( M \circ z \right)
\,.
\]
\end{defi-list}

Diese Definitionen verallgemeinern die früheren Definitionen für $\SL_2(\RR)$ (siehe \ref{DefSL2Z}). Beachte, dass weiterhin für alle $\lambda \in \RR_+$ gilt:
\[
f |_k 
\begin{pmatrix}
\lambda & 0\\
0 & \lambda
\end{pmatrix}
= f
\,.
\]

\begin{lemm-list} \label{LemmaGL2+R}
\item Die Abbildung $(M, z) \mapsto M \circ z$ definiert eine Operation von $\GL_2^+(\RR)$ auf $\HH$.
\item Man hat $f |_k M_1 M_2 = (f |_k M_1) |_k M_2$.
\end{lemm-list}

\begin{bewe-list}
\item Rechne nach und beachte hierbei, dass $\Im \Bigl( \frac{az+b}{cz+d} \Bigr) = (ad-bc) \frac{\Im z}{\abs{cz+d}^2}$.
\item Für $M = \abcd \in \GL_2^+(\RR)$ setze $j(M, z) := cz+d$. Dann gilt für beliebige Matrizen $M_1, M_2 \in \GL_2^+(\RR)$, dass
\[
j(M_1 M_2, z) = j(M_1, M_2 \circ z) \cdot j(M_2, z)
\,,
\]
woraus wegen $(cz+d)^{-k} = j(M, z)^{-k}$ die Behauptung folgt.
\end{bewe-list}

\emph{Ziel:} Definition gewisser linearer Operatoren $T \colon M_k \to M_k$ auf den Vektorräumen $M_k$ (Modulformen vom Gewicht $k \in \ZZ$) durch geeignete Mittelbildung.

\emph{Idee:} Sei $\mathcal{M} \subset \GL_2^+(\RR)$ eine Teilmenge mit folgenden Eigenschaften (mit $\cdot$ die gewöhnliche Matrizenmultiplikation): \begin{enumerate}
\item $\Gamma (1) \cdot \mathcal M \subset \mathcal M$
\item $\mathcal M \cdot \Gamma(1) \subset \mathcal M$
\item $\mathcal M$ zerfällt in endlich viele disjunkte Rechtsnebenklassen, d.h.
\[
\mathcal M = \dot{\bigcup_{M \in \linksmodulo{\Gamma(1)}{\mathcal M}}} \Gamma(1) \cdot M
\,,
\]
wobei die Vereinigung disjunkt und endlich ist. 
\end{enumerate}
Für eine Modulform $f \in M_k$ setze dann
\[
f | T_{\mathcal M} := \sum_{M \in \linksmodulo{\Gamma(1)}{\mathcal M}} f|_k M
\,.
\]
Dann ist $f | T_{\mathcal M}$ wohldefiniert, denn jede Rechtsnebenklasse $\Gamma (1) \cdot M \in \linksmodulo{\Gamma(1)}{\mathcal M}$ besteht aus Vertretern der Form $N M$ mit $N \in \Gamma (1)$ und es gilt
\[
f |_k N M = (f |_k N) |_k M = f |_k M
\,
\]
wegen Lemma \ref{LemmaGL2+R}, ii) und $f |_k N = f$ für beliebiges $N \in \Gamma (1)$, da $f \in M_k$.

\emph{Ferner:} Sei eine Matrix $N \in \Gamma(1)$ gegeben. Dann ist 
\[
(f | T_{\mathcal M}) |_k N = \sum_{M \in \linksmodulo{\Gamma(1)}{\mathcal M}} f |_k M N = \sum_{M \in \linksmodulo{\Gamma(1)}{\mathcal M}} f |_k M = f | T_{\mathcal M}
\,,
\]
denn mit $M$ durchläuft auch $MN$ ein Vertretersystem der Rechtsnebenklassen. (Begründung: Sind zwei Matrizen $M_1, M_2 \in \mathcal M$ nicht äquivalent unter Linksmultiplikation mit $\Gamma(1)$, so gilt dies trivialerweise auch für $M_1N, M_2N$. Auch ist
\[
\mathcal M N = \Bigl( \dot{\bigcup_{M \in \linksmodulo{\Gamma(1)}{\mathcal M}}} \Gamma(1) \cdot M \Bigr) N = \dot{\bigcup_{M \in \linksmodulo{\Gamma(1)}{\mathcal M}}} \Gamma(1) \cdot M N = \mathcal M
\,,
\]
denn nach Voraussetzung gilt sowohl $\mathcal M N \subset \mathcal M$ als auch $\mathcal M = \mathcal M \inv{N}N \subset \mathcal M N$.)

\emph{Folgerung:} $f | T_{\mathcal M}$ hat das Transformationsverhalten einer Modulform vom Gewicht $k$.
