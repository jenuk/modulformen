\section{Anwendungen}

\begin{beme}
Es gilt $P_0 = E_k$, wie man durch Vergleich mit \autoref{Ek_per_Gamma(1)infty} leicht einsieht.
\end{beme}

\begin{satz-list}\label{<f,Pn>}
\item Die Reihe $P_n$ konvergiert auf Kompakta in $\HH$ gleichmäßig absolut, stellt also dort eine holomorphe Funktion dar. Es gilt $P_n \in S_k$ für $n \geq 1$.
\item Es gilt 
\[
	\scalarprd {f}{P_n} = \frac{(k-2)!}{(4\pi n)^{k-1}} a_f(n)
\]
für alle $f \in S_k$ mit $f = \sum_{m\geq 1} a_f(m) q^m$.
\end{satz-list}

\begin{bewe-list}
\item Wegen $\frac{az+b}{cz+d} \in \HH$ ist
\[
	\abs{e^{2\pi i n \frac{az+b}{cz+d}}} \leq 1
\]
und daher 
\[
	\sum_{\substack{(c,d)\in\ZZ^2 \\ \ggt(c,d) = 1 \\ ad-bc = 1}} \abs{cz+d}^{-k} \cdot \abs{e^{2\pi i n \frac{az+b}{cz+d}}} \leq \sum_{\substack{(c,d)\in\ZZ^2 \\ \ggt(c,d) = 1 \\ ad-bc = 1}} \abs{cz+d}^{-k}
	\,,
\]
sodass die Reihe der Absolutbeträge nach \autoref{Ek_per_Gamma(1)infty} durch die Eisensteinreihe von Gewicht $k$ majorisiert wird. Letztere konvergiert nach FT~2 auf Kompakta in $\HH$ gleichmäßig absolut.

Zeige noch $P_n \in S_k$ für $n \geq 1$. Schreibe zunächst
\[
	P_n(z) = \frac 12 \sum_{M \in \linksmodulo{\Gamma(1)_\infty}{\Gamma(1)}} (e^n |_k M)(z)
\]
mit $e^n(z) := e^{2 \pi inz}$ und beachte, dass $e^n |_k M = e^n$ für $M \in \Gamma(1)_\infty$. Hierbei ist wie in \autoref{Ek_per_Gamma(1)infty} 
\[
	\Gamma(1)_\infty := \Set{M \in \SL_2(\ZZ) \mid M = \mymat ab0d}
	\,.
\]	
Für $P_n \in S_k$ müssen wir zeigen, dass $P_n |_k N = P_n$ für alle $N \in \SL_2(\ZZ)$ (klar, da auch $MN$ ein Vertretersystem für $\linksmodulo{\Gamma(1)_\infty}{\Gamma(1)}$ bildet) und zudem in $z = i \infty$ verschwindet. Wie im Fall der Eisensteinreihen ist hierfür zu zeigen, dass 
\[
	\lim_{z \to i\infty} P_n(z) = 0
	\text{ also }
	\lim_{\nu \to \infty} P_n(z_\nu) = 0
\]
für jede Folge von $z_\nu \in \HH$ mit $z_\nu \to i\infty$. Wegen gleichmäßiger Konvergenz gilt
\[
	\lim_{\nu \to \infty} P_n(z_\nu) = \frac 12 \sum_{\substack{(c,d)\in\ZZ^2 \\ \ggt(c,d) = 1 \\ ad-bc = 1}} \lim_{\nu \to \infty} (cz_\nu + d)^{-k} e^{2\pi in \frac{az_\nu + b}{cz_\nu + d}}
\]
und alle Grenzwerte unter der Summe sind 0. In der Tat ist der Exponentialterm wegen $\frac{az_\nu + b}{cz_\nu + d} \in \HH$ beschränkt und für $c \neq 0$ strebt $(cz_\nu + d)^{-k}$ gegen 0. Andererseits ist für $c = 0$ der vordere Term gleich $d^{-k}$ und somit beschränkt, während
\[
	\tfrac{az_\nu + b}{d} \to i\infty \quad \Ra \quad e^{2\pi in \frac{az_\nu + b}{d}} \to 0
	\,.
\]
Damit ist alles gezeigt.

\item Unter Benutzung der Darstellung
\[
	P_n(z) = \frac 12 \sum_{M \in \linksmodulo{\Gamma(1)_\infty}{\Gamma(1)}} (e^n |_k M)(z)
\]
zeigt man mit dem gleichen \glqq{}Konvolutionstrick\grqq{} wie im Beweis von \autoref{CharEk}, dass
\[
	\scalarprd {f}{P_n} = \int_0^\infty \int_{-\frac 12}^{\frac 12} f(z) e^{\conj{2\pi inz}} y^{k-2} \opd x \opd y
	\,.
\]
Man stelle sich hierzu vor, dass $\HH$ als disjunkte Vereinigung von Bildern des exakten Fundamentalbereichs unter Linksmultiplikation mit $M \in \Gamma(1)$ entsteht. Teilt man nun $\Gamma(1)_\infty$ heraus, also alle Translationen, so verbleibt noch der Streifen $\abs{x} < \frac 12$, $0 < y$. 

Es gilt weiter für beliebiges $f \in S_k$ mit Darstellung $f(z) = \sum_{m \geq 1} a(m)q^m$, wie üblich $q = \exp(2\pi iz)$ und $z = x + iy$, dass
\begin{align*}
	\scalarprd {f}{P_n} &= \int_0^\infty \int_{-\frac 12}^{\frac 12} \sum_{m \geq 1} a(m) e^{2 \pi imx}e^{-2\pi my}e^{-2\pi inx}e^{-2\pi ny}y^{k-2} \opd x \opd y\\
	&= \int_0^\infty \int_{-\frac 12}^{\frac 12} \sum_{m \geq 1} a(m) e^{2 \pi i(m-n)x}e^{-2\pi (m+n)y}y^{k-2} \opd x \opd y\,.
\end{align*}
Wegen 
\[
	\int_{- \frac 12}^{\frac 12} e^{2 \pi irx} \opd x = \delta_{r,0} := \begin{cases}1, & r = 0\\0, & r \neq 0\end{cases} \quad \text{(Kronecker-Delta)}
\]
für beliebiges $r \in \ZZ$ folgt
\begin{align*}
	\scalarprd {f}{P_n} &= \int_0^\infty \sum_{m \geq 1} a(m) \delta_{m,n} e^{-2\pi (m+n)y}y^{k-2} \opd y \\
	&= a(n) \int_0^\infty e^{-4\pi ny} y^{k-2} \opd y \\
	&= a(n) \frac{1}{(4\pi n)^{k-1}} \underbrace{\int_0^\infty e^{-y} y^{k-2} \opd y}_{= \Gamma(k-1)} \\
	&= a(n) \frac{(k-2)!}{(4\pi n)^{k-1}}
	\,.
\end{align*}
\end{bewe-list}

\begin{koro}
Die Poincaré-Reihen $\Set{ P_n \mid n \in \NN}$ zu einem festen Gewicht $k \geq 4$ mit $k$~gerade, erzeugen den Raum $S_k$.
\end{koro}

\begin{bewe}
Angenommen die $P_n$ erzeugen nicht ganz $S_k$, dann existiert ein $f \in S_k$ mit $\scalarprd {f}{P_n} = 0$ für alle $n \in \NN$. Mit \autoref{<f,Pn>}, ii) folgt hieraus aber $a(n) = 0$ für alle $n \in \NN$ und damit $f \equiv 0$.
\end{bewe}

\begin{satz}\label{satz:Pn_Fourier}
Die Reihe $P_n$ für $n \geq 1$ hat die Fourier-Entwicklung
\[
	P_n(z) = \sum_{m \geq 1} g_n(m) q^m
\]
mit
\[
	g_n(m) := \delta_{m,n} + 2\pi \cdot (-1)^{\frac k2} \cdot \left(\frac mn\right)^{\frac {k-1}2} \cdot \sum_{c \geq 1} \left[ \frac 1c \cdot K(m,n,c) \cdot J_{k-1}\left(\frac{4\pi \sqrt{mn}}c\right) \right]
	\,.
\]
Hierbei ist die Kloosterman-Summe $K$ definiert als
\[
	K(m,n,c) := \sum_{\substack{d (\operatorname{mod} c) \\ (c,d)=1}} e^{2\pi i \frac{md + n\bar{d}}{c}}
	\,,
\]
wobei $\bar d \in \ZZ$ mit $\bar d d \equiv 1 \mod c$ ist, und die Besselfunktion $J_{k-1}$ definiert als
\[
	J_{k-1}(x) := \left(\frac x2\right)^{k-1} \sum_{\ell \geq 0} \frac{(-\frac 14 x^2)^\ell}{\ell! (k-1+\ell)!}
	\,.
\]
\end{satz}

\begin{bewe}
Nach Definition ist 
\[
	P_n(z) = \frac 12 \sum_{\substack{(c,d)\in\ZZ^2 \\ \ggt(c,d) = 1 \\ ad-bc = 1}} (cz+d)^{-k} e^{2\pi in \frac{az+b}{cz+d}}
	\,.
\]
Ist $c = 0$, so folgt aus $\ggt(c,d) = 1$ bereits $d = a = \pm 1$ und unabhängig von $b \in \ZZ$ ergibt sich zweimal der Term
\[
	\frac 12 (\pm 1)^{-k} e^{2\pi in \frac{\pm z + b}{\pm 1}} = \frac 12 e^{2\pi in z} e^{\pm 2\pi in b} = \frac 12 e^{2 \pi inz}
	\,,
\]
zusammengenommen also $e^{2 \pi inz}$. Die übrigen Terme ergeben den Beitrag 
\[
	\sum_{\substack{c \geq 1, d \in \ZZ \\ \ggt(c,d) = 1 \\ ad-bc = 1}} (cz+d)^{-k} e^{2 \pi i n \frac{az+b}{cz+d}} = \sum_{c \geq 1} \sum_{\substack{d' (\operatorname{mod} c) \\ \ggt(c,d') = 1 \\ ad'-b'c = 1}} \sum_{\nu \in \ZZ} (c(z + \nu) +d')^{-k} e^{2\pi in \frac{a(z+\nu)+b'}{c(z+\nu)+d'}}
	\,.
\]
Die rechte Seite entsteht aus der linken, indem man für festes $c \geq 1$ und ein festes Vertretersystem $d' (\operatorname{mod} c)$ jedes $d \in \ZZ$ mit $\ggt(c,d) = 1$ in der Form $d = d' + c\nu$ mit $v \in \ZZ$ und $d'$ im vorgegebenen Vertretersystem schreibt. Schreibt man zudem mit geeignetem $b' \in \ZZ$ auch $b = b' + a\nu$, so wird die Bedingung $ad - bc = 1$ zu
\[
	1 = ad - bc = a(d' + c\nu) - (b' + a\nu)c = ad' - b'c
\]
und die obige Darstellung folgt durch Ausklammern von $c$ und $a$. Im Folgenden schreiben wir wieder $d$ und $b$ statt $d'$ und $b'$.

\begin{lemm}
Sei $A = \mymat abcd \in \SL_2(\RR)$ mit $c > 0$. Sei $\gamma > 0$ beliebig (insbesondere nicht unbedingt ganzzahlig). Dann gilt für alle $z \in \HH$
\[
	\sum_{\nu \in \ZZ} (c(z + \nu) +d)^{-k} e^{2\pi i\gamma \frac{a(z+\nu)+b}{c(z+\nu)+d}} 
	= 
	\tfrac{2\pi (-1)^{\frac k2}}c \sum_{m \geq 1} \left(\!\frac m\gamma\!\right)^{\!\!\frac{k-1}2} \!\! J_{k-1}\left(\!\tfrac{4\pi\sqrt{m\gamma}}c\right) e^{\frac{2\pi i}c (\gamma a + md)} e^{2\pi imz}
	\,.
\]
\end{lemm}

\begin{bewe}
Es genügt, diese Aussage nur für den Fall $A = \mymat 0{-1}10$ zu zeigen, d.h.
\[
	\sum_{\nu \in \ZZ} (z + \nu)^{-k} e^{-2\pi i\gamma \frac{1}{z+\nu}} = 2\pi (-1)^{\frac k2} \sum_{m \geq 1} \left(\!\frac m\gamma\!\right)^{\!\!\frac{k-1}2} \!\! J_{k-1}(4\pi\sqrt{m\gamma}) e^{2\pi imz}
	\,.
\]
In der Tat: Ersetzt man in dieser Gleichung $z$ durch $z + \frac dc$ und $\gamma$ durch $\frac{\gamma}{c^2}$ und multipliziert dann mit $c^{-k} e^{2\pi i \gamma \frac ac}$, so wird die linke Seite zu
\begin{align*}
	c^{-k} e^{2\pi i\gamma \frac ac} \sum_{\nu \in \ZZ} (z + \tfrac dc + \nu)^{-k} e^{-2\pi i \frac{\gamma}{c^2} \frac{1}{z + \frac dc + \nu}} 
	&= \sum_{\nu \in \ZZ} (cz + d + c\nu)^{-k} e^{2\pi i\gamma \frac ac - 2\pi i \frac{\gamma}{c} \frac{1}{cz + d + c\nu}} \\
	&= \sum_{\nu \in \ZZ} (c(z + \nu) + d)^{-k} e^{\frac{2\pi i\gamma}c \left( a - \frac{1}{c(z + \nu) + d} \right)} \\[-12pt]
	&= \sum_{\nu \in \ZZ} (c(z + \nu) + d)^{-k} e^{\frac{2\pi i\gamma}c \frac{ac(z + \nu) + \overbrace{\scriptscriptstyle ad - 1}^{= bc}}{c(z + \nu) + d}} \\
	&= \sum_{\nu \in \ZZ} (c(z + \nu) + d)^{-k} e^{2\pi i\gamma \frac{a(z + \nu) + b}{c(z + \nu) + d}}
\end{align*}
sowie die rechte Seite zu

\begin{align}\label{eq:fourierreihePn}
&c^{-k} e^{2\pi i \gamma \frac ac} 2\pi (-1)^{\frac k2} \sum_{m\geq 1} \left(\!\frac {mc^2}{\gamma}\right)^{\!\!\frac {k-1}2} \!\! J_{k-1} \left(\! 4\pi \sqrt{\frac{m\gamma}{c^2}}\right) e^{2\pi i m(z + \frac dc)} \\
&\qquad = c^{-k+2\frac{k-1}2} 2\pi (-1)^{\frac k2} \sum_{m\geq 1} \left(\!\frac {m}{\gamma}\!\right)^{\!\!\frac {k-1}2} \!\! J_{k-1} \left(\! \tfrac{4\pi \sqrt{m\gamma}}c\right) e^{2\pi i \gamma \frac ac + 2\pi i mz + 2\pi im\frac dc} \nonumber \\
&\qquad = \frac{2\pi (-1)^{\frac k2}}c \sum_{m\geq 1} \left(\!\frac {m}{\gamma}\!\right)^{\!\!\frac {k-1}2} \!\! J_{k-1} \left(\! \tfrac{4\pi \sqrt{m\gamma}}c\right) e^{\frac{2\pi i}c \left(\gamma a + md\right)} e^{2\pi i mz} \nonumber
\,.
\end{align}

Die linke Seite von \eqref{eq:fourierreihePn} konvergiert gleichmäßig absolut auf kompakten Mengen in $\HH$ und hat den Limes 0 für $q\to \infty$ (gleicher Beweis wie in \autoref{<f,Pn>}).
Sie hat daher eine Fourierentwicklung $\sum_{m\geq 1} c(m) q^m$ mit
\[
c(m) = \int_{ic}^{ic+1} \biggl( \sum_{\nu \in \ZZ} (z+\nu)^{-k} e^{-2\pi i \gamma \frac{1}{z+\nu}} \biggr) e^{-2\pi imz} \opd z
\stackrel{z \mapsto is}{=} -i^{-k+1} \int_{c-i \infty}^{c + i\infty} s^{-k} e^{-2\pi \frac{\gamma}{s}} e^{2\pi ms} \opd s
\,.
\]
Es gilt nach \myquote{Abramowitz-Stegun}, Seite 1026, Formel 29.3.80:
\[
\frac{1}{2\pi i} \int_{c-i\infty}^{c+i\infty} s^{-k} e^{-\frac{\alpha}{s}} e^{ts} \opd s = \Bigl(\frac{t}{\alpha}\Bigr)^{\frac{k-1}{2}} J_{k-1}(2\sqrt{\alpha t})
\,.
\]

Setzt man $\alpha = 2\pi \gamma$, $t = 2\pi m$, so folgt
\[
c(m) = -i^{-k+1} \cdot 2\pi i \Bigl( \frac{2\pi m}{2\pi \gamma}\Bigr)^{\frac{k-1}{2}} J_{k-1} (2\sqrt{2\pi \gamma \cdot 2 \pi m}) = (-1)^{\frac{k}{2}} \cdot 2\pi \Bigl(\frac{m}{\gamma}\Bigr)^{\frac{k-1}{2}} J_{k-1} (4\pi \sqrt{m\gamma})
\]
wie behauptet.


\end{bewe}

Nach dem Lemma folgt nun mit $\gamma = $, dass
\[
P_n(z) = e^{2\pi nz} + \sum_{c\geq1} \sum_{\substack{d \bmod c\\ \ggt(d,c) = 1}} \frac{2\pi (-1)^{\frac{k}{2}}}{c} \sum_{m\geq 1} \Bigl(\frac{m}{n}\Bigr)^{\frac{k-1}{2}} J_{k-1} \Bigl(\frac{4\pi \sqrt{mn}}{c}\Bigr) \cdot e^{\frac{2\pi i}{c}(na+md)} e^{2\pi imz}
\,.
\]
Die Behauptung folgt hieraus nach Vertauschung der Summationen über $c$ und $m$ (absolute Konvergenz) unter Beachtung von $ad = 1 + bc \equiv 1 \mod c$.

\end{bewe}

\subsection[Die Ramanujan \texorpdfstring{$\tau$}{tau}-Funktion]{Die Ramanujan {\boldmath $\tau$}-Funktion}

\begin{satz}
Sei $\Delta(z) = \sum_{n \geq 1} \tau(n) q^n \in S_{12}$ ($\tau(1) = 1$, $\tau(2) = -24, \ldots $).
Dann gilt
\begin{align*}
\tau(n) \not= 0 &\Rla P_{n,12} \not= 0 \Rla &\Rla g_n(n) \not= 0
\,,
\end{align*}
wobei $g_n(n)$ der $n$-te Fourier-Koeffizient von $P_{n,12}$ ist (siehe \autoref{satz:Pn_Fourier}).
\end{satz}
\begin{bewe}
Es gilt $P_n = c_n \cdot \Delta$ mit $c_n \in \CC \setminus \Set {0}$.
(Man kann zudem zeigen, dass $c_n \in \RR$, dies benötigen wir im Folgenden jedoch nicht.)
Aus $\scalarprd{\Delta}{P_n} \sim \tau(n)$\footnote{Hier meint $\sim$ die Proportionalität: $x \sim y \Rla x = ky$ für $k$ konstant.} (siehe \autoref{<f,Pn>} (ii)) folgt $c_n \scalarprd \Delta \Delta \sim \tau(n)$, also gilt $\tau(n) = 0$ genau dann, wenn $c_n = 0$. Aber $c_n = 0$ genau dann, wenn $P_n \equiv 0$, und dies gilt genau dann, wenn $g_n(n) \sim \scalarprd{P_n}{P_n} = 0$.
Hieraus folgt die Behauptung.
\end{bewe}

\begin{beme}
Es wird vermutet, dass $\tau(n) \not= 0$ für alle $n\in\NN$ (Lehmer).
\end{beme}

\subsection[Die Peterssonschen Formeln]{Die Peterssonschen Formeln und Abschätzungen für Fourier-Koeffizienten}

Sei $\Set{f_1, f_2, \ldots f_g}$ irgendeine orthogonale Basis von $S_k$ (nach dem Gram-Schmidt-Verfahren kann man z.\,B. jedes $f\in S_k\setminus\{0\}$ zu irgendeiner orthogonalen Basis $\Set{f, \ldots, f_g}$ ergänzen).
Dann gilt nach \autoref{<f,Pn>} (ii) für jedes $n\in\NN$
\[
P_n = \frac{(k-2)!}{(4\pi n)^{k-1}} \sum_{\nu=1}^g \frac{\conj{a_\nu(n)}}{\scalarprd{f_\nu}{f_\nu}} f_\nu
\,,
\]
wenn $f_\nu = \sum_{m\geq 1} a_\nu(m) q^m$.
Nimmt man auf beiden Seiten den $m$-ten Fourier-Koeffizienten, so erhält man
\[
g_n(m) = \frac{(k-2)!}{(4\pi n)^{k-1}} \sum_{\nu=1}^g \frac{a_\nu(m) \conj{a_\nu(n)}}{\scalarprd{f_\nu}{f_\nu}}
\,.
\]
Damit folgt
\[
g_n(n) = \frac{(k-2)!}{(4\pi n)^{k-1}} \sum_{\nu = 1}^g \frac{\abs{a_\nu(n)}^2}{\scalarprd{f_\nu}{f_\nu}}
\,.
\]
Speziell ist
\[
\abs{a_\nu(n)}^2 \leq \norm{f_\nu}^2 \frac{(4\pi n)^{k-1}}{(k-2)!} g_n(n)
\,.
\]
Für $g_n(n)$ substituiert man aus \autoref{satz:Pn_Fourier} explizite Formeln. Benutzt man $J_n(x) = \mathcal{O}(\min \{x^{-\frac{1}{2}}, x^n\})$ (einfach) und $K(n,n,c) = \mathcal{O}_\epsilon( (n,c)^{\frac{1}{2}}c^{\frac{1}{2}+\epsilon})$ (Weilsche Abschätzung, tiefliegend), so erhält man nach einigen Rechnungen
\[
g_n(n) = \mathcal{O}_\epsilon (n^{\frac{1}{2} + \epsilon})
\,,
\]
also folgt
\[
a_\nu(n) = \mathcal{O}_\epsilon (n^{\frac{k}{2}-\frac{1}{4} + \epsilon} )
\,.
\]

\begin{satz}
Sei $f \in S_k$. Dann gilt $a(n) = \mathcal{O}_\epsilon (n^{\frac{k}{2}-\frac{1}{4} + \epsilon})$, für $\epsilon > 0$.
\end{satz}

\begin{beme-list}
\item Man kann leicht zeigen, dass $a(n) \ll_f n^{\frac{k}{2}}$ (siehe \autoref{Fourierkoeff-Abschätzungen}).
\item Mit der Theorie der L-Reihen zu Modulformen kann man $a(n) \ll_{f,\epsilon} n^{\frac{k}{2} - \frac{1}{4} + \epsilon}$ für alle $\epsilon > 0$ zeigen.
\item Nach Deligne (sehr tiefliegend) gilt sogar $a(n) \ll_{f,\epsilon} n^{\frac{k}{2}-\frac{1}{2}+\epsilon}$ für alle $\epsilon > 0$ (Ramanujan-Petersson-Vermutung). 
Diese Abschätzung ist bereits bestmöglich, denn nach Rankin gilt
\[
\limsup_{n\to\infty} \frac{\abs{a_f(n)}^2}{n^{k-1}} = \infty
\,.
\]
%	\[
%		y^2 = x^3 + ax + b \mod p
%	\]

\end{beme-list}

\subsection{Hecke-Operatoren sind hermitesch}

\begin{satz}\label{Hecke(Pm)}
Sei $P_m$ für $m \in \NN$ die $m$-te Poincaré-Reihe in $S_k$.
Dann ist
\[
	P_m | T(n) = \sum_{d | (m,n)} \Bigl( \frac{n}{d} \Bigr)^{k-1} P_{\frac{mn}{d^2}}
\,.
\]
\end{satz}
\begin{bewe}
Nach Definition ist
\[
P_m = \frac{1}{2} \sum_{M\in \linksmodulo{\Gamma(1)_\infty}{\Gamma(1)}} e^m |_k M
\]
unabhängig vom Vertretersystem von $\linksmodulo{\Gamma(1)_\infty}{\Gamma(1)}$.
Es gilt
\[
2 P_m | T(n) = n^{\frac{k}{2}-1} \sum_{\substack{M \in \linksmodulo{\Gamma(1)_\infty}{\Gamma(1)}\\ N \in \linksmodulo{\Gamma(1)}{\mathcal{M}(n)}}} e^m|_k MN
= n^{\frac{k}{2}-1} \sum_{R \in \linksmodulo{\Gamma(1)_\infty}{\mathcal{M}(n)}} e^m|R
\]
Wir behaupten nun, dass die Menge 
\begin{equation*}
\begin{split}
\{\,NM \mid &N = \mymat ab0d\colon ad=n, d>0,\ b \bmod d, \\ &M = \mymat \alpha\beta\gamma\delta\colon (\gamma, \delta) \in \ZZ^2\colon \ggt(\gamma, \delta) = 1 \text{ und } (\alpha, \beta) \in \ZZ^2 \text{ fixiert s.\,d. } \alpha\delta - \beta\gamma = 1\,\}
\end{split}
\end{equation*}
ein Vertretersystem für $\linksmodulo{\Gamma(1)_\infty}{\mathcal{M}(n)}$ ist.

Wir zeigen zunächst, dass die gesamten Matrizen inäquivalent modulo $\Gamma(1)_\infty$ sind.
Angenommen
\[
\mymat 1\nu01 NM = N'M'
\]
mit $\nu \in \ZZ$ und $N$, $N'$ und $M$, $M'$ wie oben.
Daraus folgt
\[
N'^{-1} \mymat 1\nu01 N = M'M^{-1}
\,,
\]
also
\[
\mymat*{\frac{d'}{d}}{\frac{d'b-b'd+\nu d'd}{n}}{0}{\frac{d}{d'}}
= M' M^{-1}
\,.
\]
Da $M'M^{-1}$ Komponenten in $\ZZ$ hat, folgt $\frac{d}{d'}$, $\frac{d'}{d} \in \ZZ$, also $d = \pm d'$, also $d = d'$ und $a' = a$ und somit 
\[
M'M^{-1} \in \Gamma(1)_\infty
\,,
\]
d.\,h. $M' = M$, da Vertretersystem modulo $\Gamma(1)_\infty$.
Dann folgt aber $\mymat 1\nu01 N = N'$, bzw. $b' = b + \nu d$, also $b = b'$ und damit $N' = N$. Die Matrizen in der oben angegebenen Menge sind also tatsächlich inäquivalent modulo $\Gamma(1)_\infty$. 

Es verbleibt noch zu zeigen, dass sich jedes $\mymat ABCD \in \mathcal M (n)$ schreiben lässt als
\[
	\mymat* ABCD = \mymat* 1\nu 01 \mymat* ab0d \mymat* \alpha \beta \gamma \delta
	\,,
\]
d.h. 
\[
	\mymat* ABCD \mymat* \delta{-\beta}{-\gamma}\alpha = \mymat* 1\nu 01 \mymat* ab0d
\]
mit $\nu \in \ZZ$ und $ad = n, d > 0, b (\operatorname{mod} d)$ und $(\gamma, \delta) \in \ZZ^2, \ggt(\gamma, \delta) = 1, \alpha\delta - \beta\gamma = 1$. Man bestimmt zunächst $(\gamma, \delta) \in \ZZ^2$ mit $\ggt(\gamma, \delta) = 1$, sodass $C\delta - D\gamma = 0$, dann ist
\[
	\mymat* ABCD \mymat* \delta{-\beta}{-\gamma}\alpha = \mymat* {*}{*}0{*} \in \mathcal M(n)
	\,,
\]
also
\[
	\mymat* ABCD \mymat* \delta{-\beta}{-\gamma}\alpha = \mymat* a{\tilde b}0d 
\]
mit $\tilde b \in \ZZ, ad = n$. Indem man gegebenenfalls mit $-E$ multipliziert, d.h. $(\gamma, \delta)$ durch $(-\gamma, -\delta)$ ersetzt, kann man auch $d > 0$ erreichen. Wähle nun $\nu \in \ZZ$, sodass $\tilde b = b + \nu d$. Dies zeigt die Behauptung, dass die oben angegebene Menge ein Vertretersystem für $\linksmodulo{\Gamma(1)_\infty}{\mathcal{M}(n)}$ ist.

Es gilt nun
\[
	2 P_m | T(n) = n^{\frac k2 - 1} \sum_{M \in \linksmodulo {\Gamma(1)_\infty}{\Gamma(1)}} \left( \sum_{\substack{ad = n, d > 0 \\ b (\operatorname{mod} d)}} e^m |_k \mymat* ab0d \right) |_k M
	\,.
\]
Die innere Summe ist gleich 
\[
	\sum_{\substack{d|n \\ b (\operatorname{mod} d)}} n^{\frac k2} d^{-k} e^{2\pi im \left( \frac {n}{d^2} z + \frac bd \right)} = n^{\frac k2} \sum_{d | (m,n)} d^{1-k} e^{2\pi i \frac {mn}{d^2} z}
\]
wegen 
\[
	\sum_{b (\operatorname{mod} d)} e^{2\pi i \frac bd m} = \begin{cases} d, &d|m \\ 0, &\text{sonst}\end{cases}
	\,.
\]
Hieraus folgt die Behauptung.
\end{bewe}

\begin{satz}\label{T(n)herm}
Die Operatoren $T(n), n \in \NN$ eingeschränkt auf $S_k$ sind hermitesch bezüglich des Petersson-Skalarproduktes, d.h.
\[
	\scalarprd {f|T(n)}g = \scalarprd f{g|T(n)} \qquad \forall f,g \in S_k
	\,.
\]
\end{satz}

\begin{bewe}
Man zeigt dies normalerweise, indem man Modulformen zu sogenannten Kongruenzuntergruppen von $\Gamma(1)$ und deren Skalarprodukt definiert und dann gewisse Invarianzeigenschaften des Skalarproduktes (beim Übergang von einer Untergruppe zur anderen) beachtet. Wir werden hier die Behauptung unter Benutzung von \autoref{Hecke(Pm)} beweisen. Da die $P_m$ mit $m \in \NN$ den Raum $S_k$ erzeugen, genügt es zu zeigen, dass
\[
	\scalarprd {f|T(n)}{P_m} = \scalarprd f{P_m|T(n)}
	\,.
\]
Man schreibe $f = \sum_{l \geq 1} a(l) q^l$ und $f | T(n) = \sum_{l \geq 1} b(l) q^l$. Nach \autoref{<f,Pn>}, ii) ist
\begin{align*}
	\scalarprd {f|T(n)}{P_m} &= \frac {(k-2)!}{(4\pi m)^{k-1}} b(m) \\
	&= \frac {(k-2)!}{(4\pi m)^{k-1}} \sum_{d|(m,n)} d^{k-1} a\left( \frac {mn}{d^2} \right)
	\,.
\end{align*}
Andererseits ist nach \autoref{Hecke(Pm)}:
\begin{align*}
	\scalarprd f{P_m|T(n)} &= \sum_{d|(m,n)} \left( \frac nd \right)^{k-1} \scalarprd f{P_{\frac {mn}{d^2}}} \\
	&= \sum_{d|(m,n)} \left( \frac nd \right)^{k-1} \frac {(k-2)!}{(4\pi \frac {mn}{d^2})^{k-1}} a\left( \frac {mn}{d^2} \right) \\
	&= \frac {(k-2)!}{(4\pi m)^{k-1}} \sum_{d|(m,n)} d^{k-1} a\left( \frac {mn}{d^2} \right)
	\,.
\end{align*}
\end{bewe}

\begin{koro}
Die Eigenwerte von $T(n)$ sind reell.
\end{koro}
\begin{bewe}
Ist nach \autoref{T(n)herm} und LA 1 klar.
\end{bewe}

\begin{koro}\label{normEigfkt:id/orth}
Seien $f, g$ normalisierte Eigenformen in $S_k$. Dann ist entweder $f = g$ oder $\scalarprd fg = 0$.
\end{koro}

\begin{bewe}
Seien $f = \sum_{n \geq 1} a(n) q^n$ und $g = \sum_{n \geq 1} b(n) q^n$. Wegen $a(1) = b(1) = 1$ ist dann $f | T(n) = a(n) f$ und $g | T(n) = b(n) g$. Daher gilt mit \autoref{T(n)herm}
\[
	a(n) \scalarprd fg = \scalarprd {f|T(n)}g = \scalarprd f{g|T(n)} = \conj{b(n)} \scalarprd fg = b(n) \scalarprd fg
	\,.
\]
Aus $\scalarprd fg \neq 0$ folgt damit $a(n) = b(n)$ für alle $n \in \NN$, also $f = g$.
\end{bewe}

\begin{lemm}
Sei $V$ ein endlich-dimensionaler komplexer Hilbertraum mit Skalarprodukt $\scalarprd \cdot \cdot$ und sei $\Set {T_\mu}_{\mu \in I}$ eine Familie von hermiteschen, miteinander kommutierenden Endomorphismen von $V$. Dann besitzt $V$ eine orthogonale Basis bestehend aus gemeinsamen Eigenvektoren aller Operatoren $T_\mu$ mit $\mu \in I$. 
\end{lemm}

\begin{bewe}
Sei $W$ die Menge der $\CC$-linearen Endomorphismen von $V$, aufgefasst als reeller Vektorraum. Wegen $\dim_\CC V < \infty$ ist auch $\dim_\RR W < \infty$. Die $T_\mu$ erzeugen daher einen endlich-dimensionalen Unterraum von $W$, sodass es genügt, die Aussage für endlich viele Operatoren $T_1, \ldots, T_m$ zu zeigen.

Wir zeigen zunächst durch Induktion nach $m$, dass $V$ einen gemeinsamen nichttrivialen Eigenvektor von $T_1, \ldots, T_m$ enthält. Für $m = 1$ ist dies klar, da $V$ wegen $T_1$ hermitesch einen nichttrivialen Eigenvektor von $T_1$ enthält. Sei nun $m \geq 2$ und $\lambda$ ein Eigenwert von $T_1$ mit zugehörigem Eigenraum $V_\lambda := \Set {v \in V \mid T_1 v = \lambda v }$. Für alle $\mu \in \Set {2, \ldots, m}$ besteht nach Voraussetzung die Kommutativität $T_\mu T_1 = T_1 T_\mu$ und daher gilt $T_\mu V_\lambda \subset V_\lambda$. Nach Induktionsvoraussetzung besitzt nun $V_\lambda$ einen nichttrivialen gemeinsamen Eigenvektor von $T_2, \ldots, T_m$. Dieser ist nach Definition von $V_\lambda$ auch Eigenvektor von $T_1$.

Wir zeigen abschließend die Aussage des Lemmas durch Induktion nach $\dim_\CC V$. Für $\dim_\CC V = 1$ ist die Aussage klar. Sei also $m = \dim_\CC V \geq 2$. Man schreibe $V = \CC v \oplus (\CC v)^\bot$, wobei $v$ ein Eigenvektor aller $T_\mu$ mit $\mu \in \Set {1, \ldots, m}$ ist. Da die $T_\mu$ hermitesch sind und $\CC v$ invariant lassen, lassen sie auch $(\CC v)^\bot$ invariant. Nach Induktionsvoraussetzung besitzt $(\CC v)^\bot$ bereits eine orthogonale Basis von Eigenvektoren für alle $T_\mu$. Hieraus folgt die Behauptung.
\end{bewe}

\begin{koro}
Der Raum $S_k$ besitzt eine orthogonale Basis von gemeinsamen Eigenfunktionen für alle $T(n)$ mit $n \in \NN$.
\end{koro}
\begin{bewe}
Folgt direkt aus dem obigen Lemma mit $V = S_k$ und $\Set {T_\mu}_{\mu \in I} = \Set {T(n)}_{n \in \NN}$.
\end{bewe}

\begin{beme}
Nach \autoref{normEigfkt:id/orth} ist diese orthogonale Basis bis auf Permutation und Multiplikation mit Skalaren in $\CC^\times$ eindeutig bestimmt.
\end{beme}