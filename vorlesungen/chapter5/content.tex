\chapter{Die Eichler-Selberg-Spurformel auf \texorpdfstring{$\SL_2(\ZZ)$}{SL_2(Z)}}

Sei von nun an stets $k \geq 4$ gerade und wie üblich $T(m)$ mit $m \geq 1$ der $m$-te Hecke-Operator auf $M_k (\Gamma(1))$. Wir wissen bereits, dass wir $T(m)$ zu einem Endomorphismus auf $S_k$ einschränken können.

\emph{Ziel:} Bestimmung einer analytischen (einfach) und arithmetischen (schwer) Formel für die Spur $\operatorname{Tr} T(m)$ für alle $m \in \NN$.

Sei $\HH$ wie üblich die obere Halbebene und $h$ eine Funktion $h: \HH \times \HH \to \CC, (z, z') \mapsto h(z, z')$, welche in beiden Variablen eine Spitzenform von Gewicht $k$ darstellt, d.h. 
\[
	h(\cdot, z') \in S_k \quad \forall z' \in \HH \qquad \text{und} \qquad h(z, \cdot) \in S_k \quad \forall z \in \HH
	\,.
\]
Für $f \in S_k$ definieren wir dann $f \ast h$ als die Funktion
\[
	f \ast h \colon \HH \to \CC, z' \mapsto (f \ast h)(z') := \int_{\mathcal F} f(z) \conj{ h(z, \conj{-z'})} y^{k-2} \opd x \opd y \qquad (z = x + iy)
	\,.
\]
Dies ist im Wesentlichen das Petersson-Skalarprodukt $\scalarprd f{h(\cdot, \conj{-z'})}$. Wir wollen zunächst zeigen, dass $T(m) \colon S_k \to S_k$ als ein Integral dieses Typs geschrieben werden kann mit einem bestimmten Kern $h = h_m$ (bis auf eine Konstante). Aus diesen Überlegungen folgt dann auch sogleich eine analytische Formel für $\operatorname{Tr} T(m)$.
