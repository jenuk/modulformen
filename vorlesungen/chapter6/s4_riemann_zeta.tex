\section{Die Riemannsche Zetafunktion}

Die einfachste und wichtigste Dirichletreihe ist die Riemannsche Zetafunktion
\[
	\zeta(s) 
	:= \sum_{n=1}^\infty \frac 1{n^s}
	= \prod_{p \in \PP} \frac 1{1 - p^{-s}}
	= \frac 1{\Gamma(s)} \int_0^\infty \frac {x^{s-1}}{e^x - 1} \opd x
	\,,
\]
wobei alle drei Darstellungen nur für $s \in \CC$ mit $\Re(s) > 1$ gültig sind. Die wichtigsten bisher bewiesenen Eigenschaften der Zetafunktion sind im folgenden Satz zusammengefasst:

\begin{satz}\label{Zeta-Fakten}
Die auf $\Set {z \in \CC \mid \Re(s) > 1}$ durch $\zeta(s) := \sum_{n=1}^\infty \frac 1{n^s}$ definierte Funktion $\zeta$ besitzt eine meromorphe Fortsetzung in die komplexe Zahlenebene $\CC$ mit einem einfachen Pol an der Stelle $s = 1$ mit Residuum $1$. Dies ist zugleich ihre einzige Polstelle. 

Die Werte der Zetafunktion bei nichtpositiven ganzen Zahlen sind rational, genauer:
\begin{align*}
	\zeta(0) &= - \frac 12 \\
	\zeta(-2n) &= 0 && \forall \, n \in \NN \\
	\zeta(1-2n) &= - \frac {B_{2n}}{2n} && \forall \, n \in \NN
	\,,
\end{align*}
wobei die rationalen Zahlen $B_2 = \frac 16$, $B_4 = - \frac 1{30}$, \ldots die durch
\[
	\frac t{e^t - 1} = \sum_{k=0}^\infty \frac {B_k}{k!} t^k \quad \text{ für } t \in U_{2\pi}(0)
\]
definierten \myemph{Bernoulli-Zahlen} sind. 

Die Werte der Zetafunktion bei positiven geraden Zahlen sind durch
\[
	\zeta(2n) = \frac {(-1)^{n-1} 2^{2n-1} B_{2n}}{(2n)!} \pi^{2n}, \quad n \in \NN
\]
gegeben.
\end{satz}

\begin{bewe}
Wir entwickeln zunächst
\[
	\frac t{e^t - 1} = \frac t{t + \frac{t^2}{2!} + \frac {t^3}{3!} + \ldots} = 1 - \frac t2 + \frac {t^2}{12} - \frac {t^4}{720} \pm \ldots
\]
und definieren $B_n$ als das $n!$-fache des Koeffizienten von $t^n$ auf der rechten Seite. Aus
\[
	\frac t{e^t - 1} - \frac {-t}{e^{-t} - 1} = -t
\]
folgt, dass abgesehen von $B_1 = - \frac 12$ alle $B_n$ mit $n$ ungerade verschwinden. Setze nun für beliebiges $n \in \NN$
\[
	f_n(t) := \sum_{k=0}^n (-1)^k \frac {B_k}{k!} t^k = 1 + \frac t2 + \frac {B_2}{2!}t^2 + \ldots + \frac {B_n}{n!}t^n
\]
unter Beachtung von $(-1)^k B_k = B_k$ für $k > 1$ wegen $B_k = 0 = - B_k$ für ungerade $k > 1$. 

Dann gilt für alle $s \in \CC$ mit $\sigma := \Re(s) > 1$ und für $n \in \NN$ beliebig:
\begin{align*}
	\Gamma(s) \zeta(s) 
	&\overset{\eqref{eq:ZetaMellinInt}}= \int_0^\infty \frac {t^{s-1}}{e^t - 1} \opd t \\
	&= \int_0^\infty \frac {te^t}{e^t - 1} e^{-t} t^{s-2} \opd t \\
	&= \underbrace{\int_0^\infty \left( \frac {te^t}{e^t - 1} - f_n(t) \right) e^{-t} t^{s-2} \opd t}_{=: I_1(s)} + \underbrace{\int_0^\infty f_n(t) e^{-t} t^{s-2} \opd t}_{=: I_2(s)}
	\,,
\end{align*}
Die Funktion $t \mapsto \frac {te^t}{e^t - 1}$ ist lokal um $t = 0$ holomorph und hat dort die Taylorentwicklung
\[
	\frac {te^t}{e^t - 1} = \frac {-t}{e^{-t} - 1} = \sum_{k=0}^\infty \frac {B_k}{k!} (-t)^k = \sum_{k=0}^\infty (-1)^k \frac {B_k}{k!} t^k = f_\infty(t)
	\,,
\]
sodass für $t \to 0$
\[
	\frac {te^t}{e^t - 1} - f_n(t) = \mathcal O (t^{n+1})
\]
gilt. Somit ist der Integrand von $I_1(s)$ für $t \to 0$ in $\mathcal O (t^{n+\sigma-1})$ und fällt für $t \to \infty$ ohnehin exponentiell ab. Fixiere nun ein $n \in \NN$ und betrachte die Halbebene $\HH_{-n} := \Set {s \in \CC \mid \sigma := \Re(s) > -n}$. Dort stellt das Integral wegen $n+\sigma-1 > -1$ somit eine holomorphe Funktion dar.

Das zweite Integral ist zwar nur für $\sigma > 1$ konvergent, lässt sich aber elementar berechnen zu
\begin{align*}
	I_2(s) 
	&= \int_0^\infty f_n(t) e^{-t} t^{s-2} \opd t \\
	&= \int_0^\infty \left( 1 + \frac t2 + \frac {B_2}{2!}t^2 + \ldots + \frac {B_n}{n!}t^n \right) e^{-t} t^{s-2} \opd t \\
	&= \Gamma(s-1) + \frac 12 \Gamma(s) + \sum_{k=2}^n \frac {B_k}{k!} \Gamma(s+k-1)
	\,.
\end{align*}

Da dies eine auf ganz $\CC$ meromorphe Funktion ist, folgt wegen $n \in \NN$ beliebig, dass $\zeta (s)$ eine in ganz $\CC$ meromorphe Fortsetzung besitzt. Genauer erhalten wir durch Einsetzen der Integrale $I_1(s)$ und $I_2(s)$ sowie durch Ausnutzen der Funktionalgleichung von $\Gamma(s)$, dass insgesamt gilt:
\begin{align*}
	\zeta(s) 
	&= \frac 1{\Gamma(s)} \left( I_1(s) + I_2(s) \right) \\
	&= \frac {I_1(s)}{\Gamma(s)} + \frac 1{\Gamma(s)} \left( \Gamma(s-1) + \frac 12 \Gamma(s) + \sum_{k=2}^n \frac {B_k}{k!} \Gamma(s+k-1) \right) \\
	&= \frac {I_1(s)}{\Gamma(s)} + \frac 1{s-1} + \frac 12 + \sum_{k=2}^n \frac{B_k}{k!} s(s+1)(s+2)\ldots(s+k-2)
\end{align*}
Da $I_1(s)$ für $n \in \NN$ beliebig auf $\HH_{-n}$ holomorph und $\Gamma$ zudem nullstellenfrei ist, zeigt diese Formel zugleich, dass $\zeta(s) - \frac 1{s-1}$ auf ganz $\CC$ holomorph ist. Somit ist auch der zweite Teil der Behauptung bewiesen und nur die konkreten Werte der Zetafunktion verbleiben noch zu zeigen.

Sei dazu $s \in \ZZ$ mit $-n < s \leq 0$, dann ist $\frac {I_1(s)}{\Gamma(s)}$ wegen des Pols von $\Gamma(s)$ an dieser Stelle gleich Null und es folgt
\begin{align*}
	\zeta(s) 
	&= \frac 1{s-1} + \frac 12 + \sum_{k=2}^n \frac{B_k}{k!} s(s+1)(s+2)\ldots(s+k-2) \\
	&= \frac 1{s-1} + \frac 12 + \frac s{12} - \frac {s(s+1)(s+2)}{720} + \frac {s(s+1)(s+2)(s+3)(s+4)}{30240} \mp \ldots
	\,.
\end{align*}
Dies zeigt, dass 
\begin{align*}
	\zeta(0) &= \frac 1{-1} + \frac 12 = - \frac 12 \,, \\
	\zeta(-1) &= \frac 1{-2} + \frac 12 - \frac 1{12} = -\frac 1{12} \,, \\
	\zeta(-2) &= \frac 1{-3} + \frac 12 - \frac 16 = 0 \,, \\
	\zeta(-3) &= \frac 1{-4} + \frac 12 - \frac 14 + \frac 1{120} = \frac 1{120} \,.
\end{align*}

Es ist klar, dass man dieses Verfahren für beliebig große $n$ fortsetzen könnte, um $\zeta(-n)$ zu berechnen. Jedoch kann man auch eine geschlossene Form entwickeln: Aus dem erarbeiteten Ausdruck 
\[
	\zeta(s)
	= \frac 1{s-1} + \frac 12 + \sum_{r=2}^n \frac{B_r}{r!} s(s+1)(s+2)\ldots(s+r-2)
\]
erhält man für $s = -k$ und beliebiges $n > k$ (z.B. $n = k+1$) den Ausdruck
\begin{align*}
	\zeta(-k)
	&= \frac 1{-k-1} + \frac 12 + \sum_{r=2}^n \frac {B_r}{r!} (-k)(-k+1) \ldots (-k+r-2) \\
	&= - \frac 1{k+1} + \frac 12 + \sum_{r=2}^{k+1} (-1)^{r-1} \frac {B_r}{r!} \frac {k!}{(k+1-r)!} \\
	&= - \frac 1{k+1} \sum_{r=0}^{k+1} \binom {k+1}r B_r
	\,.
\end{align*}
Die Bernoulli-Zahlen erfüllen nun für beliebiges $n \in \NN$ die Beziehung
\[
	\sum_{r=0}^n \binom nr B_r = (-1)^n B_n
	\,,
\]
was per Koeffizientenvergleich und mit $k = n-r$ aus
\begin{align*}
	\sum_{n=0}^\infty \left( \sum_{r=0}^n \binom nr B_r \right) \frac {t^n}{n!}
	&= \sum_{r=0}^\infty \sum_{k=0}^\infty \frac {B_r t^{r+k}}{r!k!} \\
	&= \underbrace{\left( \sum_{r=0}^\infty \frac {B_r}{r!} t^r \right)}_{= \frac t{e^t - 1}} \underbrace{\left( \sum_{k=0}^\infty \frac {t^k}{k!} \right)}_{= e^t} \\
	&= \frac {-t}{e^{-t} - 1}
	= \sum_{k=0}^\infty \frac {B_n}{n!} (-t)^n
	= \sum_{k=0}^\infty (-1)^n B_n \frac {t^n}{n!}
\end{align*}
folgt. Damit ergibt sich wie behauptet
\[
	\zeta(-k) = - \frac 1{k+1} \sum_{r=0}^{k+1} \binom {k+1}r B_r = - \frac {B_{k+1}}{k+1}
	\,.
\]

Zuletzt verbleibt noch, die Behauptung für die Werte $\zeta(2n)$ mit $n \in \NN$ zu beweisen. Unter Benutzung des Eulerschen Ergänzungssatzes
\[
	\Gamma(1-s) \Gamma(s) = \frac \pi{\sin \pi s}
\]
erhalten wir
\begin{align*}
	\sum_{n=1}^\infty (-1)^{n-1} 2^{2n-1} \pi^{2n} \frac {B_{2n}}{(2n)!} s^{2n}
	&= - \frac 12 \sum_{n=1}^\infty \frac {B_{2n}}{(2n)!} (2\pi is)^{2n} \\
	&= - \frac 12 \left[ \sum_{n=0}^\infty \frac {B_{2n}}{(2n)!} (2\pi is)^{2n} - 1 \right] \\
	&= - \frac 12 \left[ \sum_{n=0}^\infty \frac {B_n}{n!} (2\pi is)^n - 1 - 2 B_1 \pi i s \right] \\
	&= - \frac 12 \left[ \frac {2\pi is}{e^{2\pi is} - 1} - 1 + \pi is \right] \\
	&= \frac 12 - \frac {\pi is}2 \cdot \left( \frac 2{e^{2\pi is} - 1} + 1 \right) \\
	&= \frac 12 - \frac {\pi is}2 \cdot \left( \frac {2 + e^{2\pi is} - 1}{e^{2\pi is} - 1} \right) \\
	&= \frac 12 - \frac {\pi is}2 \cdot \frac {e^{\pi is} + e^{-\pi is}}{e^{\pi is} - e^{-\pi is}} \\
	&= \frac 12 \left( 1 - \frac {\pi s}{\tan \pi s} \right) \\
	&= \frac s2 \left( \frac 1s - \frac \pi{\tan \pi s} \right) && \Big| \; \text{nachrechnen!} \\
	&= \frac s2 \frac {\opd}{\opd s} \Log \frac {\pi s}{\sin \pi s} && \Big| \; \text{Ergänzungssatz} \\
	&= \frac s2 \frac {\opd}{\opd s} \Log \big( \Gamma(1+s) \Gamma(1-s) \big) && \Big| \; \text{\autoref{LogGamma(s+1)}} \\
	&= \frac s2 \frac {\opd}{\opd s} \left[ \sum_{n=2}^\infty \left( (-1)^n \frac {\zeta(n)}n + \frac {\zeta(n)}n \right) s^n \right] \\
	&= \frac s2 \frac {\opd}{\opd s} \left[ \sum_{n=1}^\infty 2\frac {\zeta(2n)}{2n} s^{2n} \right] \\
	&= \sum_{n=1}^\infty \zeta(2n) s^{2n}
	\,.
\end{align*}
Hieraus folgt über Koeffizientenvergleich wie behauptet für $n \in \NN$ beliebig
\[
	\zeta(2n) = \frac {(-1)^{n-1} 2^{2n-1} B_{2n}}{(2n)!} \pi^{2n} 
	\,.
\]
\end{bewe}

Die Tatsache, dass die Werte von $\zeta(2n)$ und $\zeta(1-2n)$ dieselben Bernoulli-Zahlen enthalten, lässt erahnen, dass es eine Beziehung zwischen beiden Werten geben könnte. Dies in der Tat der Fall: Setzen wir 
\[
	\xi(s) := \pi^{- \frac s2} \cdot \Gamma \! \left( \frac s2 \right) \cdot \zeta(s)
	\,,
\]
so gilt für alle $s \in \CC \setminus \Set {0,1}$ die Gleichheit
\begin{equation}\label{eq:Xi-Beziehung}
	\xi(1-s) = \xi(s)
	\,.
\end{equation}
Diese Relation wurde zuerst von Euler vermutet und schließlich von Riemann bewiesen.

Für $\sigma > 1$ ist die rechte Seite der Gleichung \eqref{eq:Xi-Beziehung} von $0$ verschieden, was sich mit der Darstellung von $\zeta$ als Eulerprodukt leicht einsehen lässt. Es folgt dann aus \eqref{eq:Xi-Beziehung}, dass für $\sigma < 0$ nur die in \autoref{Zeta-Fakten} bereits ermittelten \glqq{}trivialen Nullstellen\grqq{} $s = -2n$ mit $n \in \NN$ als Nullstellen von $\zeta$ infrage kommen. Zudem kann man zeigen, dass $\zeta$ auf den Geraden $\Re(s) = 0$ und $\Re(s) = 1$ keine Nullstellen besitzt. Die einzigen \glqq{}nicht-trivialen\grqq{} Nullstellen von $\zeta$ liegen somit im sogenannten \glqq{}kritischen Streifen\grqq{} $\Set {s \in \CC \mid 0 \leq \sigma := \Re(s) \leq 1}$. Die \glqq{}ersten\grqq{} hiervon haben die Form
\begin{align*}
	\tfrac 12 &\pm 14,134725\ldots i \,, \\
	\tfrac 12 &\pm 21,022040\ldots i \,, \\
	\tfrac 12 &\pm 25,010852\ldots i \,.
\end{align*}
Es ist bekannt, dass $\zeta$ unendlich viele nicht-triviale Nullstellen besitzt. Die bis heute ungelöste \myemph{Riemann-Vermutung} besagt, dass all diese Nullstellen den Realteil $\frac 12$ haben.