\section{Formale Eigenschaften von Dirichletreihen}

Ab jetzt bezeichnen wir gewöhnliche Dirichletreihen als Dirichletreihen.
Die Regeln für die Handhabung von Dirichletreihen sind anders als die bei Potenzreihen, daher wollen wir diese jetzt näher erläutern.

Es ist klar, dass die Summe zweier Dirichletreihen die Reihe ist, deren allgemeiner Koeffizient die Summe der Koeffizienten der einzelnen Reihen ist.
Aber wie bildet man das Produkt?

Seien
\[
	F(s) = \sum_{n=1}^\infty a(n)n^{-s}\,, \qquad G(s) = \sum_{n=1}^\infty b(n)n^{-s}
\]
zwei in einer offenen Menge $U \subset \CC$ durch absolut konvergente Dirichletreihen gegebene Funktionen, dann ist in $U$:
\[
	F(s) G(s)
	&= \sum_{n=1}^\infty \sum_{m=1}^\infty a(n)b(m) n^{-s}m^{-s} \\
	&= \sum_{n,m=1}^\infty a(n)b(m) (nm)^{-s} \\
	&= \sum_{k=1}^\infty \biggl(\underbrace{\sum_{d|k} a(d)b\Bigl(\frac{k}{d}\Bigr)}_{=:c(k)}\biggr) k^{-s}\,.
\]
Das heißt die additive Faltung $\sum_{n+m=k} a(n)b(m)$, die die Multiplikation von Potenzreihen beschreibt, wird durch die multiplikative Faltung $\sum_{d|k} a(d)b(\frac{k}{d})$ bei Dirichletreihen ersetzt.
Diese Tatsache ist für große Bedeutung der Dirichletreihen in der Zahlentheorie verantwortlich.

\begin{bsp-list}
	\item Es sei $d(n)$ die Anzahl der Teiler von $n$.
	Dann gilt für alle $s \in \CC$ mit $\Re s > 1$
	\[
	\sum_{n=1}^\infty \frac{d(n)}{n^s}
	&= \sum_{n=1}^\infty \Bigl(\sum_{d|n} 1\cdot 1\Bigr) n^{-s} \\
	&= \Bigl(\sum_{n=1}^\infty \frac{1}{n^s}\Bigr) \cdot \Bigl(\sum_{n=1}^\infty \frac{1}{n^s}\Bigr) \\
	&= \zeta(s)^2
	\,.
	\]
	
	\item Allgemein gilt für $\sigma_k(n) = \sum_{d|n} d^k$ für $k \in \ZZ$, deshalb ist
	\[
	\sum_{n=1}^\infty \frac{\sigma_k(n)}{n} = \zeta(s) \cdot \zeta(s-k)\,.
	\]
\end{bsp-list}

In beiden Beispielen haben die Koeffizienten die Eigenschaft multiplikativ zu sein.
Eine \myemph{multiplikative Funktion} $f\colon \NN\ra \CC$ ist durch die Eigenschaft $f\not\equiv 0$ und $f(n \cdot m) = f(n) \cdot f(m)$ für alle $n$, $m$ mit $\ggt(n,m) = 1$ charakterisiert.
Gilt dies sogar für alle $n$, $m \in \NN$, so nennen wir $f$ \myemph[multiplikative Funktion!streng]{streng multiplikativ} oder \myemph[multiplikative Funktion!vollständig multiplikativ]{vollständig multiplikativ}.

Diese Eigenschaft wirkt sich wie folgt auf die entsprechenden Dirichletreihen aus: ist $f$ multiplikativ, so ist $f(1) = 1$ und ist $n = p_1^{r_1} \cdot \ldots \cdot p_k^{r_k}$, so gilt
\[
f(n) = f(p_1^{r_1}) \cdot \ldots \cdot f(p_k^{r_k})\,.
\]
Es ist also im Bereich der absoluten Konvergenz von $F(s) = \sum_{n=1}^\infty \frac{f(n)}{n^s}$:
\[
F(s)
&= \sum_{0 \leq r_1, r_2, \ldots} f(2^{r_1} \cdot 3^{r_2} \cdot 5^{r_3} \ldots) \cdot (2^{r_1} \cdot 3^{r_2} \cdot 5^{r_3} \ldots)^{-s} \\
&= \sum_{r_1=0}^\infty \sum_{r_2=0}^\infty \ldots \sum_{r_j=0}^\infty \ldots \frac{f(2^{r_1}) \cdot f(3^{r_2}) \cdot \ldots \cdot f(p_j^{r_j}) \cdot \ldots }{(2^{r_1} \cdot 3^{r_2}  \cdot \ldots \cdot p_j^{r_j} \cdot \ldots)^s} \\
&= \prod_{p \in \PP} \sum_{r=0}^\infty \frac{f(p^r)}{p^{rs}}
\,,
\]
wobei sich das Produkt über alle Primzahlen erstreckt.
Der folgende Satz präzisiert diese Aussage.

\begin{satz}
	Sei $f:\NN \ra \CC$ eine multiplikative Funktion und sei die Dirichletreihe
	\[
	F(s) = \sum_{n=1}^\infty \frac{f(n)}{n^s}
	\]
	absolut konvergent. Dann ist $F(s)$ gleich dem \myemph{Euler-Produkt}
	\[
	F(s) = \prod_{p \in \PP} \Bigl(1 + \frac{f(p)}{p^s} + \frac{f(p^2)}{p^{2s}} + \ldots \Bigr)
	\,.
	\]
\end{satz}

\begin{bsp-list}
	\item Für $\zeta(s) $ sind die Koeffizienten alle gleich 1, es gilt also für $\Re s > 1$
	\[
	\zeta(s) = \prod_{p \in \PP} \Bigl(1 + \frac{1}{p^s} + \frac{1}{p^{2s}} + \ldots\Bigr)
	= \prod_{p \in \PP} \frac{1}{1-p^{-s}}
	\,.
	\]
	Diese von Euler entdeckte Formel ist der Grund für die große Rolle, die die Zetafunktion in der Zahlentheorie spielt.
	
	\item Man erhält 
	\[
	\sum_{n=1}^\infty \frac{d(n)}{n^s}
	= \zeta(s)^2
	&= \prod_{p \in \PP} (1-p^{-s})^{-2}
	= \prod_{p \in \PP} \bigl(1 + 2p^{-s} + 3p^{-2s} + \ldots\bigr) \\
	&= \prod_{p \in \PP} \Bigl(1+ \frac{d(p)}{p^s} + \frac{d(p^2)}{p^{2s}} + \ldots \Bigr)
	\,.
	\]
	
	\item Für den Kehrwert $F(s) = \frac{1}{\zeta(s)}$ der Zetafunktion erhalten wir
	\[
	\frac{1}{\zeta(s)}
	= \prod_{p \in \PP} \bigl(1-p^{-s}\bigr) = \sum_{n=1}^\infty \frac{\mu(n)}{n^s}
	\,,
	\]
	wobei
	\[
	\mu(n)
	= \begin{cases}
	(-1)^L & \parbox[t]{.6\textwidth}{wobei $L$ die Anzahl der verschiedenen Primteiler von $n$, falls $n$ quadratfrei,} \\
	1 & \text{falls } n=1,\\
	0 & \text{sonst.}
	\end{cases}
	\]
	Diese Funktion heißt auch \myemph{Möbiussche $\mu$-Funktion}.
	Über die Faltungsformel folgt die wichtige Eigenschaft
	\[
	\sum_{d | n} \mu(d)
	= \begin{cases}
	1  & \text{falls } n=1 \\
	0 & \text{sonst}
	\end{cases}
	\,.
	\]
	Die Möbiussche $\mu$-Funktion ist für folgende Formel wichtig.
\end{bsp-list}

\begin{satz}
	Seien $f$ und $g$ zwei Funktionen von $\NN$ nach $\CC$.
	Ist für alle $n\in\NN$
	\[
	f(n) = \sum_{d|n} g(d)
	\,,
	\]
	so ist für alle $n \in \NN$
	\begin{equation}\label{eq:moebiusMuFaltung}
	g(n) = \sum_{d|n} f(d) \mu\Bigl(\frac{n}{d}\Bigr)
	\,.
	\end{equation}
	
	Stehen $f$ und $g$ in dieser Beziehung zueinander, so ist $f$ multiplikativ genau dann, wenn auch $g$ multiplikativ ist.
\end{satz}
\begin{bewe}
	Die Gleichungen 
	\begin{align*}
	f(1) &= g(1) \\
	f(2) &= g(1) + g(2) \\
	f(3) &= g(1) + g(3) \\
	f(4) &= g(1) + g(2) + g(4) \\
	&\ldots 
	\end{align*}
	lassen sich induktiv für $g$ lösen
	\begin{align*}
	g(1) &= f(1) \\
	g(2) &= f(2) - f(1) \\
	g(3) &= f(3) - f(1) \\
	g(4) &= f(4) - f(2) \\
	&\ldots 
	\,.
	\end{align*}
	Es ist also klar, dass es eine solche Beziehung wie in \eqref{eq:moebiusMuFaltung} unabhängig von $f$ und $g$ existieren muss.
	Insbesondere genügt es, die Umkehrformel für Folgen \myquote{langsamen Wachstums} zu beweisen.
	Wir nehmen daher an, die Dirichletreihen
	\[
	F(s) = \sum_{n=1}^\infty \frac{f(n)}{n^s} \qquad \text{und} \qquad G(s) = \sum_{n=1}^\infty \frac{g(n)}{n^s}
	\]
	seien absolut konvergent.
	Dann gilt wegen der Faltungsformel und $\frac{1}{\zeta(s)} = \sum_{n=1}^\infty \frac{\mu(n)}{n^s}$ für alle $n\in\NN$:
	\[
	f(n) = \sum_{d|n} g(d)
	&\Rla F(s) = \Bigl(\sum_{n=1}^\infty n^{-s} \Bigr) \cdot \Bigl(\sum_{m=1}^\infty g(m) m^{-s}\Bigr) \\
	&\Rla G(s) = \zeta(s)^{-1} F(s) = \sum_{n=1}^\infty \mu(n)n^{-s} \sum_{m=1}^\infty f(m) m^{-s} \\
	&\Rla g(n) = \sum_{d|n} f(d) \mu\Bigl(\frac{n}{d}\Bigr)
	\,.
	\]
	
	Gelten nun diese Beziehung, so folgt weiter
	\[
	g \text{ multiplikativ}
	&\Rla G(s) \text{ besitzt Euler-Produkt} \\
	&\Rla F(s) = \zeta(s)G(s) \text{ besitzt Euler-Produkt} \\
	&\Rla f \text{ multiplikativ}
	\,.
	\] 
\end{bewe}

Als abschließendes Beispiel dieses Abschnitts sei
\[
r(n) = \# \Set{ (a,b) \in \ZZ^2 \mid a^2 + b^2 = n}
\]
die Anzahl der Darstellungen von $n$ als Summe zweier ganzer Quadrate.
Dann ist die Funktion $\frac{1}{4}r(n)$ multiplikativ und die entsprechende Dirichletreihe
\[
\sum_{n=1}^\infty \frac{1}{4} r(n) n^{-s}
= \zeta(s) \Bigl( 1 - \frac{1}{3^s} + \frac{1}{5^s} - \frac{1}{7^s} + \frac{1}{9^s} - \ldots \Bigr)
\]
ist für $\Re s > 1$ absolut konvergent. Dies führt mit 
\[
\chi(n)
= \begin{cases}
1 & n \equiv 1 \mod 4 \\
-1 & n \equiv 3 \mod 4 \\
0 & \text{sonst}
\end{cases}
\]
zu der nicht-trivialen Beziehung
\[
r(n) = 4\sum_{d|n} \chi(d)
\,.
\]