\section{Formale Eigenschaften von Dirichletreihen}

Ab jetzt bezeichnen wir gewöhnliche Dirichletreihen als Dirichletreihen.
Die Regeln für die Handhabung von Dirichletreihen sind anders als die bei Potenzreihen, daher wollen wir diese jetzt näher erläutern.

Es ist klar, dass die Summe zweier Dirichletreihen die Reihe ist, deren allgemeiner Koeffizient die Summe der Koeffizienten der einzelnen Reihen ist.
Aber wie bildet man das Produkt?

Seien
\[
	F(s) = \sum_{n=1}^\infty a(n)n^{-s}\,, \qquad G(s) = \sum_{n=1}^\infty b(n)n^{-s}
\]
zwei in einer offenen Menge $U \subset \CC$ durch absolut konvergente Dirichletreihen gegebene Funktionen, dann ist in $U$:
\[
	F(s) G(s)
	&= \sum_{n=1}^\infty \sum_{m=1}^\infty a(n)b(m) n^{-s}m^{-s} \\
	&= \sum_{n,m=1}^\infty a(n)b(m) (nm)^{-s} \\
	&= \sum_{k=1} \biggl(\underbrace{\sum_{d|k} a(d)b\Bigl(\frac{k}{d}\Bigr)}_{=:c(k)}\biggr) k^{-s}\,.
\]
Das heißt die additive Faltung $\sum_{n+m=k} a(n)b(m)$, die die Multiplikation von Potenzreihen beschreibt, wird durch die multiplikative Faltung $\sum_{d|k} a(d)b(\frac{k}{d})$ bei Dirichletreihen ersetzt.
Diese Tatsache ist für große Bedeutung der Dirichletreihen in der Zahlentheorie verantwortlich.