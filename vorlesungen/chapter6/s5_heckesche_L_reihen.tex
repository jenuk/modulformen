\section{Heckesche L-Reihen}

Einer Modulform $f = \sum_{n=0}^\infty a(n) q^n \in M_k$ ordnet man die L-Reihe
\[
	L(f,s) := \sum_{n=1}^\infty a(n)n^{-s}
\]
zu. Nach Hecke impliziert das Transformationsverhalten von $f$ \glqq{}gute\grqq{} analytische Eigenschaften für $L(f,s)$, wie zum Beispiel die Existenz einer meromorphen Fortsetzung nach $\CC$ oder die Gültigkeit einer Funktionalgleichung. Der Übergang von $f$ zu $L(f,s)$ erfolgt mittels Mellin-Transformation.

Konvention: Sei im Folgenden $k \geq 4$ stets gerade. 

\begin{defi}
Sei $f = \sum_{n=0}^\infty a(n) q^n \in M_k$. Dann heißt die Reihe 
\[
	L(f,s) := \sum_{n=1}^\infty a(n)n^{-s}
\]
die \myemph{Heckesche L-Reihe} zu $f$. 
\end{defi}

\begin{satz}\label{Fourierkoeff-Abschätzungen}
Sei $f = \sum_{n=0}^\infty a(n) q^n \in M_k$. Dann gilt:
\begin{enumerate}
\item $a(n) = \mathcal O(n^{k-1})$.
\item Ist $f \in S_k$, so gilt sogar $a(n) = \mathcal O(n^{\frac k2})$.
\end{enumerate}
\end{satz}

\begin{bewe}
Wegen $M_k = \CC E_k \oplus S_k$ und
\[
	E_k \propto G_k = 2 \zeta(k) + \frac {2(2\pi i)^k}{(k-1)!} \sum_{n=1}^\infty \sigma_{k-1}(n) q^n
\]
folgt die Aussage (i) mit (ii) und
\[
	\sigma_{k-1}(n) := \sum_{d|n} d^{k-1} = n^{k-1} \sum_{d|n} \left( \frac dn \right)^{k-1} \leq n^{k-1} \underbrace{\sum_{l=1}^\infty \frac 1{l^{k-1}}}_{< \infty} = \mathcal O(n^{k-1})
	\,.
\]
Wir müssen also nur noch (ii) zeigen. Sei dazu $f \in S_k$. Nach Definition ist
\[
	a(n) = \int_{ci}^{ci + 1} f(z) e^{-2\pi inz} \opd z
\]
mit $c \in \RR_{>0}$ beliebig. Man schreibe $f(z) = y^{- \frac k2} y^{\frac k2} f(z)$. Wie schon früher gezeigt, ist $g(z) := y^{\frac k2} f(z)$ auf ganz $\HH$ beschränkt. Es folgt somit
\[
	a(n) = \int_{ci}^{ci + 1} y^{-\frac k2} g(z) e^{-2\pi inz} \opd z = \int_0^1 c^{- \frac k2} g(t+ic) e^{2\pi nc} e^{-2\pi int} \opd t
\]
und damit $\abs{a(n)} \leq c^{- \frac k2} e^{2\pi nc} M$, wobei $M > 0$ nicht mehr von $n$ abhängt. Man wähle $c = \frac 1n$ und folgere $\abs{a(n)} = \mathcal O(n^{\frac k2})$. Damit ist alles gezeigt.
\end{bewe}

\begin{koro}
Sei $f \in M_k$. Dann gilt $\sigma_a (L(f,s)) \leq k$. Ist zudem $f \in S_k$, so gilt sogar $\sigma_a (L(f,s)) \leq \frac k2 + 1$. Die Funktion $s \mapsto L(f,s)$ ist in der Halbebene $\Re(s) > \sigma_c (L(f,s))$ holomorph.
\end{koro}

\begin{satz}[Hecke]\label{satz:hecke}
Sei $f = \sum_{n=0}^\infty a(n) q^n \in M_k$. Setzt man für $\Re(s) > k$
\[
	L^*(f,s) := (2\pi)^{-s} \Gamma(s) L(f,s)
	\,,
\]
dann gilt: Die Funktion 
\[
	s \mapsto L^*(f,s) + \frac {a(0)}s + \frac {(-1)^{\frac k2} a(0)}{k-s}
\]
hat eine holomorphe Fortsetzung auf ganz $\CC$, ist beschränkt in jedem Vertikalstreifen $\Set {s \in \CC \mid \nu_1 \leq \Re(s) \leq \nu_2}$ und $L^*$ erfüllt die Funktionalgleichung
\[
	L^*(f,k-s) = (-1)^{\frac k2} L^*(f,s)
	\,.
\]
Ist $f \in S_k$, so ist $a(0) = 0$ und daher sogar $s \mapsto L^*(f,s)$ selbst bereits eine ganze Funktion.
\end{satz}

\begin{bewe}
Nach \autoref{Mellin-Trafo} (Mellin-Transformation) ist
\[
	L(f,s) 
	= \frac 1{\Gamma(s)} \int_0^\infty \sum_{n=1}^\infty a(n) (e^{-x})^n x^{s-1} \opd x
	= \frac 1{\Gamma(s)} \int_0^\infty \Bigg( \underbrace{\sum_{n=0}^\infty a(n) e^{-nx}}_{= f\bigl(\frac{ix}{2\pi}\bigr)} - a(0) \Bigg) x^{s-1} \opd x
	\,,
\]
sodass nach Substitution $x \mapsto y := 2\pi x$ für alle $s \in \CC$ mit $\Re(s) > k$ gilt:
\begin{align*}
	L^*(f,s) 
	&= (2\pi)^{-s} \Gamma(s) L(f,s) \\
	&= \int_0^\infty \left( f(iy) - a(0) \right) y^{s-1} \opd y \\
	&= \underbrace{\int_0^1 \left( f(iy) - a(0) \right) y^{s-1} \opd y}_{=: I_1(s)} + \underbrace{\int_1^\infty \left( f(iy) - a(0) \right) y^{s-1} \opd y}_{=: I_2(s)}
	\,.
\end{align*}
Wie man schnell sieht, ist $L^*(f,s)$ für $\Re(s) > k$ holomorph. Darüber hinaus sieht man schnell ein, dass $I_2(s)$ für alle $s \in \CC$ konvergent (also eine ganze Funktion) und außerdem auf jedem Vertikalstreifen beschränkt ist. Es verbleibt somit nur noch das Studium von
\[
	I_1(s) = - a(0) \int_0^1 y^{s-1} \opd y + \int_0^1 f(iy) y^{s-1} \opd y
	\,.
\]
Nun gilt für den ersten Summanden
\[
	\int_0^1 y^{s-1} \opd y = \left[ \frac {y^s}s \right]_0^1 = \frac 1s
\]
und für den zweiten Summanden nach Substitution $y \mapsto \inv y$ mit $\opd (\inv y) = -y^{-2} \opd y$:
\begin{align*}
	\int_0^1 f(iy) y^{s-1} \opd y
	&= \int_\infty^1 f(i \inv y) y^{-s+1} \opd (\inv y) \\
	&= \int_1^\infty f \big( S \circ (iy) \big) y^{-s-1} \opd y \qquad \qquad \qquad \qquad \Big| \; f \in M_k \\
	&= \int_1^\infty (iy)^k f(iy) y^{-s-1} \opd y \\
	&= (-1)^{\frac k2} \int_1^\infty f(iy) y^{k-s-1} \opd y \\
	&= (-1)^{\frac k2} \left( \int_1^\infty \left( f(iy) - a(0) \right) y^{k-s-1} \opd y + a(0) \int_1^\infty y^{k-s-1} \opd y \right) \\
	&= (-1)^{\frac k2} \left( I_2(k-s) - \frac {a(0)}{k-s} \right)
	\,.
\end{align*}

Insgesamt erhalten wir also
\begin{align}
	\label{L*-SummeGanz}
	\notag{}
	L^*(f,s) + \frac {a(0)}s + \frac {(-1)^{\frac k2} a(0)}{k-s}
	&= I_1(s) + I_2(s) + \frac {a(0)}s + \frac {(-1)^{\frac k2} a(0)}{k-s} \\
	&= I_2(s) + (-1)^{\frac k2} I_2(k-s)
	\,.	
\end{align}
Da $I_2(s)$ ganz und auf jedem Vertikalstreifen beschränkt ist, müssen wir nur noch die Funktionalgleichung nachrechnen. Hierzu ersetzen wir in \eqref{L*-SummeGanz} das $s$ durch $k-s$ und beobachten ($k$ gerade):
\[
	L^*(f,k-s) + \frac {a(0)}{k-s} + \frac {(-1)^{\frac k2} a(0)}s
	&\overset{\eqref{L*-SummeGanz}}= I_2(k-s) + (-1)^{\frac k2} I_2(s) \\
	&= (-1)^{\frac k2} \left( (-1)^{\frac k2} I_2(k-s) + I_2(s) \right) \\
	&\overset{\eqref{L*-SummeGanz}}= (-1)^{\frac k2} \left( L^*(f,s) + \frac {a(0)}s + \frac {(-1)^{\frac k2} a(0)}{k-s} \right) \\
	&= (-1)^{\frac k2} L^*(f,s) + \frac {(-1)^{\frac k2} a(0)}s + \frac {a(0)}{k-s}
	\,,
\]
was nach Subtraktion von $\frac {a(0)}{k-s} + \frac {(-1)^{\frac k2} a(0)}s$ genau die Behauptung ergibt.
\end{bewe}


\begin{bsp-list}
	\item Sei $f = 1 - \frac{2k}{B_k} \sum_{m=1}^\infty \sigma_{k-1}(m)q^m$ die normierte Eisensteinreihe in $M_k$.
	Es ist
	\[
	\sum_{m=1}^\infty \frac{\sigma_{k-1}(m)}{m^s}
	= \zeta(s) \zeta(s-k+1)
	\,.
	\]
	Es folgt, dass $(\zeta(s)\zeta(s-k+1))^*$ (bis auf eine Konstante $- \frac{2k}{B_k}$) die vervollständigte Heckesche L-Reihe $L^*(E_k,s)$ ist und $(-1)^{\frac{k}{2}}$ invariant unter $s \mapsto k-s$.
	
	\item $f = \Delta \in S_{12}$.
	Dann hat $L^*(\Delta, s) = (2\pi)^{-s} \Gamma(s)L(\Delta, s)$ eine holomorphe Fortsetzung auf $\CC$ und ist invariant unter $s \mapsto 12-s$.
\end{bsp-list}

	\emph{Kuriose Anwendung}:
	Sei $f = \sum_{n \geq 1} a(n)q^n \in S_k$ und seien fast alle $a(n)$ gleich 0.
	Dann ist $f \equiv 0$.
	\begin{bewe}
		Die Funktion $L^*(f, s) = (2\pi)^{-s} \Gamma(s) L(f, s)$ ist holomorph in $\CC$.
		Also folgt $L(f,s) = 0$ für $s = 0$, $-1$, $-2$, \ldots{}.
		
		Nach Voraussetzung
		\[
		&\phantom{\Rla}\qquad f = \sum_{n=1}^N a(n) q^n \\
		&\Ra \qquad L(f,s) = \sum_{n=1}^N a(n) n^{-s} \\
		&\Ra \qquad L(f, -\nu) = \sum_{n=1}^N a(n) n^\nu = 0 \qquad \forall \nu = 0, 1, 2, \ldots \\
		&\Ra \qquad \begin{pmatrix}
		1 & 1 &\ldots & 1 \\
		\vdots & \vdots & \ddots & \vdots \\
		1^{N-1} & 2^{N-1} & \ldots & & N^{N-1}
		\end{pmatrix}
		\begin{pmatrix}
		a(1) \\
		\vdots \\
		a(N)
		\end{pmatrix}
		= 0 \\
		&\Ra a = 0 \\
		&\Ra f \equiv 0
		\,.
		\]
	\end{bewe}