\section{Der Heckesche Umkehrsatz}

\autoref{satz:hecke} hat Umkehrung:

\begin{satz}[Hecke]
	Seien $k \in 2\NN$, $\alpha > 0$ gegeben.
	Sei $(a(n))_{n\in\NN_0}$ eine Folge komlexer Zahlen mit $a(n) = \mathcal{O}(n^\alpha)$.
	Sei
	\[
		L(s) = \sum_{n=1}^\infty \frac{a(n)}{n^s}
		\qquad (\sigma > \alpha +1)
	\]
	Sei $L^*(s) = (2\pi)^{-s} \Gamma(s) L(s)$ und $L^*(s) + \frac{a(0)}{s} + \frac{(-1)^{\frac{k}{2}} a(0)}{k-s}$ besitzen holomorphe Fortsetzung nach $\CC$ und $L^*$ erfülle die Funktionalgleichung
	\[
		L^*(k-s) = (-1)^{\frac{k}{2}L^*(s)}\,.
	\]
	
	Ist zudem $L^*(s) + \frac{a(0)}{s} + \frac{(-1)^{\frac{k}{2}} a(0)}{k-s}$ beschränkt in jedem Vertikalstreifen $\nu_1 \leq Re(s) \leq \nu_2$, so gilt $f(z) = \sum_{n=0}^\infty a(n)q^n \in M_k$.
\end{satz}

\begin{bewe}
	Nutze zuerst
	\begin{satz}[Mellinscher Umkehrsatz]
		Sei $x \in \CC$ mit $\Re(x) > 0$ und $c > 0$ eine reelle Zahl.
		Dann gilt
		\[
			e^{-x}
			= \frac{1}{2\pi i} \int_{c-i\infty}^{c+i\infty} \Gamma(s)x^{-s} \opd s
			\,.
		\]
	\end{satz}
	\begin{bewe}
		Betrachte das Kurvenintegral über das Rechteck mit den Ecken $c+iT$, $-\frac{1}{2}-N+iT$, $-\frac{1}{2}-N-iT$, $c-iT$ mit $T > 0$ reell und $N > 0$ ganz, beide groß.
		
		Die horizontalen Integrale verschwinden für $T \to \infty$ (das folgt aus der Sterlingformel) und das linke vertikale Integral für $N \to \infty$ (das folgt aus der Funktionalgleichung von $\Gamma(s)$).
		Damit folgt aus dem Residuensatz
		\[
			\frac{1}{2\pi i} \int_{c-i\infty}^{c+i\infty} \Gamma(s)x^{-s} \opd s
			= \lim_{N \to \infty} \sum_{j=0}^N \frac{(-1)^j}{j!} x^j = e^{-x}\,.
		\]
	\end{bewe}

	Außerdem brauchen wir noch
	\begin{lemm}[Phragmen-Lindelöf]
		Für zwei reelle Zahlen $\nu_1 < \nu_2$ setze $F = \Set{s \in \CC\mid \nu_1 \leq \Re(s) \leq \nu_2}$.
		Sei $\phi$ eine holomorphe Funktion auf einer offenen Menge $U \supseteq F$, welche
		\[
			\abs{\phi(s)} = \mathcal{O}(e^{\abs{\tau}^\delta})
			\qquad (\abs \tau \to \infty)\quad s= \sigma + i\tau
		\]
		gleichmäßig auf $F$ mit einem $\delta > 0$ erfüllt.
		
		Gibt es dann eine reelle Zahl $b$ mit
		\[
			\abs{\phi(s)} = \mathcal{O}(\abs{\tau}^b) \qquad \text{auf } \partial F
		\]
		so folgt bereits
		\[
			\abs{\phi(s)} = \mathcal{O}(\abs{\tau}^b)
		\]
		gleichmäßig in $F$.
	\end{lemm}
	\begin{bewe}
		Nach Vorausetzung existiert ein $L > 0$ so, dass $\abs{\phi(s)} \leq L e^{\abs \tau ^\delta}$ in $F$.
		
		Betrachte zunächst den Fall $b=0$.
		Dann gibt es eine Konstante $M > 0$ so, dass $\abs{\phi(s)} \leq M$ in $\partial F$.
		Sei $m$ eine positive ganze Zahl mit $m \equiv 2 \mod 4$.
		Setze $s = \sigma +i\tau$.
		Da $\Re(s^m) = \Re((\sigma + i\tau)^m) $ ein Polynom in $\sigma$ und $\tau$ ist und der nächste höchste Term von $\tau$ durch $-\tau^m$ gegeben ist, haben wir
		\[
			\Re(s^m) = -\tau^m + \mathcal{O} (\abs{\tau}^{m-1})\,,
		\]
		gleichmäßig auf $F$.
		Es folgt, dass $\Re(s^m)$ eine obere Grenze $N$ in $F$ besitzt.
		Indem wir $m > \delta$ wählen, erhalten wir für $\epsilon > 0$
		\[
			\abs{\phi(s)e^{\epsilon s^m}} \leq Me^{\epsilon N}
			\qquad \text{auf } \partial F
		\]
		und damit
		\[
			\abs{\phi(s) e^{\epsilon s^m}}
			= \mathcal{O} \Bigl( e^{\abs{\tau}^\delta - \epsilon\abs{\tau}^m + K\abs\tau^{m-1}}\Bigr)
			\xto{\abs\tau \to \infty} 0
		\]
		gleichmäßig auf $F$.
		Nach dem Maximumsprinzip folgt damit 
		\[
			\abs{\phi(s)e^{\epsilon s^m}}
			\leq Me^{\epsilon N}
			\qquad (s\in F)
			\,.
		\]
		Da $\epsilon$ unabhängig von $M$ und $N$ bekommen wir mit $\epsilon \to 0$ $\abs{\phi(s)} \leq M$.
		Also $\abs{\phi(s)} = \mathcal{O}(\abs\tau^0)$ in $F$.
		\medskip
		
		Nehmen wir jetzt $b \not= 0$ an.
		Wir definieren eine lokal um $F$ holomorphe Hilfsfunktion $\psi(s) = (s-\nu_1 +1)^b = e^{b\log(s-\nu_1+1)}$ mit dem Hauptzweig des Logarithmus.
		Da nun
		\[
			\Re(\log(s-\nu_1 + 1))
			= \log\abs{s-\nu_1+1}
		\]
		haben wir gleichmäßig auf $F$
		\[
			\abs{\psi(s)} = \abs{s - \nu_1 + 1}^b \sim \abs\tau^b \qquad (\abs\tau \to \infty)
		\]
		Setze $\phi_1(s) = \frac{\phi(s)}{\psi(s)}$.
		Dann erfüllt $\phi_1(s)$ die gleichen Voraussetzungen wie $\phi$ mit $b=0$ und ist damit beschränkt in $F$.
		Es folgt $\abs{\phi(s)} = \mathcal{O}(\abs\tau^b)$ in $F$.
	\end{bewe}

	Betrachte für $c > \alpha + 1$ das Integral
	\[
		I(s)
		= \frac{1}{2\pi i} \int_{c-i\infty}^{c+i\infty} L^*(s)x^{-s} \opd s
		\,.
	\]
	Nach Voraussetzung ist $L^*(s)$ (bis auf die vernachlässigbaren Terme $\frac{a(0)}{s} + \frac{(-1)^{\frac{k}{2}} a(0)}{k-s}$) beschränkt auf jedem Vertikalstreifen $\nu_1 \leq \Re(s) \leq \nu_2$.
	Nach der Sterlingformel gilt nun aber
	\[
		L^*(s) = \mathcal{O} (\abs\tau^{-M}) \qquad \text{für } \abs\tau \to \infty
	\]
	auf jeder Geraden $\Re (s) = \sigma_1 > \alpha+1$ und $\Re(s) = \sigma_2 < k- \alpha -1$ wegen der Funktionalgleichung.
	Betrachte nun die Funktion
	\[
		\frac{1}{2\pi i} \int_{c-i\infty}^{c+i\infty} L^*(s)x^{-s}\opd s
	\]
	Dann gilt
	\[
		f(iy) - a(0)
		= \frac{1}{2\pi i} \int_{c-i\infty}^{c+i\infty} L^*(s)y^{-s}\opd s
	\]
	nach \autoref{Mellin-Trafo}.
	Man verschiebe den Integrationsweg nach links auf die Gerade $k-c < 0$ (wähle $c > k$).
	Auf diese Weise erhält man Beträge von Residuen der Polstellen bei $s=0$ und $s=k$ als $-a(0)$ beziehungsweise $(-1)^{\frac{k}{2}} a(0)y^{-k}$.
	Betrachte dafür für $T > 0$ reell das Kurvenintegral über das Rechteck mit Ecken $c +iT$, $c-iT$, $k-c+iT$, $k-c-iT$.
	Nach dem Prinzip von Phragmen-Lindelöf folgt, dass die Integrale über die horizontalen Streifen für $T \to \infty$ verschwinden.
	Damit folgt
	\[
		-a(0) + a(0)(-1)^{\frac{k}{2}} y^{-k}
		= \frac{1}{2\pi i} \int_{c-i\infty}^{c+i\infty} L^*(s) y^{-s} \opd s - \frac{1}{2\pi i} \int_{k-c-i\infty}^{k-c+i\infty} L^*(s)y^{-s} \opd s\,.
	\]
	Durch Substitution $s \mapsto k-s$ in das zweite Integral erhält man
	\[
		-\frac{1}{2\pi i} \int_{k-c-i\infty}^{k-c+i\infty} L^*(s) y^{-s} \opd s
		&= -\frac{1}{2\pi i} \int_{c+i\infty}^{c+i\infty} L^*(k-s) y^{s-k} \opd (k-s) \\
		&= - (-1)^{\frac{k}{2}}y^{-k} \frac{1}{2\pi i} \int_{c-i\infty}^{c+i\infty} L^*(s) y^s \opd s
		\,.
	\]
	Damit folgt
	\[
		-a(0) + (-1)^{\frac{k}{2}} a(0) y^{-k}
		= f(iy) - a(0) - (-1)^{\frac{k}{2}}y^{-k} \biggl(f\Bigl(\frac{1}{y}\Bigr) - a(0)\biggl)
	\]
	Damit erhalten wir
	\[
		0 = f(iy) - (iy)^k f\Bigl(\frac{1}{y}\Bigr)
	\]
	Also $f \in M_k$.
\end{bewe}