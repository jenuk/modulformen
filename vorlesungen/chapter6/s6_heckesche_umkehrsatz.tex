\section{Der Heckesche Umkehrsatz}

\autoref{satz:hecke} hat eine Umkehrung:

\begin{satz}[Hecke]
	Seien $k \in 2\NN$ und $\alpha > 0$ gegeben.
	Sei $(a(n))_{n\in\NN_0}$ eine Folge komplexer Zahlen mit $a(n) = \mathcal{O}(n^\alpha)$.
	Sei
	\[
		L(s) = \sum_{n=1}^\infty \frac{a(n)}{n^s}
		\qquad (\sigma > \alpha +1)
	\]
	und die Funktion $L^*(s) = (2\pi)^{-s} \Gamma(s) L(s)$ erfülle die Funktionalgleichung
	\[
		L^*(k-s) = (-1)^{\frac{k}{2}}L^*(s)
		\,.
	\]
	Die Funktion 
	\[
	s \mapsto L^*(s) + \frac{a(0)}{s} + \frac{(-1)^{\frac{k}{2}} a(0)}{k-s}
	\]
	besitze eine holomorphe Fortsetzung nach $\CC$ und sei zudem in jedem Vertikalstreifen $\nu_1 \leq \Re(s) \leq \nu_2$ beschränkt. Dann gilt $f(z) = \sum_{n=0}^\infty a(n)q^n \in M_k$.
\end{satz}

\begin{bewe}
	Wir benötigen zum Beweis zwei weitere Sätze. Zunächst: 
	\begin{satz}[Mellinscher Umkehrsatz]\label{Mellin-Umkehrsatz}
		Sei $x \in \CC$ mit $\Re(x) > 0$ und $c > 0$ eine reelle Zahl.
		Dann gilt
		\[
			e^{-x}
			= \frac{1}{2\pi i} \int_{c-i\infty}^{c+i\infty} \Gamma(s)x^{-s} \opd s
			\,.
		\]
	\end{satz}
	\begin{bewe}
		Betrachte das Kurvenintegral über das Rechteck mit den Ecken $c+iT$, $-\frac{1}{2}-N+iT$, $-\frac{1}{2}-N-iT$, $c-iT$ mit $T > 0$ reell und $N > 0$ ganz, beide groß.
		
		Die horizontalen Integrale verschwinden für $T \to \infty$ (das folgt aus der Stirlingformel, siehe \autoref{Stirling}) und das linke vertikale Integral für $N \to \infty$ (das folgt aus der Funktionalgleichung von $\Gamma(s)$).
		Damit folgt aus dem Residuensatz
		\[
			\frac{1}{2\pi i} \int_{c-i\infty}^{c+i\infty} \Gamma(s)x^{-s} \opd s
			= \lim_{N \to \infty} \sum_{j=0}^N \frac{(-1)^j}{j!} x^j = e^{-x}\,.
		\]
	\end{bewe}

	Darüber hinaus brauchen wir noch:
	\begin{lemm}[Phragmen-Lindelöf]\label{Phragmen-Lindeloef}
		Für zwei reelle Zahlen $\nu_1 < \nu_2$ setze $F = \Set{s \in \CC\mid \nu_1 \leq \Re(s) \leq \nu_2}$.
		Sei $\phi$ eine holomorphe Funktion auf einer offenen Menge $U \supseteq F$, welche für $s = \sigma + i\tau$
		\[
			\abs{\phi(s)} = \mathcal{O}(e^{\abs{\tau}^\delta})
			\qquad (\abs \tau \to \infty)
		\]
		gleichmäßig auf $F$ mit einem $\delta > 0$ erfüllt.
		
		Gibt es dann eine reelle Zahl $b$ mit $\abs{\phi(s)} = \mathcal{O}(\abs{\tau}^b)$ auf $\partial F$, so folgt bereits $\abs{\phi(s)} = \mathcal{O}(\abs{\tau}^b)$ gleichmäßig auf $F$.
	\end{lemm}
	\begin{bewe}
		Nach Vorausetzung existiert ein $L > 0$, sodass $\abs{\phi(s)} \leq L e^{\abs \tau ^\delta}$ auf $F$.
		
		Betrachte zunächst den Fall $b=0$.
		Dann gibt es eine Konstante $M > 0$, sodass $\abs{\phi(s)} \leq M$ auf $\partial F$.
		Sei $m$ eine positive ganze Zahl mit $m \equiv 2 \mod 4$.
		Setze $s = \sigma +i\tau$.
		Da $\Re(s^m) = \Re((\sigma + i\tau)^m) $ ein Polynom in $\sigma$ und $\tau$ ist, dessen höchste Potenz von $\tau$ durch $-\tau^m$ gegeben ist, haben wir
		\[
			\Re(s^m) = -\tau^m + \mathcal{O} (\abs{\tau}^{m-1})
		\]
		gleichmäßig auf $F$.
		Es folgt, dass $\Re(s^m)$ eine obere Grenze $N$ auf $F$ besitzt.
		Indem wir $m > \delta$ wählen, erhalten wir für $\epsilon > 0$
		\[
			\abs{\phi(s)e^{\epsilon s^m}} \leq Me^{\epsilon N}
			\qquad \text{auf } \partial F
		\]
		und damit
		\[
			\abs{\phi(s) e^{\epsilon s^m}}
			= \mathcal{O} \Bigl( e^{\abs{\tau}^\delta - \epsilon\abs{\tau}^m + K\abs\tau^{m-1}}\Bigr)
			\xto{\abs\tau \to \infty} 0
		\]
		gleichmäßig auf $F$.
		Nach dem Maximumsprinzip folgt damit bereits
		\[
			\abs{\phi(s)e^{\epsilon s^m}}
			\leq Me^{\epsilon N}
		\]
		auf $F$.
		Da $\epsilon$ unabhängig von $M$ und $N$ ist, bekommen wir mit $\epsilon \to 0$ wie behauptet $\abs{\phi(s)} \leq M$, also $\abs{\phi(s)} = \mathcal{O}(\abs\tau^0)$, auf $F$.
		\medskip
		
		Nehmen wir jetzt $b \not= 0$ an.
		Wir definieren eine lokal um $F$ holomorphe Hilfsfunktion $\psi(s) := (s-\nu_1 +1)^b = e^{b\Log(s-\nu_1+1)}$ mit $\Log$ dem Hauptzweig des komplexen Logarithmus.
		Da nun
		\[
			\Re(\Log(s-\nu_1 + 1))
			= \log\abs{s-\nu_1+1}
			\,,
		\]
		haben wir gleichmäßig auf $F$
		\[
			\abs{\psi(s)} = \abs{s - \nu_1 + 1}^b \sim \abs\tau^b \qquad (\abs\tau \to \infty)
			\,.
		\]
		Setze $\phi_1(s) = \frac{\phi(s)}{\psi(s)}$.
		Dann erfüllt $\phi_1$ die Voraussetzungen des Satzes mit $b=0$ und ist damit wie gerade gezeigt gleichmäßig beschränkt auf $F$.
		Es folgt $\abs{\phi(s)} = \mathcal{O}(\abs\tau^b)$ gleichmäßig auf $F$.
	\end{bewe}

	Nach Voraussetzung ist $L^*(s)$ (bis auf die vernachlässigbaren Terme $\frac{a(0)}{s} + \frac{(-1)^{\frac{k}{2}} a(0)}{k-s}$) beschränkt auf jedem Vertikalstreifen $\nu_1 \leq \Re(s) \leq \nu_2$.
	Nach der Stirlingformel gilt nun
	\[
		L^*(s) = \mathcal{O} (\abs\tau^{-M}) \qquad \text{für } \abs\tau \to \infty
	\]
	auf jeder Geraden $\Re (s) = \sigma_1 > \alpha+1$ und wegen der Funktionalgleichung genauso auf jeder Geraden $\Re(s) = \sigma_2 < k- \alpha -1$.
	Betrachte für $c > \max \Set {\alpha + 1, k}$ und $y > 0$ beliebig das Integral
	\[
		\frac{1}{2\pi i} \int_{c-i\infty}^{c+i\infty} L^*(s)y^{-s} \opd s
		= f(iy) - a(0)
	\]
	nach dem Mellinschen Umkehrsatz (\autoref{Mellin-Umkehrsatz}):
	Man verschiebe den Integrationsweg nach links auf die Gerade $\Re (s) = k-c < 0$.
	Auf diese Weise erhält man Beiträge von Residuen der Polstellen von $L^*(s)y^{-s}$ bei $s=0$ und $s=k$ als $-a(0)$ beziehungsweise $(-1)^{\frac{k}{2}} a(0)y^{-k}$.
	Betrachte nun für $T > 0$ reell das Kurvenintegral über das Rechteck mit Ecken $c +iT$, $c-iT$, $k-c+iT$, $k-c-iT$.
	Nach dem Prinzip von Phragmen-Lindelöf (\autoref{Phragmen-Lindeloef}) folgt dann, dass die Integrale über die horizontalen Streifen für $T \to \infty$ verschwinden, also verbleibt nur noch
	\[
		-a(0) + a(0)(-1)^{\frac{k}{2}} y^{-k}
		= \frac{1}{2\pi i} \int_{c-i\infty}^{c+i\infty} L^*(s) y^{-s} \opd s - \frac{1}{2\pi i} \int_{k-c-i\infty}^{k-c+i\infty} L^*(s)y^{-s} \opd s
		\,.
	\]
	Durch Substitution $s \mapsto k-s$ in das zweite Integral erhält man
	\[
		-\frac{1}{2\pi i} \int_{k-c-i\infty}^{k-c+i\infty} L^*(s) y^{-s} \opd s
		&= -\frac{1}{2\pi i} \int_{c+i\infty}^{c-i\infty} L^*(k-s) y^{s-k} \opd (k-s) \\
		&= - (-1)^{\frac{k}{2}}y^{-k} \frac{1}{2\pi i} \int_{c-i\infty}^{c+i\infty} L^*(s) y^s \opd s
		\,.
	\]
	Damit folgt
	\[
		-a(0) + (-1)^{\frac{k}{2}} a(0) y^{-k}
		= f(iy) - a(0) - (-1)^{\frac{k}{2}}y^{-k} \biggl(f\Bigl(\frac{i}{y}\Bigr) - a(0)\biggl)
	\]
	und somit
	\[
		0 = f(iy) - (iy)^k f\Bigl(\frac{i}{y}\Bigr)
		\,.
	\]
	Also gilt wie behauptet $f \in M_k$.
\end{bewe}

Allgemeiner gilt sogar Folgendes:

\begin{satz}\label{satz:Lreihe-quasiMf}
	Seien $k > 0$, $\lambda >0$ reelle Zahlen und $C \in \CC^\times$. Sei $a(n)$ eine Folge sodass $a(n) = \mathcal O(n^\alpha)$ für ein $\alpha > 0$. Sei
	\begin{align*}
	L(s) &= \sum_{n=1}^\infty a(n)n^{-s}\,,\\
	L^*(s) &= \left( \frac{2\pi}{\lambda}\right)^{-s} \Gamma(s) L(s)\,,\\
	f(z) &= \sum_{n=1}^\infty a(n)e^{2\pi i nz/\lambda}\,.
	\end{align*}
	Dann sind äquivalent: 
	\begin{enumerate}
		\item $f(-1/z) = C (z/i)^{k} f(z)$ für alle $z \in \HH$.
		\item $L^*(s) + \frac{a(0)}{s} + \frac{Ca(0)}{k - s}$ hat holomorphe Fortsetzung, ist beschränkt in $\nu_1 \leq \Re(s) \leq \nu_2$ und es gilt 
		\[L^*(k - s) = C L^*(s).\]
	\end{enumerate}
\end{satz}


\begin{bsp}
	Theta-Transformationsformel: $A \in M_m(\RR)$, $A = A^t$, $A > 0$, setze
	\[
	\theta_A(z) = \sum_{g \in \ZZ^m} e^{\pi i A[g] z}, \qquad z \in \HH\,.
	\]
	Dann gilt 
	\[
	\theta_{A^{-1}}(-1/z) = \sqrt{\det A} (z / i)^{m/2} \theta_A(z)\,.
	\]
	Ist speziell $m=1$ und $A = 1$, so folgt 
	\[
	\theta_1(z) = 1 + 2\sum_{n=1}^\infty e^{\pi i n^2 z}\,.
	\]
	Sei $f = \frac12 \theta_1$. Dann gilt
	\[
	f(-1/z) = (z/i)^{1/2} f(z).
	\]
	Man wende nun \autoref{satz:Lreihe-quasiMf} auf $f$ mit den Parametern $k = \frac12$, $\lambda = 2$ und $C = 1$, also $a(0) = \frac12$. Dann
	\[
	L(s) = \sum_{n=1}^\infty (n^{2})^{-s} = \zeta(2s).
	\]
	Es folgt $\pi^{-s} \Gamma(s)\zeta(2s) + \frac{1}{2s} + \frac{1}{1-2s}$ hat holomorphe Fortsetzung auf $\CC$, ist beschränkt auf Vertikalstreifen $\nu_1 \leq \Re(s) \leq \nu_2$ und es gilt die Funktionalgleichung unter $s \mapsto \frac12 - s$. Hieraus erhält man die schon bekannten analytischen Eigenschaften von $\zeta(s)$, üblicherweise formuliert wie folgt: $\pi^{- \frac s2}\Gamma(\frac s2)\zeta(s)$ hat holomorphe Fortsetzung auf $\CC \setminus \Set {0,1}$, einfache Pole in $s=0$ und $s=1$ und ist invariant  unter $s \mapsto 1 - s$.
\end{bsp} 