\section{Die Mellin-Transformation}

\begin{erin}
	Die Gammafunktion $\Gamma(s)$ ist definiert durch
	\[
		\Gamma(s) = \lim_{N\to\infty} \frac{(N-1)!N^s}{s(s+1) \ldots (s+N-1)}\,.
	\]
	Es gilt der folgende Satz:
	\begin{satz}
		Die Funktion $\Gamma(s)$ ist eine in ganz $\CC\setminus-\NN_0$ holomorphe Funktion mit einfachen Polen in $s=0$, $-1$, $-2$, \ldots{}.
		Es gilt die Funktionalgleichung
		\[
			s\Gamma(s) = \Gamma(s+1) \qquad \forall s \in \CC\setminus-\NN_0\,.
		\]
		Weiter gilt $\Gamma(n) = (n-1)!$ für $n \in \NN$ und
		\[
			\res_{s=-n} \Gamma(s) = \frac{(-1)^n}{n!}
			\qquad \forall n \in \NN_0\,.
		\]
	\end{satz}
	\begin{bewe}
		Siehe FT 2 oder Busam-Freitag FT 1.
	\end{bewe}

	Da $\Gamma(s) \not= 0$ für alle $s \in \CC$, ist $\frac{1}{\Gamma(s)}$ eine ganze Funktion und es gilt die Produktentwicklung
	\[
		\frac{1}{\Gamma(s)} = s e^{\gamma s} \prod_{j=1}^\infty \Bigl(1 + \frac{s}{j}\Bigr)e^{-\frac{s}{j}}
		\,,
	\]
	wobei $\gamma = \lim_{N \to \infty} 1 + \frac{1}{2} + \ldots + \frac{1}{N} - \log(N)$ die Euler-Mascheroni-Konstante ist.
	
	Eine der wichtigen Formeln für die Gammafunktion ist gegeben durch:
	\begin{satz}[Stirling-Formel]\label{Stirling}
		Es gilt in jedem Winkelbereich $W_\delta = \Set{s \in \CC \mid -\pi + \delta < \Arg(s) < \pi - \delta}$
		\[
			\Gamma(s)
			= \sqrt{2\pi} \cdot s^{s - \frac{1}{2}} e^{-s} e^{H(s)}
			\,,
		\]
		wobei $H(s)$ eine in $\CC_-$ holomorphe Funktion ist mit der Eigenschaft
		\[
			\lim_{\substack{\abs s \to \infty \\ s \in W_\delta}} H(s) = 0
			\,.
		\]
	\end{satz}
	Diese ist vor allem Dingen dafür geeignet, den exponentiellen Abfall der Funktion $\Gamma(s)$ auf vertikalen Streifen $\sigma_1 < \Re s < \sigma_2$ zu beweisen.
\end{erin}

Eine für uns sehr wichtige Darstellung der Gammafunktion ist die Integraldarstellung von Euler:

\begin{satz}
Es gilt für alle $s \in \CC$ mit $\Re (s) > 0$:
\begin{equation}\label{eq:GammaEulerIntegral}
	\Gamma(s) = \int_0^\infty e^{-x} x^{s-1} \opd x
	\,.
\end{equation}
\end{satz}

\begin{bewe}
Siehe FT 2: Man verifiziert die Funktionalgleichung über partielle Integration und nutzt anschließend den Satz von Wielandt. 
\end{bewe}

Die enorme Bedeutung der Gammafunktion in der Zahlentheorie wird in ihrem Zusammenspiel mit Dirichletreihen deutlich:

\begin{satz}[Mellin-Transformation]\label{Mellin-Trafo}
Es sei $F(s) = \sum_{n=1}^\infty \frac {a(n)}{n^s}$ eine Dirichletreihe, welche irgendwo konvergiert. Dann gilt für die zugehörige Potenzreihe $P(z) = \sum_{n=1}^\infty a(n) z^n$ und für alle $s \in \CC$ mit $\Re (s) > \max \Set {0, \sigma_a(F)}$:
\[
	F(s) = \frac 1{\Gamma(s)} \int_0^\infty P(e^{-x}) \underbrace{x^{s-1}}_{\text{Mellin-Kern}} \opd x
	\,.
\]
\end{satz}

\begin{bewe}
Für beliebiges $n \in \NN$ machen wir in \eqref{eq:GammaEulerIntegral} die Substitution $x = ny$ und sehen, dass
\[
	\Gamma(s) = n^s \int_0^\infty e^{-ny} y^{s-1} \opd y
	\,.
\]
Daraus folgt für alle $s \in \CC$ mit $\Re (s) > \max \Set {0, \sigma_a(F)}$, dass
\begin{align*}
	F(s) = \sum_{n=1}^\infty \frac {a(n)}{n^s}
	&= \sum_{n=1}^\infty \frac {a(n)}{\Gamma(s)} \int_0^\infty e^{-ny} y^{s-1} \opd y \\
	&= \frac 1{\Gamma(s)} \sum_{n=1}^\infty \int_0^\infty a(n) e^{-ny} y^{s-1} \opd y \\
	&\overset {(\ast)}= \frac 1{\Gamma(s)} \int_0^\infty y^{s-1} \sum_{n=1}^\infty a(n) e^{-ny} \opd y \\
	&= \frac 1{\Gamma(s)} \int_0^\infty P(e^{-y}) y^{s-1} \opd y
	\,.
\end{align*}
Die Vertauschung von Integral und Summe bei $(\ast)$ ist nach Satz von Lebesgue gerechtfertigt wegen $\sigma := \Re (s) > \max \Set {0, \sigma_a(F)}$ nach Voraussetzung und daher
\begin{align*}
	\sum_{n=1}^\infty \int_0^\infty \abs{a(n) e^{-ny} y^{s-1} } \opd y
	&= \sum_{n=1}^\infty \abs{a(n)} \int_0^\infty e^{-ny} y^{\sigma-1} \opd y && \big| \; \sigma > 0 \\
	&= \sum_{n=1}^\infty \abs{a(n)} n^{-\sigma} \Gamma(\sigma) \\
	&= \Gamma(\sigma) \sum_{n=1}^\infty \abs{\frac {a(n)}{n^{\sigma}}} && \big| \; \sigma > \sigma_a(F) \\
	&< \infty
	\,.
\end{align*}
\end{bewe}

\begin{bsp-list}
\item Ist $F(s) = \zeta(s)$, so erhalten wir für $s \in \CC$ mit $\Re (s) > 1$:
\begin{equation}\label{eq:ZetaMellinInt}
	F(s) = \zeta(s) = \frac 1{\Gamma(s)} \int_0^\infty \frac {x^{s-1}}{e^x - 1} \opd x
	\,.
\end{equation}
\item Ist $G(s) = 1 - \frac 1{2^s} + \frac 1{3^s} - \frac 1{4^s} \pm \ldots = (1 - 2^{1-s}) \zeta(s)$, so gilt sogar für alle $s \in \CC$ mit $\Re (s) > 0$ (folgt nicht vollständig aus \autoref{Mellin-Trafo}, lässt sich aber beweisen):
\[
	G(s) = \frac 1{\Gamma(s)} \int_0^\infty \frac {x^{s-1}}{e^x + 1} \opd x
	\,.
\]
\end{bsp-list}

Zum Schluss beweisen wir noch eine nützliche Darstellung von $\Log \Gamma(s+1)$ in Form der expliziten Taylor-Entwicklung im Bereich $\abs{s} < 1$: 

\begin{satz}\label{LogGamma(s+1)}
Es gilt für alle $s \in \EE := U_1(0)$:
\[
	\Log \Gamma(s+1) = - \gamma s + \sum_{n=2}^\infty (-1)^n \frac {\zeta(n)}n s^n
	\,.
\]
\end{satz}

\begin{bewe}
Es gilt
\[
	\Gamma(s+1) = s \Gamma(s) = \lim_{N \to \infty} \frac {N^s}{\left( 1 + \frac s1 \right) \left( 1 + \frac s2 \right) \cdots \left( 1 + \frac s{N-1} \right)}
\]
und somit folgt durch Logarithmieren:
\begin{align*}
	\Log \Gamma(s+1) 
	&= \lim_{N \to \infty} \left[ s \Log N - \sum_{n=1}^{N-1} \Log \left( 1 + \frac sn \right) \right] \\
	&= \lim_{N \to \infty} \left[ s \log N - \sum_{n=1}^{N-1} \left( \frac sn - \frac {s^2}{2n^2} + \frac {s^3}{3n^3} \mp \ldots \right) \right] \\
	&= \lim_{N \to \infty} \Bigg[ s \left( \log N - \left( 1 + \frac 12 + \frac 13 + \ldots + \frac 1{N-1} \right) \right) \\
	&\qquad\qquad + \frac {s^2}2 \left( 1 + \frac 1{2^2} + \frac 1{3^2} + \ldots + \frac 1{(N-1)^2} \right) \\
	&\qquad\qquad - \frac {s^3}3 \left( 1 + \frac 1{2^3} + \frac 1{3^3} + \ldots + \frac 1{(N-1)^3} \right) \\
	&\qquad\qquad \pm \ldots \, \Bigg] \\
	&= - \gamma s + \frac {\zeta(2)}2 s^2 - \frac {\zeta(3)}3 s^3 \pm \ldots
	\,.
\end{align*}
Da die Folgen $a_r(N) = \sum_{n=1}^N \frac 1{n^r}$ für $r \geq 2$ und $N \to \infty$ gleichmäßig gegen die Grenzwerte $\zeta(r)$ konvergieren, ist die Vertauschung von Limes und Summation ($\pm \ldots$) im letzten Schritt erlaubt.
\end{bewe}