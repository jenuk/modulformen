\section{Die Mellin-Transformation}

\begin{erin}
	Die Gammafunktion $\Gamma(s)$ ist definiert durch
	\[
		\Gamma(s) = \lim_{N\to\infty} \frac{(N-1)!N^s}{s(s+1) \ldots (s+N-1)}\,.
	\]
	Es gilt der folgende Satz
	\begin{satz}
		Die Funktion $\Gamma(s)$ ist in ganz $\CC\setminus-\NN_0$ holomorphe Funktion mit einfachen Polen in $s=0$, $-1$, $-2$, \ldots{}
		Es gilt die Funktionalgleichung
		\[
			s\Gamma(s) = \Gamma(s+1) \qquad \forall s \in \CC\setminus-\NN_0\,.
		\]
		Weiter gilt $\Gamma(n) = (n-1)!$ für $n \in \NN$ und
		\[
			\res_{s=-n} \Gamma(s) = \frac{(-1)^n}{n!}
			\qquad \forall n \in \NN_0\,.
		\]
	\end{satz}
	\begin{bewe}
		Siehe FT 2 oder Busam-Freitag FT 1.
	\end{bewe}

	Da $\Gamma(s) \not= 0$ für alle $s \in \CC$ ist $\frac{1}{\Gamma(s)}$ eine ganze Funktion und es gilt die Produktentwicklung
	\[
		\frac{1}{\Gamma(s)} = s e^{\gamma s} \prod_{j=1}^\infty \Bigl(1 + \frac{s}{j}\Bigr)e^{-\frac{s}{j}}
	\]
	wobei $\gamma = \lim_{N \to \infty} 1 + \frac{1}{2} + \ldots + \frac{1}{N} - \log(N)$.
	
	Eine der wichtigen Formeln für die Gammafunktion ist gegeben durch
	\begin{satz}
		Es gilt in jedem Winkelbereich $W_\delta = \Set{s \in \CC \mid -\pi + \delta < \Arg(s) < \pi - \delta}$
		\[
			\Gamma(s)
			= \sqrt{2\pi} \cdot s^{s - \frac{1}{2}} e^{-s} e^{H(s)}
		\]
		wobei $H(s)$ eine in $\CC_-$ holomorphe Funktion ist mit der Eigenschaft
		\[
			\lim_{\substack{\abs s \to \infty \\ s \in W_\delta}} H(s) = 0\,.
		\]
	\end{satz}
	Diese ist vor allem Dingen dafür geeignet den exponentiellen Abfall der Funktion $\Gamma(s)$ auf vertikalen Streifen $\sigma_1 < \Re s < \sigma_2$ beweisen.
\end{erin}
