\section{Die Mellin-Transformation}

\begin{erin}
	Die Gammafunktion $\Gamma(s)$ ist definiert durch
	\[
		\Gamma(s) = \lim_{N\to\infty} \frac{(N-1)!N^s}{s(s+1) \ldots (s+N-1)}\,.
	\]
	Es gilt der folgende Satz
	\begin{satz}
		Die Funktion $\Gamma(s)$ ist in ganz $\CC\setminus-\NN_0$ holomorphe Funktion mit einfachen Polen in $s=0$, $-1$, $-2$, \ldots{}
		Es gilt die Funktionalgleichung
		\[
			s\Gamma(s) = \Gamma(s+1) \qquad \forall s \in \CC\setminus-\NN_0\,.
		\]
		Weiter gilt $\Gamma(n) = (n-1)!$ für $n \in \NN$ und
		\[
			\res_{s=-n} \Gamma(s) = \frac{(-1)^n}{n!}
			\qquad \forall n \in \NN_0\,.
		\]
	\end{satz}
	\begin{bewe}
		Siehe FT 2 oder Busam-Freitag FT 1.
	\end{bewe}

	Da $\Gamma(s) \not= 0$ für alle $s \in \CC$ ist $\frac{1}{\Gamma(s)}$ eine ganze Funktion und es gilt die Produktentwicklung
	\[
		\frac{1}{\Gamma(s)} = s e^{\gamma s} \prod_{j=1}^\infty \Bigl(1 + \frac{s}{j}\Bigr)e^{-\frac{s}{j}}
	\]
	wobei $\gamma = \lim_{N \to \infty} 1 + \frac{1}{2} + \ldots + \frac{1}{N} - \log(N)$.
	
	Eine der wichtigen Formeln für die Gammafunktion ist gegeben durch
	\begin{satz}
		Es gilt in jedem Winkelbereich $W_\delta = \Set{s \in \CC \mid -\pi + \delta < \Arg(s) < \pi - \delta}$
		\[
			\Gamma(s)
			= \sqrt{2\pi} \cdot s^{s - \frac{1}{2}} e^{-s} e^{H(s)}
		\]
		wobei $H(s)$ eine in $\CC_-$ holomorphe Funktion ist mit der Eigenschaft
		\[
			\lim_{\substack{\abs s \to \infty \\ s \in W_\delta}} H(s) = 0\,.
		\]
	\end{satz}
	Diese ist vor allem Dingen dafür geeignet den exponentiellen Abfall der Funktion $\Gamma(s)$ auf vertikalen Streifen $\sigma_1 < \Re s < \sigma_2$ beweisen.
\end{erin}

Eine für uns sehr wichtige Darstellung der Gammafunktion ist die Integraldarstellung von Euler:

\begin{satz}
Es gilt für alle $s \in \CC$ mit $\Re (s) > 0$:
\begin{equation}\label{eq:GammaEulerIntegral}
	\Gamma(s) = \int_0^\infty e^{-x} x^{s-1} \opd x
	\,.
\end{equation}
\end{satz}

\begin{bewe}
Siehe FT 2: Man verifiziert die Funktionalgleichung über partielle Integration und nutzt anschließend den Satz von Wielandt. 
\end{bewe}

Die enorme Bedeutung der Gammafunktion in der Zahlentheorie wird in ihrem Zusammenspiel mit Dirichletreihen deutlich:

\begin{satz}[Mellin-Transformation]\label{Mellin-Trafo}
Es sei $F(s) = \sum_{n=1}^\infty \frac {a(n)}{n^s}$ eine Dirichletreihe, welche irgendwo konvergiert. Dann gilt für die zugehörige Potenzreihe $P(z) = \sum_{n=1}^\infty a(n) z^n$ und für alle $s \in \CC$ mit $\Re (s) > \max \Set {0, \sigma_a(F)}$:
\[
	F(s) = \frac 1{\Gamma(s)} \int_0^\infty P(e^{-x}) \underbrace{x^{s-1}}_{\text{Mellin-Kern}} \opd x
	\,.
\]
\end{satz}

\begin{bewe}
Für beliebiges $n \in \NN$ machen wir in \eqref{eq:GammaEulerIntegral} die Substitution $x = ny$ und sehen, dass
\[
	\Gamma(s) = n^s \int_0^\infty e^{-ny} y^{s-1} \opd y
	\,.
\]
Daraus folgt für alle $s \in \CC$ mit $\Re (s) > \max \Set {0, \sigma_a(F)}$, dass
\begin{align*}
	F(s) = \sum_{n=1}^\infty \frac {a(n)}{n^s}
	&= \sum_{n=1}^\infty \frac {a(n)}{\Gamma(s)} \int_0^\infty e^{-ny} y^{s-1} \opd y \\
	&= \frac 1{\Gamma(s)} \sum_{n=1}^\infty \int_0^\infty a(n) e^{-ny} y^{s-1} \opd y \\
	&\overset {(\ast)}= \frac 1{\Gamma(s)} \int_0^\infty y^{s-1} \sum_{n=1}^\infty a(n) e^{-ny} \opd y \\
	&= \frac 1{\Gamma(s)} \int_0^\infty P(e^{-y}) y^{s-1} \opd y
	\,.
\end{align*}
Die Vertauschung von Integral und Summe bei $(\ast)$ ist nach Satz von Lebesgue gerechtfertigt wegen $\sigma := \Re (s) > \max \Set {0, \sigma_a(F)}$ nach Voraussetzung und daher
\begin{align*}
	\sum_{n=1}^\infty \int_0^\infty \abs{a(n) e^{-ny} y^{s-1} } \opd y
	&= \sum_{n=1}^\infty \abs{a(n)} \int_0^\infty e^{-ny} y^{\sigma-1} \opd y && \big| \; \sigma > 0 \\
	&= \sum_{n=1}^\infty \abs{a(n)} n^{-\sigma} \Gamma(\sigma) \\
	&= \Gamma(\sigma) \sum_{n=1}^\infty \abs{\frac {a(n)}{n^{\sigma}}} && \big| \; \sigma > \sigma_a(F) \\
	&< \infty
	\,.
\end{align*}
\end{bewe}

\begin{bsp-list}
\item Ist $F(s) = \zeta(s)$, so erhalten wir für $s \in \CC$ mit $\Re (s) > 1$:
\begin{equation}\label{eq:ZetaMellinInt}
	F(s) = \zeta(s) = \frac 1{\Gamma(s)} \int_0^\infty \frac {x^{s-1}}{e^x - 1} \opd x
	\,.
\end{equation}
\item Ist $G(s) = 1 - \frac 1{2^s} + \frac 1{3^s} - \frac 1{4^s} \pm \ldots = (1 - 2^{1-s}) \zeta(s)$, so gilt sogar für alle $s \in \CC$ mit $\Re (s) > 0$ (folgt nicht vollständig aus \autoref{Mellin-Trafo}, lässt sich aber beweisen):
\[
	G(s) = \frac 1{\Gamma(s)} \int_0^\infty \frac {x^{s-1}}{e^x + 1} \opd x
	\,.
\]
\end{bsp-list}

Zum Schluss beweisen wir noch eine nützliche Darstellung von $\Log \Gamma(s+1)$ in Form der expliziten Taylor-Entwicklung im Bereich $\abs{s} < 1$: 

\begin{satz}\label{LogGamma(s+1)}
Es gilt für alle $s \in U_1(0)$:
\[
	\Log \Gamma(s+1) = - \gamma s + \sum_{n=2}^\infty (-1)^n \frac {\zeta(n)}n s^n
	\,.
\]
\end{satz}

\begin{bewe}
Es gilt
\[
	\Gamma(s+1) = s \Gamma(s) = \lim_{N \to \infty} \frac {N^s}{\left( 1 + \frac s1 \right) \left( 1 + \frac s2 \right) \cdots \left( 1 + \frac s{N-1} \right)}
\]
und somit folgt durch Logarithmieren:
\begin{align*}
	\Log \Gamma(s+1) 
	&= \lim_{N \to \infty} \left[ s \Log N - \sum_{n=1}^{N-1} \Log \left( 1 + \frac sn \right) \right] \\
	&= \lim_{N \to \infty} \left[ s \log N - \sum_{n=1}^{N-1} \left( \frac sn - \frac {s^2}{2n^2} + \frac {s^3}{3n^3} \mp \ldots \right) \right] \\
	&= \lim_{N \to \infty} \Bigg[ s \left( \log N - \left( 1 + \frac 12 + \frac 13 + \ldots + \frac 1{N-1} \right) \right) \\
	&\qquad\qquad + \frac {s^2}2 \left( 1 + \frac 1{2^2} + \frac 1{3^2} + \ldots + \frac 1{(N-1)^2} \right) \\
	&\qquad\qquad - \frac {s^3}3 \left( 1 + \frac 1{2^3} + \frac 1{3^3} + \ldots + \frac 1{(N-1)^3} \right) \\
	&\qquad\qquad \pm \ldots \, \Bigg] \\
	&= - \gamma s + \frac {\zeta(2)}2 s^2 - \frac {\zeta(3)}3 s^3 \pm \ldots
	\,.
\end{align*}
Da die Folgen $a_r(N) = \sum_{n=1}^N \frac 1{n^r}$ für $r \geq 2$ und $N \to \infty$ gleichmäßig gegen die Grenzwerte $\zeta(r)$ konvergieren, ist die Vertauschung von Limes und Summation ($\pm \ldots$) im letzten Schritt erlaubt.
\end{bewe}

\subsection{Die Riemannsche Zetafunktion}

Die einfachste und wichtigste Dirichletreihe ist die Riemannsche Zetafunktion
\[
	\zeta(s) 
	:= \sum_{n=1}^\infty \frac 1{n^s}
	= \prod_{p \in \PP} \frac 1{1 - p^{-s}}
	= \frac 1{\Gamma(s)} \int_0^\infty \frac {x^{s-1}}{e^x - 1} \opd x
	\,,
\]
wobei alle drei Darstellungen nur für $s \in \CC$ mit $\Re(s) > 1$ gültig sind. Die wichtigsten bisher bewiesenen Eigenschaften der Zetafunktion sind im folgenden Satz zusammengefasst:

\begin{satz}\label{Zeta-Fakten}
Die auf $\Set {z \in \CC \mid \Re(s) > 1}$ durch $\zeta(s) := \sum_{n=1}^\infty \frac 1{n^s}$ definierte Funktion $\zeta$ besitzt eine meromorphe Fortsetzung in die komplexe Zahlenebene $\CC$ mit einem einfachen Pol an der Stelle $s = 1$ mit Residuum $1$. Dies ist zugleich ihre einzige Polstelle. 

Die Werte der Zetafunktion bei nichtpositiven ganzen Zahlen sind rational, genauer:
\begin{align*}
	\zeta(0) &= - \frac 12 \\
	\zeta(-2n) &= 0 && \forall \, n \in \NN \\
	\zeta(1-2n) &= - \frac {B_{2n}}{2n} && \forall \, n \in \NN
	\,,
\end{align*}
wobei die rationalen Zahlen $B_2 = \frac 16$, $B_4 = - \frac 1{30}$, \ldots die durch
\[
	\frac t{e^t - 1} = \sum_{k=0}^\infty \frac {B_k}{k!} t^k \quad \text{ für } t \in U_{2\pi}(0)
\]
definierten \myemph{Bernoulli-Zahlen} sind. 

Die Werte der Zetafunktion bei positiven geraden Zahlen sind durch
\[
	\zeta(2n) = \frac {(-1)^{n-1} 2^{2n-1} B_{2n}}{(2n)!} \pi^{2n}, \quad n \in \NN
\]
gegeben.
\end{satz}

\begin{bewe}
Wir entwickeln zunächst
\[
	\frac t{e^t - 1} = \frac t{t + \frac{t^2}{2!} + \frac {t^3}{3!} + \ldots} = 1 - \frac t2 + \frac {t^2}{12} - \frac {t^4}{720} \pm \ldots
\]
und definieren $B_n$ als das $n!$-fache des Koeffizienten von $t^n$ auf der rechten Seite. Aus
\[
	\frac t{e^t - 1} - \frac {-t}{e^{-t} - 1} = -t
\]
folgt, dass abgesehen von $B_1 = - \frac 12$ alle $B_n$ mit $n$ ungerade verschwinden. Setze nun für beliebiges $n \in \NN$
\[
	f_n(t) := \sum_{k=0}^n (-1)^k \frac {B_k}{k!} t^k = 1 + \frac t2 + \frac {B_2}{2!}t^2 + \ldots + \frac {B_n}{n!}t^n
\]
unter Beachtung von $(-1)^k B_k = B_k$ für $k > 1$ wegen $B_k = 0 = - B_k$ für ungerade $k > 1$. 

Dann gilt für alle $s \in \CC$ mit $\sigma := \Re(s) > 1$ und für $n \in \NN$ beliebig:
\begin{align*}
	\Gamma(s) \zeta(s) 
	&\overset{\eqref{eq:ZetaMellinInt}}= \int_0^\infty \frac {t^{s-1}}{e^t - 1} \opd t \\
	&= \int_0^\infty \frac {te^t}{e^t - 1} e^{-t} t^{s-2} \opd t \\
	&= \underbrace{\int_0^\infty \left( \frac {te^t}{e^t - 1} - f_n(t) \right) e^{-t} t^{s-2} \opd t}_{=: I_1(s)} + \underbrace{\int_0^\infty f_n(t) e^{-t} t^{s-2} \opd t}_{=: I_2(s)}
	\,,
\end{align*}
Die Funktion $t \mapsto \frac {te^t}{e^t - 1}$ ist lokal um $t = 0$ holomorph und hat dort die Taylorentwicklung
\[
	\frac {te^t}{e^t - 1} = \frac {-t}{e^{-t} - 1} = \sum_{k=0}^\infty \frac {B_k}{k!} (-t)^k = \sum_{k=0}^\infty (-1)^k \frac {B_k}{k!} t^k = f_\infty(t)
	\,,
\]
sodass für $t \to 0$
\[
	\frac {te^t}{e^t - 1} - f_n(t) = \mathcal O (t^{n+1})
\]
gilt. Somit ist der Integrand von $I_1(s)$ für $t \to 0$ in $\mathcal O (t^{n+\sigma-1})$ und fällt für $t \to \infty$ ohnehin exponentiell ab. Fixiere nun ein $n \in \NN$ und betrachte die Halbebene $\HH_{-n} := \Set {s \in \CC \mid \sigma := \Re(s) > -n}$. Dort stellt das Integral wegen $n+\sigma-1 > -1$ somit eine holomorphe Funktion dar.

Das zweite Integral ist zwar nur für $\sigma > 1$ konvergent, lässt sich aber elementar berechnen zu
\begin{align*}
	I_2(s) 
	&= \int_0^\infty f_n(t) e^{-t} t^{s-2} \opd t \\
	&= \int_0^\infty \left( 1 + \frac t2 + \frac {B_2}{2!}t^2 + \ldots + \frac {B_n}{n!}t^n \right) e^{-t} t^{s-2} \opd t \\
	&= \Gamma(s-1) + \frac 12 \Gamma(s) + \sum_{k=2}^n \frac {B_k}{k!} \Gamma(s+k-1)
	\,.
\end{align*}

Da dies eine auf ganz $\CC$ meromorphe Funktion ist, folgt wegen $n \in \NN$ beliebig, dass $\zeta (s)$ eine in ganz $\CC$ meromorphe Fortsetzung besitzt. Genauer erhalten wir durch Einsetzen der Integrale $I_1(s)$ und $I_2(s)$ sowie durch Ausnutzen der Funktionalgleichung von $\Gamma(s)$, dass insgesamt gilt:
\begin{align*}
	\zeta(s) 
	&= \frac 1{\Gamma(s)} \left( I_1(s) + I_2(s) \right) \\
	&= \frac {I_1(s)}{\Gamma(s)} + \frac 1{\Gamma(s)} \left( \Gamma(s-1) + \frac 12 \Gamma(s) + \sum_{k=2}^n \frac {B_k}{k!} \Gamma(s+k-1) \right) \\
	&= \frac {I_1(s)}{\Gamma(s)} + \frac 1{s-1} + \frac 12 + \sum_{k=2}^n \frac{B_k}{k!} s(s+1)(s+2)\ldots(s+k-2)
\end{align*}
Da $I_1(s)$ für $n \in \NN$ beliebig auf $\HH_{-n}$ holomorph und $\Gamma$ zudem nullstellenfrei ist, zeigt diese Formel zugleich, dass $\zeta(s) - \frac 1{s-1}$ auf ganz $\CC$ holomorph ist. Somit ist auch der zweite Teil der Behauptung bewiesen und nur die konkreten Werte der Zetafunktion verbleiben noch zu zeigen.

Sei dazu $s \in \ZZ$ mit $-n < s \leq 0$, dann ist $\frac {I_1(s)}{\Gamma(s)}$ wegen des Pols von $\Gamma(s)$ an dieser Stelle gleich Null und es folgt
\begin{align*}
	\zeta(s) 
	&= \frac 1{s-1} + \frac 12 + \sum_{k=2}^n \frac{B_k}{k!} s(s+1)(s+2)\ldots(s+k-2) \\
	&= \frac 1{s-1} + \frac 12 + \frac s{12} - \frac {s(s+1)(s+2)}{720} + \frac {s(s+1)(s+2)(s+3)(s+4)}{30240} \mp \ldots
	\,.
\end{align*}
Dies zeigt, dass 
\begin{align*}
	\zeta(0) &= \frac 1{-1} + \frac 12 = - \frac 12 \,, \\
	\zeta(-1) &= \frac 1{-2} + \frac 12 - \frac 1{12} = -\frac 1{12} \,, \\
	\zeta(-2) &= \frac 1{-3} + \frac 12 - \frac 16 = 0 \,, \\
	\zeta(-3) &= \frac 1{-4} + \frac 12 - \frac 14 + \frac 1{120} = \frac 1{120} \,.
\end{align*}

Es ist klar, dass man dieses Verfahren für beliebig große $n$ fortsetzen könnte, um $\zeta(-n)$ zu berechnen. Jedoch kann man auch eine geschlossene Form entwickeln: Aus dem erarbeiteten Ausdruck 
\[
	\zeta(s)
	= \frac 1{s-1} + \frac 12 + \sum_{r=2}^n \frac{B_r}{r!} s(s+1)(s+2)\ldots(s+r-2)
\]
erhält man für $s = -k$ und beliebiges $n > k$ (z.B. $n = k+1$) den Ausdruck
\begin{align*}
	\zeta(-k)
	&= \frac 1{-k-1} + \frac 12 + \sum_{r=2}^n \frac {B_r}{r!} (-k)(-k+1) \ldots (-k+r-2) \\
	&= - \frac 1{k+1} + \frac 12 + \sum_{r=2}^{k+1} (-1)^{r-1} \frac {B_r}{r!} \frac {k!}{(k+1-r)!} \\
	&= - \frac 1{k+1} \sum_{r=0}^{k+1} \binom {k+1}r B_r
	\,.
\end{align*}
Die Bernoulli-Zahlen erfüllen nun für beliebiges $n \in \NN$ die Beziehung
\[
	\sum_{r=0}^n \binom nr B_r = (-1)^n B_n
	\,,
\]
was per Koeffizientenvergleich und mit $k = n-r$ aus
\begin{align*}
	\sum_{n=0}^\infty \left( \sum_{r=0}^n \binom nr B_r \right) \frac {t^n}{n!}
	&= \sum_{r=0}^\infty \sum_{k=0}^\infty \frac {B_r t^{r+k}}{r!k!} \\
	&= \underbrace{\left( \sum_{r=0}^\infty \frac {B_r}{r!} t^r \right)}_{= \frac t{e^t - 1}} \underbrace{\left( \sum_{k=0}^\infty \frac {t^k}{k!} \right)}_{= e^t} \\
	&= \frac {-t}{e^{-t} - 1}
	= \sum_{k=0}^\infty \frac {B_n}{n!} (-t)^n
	= \sum_{k=0}^\infty (-1)^n B_n \frac {t^n}{n!}
\end{align*}
folgt. Damit ergibt sich wie behauptet
\[
	\zeta(-k) = - \frac 1{k+1} \sum_{r=0}^{k+1} \binom {k+1}r B_r = - \frac {B_{k+1}}{k+1}
	\,.
\]

Zuletzt verbleibt noch, die Behauptung für die Werte $\zeta(2n)$ mit $n \in \NN$ zu beweisen. Unter Benutzung des Eulerschen Ergänzungssatzes
\[
	\Gamma(1-s) \Gamma(s) = \frac \pi{\sin \pi s}
\]
erhalten wir
\begin{align*}
	\sum_{n=1}^\infty (-1)^{n-1} 2^{2n-1} \pi^{2n} \frac {B_{2n}}{(2n)!} s^{2n}
	&= - \frac 12 \sum_{n=1}^\infty \frac {B_{2n}}{(2n)!} (2\pi is)^{2n} \\
	&= - \frac 12 \left[ \sum_{n=0}^\infty \frac {B_{2n}}{(2n)!} (2\pi is)^{2n} - 1 \right] \\
	&= - \frac 12 \left[ \sum_{n=0}^\infty \frac {B_n}{n!} (2\pi is)^n - 1 - 2 B_1 \pi i s \right] \\
	&= - \frac 12 \left[ \frac {2\pi is}{e^{2\pi is} - 1} - 1 + \pi is \right] \\
	&= \frac 12 - \frac {\pi is}2 \cdot \left( \frac 2{e^{2\pi is} - 1} + 1 \right) \\
	&= \frac 12 - \frac {\pi is}2 \cdot \left( \frac {2 + e^{2\pi is} - 1}{e^{2\pi is} - 1} \right) \\
	&= \frac 12 - \frac {\pi is}2 \cdot \frac {e^{\pi is} + e^{-\pi is}}{e^{\pi is} - e^{-\pi is}} \\
	&= \frac 12 \left( 1 - \frac {\pi s}{\tan \pi s} \right) \\
	&= \frac s2 \left( \frac 1s - \frac \pi{\tan \pi s} \right) && \Big| \; \text{nachrechnen!} \\
	&= \frac s2 \frac {\opd}{\opd s} \Log \frac {\pi s}{\sin \pi s} && \Big| \; \text{Ergänzungssatz} \\
	&= \frac s2 \frac {\opd}{\opd s} \Log \big( \Gamma(1+s) \Gamma(1-s) \big) && \Big| \; \text{\autoref{LogGamma(s+1)}} \\
	&= \frac s2 \frac {\opd}{\opd s} \left[ \sum_{n=2}^\infty \left( (-1)^n \frac {\zeta(n)}n + \frac {\zeta(n)}n \right) s^n \right] \\
	&= \frac s2 \frac {\opd}{\opd s} \left[ \sum_{n=1}^\infty 2\frac {\zeta(2n)}{2n} s^{2n} \right] \\
	&= \sum_{n=1}^\infty \zeta(2n) s^{2n}
	\,.
\end{align*}
Hieraus folgt über Koeffizientenvergleich wie behauptet für $n \in \NN$ beliebig
\[
	\zeta(2n) = \frac {(-1)^{n-1} 2^{2n-1} B_{2n}}{(2n)!} \pi^{2n} 
	\,.
\]
\end{bewe}

Die Tatsache, dass die Werte von $\zeta(2n)$ und $\zeta(1-2n)$ dieselben Bernoulli-Zahlen enthalten, lässt erahnen, dass es eine Beziehung zwischen beiden Werten geben könnte. Dies in der Tat der Fall: Setzen wir 
\[
	\xi(s) := \pi^{- \frac s2} \cdot \Gamma \! \left( \frac s2 \right) \cdot \zeta(s)
	\,,
\]
so gilt für alle $s \in \CC \setminus \Set {0,1}$ die Gleichheit
\begin{equation}\label{eq:Xi-Beziehung}
	\xi(1-s) = \xi(s)
	\,.
\end{equation}
Diese Relation wurde zuerst von Euler vermutet und schließlich von Riemann bewiesen.

Für $\sigma > 1$ ist die rechte Seite der Gleichung \eqref{eq:Xi-Beziehung} von $0$ verschieden, was sich mit der Darstellung von $\zeta$ als Eulerprodukt leicht einsehen lässt. Es folgt dann aus \eqref{eq:Xi-Beziehung}, dass für $\sigma < 0$ nur die in \autoref{Zeta-Fakten} bereits ermittelten \glqq{}trivialen Nullstellen\grqq{} $s = -2n$ mit $n \in \NN$ als Nullstellen von $\zeta$ infrage kommen. Zudem kann man zeigen, dass $\zeta$ auf den Geraden $\Re(s) = 0$ und $\Re(s) = 1$ keine Nullstellen besitzt. Die einzigen \glqq{}nicht-trivialen\grqq{} Nullstellen von $\zeta$ liegen somit im sogenannten \glqq{}kritischen Streifen\grqq{} $\Set {s \in \CC \mid 0 \leq \sigma := \Re(s) \leq 1}$. Die \glqq{}ersten\grqq{} hiervon haben die Form
\begin{align*}
	\tfrac 12 &\pm 14,134725\ldots i \,, \\
	\tfrac 12 &\pm 21,022040\ldots i \,, \\
	\tfrac 12 &\pm 25,010852\ldots i \,.
\end{align*}
Es ist bekannt, dass $\zeta$ unendlich viele nicht-triviale Nullstellen besitzt. Die bis heute ungelöste \myemph{Riemann-Vermutung} besagt, dass all diese Nullstellen den Realteil $\frac 12$ haben.

\subsection{Heckesche L-Reihen}

Einer Modulform $f = \sum_{n=0}^\infty a(n) q^n \in M_k$ ordnet man die L-Reihe
\[
	L(f,s) := \sum_{n=1}^\infty a(n)n^{-s}
\]
zu. Nach Hecke impliziert das Transformationsverhalten von $f$ \glqq{}gute\grqq{} analytische Eigenschaften für $L(f,s)$, wie zum Beispiel die Existenz einer meromorphen Fortsetzung nach $\CC$ oder die Gültigkeit einer Funktionalgleichung. Der Übergang von $f$ zu $L(f,s)$ erfolgt mittels Mellin-Transformation.

Konvention: Sei im Folgenden $k \geq 4$ stets gerade. 

\begin{defi}
Sei $f = \sum_{n=0}^\infty a(n) q^n \in M_k$. Dann heißt die Reihe 
\[
	L(f,s) := \sum_{n=1}^\infty a(n)n^{-s}
\]
die \myemph{Heckesche L-Reihe} zu $f$. 
\end{defi}

\begin{satz}
Sei $f = \sum_{n=0}^\infty a(n) q^n \in M_k$. Dann gilt:
\begin{enumerate}
\item $a(n) = \mathcal O(n^{k-1})$.
\item Ist $f \in S_k$, so gilt sogar $a(n) = \mathcal O(n^{\frac k2})$.
\end{enumerate}
\end{satz}

\begin{bewe}
Wegen $M_k = \CC E_k \oplus S_k$ und
\[
	E_k \propto G_k = 2 \zeta(k) + \frac {2(2\pi i)^k}{(k-1)!} \sum_{n=1}^\infty \sigma_{k-1}(n) q^n
\]
folgt die Aussage (i) mit (ii) und
\[
	\sigma_{k-1}(n) := \sum_{d|n} d^{k-1} = n^{k-1} \sum_{d|n} \left( \frac dn \right)^{k-1} \leq n^{k-1} \underbrace{\sum_{l=1}^\infty \frac 1{l^{k-1}}}_{< \infty} = \mathcal O(n^{k-1})
	\,.
\]
Wir müssen also nur noch (ii) zeigen. Sei dazu $f \in S_k$. Nach Definition ist
\[
	a(n) = \int_{ci}^{ci + 1} f(z) e^{-2\pi inz} \opd z
\]
mit $c \in \RR_{>0}$ beliebig. Man schreibe $f(z) = y^{- \frac k2} y^{\frac k2} f(z)$. Wie schon früher gezeigt, ist $g(z) := y^{\frac k2} f(z)$ auf ganz $\HH$ beschränkt. Es folgt somit
\[
	a(n) = \int_{ci}^{ci + 1} y^{-\frac k2} g(z) e^{-2\pi inz} \opd z = \int_0^1 c^{- \frac k2} g(t+ic) e^{2\pi nc} e^{-2 int} \opd t
\]
und damit $\abs{a(n)} \leq c^{- \frac k2} e^{2\pi nc} M$, wobei $M > 0$ nicht mehr von $n$ abhängt. Man wähle $c = \frac 1n$ und folgere $\abs{a(n)} = \mathcal O(n^{\frac k2})$. Damit ist alles gezeigt.
\end{bewe}

\begin{koro}
Sei $f \in M_k$. Dann gilt $\sigma_a (L(f,s)) \leq k$. Ist zudem $f \in S_k$, so gilt sogar $\sigma_a (L(f,s)) \leq \frac k2 + 1$. Die Funktion $s \mapsto L(f,s)$ ist in der Halbebene $\Re(s) > \sigma_c (L(f,s))$ holomorph.
\end{koro}

\begin{satz}[Hecke]\label{satz:hecke}
Sei $f = \sum_{n=0}^\infty a(n) q^n \in M_k$. Setzt man für $\Re(s) > k$
\[
	L^*(f,s) := (2\pi)^{-s} \Gamma(s) L(f,s)
	\,,
\]
dann gilt: Die Funktion 
\[
	s \mapsto L^*(f,s) - \frac {a(0)}s + \frac {(-1)^{\frac k2} a(0)}{k-s}
\]
hat eine holomorphe Fortsetzung auf ganz $\CC$, ist beschränkt in jedem Vertikalstreifen $\Set {s \in \CC \mid \nu_1 \leq \Re(s) \leq \nu_2}$ und erfüllt die Funktionalgleichung
\[
	L^*(f,k-s) = (-1)^{\frac k2} L^*(f,s)
	\,.
\]
Ist $f \in S_k$, so ist $a(0) = 0$ und daher sogar $s \mapsto L^*(f,s)$ selbst bereits eine ganze Funktion.
\end{satz}

\begin{bewe}
Nach \autoref{Mellin-Trafo} (Mellin-Transformation) ist
\[
	L(f,s) 
	= \frac 1{\Gamma(s)} \int_0^\infty \sum_{n=1}^\infty a(n) (e^{-x})^n x^{s-1} \opd x
	= \frac 1{\Gamma(s)} \int_0^\infty \Bigg( \underbrace{\sum_{n=0}^\infty a(n) e^{-nx}}_{= f(ix)} - a(0) \Bigg) x^{s-1} \opd x
	\,,
\]
sodass nach Substitution $x \mapsto y := 2\pi x$ für alle $s \in \CC$ mit $\Re(s) > k$ gilt:
\begin{align*}
	L^*(f,s) 
	&= (2\pi)^{-s} \Gamma(s) L(f,s) \\
	&= \int_0^\infty \left( f(iy) - a(0) \right) y^{s-1} \opd y \\
	&= \underbrace{\int_0^1 \left( f(iy) - a(0) \right) y^{s-1} \opd y}_{=: I_1(s)} + \underbrace{\int_1^\infty \left( f(iy) - a(0) \right) y^{s-1} \opd y}_{=: I_2(s)}
	\,.
\end{align*}
Wie man schnell sieht, ist $L^*(f,s)$ für $\Re(s) > k$ holomorph. Darüber hinaus sieht man schnell ein, dass $I_2(s)$ für alle $s \in \CC$ konvergent (also eine ganze Funktion) und außerdem auf jedem Vertikalstreifen beschränkt ist. Es verbleibt somit nur noch das Studium von
\[
	I_1(s) = - a(0) \int_0^1 y^{s-1} \opd y + \int_0^1 f(iy) y^{s-1} \opd y
	\,.
\]
Nun gilt für den ersten Summanden
\[
	\int_0^1 y^{s-1} \opd y = \left[ \frac {y^s}s \right]_0^1 = \frac 1s
\]
und für den zweiten Summanden nach Substitution $y \mapsto \inv y$ mit $\opd (\inv y) = -y^{-2} \opd y$:
\begin{align*}
	\int_0^1 f(iy) y^{s-1} \opd y
	&= \int_\infty^1 f(i \inv y) y^{-s+1} \opd (\inv y) \\
	&= \int_1^\infty f \big( S \circ (iy) \big) y^{-s-1} \opd y \qquad \qquad \qquad \qquad \Big| \; f \in M_k \\
	&= \int_1^\infty (iy)^k f(iy) y^{-s-1} \opd y \\
	&= (-1)^{\frac k2} \int_1^\infty f(iy) y^{k-s-1} \opd y \\
	&= (-1)^{\frac k2} \left( \int_1^\infty \left( f(iy) - a(0) \right) y^{k-s-1} \opd y + a(0) \int_1^\infty y^{k-s-1} \opd y \right) \\
	&= (-1)^{\frac k2} \left( I_2(k-s) - \frac {a(0)}{k-s} \right)
	\,.
\end{align*}

Insgesamt erhalten wir also
\begin{align}
	\label{L*-SummeGanz}
	\notag{}
	L^*(f,s) + \frac {a(0)}s + \frac {(-1)^{\frac k2} a(0)}{k-s}
	&= I_1(s) + I_2(s) + \frac {a(0)}s + \frac {(-1)^{\frac k2} a(0)}{k-s} \\
	&= I_2(s) + (-1)^{\frac k2} I_2(k-s)
	\,.	
\end{align}
Da $I_2(s)$ ganz und auf jedem Vertikalstreifen beschränkt ist, müssen wir nur noch die Funktionalgleichung nachrechnen. Hierzu ersetzen wir in \eqref{L*-SummeGanz} das $s$ durch $k-s$ und beobachten ($k$ gerade):
\[
	L^*(f,k-s) + \frac {a(0)}{k-s} + \frac {(-1)^{\frac k2} a(0)}s
	&\overset{\eqref{L*-SummeGanz}}= I_2(k-s) + (-1)^{\frac k2} I_2(s) \\
	&= (-1)^{\frac k2} \left( (-1)^{\frac k2} I_2(k-s) + I_2(s) \right) \\
	&\overset{\eqref{L*-SummeGanz}}= (-1)^{\frac k2} \left( L^*(f,s) + \frac {a(0)}s + \frac {(-1)^{\frac k2} a(0)}{k-s} \right) \\
	&= (-1)^{\frac k2} L^*(f,s) + \frac {(-1)^{\frac k2} a(0)}s + \frac {a(0)}{k-s}
	\,,
\]
was nach Subtraktion von $\frac {a(0)}{k-s} + \frac {(-1)^{\frac k2} a(0)}s$ genau die Behauptung ergibt.
\end{bewe}


\begin{bsp-list}
	\item Sei $f = 1 - \frac{2k}{B_k} \sum_{m=1}^\infty \sigma_{k-1}(m)q^m$ die normierte Eisensteinreihe in $M_k$.
	Es ist
	\[
	\sum_{m=1}^\infty \frac{\sigma_{k-1}(m)}{m^s}
	= \zeta(s) \zeta(s-k+1)
	\,.
	\]
	Es folgt, dass $(\zeta(s)\zeta(s-k+1))^*$ die vervollsändigte Heckesche L-Reihe $L^*(E_k,s)$ ist und $(-1)^{\frac{k}{2}}$ invariant unter $s \mapsto k-s$.
	
	\item $f = \Delta \in S_{12}$.
	Dann hat $L^*(\Delta, s) = (2\pi)^{-s} \Gamma(s)L(\Delta, s)$ eine holomorphe Fortsetzung auf $\CC$ und ist invariant unter $s \mapsto 12-s$.
	
	\emph{Kuriose Anwendung}:
	Sei $f = \sum_{n \geq 1} a(n)q^n \in S_k$ und seien fast alle $a(n)$ gleich 0.
	Dann ist $f = 0$.
	\begin{bewe}
		Die Funktion $L^*(f, s) = (s\pi)^{-s} \Gamma(s) L(f, s)$ ist holomorph in $\CC$.
		Also folgt $L(f,s) = 0$ für $s = 0$, $-1$, $-2$, \ldots
		
		Nach Voraussetzung
		\[
		&\phantom{\Rla}\qquad f = \sum_{n=1}^N a(n) q^n \\
		&\Ra \qquad L(f,s) = \sum_{n=1}^N a(n) n^{-s} \\
		&\Ra \qquad L(f, -\nu) = \sum_{n=1}^N a(n) n^\nu = 0 \qquad \forall \nu = 0, 1, 2, \ldots \\
		&\Ra \qquad \begin{pmatrix}
		1 & 1 &\ldots & 1 \\
		\vdots & \vdots & \ddots & \vdots \\
		1^{N-1} & 2^{N-1} & \ldots & & N^{N-1}
		\end{pmatrix}
		\begin{pmatrix}
		a(1) \\
		\vdots \\
		a(N)
		\end{pmatrix}
		= 0 \\
		&\Ra a = 0 \\
		&\Ra f \equiv 0
		\,.
		\]
	\end{bewe}
\end{bsp-list}