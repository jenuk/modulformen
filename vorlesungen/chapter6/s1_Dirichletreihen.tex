\section{Dirichletreihen}

\emph{Ziel}: Angabe elementarer Eigenschaften von \myquote{Dirichletreihen}, welche eine besondere Stellung in der analytischen Theorie der Zahlen einnehmen.

\begin{defi}\label{def:dirichletreihe}
	Formell ist eine \myemph{Dirichletreihe} eine Reihe der Form
	\[
		\sum_{n=1}^{\infty} a(n) e^{-\lambda_n s}
		\,,
	\]
	wobei die $\lambda_n$ reelle Zahlen sind mit $\lambda_1 < \lambda_2 < \lambda_3 < \ldots \xto{n\to\infty} \infty$ und $s= \sigma +it$ eine komplexe Zahl ist.
\end{defi}

\begin{beme-list}
	\item $\lambda_n = n$ ist naheliegend führt aber mit $z = e^{-s}$ zur Theorie der Potenzreihen, die wir schon ausgiebig in der Funktionentheorie I studiert haben.
	
	\item $\lambda_n = \log (n)$: Mit dieser Wahl lässt sich die obere Reihe \myquote{schöner} schreiben als
	\begin{equation}\label{eq:gewDirichtletreihe}
		\sum_{n=1}^\infty a(n) n^{-s}
		\,.
	\end{equation}
	Das ist der für die Zahlentheorie relevante Fall.
	Eine Reihe der Gestalt \eqref{eq:gewDirichtletreihe} heißt \myemph[Dirichletreihe!Gewöhnliche Dirichletreihe]{gewöhnliche Dirichletreihe}.
	
	\item Im Gegensatz zu Potenzreihen weisen Dirichletreihen ein anderes Konvergenzverhalten auf als das \myquote{auf Kreisscheiben} auf.
	Während also für eine Potenzreihe $f(z) = \sum_{n=0}^\infty a(n) z^n$ stets ein $0 \leq R \leq \infty$ existiert, so dass $f$ für $z \in \CC$ mit $\abs z < R$ konvergiert und für $\abs z > R$ divergiert, \myquote{konvergieren Dirichletreihen auf Halbebenen statt Kreisscheiben}.
	Dies präzisiert der nächste Satz
\end{beme-list}

\begin{satz}
	Es sei $F(s) := \sum_{n=1}^\infty a(n) e^{-\lambda_n s}$ eine Dirichletreihe wie in \autoref{def:dirichletreihe} definiert.
	Ist diese für ein $s = s_0$ konvergent, so konvergiert sie auch für alle $Re(s) > \sigma_0\ (= \Re(s_0))$ und gleichmäßig auf Kompakten Mengen.
	
	Somit existiert eine reelle Zahl $\sigma_0$, so dass die Reihe für alle $s$ mit $\Re s > \sigma_0$ konvergiert und für alle $\Re(s) < \sigma_0$ divergiert (falls überall konvergent, setze $\sigma_0 = -\infty$, falls überall divergent setze $\sigma_0 = \infty$).
	
	Die in dem Gebiet $\Set{s \in \CC \mid \Re(s) > \sigma_0}$ durch $F(s)$ definierte Funktion ist dort holomorph, die Ableitungen sind gegeben durch
	\[
		F^{(k)}(s) = (-1)^k \sum_{n=1}^\infty \lambda_n^k a(n) e^{-\lambda_n s}\,,
	\]
	wobei die rechts stehende Dirichletreihe auch für $\sigma > \sigma_0$ konvergiert.
	
	Die Zahl $\sigma_0$ heißt \myemph{Konvergenzabszisse} der Dirichletreihe $F(s)$.
\end{satz}

\begin{bewe}
	Es genügt die gleichmäßige Konvergenz im gesamten Bereich zu zeigen, da damit die Existenz eines $\sigma_0$ folgt und die Holomorphieaussagen aus der gleichmäßigen Konvergenz über den Satz von Weierstraß ersichtlich sind.
	
	Wir zeigen, dass die Reihe in jedem Gebiet
	\[
		\abs{\Arg (s - s_0)} \leq \frac{\pi}{2} - \epsilon < \frac{\pi}{2}
	\]
	gleichmäßig konvergiert.
	Das ist stärker als die Aussage des Satzes, da jede kompakte Menge $K \subset \Set{s \in \CC \mid \Re(s) > \sigma_0}$ in einem solchen Gebiet liegt.
\end{bewe}