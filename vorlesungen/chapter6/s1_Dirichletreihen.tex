\section{Dirichletreihen}

\emph{Ziel}: Angabe elementarer Eigenschaften von \myquote{Dirichletreihen}, welche eine besondere Stellung in der analytischen Theorie der Zahlen einnehmen.

\begin{defi}\label{def:dirichletreihe}
	Formell ist eine \myemph{Dirichletreihe} eine Reihe der Form
	\[
		\sum_{n=1}^{\infty} a(n) e^{-\lambda_n s}
		\,,
	\]
	wobei die $\lambda_n$ reelle Zahlen sind mit $\lambda_1 < \lambda_2 < \lambda_3 < \ldots \xto{n\to\infty} \infty$ und $s= \sigma +it$ eine komplexe Zahl ist.
\end{defi}

\begin{beme-list}
	\item $\lambda_n = n$ ist naheliegend führt aber mit $z = e^{-s}$ zur Theorie der Potenzreihen, die wir schon ausgiebig in der Funktionentheorie I studiert haben.
	
	\item $\lambda_n = \log (n)$: Mit dieser Wahl lässt sich die obere Reihe \myquote{schöner} schreiben als
	\begin{equation}\label{eq:gewDirichtletreihe}
		\sum_{n=1}^\infty a(n) n^{-s}
		\,.
	\end{equation}
	Das ist der für die Zahlentheorie relevante Fall.
	Eine Reihe der Gestalt \eqref{eq:gewDirichtletreihe} heißt \myemph[Dirichletreihe!Gewöhnliche Dirichletreihe]{gewöhnliche Dirichletreihe}.
	
	\item Im Gegensatz zu Potenzreihen weisen Dirichletreihen ein anderes Konvergenzverhalten auf als das \myquote{auf Kreisscheiben} auf.
	Während also für eine Potenzreihe $f(z) = \sum_{n=0}^\infty a(n) z^n$ stets ein $0 \leq R \leq \infty$ existiert, so dass $f$ für $z \in \CC$ mit $\abs z < R$ konvergiert und für $\abs z > R$ divergiert, \myquote{konvergieren Dirichletreihen auf Halbebenen statt Kreisscheiben}.
	Dies präzisiert der nächste Satz
\end{beme-list}

\begin{satz}\label{satz:konvergenzDirichletreihen}
	Es sei $F(s) := \sum_{n=1}^\infty a(n) e^{-\lambda_n s}$ eine Dirichletreihe wie in \autoref{def:dirichletreihe} definiert.
	Ist diese für ein $s = s_0$ konvergent, so konvergiert sie auch für alle $Re(s) > \sigma_0\ (= \Re(s_0))$ und gleichmäßig auf Kompakten Mengen.
	
	Somit existiert eine reelle Zahl $\sigma_0$, so dass die Reihe für alle $s$ mit $\Re s > \sigma_0$ konvergiert und für alle $\Re(s) < \sigma_0$ divergiert (falls überall konvergent, setze $\sigma_0 = -\infty$, falls überall divergent setze $\sigma_0 = \infty$).
	
	Die in dem Gebiet $\Set{s \in \CC \mid \Re(s) > \sigma_0}$ durch $F(s)$ definierte Funktion ist dort holomorph, die Ableitungen sind gegeben durch
	\[
		F^{(k)}(s) = (-1)^k \sum_{n=1}^\infty \lambda_n^k a(n) e^{-\lambda_n s}\,,
	\]
	wobei die rechts stehende Dirichletreihe auch für $\sigma > \sigma_0$ konvergiert.
	
	Die Zahl $\sigma_0$ heißt \myemph{Konvergenzabszisse} der Dirichletreihe $F(s)$.
\end{satz}

\begin{bewe}
	
	Es genügt die gleichmäßige Konvergenz im gesamten Bereich zu zeigen, da damit die Existenz eines $\sigma_0$ folgt und die Holomorphieaussagen aus der gleichmäßigen Konvergenz über den Satz von Weierstraß ersichtlich sind.
	
	Wir zeigen, dass die Reihe in jedem Gebiet (siehe \autoref{fig:winkelbereich})
	\begin{equation}\label{eq:winkelbereich}
		\abs{\Arg (s - s_0)} \leq \frac{\pi}{2} - \epsilon < \frac{\pi}{2}
	\end{equation}
	gleichmäßig konvergiert.
	Das ist stärker als die Aussage des Satzes, da jede kompakte Menge $K \subset \Set{s \in \CC \mid \Re(s) > \sigma_0}$ in einem solchen Gebiet liegt.
	
	\begin{figure}
		\begin{center}
			\includestandalone[scale=.8]{images/chapter6/winkelbereich}
			\caption{Das Gebiet aus \eqref{eq:winkelbereich}}
			\label{fig:winkelbereich}
		\end{center}
	\end{figure}

	Wir führen die Bezeichnungen
	\[
	A(N) = \sum_{n=1}^N a(n)\,, \qquad A(M,N) = \sum_{n=M}^N a(n)\,, \qquad A(M,M-1) = 0
	\]
	ein, die in diesem Paragraphen mehrmals benutzt werden. Ohne Einschränkung können wir $s_0 = 0$ voraussetzen (indem wir $s$ durch $s+s_0$ und $a(n)$ durch $a(n)e^{-\lambda_ns_0}$ ersetzen).
	Dann ist $\sum_{n=1}^\infty a(n)$ konvergent und es gibt zu vorgegebenen $\epsilon > 0$ ein $N_0$, so dass $\abs{A(M,N)} \leq \epsilon$ für alle $N_0 \leq M < N$.
	Dann gilt für $N > M \geq N_0$:\footnote{dieses Verfahren wir auch als abelsche Summation bezeichnet}
	\[
	\sum_{n=M}^N a(n)e^{-\lambda_ns}
	&= \sum_{n=M}^N \bigl(A(M,n) - A(M,n-1)\bigr)e^{-\lambda_ns} \\
	&= A(M,M)e^{-\lambda_Ms} - A(M,M) e^{-\lambda_{M+1}s} + A(M, M+1)e^{-\lambda_{M+1}s} \\ &\qquad + \ldots + A(M,N-1) e^{-\lambda_{N-1}s} - A(M,N-1)e^{-\lambda_Ns} + A(M,N) e^{-\lambda_ns} \\
	&= \sum_{n=M}^{N-1} A(M,n) \bigl(e^{-\lambda_ns} - e^{-\lambda_{n+1}s}\bigr) + A(M,N)e^{-\lambda_Ns}\,.
	\]
	
	Es ist
	\[
	\abs{e^{-\lambda s} - e^{-\lambda_{n+1}s}}
	&= \abs{s \int_{\lambda_n}^{\lambda_{n+1}} e^{-sn} \opd n} \\
	&\leq \abs s \int_{\lambda_n}^{\lambda_{n+1}} \abs{e^{-sn}} \opd n \\
	&= \frac{\abs s}{\sigma} \bigl(e^{-\lambda \sigma} - e^{-\lambda_{n+1}\sigma}\bigr) \qquad (\sigma = \Re s) \,.
	\]
	
	Die Größe $\frac {\abs s}\sigma$ ist in den Bereichen $\abs{\Arg(s)} \leq \frac{\pi}{2} - \delta$ durch eine Konstante $C_\delta > 0$ beschränkt.
	Somit ist für $\sigma > 0$:
	\[
	\abs{ \sum_{n=M}^N a(n) e^{-\lambda_ns} }
	& \leq \sum_{n=M}^{N-1} \abs{A(M,n)} \cdot \abs*{e^{-\lambda_n s} - e^{-\lambda_{n+1}s}} + \abs{A(M,N)} \cdot \abs*{e^{-\lambda_Ns}} \\
	&\leq C_\delta \epsilon \sum_{n=M}^{N-1} \bigl(e^{-\lambda_n\sigma} - e^{-\lambda_{n+1}\sigma}\bigr) + \epsilon e^{-\lambda_N \sigma} \\
	&\leq C_\delta \epsilon e^{-\lambda_M\sigma} + \epsilon e^{-\lambda_N \sigma} \\
	& \leq (C_\delta + 1)e^{-\lambda_{N_0}\sigma} \epsilon
	\,,
	\]
	womit die gleichmäßige Konvergenz in diesem Bereich folgt.

\end{bewe}

Analog zur bedingten Konvergenzabszissen kann man (bei gewöhnlichen Dirichletreihen) die absoluten Konvergenzabszisse definieren als die bedingten Abszisse von
\[
\sum_{n=1}^\infty \abs{a(n)} n^{-s}
\,.
\]
Wir bezeichnen $\sigma_a$ als die absoluten und $\sigma_c$ als die bedingten Konvergenzabszisse.

\begin{satz}
	Ist $F$ eine gewöhnliche Dirichletreihe mit $\sigma_c \in \RR$, so gilt
	\[
		\sigma_c \leq \sigma_a \leq \sigma_c + 1
	\]
\end{satz}

\begin{bewe}
	Übung!
\end{bewe}

\begin{beme}
	Wie in der Theorie der Potenzreihen gibt es auch in der Theorie der Dirichletreihen eine Methode zur Berechnung der Konvergenzsabszisse.
	Ist $\sum_{n=1}^\infty a(n)$ divergent, so folgt für $F(s) = \sum_{n=1}^\infty a(n) e^{-\lambda_n s}$ die Formel
	\[
		\sigma_c = \limsup_{n\to\infty} \frac{\log{A(n)}}{\lambda_n}
		\,.
	\]
	
	Es gibt einen noch viel wichtigeren Unterschied zwischen Dirichletreihen und den uns geläufigen Potenzreihen.
	Bei den Potenzreihen kann man den Konvergenzradius nicht nur in Abhängigkeit der Koeffizienten, sondern auch durch das Verhalten der durch die Reihe dargestellte lokal analytische Funktion bestimmen, nämlich als kleinster Absolutbetrag der singulären Punkte (ohne Einschränkung ist der Entwicklungspunkt $z_0 = 0$): stellt die Reihe $\sum_{n=0}^\infty a(n)z^n$ eine Funktion dar, die such auf eine Kreisscheibe $\abs z < r$ holomorph fortsetzen lässt, so ist sie in diesem Bereich auch absolut konvergent.
	
	Für Dirichletreihen stimmt das nicht.
	Beispielsweise hat die Funktion
	\[
		F(s) = 1 - \frac 1{2^s} + \frac 1{3^s} - \frac 1{4^s} + \ldots
	\]
	eine Forsetzung zu einer ganzen Funktion (wie wir noch sehen werden!), aber die Reihe konvergiert nur für Werte $s \in \CC$ mit $\Re s > 0$.
	
	Nur in Spezialfällen können wir auf die Existenz von singulären Punkten auf dem Rand der Konvergenzhalbebene schließen.
\end{beme}

\begin{satz}[Landau]
	Sei $\sum_{n=1}^\infty a(n)n^{-s}$ eine gewöhnliche Dirichletreihe mit nicht-negativen Koeffizienten und Konvergenzabsziss $\sigma_c$.
	Dann hat die durch $F(s) = \sum_{n=1}^\infty a(n)n^{-s}$ definierte Funktion in $s=\sigma_c$ einen singulären Punkt.
\end{satz}
\begin{bewe}
	Ohne Einschränkung sei $\sigma_c = 0$.
	Nehmen wir an, die Funktion $F(s)$ wäre in einer Umgebung $U_\epsilon(0)$ holomorph fortsetzbar. Dann würde sie um $s=1$ eine Taylorentwicklung haben mit Konvergenzradius $R > 1$, da kein Punkt in $\partial U_1(1)$ singulär ist.
	Also wäre für geeignetes $\delta > 0$ die Reihe Reihe $\sum_{k=1}^\infty \frac{(-1-\delta)^{k}}{k!} F^{(k)}(1)$ konvergent und gleich $F(-\delta)$.
	Nach \autoref{satz:konvergenzDirichletreihen} ist
	\[
	\sum_{k=0}^\infty \frac{(-1-\delta)^k}{k!} F^{(k)}(1)
	&\stackrel{\phantom{a(n) \geq 0}}{=} \sum_{k=0}^\infty \frac{(1+\delta)^k}{k!} \sum_{n=1}^\infty \frac{(\log n)^k a(n)
	}{n} \\
	&\stackrel{a(n) \geq 0}{=} \sum_{n=1}^\infty \frac{a(n)}{n}  \sum_{k=0}^\infty \frac{(\log n)^k (1+\delta)^k}{k!} \\
	&\stackrel{\phantom{a(n) \geq 0}}{=} \sum_{n=1}^\infty \frac{a(n)}{n} e^{(1+\delta)\log(n)}
	= \sum_{n=1}^\infty a(n)n^\delta
	\,.
	\]
	Damit gilt $\sigma_c \leq -\delta < \sigma_c$. \blitz
\end{bewe}

\begin{satz}
	Seien $\sum_{n=1}^\infty a(n)e^{-\lambda_ns}$ und $\sum_{n=1}^\infty b(n)n^{-\lambda_ns}$ zwei Dirichletreihen, die in einem Gebiet $U \subset \CC$ konvergieren und dort die selbe analytische Funktion darstellen.
	Dann ist $a(n) = b(n)$ für alle $n\in\NN$.
\end{satz}
\begin{bewe}
	Nehmen wir an, dies sei nicht der Fall.
	Sei $m$ der kleinste Index mit $a(m) \not= b(m)$.
	Dann gilt für $\sigma$ groß genug (Identitätssatz)
	\[
	0 &= e^{\lambda_m \sigma} \biggl( \sum_{n=m}^\infty a(n) e^{-\lambda_n \sigma} - \sum_{n=m}^\infty b(n) e^{-\lambda_n \sigma} \biggl) \\
	&= a(m) - b(m) + \sum_{n=m+1}^\infty \bigl(a(n)-b(n)\bigr) e^{-(\lambda_n-\lambda_m)\sigma}
	\]
	In der Tat hat jedes Glied in der Reihe wegen $\lambda_n > \lambda_m$ den Limes 0 und die gleichmäßige Konvergenz impliziert, dass die Reihe für $\sigma\to\infty$ gegen 0 strebt, was $a(m) \not= b(m)$ widerspricht.
\end{bewe}