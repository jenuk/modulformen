\section{Spezielle Werte von L-Funktionen}

\begin{erin}
	Es sei $f \in S_k$ eine normalisierte Hecke-Eigenform. Dann ist
	\[
		L^*(f, s) = (2\pi)^{-s} \Gamma(s) L(f, s)
	\]
	eine ganze Funktion und $(-1)^{k/2}$-invariant unter $s \mapsto k - s$. Ferner gilt 
	\[
		L(f,s) = \prod_{p \in \PP} \bigl(1 - a(p)p^{-s} + p^{k-1-2s}\bigr)^{-1}
		\,.
	\]
\end{erin}

\begin{defi}
	Eine ganze Zahl $s_0$ heißt \myemph{kritisch} bezüglich $L(f,s)$ ($f$ weiterhin Hecke-Eigenform), falls $1 \leq s_0 \leq k-1$. 
	
	Idee: $s_0$ ist genau dann kritisch, falls es weder Pol von $\Gamma(s)$ noch von $\Gamma(k-s)$ ist.
\end{defi}

Es gibt folgende Philosophie von Deligne: Sei 
\[
	L(s) = \sum_{n=1}^\infty a(n)n^{-s}
\]
eine \glqq{}motivierte\grqq{} Dirichletreihe (d.\,h. $L$ kommt von einem natürlichen mathematischen Objekt wie einer Varietät, einer Modulform, einer Galois-Darstellung, einem Zahlkörper, \ldots). Man setzt voraus, dass $L$ ein Euler-Produkt besitzt und sich zu einer ganzen Funktion fortsetzen lässt (oder zumindest einer meromorphen Funktion mit endlich vielen Polstellen) und ihre Vervollständigung $L^*(s) = \gamma(s)L(s)$ (mit einem Gamma-Faktor $\gamma(s)$, z.\,B. $(2\pi)^{-s} \Gamma(s)$ für Modulformen) eine Funktionalgleichung $L^*(k - s) = \epsilon L^*(s)$ erfüllt mit $k > 0$ und $\epsilon \in \{\pm1\}$. Ist $s_0$ kritisch, so ist es weder Pol von $\gamma(s)$ noch von $\gamma(k-s)$. Dann soll eine geschlossene Formel
\[
	L^*(s_0) = B(s_0) \Omega
\]
gelten mit $B(s_0) \in \mathds{A}$ wobei $\mathds{A} \subset \CC$ der algebraische Abschluss von $\QQ$ ist und $\Omega$ \glqq{}im Wesentlichen unabhängig von $s_0$\grqq{} (\glqq Periode\grqq).

\begin{bsp}
	Für $\zeta(s)$ sind genau die positiven geraden und die negativen ungeraden Zahlen kritisch.
\end{bsp}

\begin{satz}[Eichler-Shimura]
	Sei $f \in S_k$ eine Hecke-Eigenform.
	Dann existieren $\omega_+, \omega_- \in \RR_+$ derart, dass die Werte $L^*(f, s_0)/\omega_+$ für $s_0$ kritisch und ungerade beziehungsweise $L^*(f, s_0)/\omega_-$ für $s_0$ kritisch und gerade, algebraisch sind. Man kann $f$ so normalisieren, dass
	\[
		\omega_+ \omega_- = \scalarprd ff
		\,.
	\]
\end{satz}

\begin{bsp}
	Ist $f = \Delta \in S_{12}$, so gilt
	\begin{align*}
		L^*(\Delta, 1) &= L^*(\Delta, 11) = \frac{192}{691} \omega_+ \,, \\
		L^*(\Delta, 3) &= L^*(\Delta, \phantom{11}\makebox[0pt][r]{9}) = \frac{16}{135} \omega_+ \,, \\
		L^*(\Delta, 5) &= L^*(\Delta, \phantom{11}\makebox[0pt][r]{7}) = \frac{8}{105} \omega_+ \,.
	\end{align*}
	Ähnliche Formeln erhält man auch für $\omega_-$.
\end{bsp}