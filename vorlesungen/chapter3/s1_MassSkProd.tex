\section{Invariantes Maß und Skalarprodukt}

\emph{Ziel:} Definition eines \glqq{}natürlichen\grqq{} Skalarprodukts auf $S_k$. Hierzu benötigt man zunächst ein $\Gamma(1)$-invariantes Maß auf $\RR^2$.

\begin{defi}
Für $z = x + iy \in \HH$ setze man
\[
\opd \omega(z) := \frac {\opd x \opd y}{y^2}
\]
Dann gilt für alle $M \in \SL_2(\RR)$
\[
\opd \omega(M \circ z) = \opd \omega(z) 
\]
d.h. $\opd \omega(z)$ ist $\SL_2(\RR)$-invariant.
\end{defi}

\begin{bewe}
Es gilt $\opd \omega(z) = \frac {i}{2y^2} \opd z \conj{\opd z}$, denn
\begin{align*}
\opd z \conj{\opd z} &= (\opd x + i \opd y)(\opd x - i \opd y)\\
&= \opd x \opd x - i \opd x \opd y + i \opd y \opd x + \opd y \opd y\\
&= 0 - i \opd x \opd y - i \opd x \opd y + 0\\
&= -2i \opd x \opd y\,.
\end{align*}
Sei nun $M = \abcd \in \SL_2(\RR)$, dann gilt unter Verwendung von
\[
\frac {\opd (M \circ z)}{\opd z} = \frac {\opd \frac{az + b}{cz + d}}{\opd z} = \frac {a(cz+d) - (az+b)c}{(cz+d)^2} = \frac{1}{(cz+d)^2}
\]
die Behauptung nach
\begin{align*}
\opd \omega(M \circ z) &= \frac {i}{2 (\Im \left(M \circ z)^2\right)} \opd (M \circ z) \conj{\opd (M \circ z)}\\
&= \frac {i}{2 \frac{y^2}{\abs{cz+d}^4}} \frac{\opd z}{(cz+d)^2} \conj{\frac {\opd z} {(cz+d)^2}}\\
&= \frac {i}{2 \frac{y^2}{\abs{cz+d}^4}} \frac{\opd z}{(cz+d)^2} \frac {\conj{\opd z}} {\conj{(cz+d)^2}}\\
&= \frac{i \abs{cz+d}^4}{2 y^2} \cdot \frac {1}{\abs{cz+d}^4} \cdot \opd z \conj{\opd z}\\
&= \frac {i}{2y^2} \opd z \conj{\opd z}\\
&= \opd \omega(z)\,.
\end{align*}
\end{bewe}

\emph{Ansatz:} $f, g \in S_k$, setze:
\[
<f,g> := \int_{\closure{\mathcal F}} y^k f(z) \conj{g(z)} \opd \omega
\]
wobei $\mathcal F$ ein Fundamentalbereich ist.
