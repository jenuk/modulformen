\section{Anwendung: Eine Charakterisierung der Eisensteinreihen}

\begin{satz}\label{CharEk}
	Sei $k \in 2\ZZ$, $k \geq 4$. Sei $C_k := \Set{f \in M_k \mid \scalarprd{f}{g} = 0 \ \forall g \in S_k}$ ein Unterraum von $M_k$.
	Dann gilt $C_k = \CC E_k$.
\end{satz}


\begin{bewe}

	Der Beweis erfolgt in mehreren Schritten:
	\vspace{-2em}

\begin{lemm}
	Es gilt $M_k = C_k \oplus S_k$ (und bekanntermaßen $M_k = \CC E_k \oplus S_k$).
\end{lemm}

\begin{bewe}
	Sei $f \in C_k \cap S_k$. Dann $\scalarprd ff = 0$, also $f \equiv 0$.
	
	Sei $f \in M_k$.
	Die Abbildung $S_k \ra \CC$, $g \mapsto \scalarprd gf$ ist ein lineares Funktional.
	Nach dem Satz von Riesz existiert daher ein eindeutig bestimmtes Element $g_0 \in S_k$, so dass $\scalarprd gf = \scalarprd g{g_0}$ für alle $g \in S_k$.
	Daher $\scalarprd g{f-g_0} = 0$ für alle $g\in S_k$, d.\,h. $\scalarprd {f-g_0}g = 0$ für alle $g \in S_k$.
	Also ist $f-g_0 \in C_k$ nach Definition und somit
	\[
		f = \underbrace{(f-g_0)}_{\in C_k} + \underbrace{g_0}_{\in S_k} .
	\]
\end{bewe}

Es folgt
\[
	\dim C_k = \dim M_k - \dim S_k = 1 + \dim S_k - \dim S_k = 1\,.
\]
Daher genügt es zu zeigen, dass $E_k \in C_k$, d.\,h. $\scalarprd{E_k}g = 0$ für alle $g \in S_k$.

\begin{lemm}\label{Ek_per_Gamma(1)infty}
	Es gilt
	\[
		E_k(z) = \frac{1}{2} \sum_{M \in \linksmodulo{\Gamma(1)_\infty}{\Gamma(1)}} (1|_k M)(z)\,,
	\]
	wobei $\Gamma(1)_\infty = \Set{\mymat 1n01 \mid n \in \ZZ}$ und $(1|_k M) = (cz+d)^{-k}$.
\end{lemm}

\begin{bewe}
	Es gilt $E_k = \frac{1}{2\zeta(k)} G_k$ mit $G_k = \sumprime_{m,n} \frac{1}{(mz+n)^k}$.
	Ist $(m,n) \in \ZZ^2\setminus\Set{(0,0)}$, so schreibe $(m,n) = \lambda (c,d)$ wobei $\lambda = \ggt(m,n) \in \NN$ und $(c,d) \in \ZZ^2$ mit $\ggt(c,d) = 1$.
	Also
	\[
		G_k(z)
		= \underbrace{\zeta(k)}_{= \sum\limits_{\lambda=1}^\infty \frac{1}{\lambda^k}}\ \cdot\ \sum_{\substack{(c,d) \in \ZZ^2 \\ \ggt(c,d) = 1}} (cz+d)^{-k}
	\]
	und damit
	\[
		E_k = \frac{1}{2} \sum_{\substack{(c,d) \in \ZZ^2 \\ \ggt(c,d) = 1}} (cz+d)^{-k}
		\,.
	\]
	Daher genügt es zu zeigen, dass
	\[
		\sum_{M \in \linksmodulo{\Gamma(1)_\infty}{\Gamma(1)}} (1|_k M)(z)
		= \sum_{\substack{(c,d) \in \ZZ^2 \\ \ggt(c,d) = 1}} (cz+d)^{-k}
		\,.
	\]
	
	Jeder Summand links hat die Gestalt $(cz+d)^{-k}$ mit $\ggt(c,d) = 1$.
	Umgekehrt ist zu zeigen:
	Jedes $(c,d) \in \ZZ^2$ mit $\ggt(c,d) = 1$ lässt sich vervollständigen zu $M = \mymat abcd \in \Gamma(1)$ eindeutig bis auf Links-Multiplikation eines Elementes in $\Gamma(1)_\infty$.
	Es gilt:
	
	\begin{itemize}
		\item $\ggt(c,d) = 1$, also exisitieren $a$, $b \in \ZZ$ mit $ad - bc = 1$, denn $\ZZ$ ist ein Hauptidealring.
		Also $M = \mymat abcd \in \Gamma(1)$.
		
		\item Seien $\mymat abcd$, $\mymat {a'}{b'}cd \in \Gamma(1)$. Dann $ad-bc = 1 = a'd-b'c$.
		Also $(a-a') d = (b-b')c$, also $\frac{c}{d} = \frac{a-a'}{b-b'}$.
		Da $\ggt(c,d) = 1$, folgt $a-a' = nc$ und $b-b' = nd$ mit $n \in \ZZ$.
		Das heißt $\mymat 1n01 \mymat{a'}{b'}cd = \mymat {a'+nc}{b'+nd}c{d} = \mymat abcd$.
	\end{itemize}
\end{bewe}

	Man kann dies schreiben als
	\[
		E_k(z)
		= \sum_{M \in \linksmodulo{\Gamma(1)_\infty'}{\Gamma(1)'}} (1|_k M)(z)\,,
	\]
	wobei $\Gamma(1)' = \modulo{\Gamma(1)}{\Set{\pm E}}$ und $\Gamma(1)'_\infty = \modulo{\Set{\mymat{\pm1}n0{\pm1} \mid n \in \ZZ}}{\Set{\pm E}}$.

	Sei nun $g \in S_k$, dann ist zu zeigen, dass $\scalarprd{E_k}g = 0$. 
	Nach Definition ist
	\begin{align*}
		\scalarprd{E_k}g &= \int_{\closure \F} E_k(z) \conj{g(z)} y^k \opd w
		= \int_{\closure \F} \biggl(\sum_{M \in \linksmodulo{\Gamma(1)_\infty'}{\Gamma(1)'}} (1|_kM)(z) \conj{g(z)} (\Im z)^k \biggr) \opd w(z)
		\,.
	\end{align*}
	
	Beachte nun
	\begin{align*}
		(1|_k M)(z) \conj{g(z)}(\Im z)^k
		&= (1|_kM)(z) \conj{g(M \circ z) (1|_kM)(z)} (\Im M \circ z)^k \abs{(1|_kM)(z)}^{-2}\\
		&= \conj{g(M \circ z)} (\Im M \circ z)^k
		\,.
	\end{align*}
	Sei $\tilde\F := \bigcup_{M \in \linksmodulo{\Gamma(1)_\infty'}{\Gamma(1)'}} M \circ \closure\F$ ein Fundamentalbereich für die Untergruppe $\Gamma(1)'_\infty \subset \Gamma(1)'$, welche durch $z \mapsto z+n$ operiert.
	Dann folgt durch Substitution und Vertauschung (aufgrund der absoluten Konvergenz gerechtfertigt)
	\begin{align*}
		\scalarprd{E_k}g
		&= \sum_{M \in \linksmodulo{\Gamma(1)_\infty'}{\Gamma(1)'}} \int_{M \circ \closure \F} \conj{g(z)} y^{k-2} \opd x \opd y % Substituiere, Vertauschung wegen absoluter Konvergenz ok
		= \int_{\tilde\F} \conj{g(z)} y^{k-2} \opd x \opd y
		% = \int_{\underbrace{\bigcup_{M \in \linksmodulo{\Gamma(1)_\infty'}{\Gamma(1)'}} M \circ \closure\F}_{\text{Fundamentalbereich für die Untergruppe $\Gamma(1)'_\infty \subset \Gamma(1)'$, welche durch $z \mapsto z+n$ operiert}}} \conj{g(z)} y^{k-2} \opd x \opd y
		\,.
	\end{align*}
	
	Man zeigt formal: Das Integral ist unabhängig von der Auswahl des Fundamentalbereichs $\mathcal G$.
	Man wähle für $\mathcal G$ einen Steifen der Breite 1, etwa $\mathcal G = \Set{z = x + iy \in \HH \mid \abs x < \frac{1}{2}}$.
	Dann ist
	\[
		\scalarprd{E_k}g = \int_0^\infty \int_{-\frac{1}{2}}^{\frac{1}{2}} \conj{g(z)} y^{k-2} \opd x \opd y
		\,.
	\]
	Sei nun $g(z) = \sum_{n\geq 1} a(n) e^{2\pi inz} = \sum_{n \geq 1} a(n) e^{2\pi inx} e^{-2\pi ny} \in S_k$ und daher $\conj{g(z)} = \sum_{n \geq 1} \conj{a(n)} e^{-2\pi ny} e^{-2\pi inx}$, dann folgt
	
	\begin{align*}
		\scalarprd {E_k}g
		= \int_0^\infty \int_{-\frac{1}{2}}^{\frac{1}{2}} \Bigl( \sum_{n\geq 1} \conj{a(n)} e^{-2\pi ny} y^{k-2} e^{-2\pi inx}\Bigr) \opd x \opd y
		= 0
	\end{align*}
	
	durch Vertauschen von Summe und innerem Integral (erlaubt, denn $g \in S_k$) unter Beachtung von $\int_{-\frac{1}{2}}^{\frac{1}{2}} e^{-2\pi inx} \opd x = 0$ für $n \not= 0$.
\end{bewe}