\chapter{Exkurs: Produktdarstellung der Diskriminantenfunktion}

\begin{satz}
Für die Diskriminantenfunktion $\Delta$ gilt die Produktentwicklung
\[
\Delta (z) := \frac{1}{1728} \bigl( E_4^3 (z) - E_6^2 (z) \bigr) = q \prod_{m=1}^\infty \bigl( 1 - q^m \bigr)^{24}
\,,
\]
wobei wie üblich $q := \exp (2\pi iz)$ ist.
\end{satz}

\begin{bewe}
Ein erster Beweis dieser Identität stammt von Jacobi; ein weiterer Beweis, der allein mit elementaren Mitteln auskommt, wird auf Übungsblatt 4 geführt werden. Im Folgenden soll ein vergleichsweise einfacher Beweis von Professor Kohnen selbst vorgestellt werden, der unter anderem auf die Hecke-Operatoren zurückgreift.

1. Schritt: Wir leiten eine zur Behauptung äquivalente Aussage her. Nehme also an, die Produktdarstellung gelte, dann können wir die logarithmische Ableitung bilden (verifiziere durch Nachrechnen unter Beachtung der Produktregel):
\begin{align*}
\frac {\Delta'}{\Delta} &= 2\pi i - 2\pi i \cdot 24 \sum_{m=1}^\infty m \frac {q^m}{1-q^m} \\
&= 2\pi i \Bigl( 1 - 24 \sum_{m=1}^\infty m \sum_{a=1}^\infty q^{ma} \Bigr) \\
&= 2\pi i \Bigl( 1 - 24 \sum_{n=1}^\infty \sigma_1(n) q^n \Bigr)
\,.
\end{align*}
Es genügt also, folgende Aussage zu zeigen:
\[
\frac {\Delta'}{\Delta} = 2\pi i E_2
\,,
\]
wobei $E_2(z) := 1 - 24 \sum_{n=1}^\infty \sigma_1(n) q^n$. Dies ist der \ldots

2.Schritt: Für $n \in \NN$ betrachte nun wie in \autoref{defiM(n)}
\[
\mathcal{M}(n) := \Set{M \in \ZZ^{2 \times 2} \mid \det (M) = n}
\,.
\]
Wohl bekannt ist aus \autoref{lemma:Mn_schoen}, dass 
\[
\mathcal{M}(n) = \dot{\bigcup_{\substack{ad = n\\ d > 0\\
b (\operatorname{mod} d)}}} \Gamma(1) \cdot 
\begin{pmatrix}
a & b\\
0 & d
\end{pmatrix}
\]
und damit $\# \linksmodulo{\Gamma(1)}{\mathcal{M}(n)} = \sigma_1(n)$. Definiere nun einen \glqq{}multiplikativen Hecke-Operator\grqq{} $\mathfrak{M}_n$, der eine Modulform $f$ vom Gewicht $k$ bezüglich $\Gamma(1)$ auf eine solche von Gewicht $\sigma_1 (n) \cdot k$ abbildet durch
\begin{equation}
\label{Gl 1}
\mathfrak{M}_n (f) := \prod_{\gamma \in \linksmodulo{\Gamma(1)}{\mathcal M(n)}} f |_k \gamma = \prod_{\substack{ad = n\\ d > 0\\b (\operatorname{mod} d)}} f |_k \mymat{a}{b}{0}{d}
\,.
\end{equation}
Dies ist wohldefiniert (argumentiere dazu wie bei $T(n)$ in \autoref{VorbemerkungHecke}). Wendet man dies nun auf $f = \Delta$ an, dann ist $\mathfrak{M}_n(f) = \mathfrak{M}_n(\Delta)$ eine Modulform vom Gewicht $12 \sigma_1(n)$ ohne Nullstellen in $\HH$ und mit $\ord_\infty \bigl( \mathfrak{M}_n(\Delta) \bigr) = \sigma_1(n)$.

Aus der Valenzformel folgt jetzt $\mathfrak{M}_n(\Delta) = c \cdot \Delta^{\sigma_1(n)}$ für ein $c \in \CC^\times$. Durch logarithmisches Ableiten beider Seiten von \autoref{Gl 1} erhalten wir mit $f  = \Delta$ und $\mathfrak{M}_n(\Delta) = c \cdot \Delta^{\sigma_1(n)}$, dass
\begin{equation}
\label{Gl 2}
\sigma_1(n) \frac {\Delta'}{\Delta} = \sum_{\substack{ad = n\\ d > 0\\ b (\operatorname{mod} d)}} d^{-2}n \frac{\Delta'}{\Delta} \Bigl( \frac{az+b}{d} \Bigr) = \sum_{\substack{ad = n\\ d > 0\\ b (\operatorname{mod} d)}} \frac{\Delta'}{\Delta} \Big|_2 \mymat{a}{b}{0}{d}
\,,
\end{equation}
denn die Ableitung von
\[
f |_k \mymat{a}{b}{0}{d} = n^{\frac k2} d^{-k} f\Bigl( \frac{az+b}{d} \Bigr)
\]
ist für beliebiges $f \colon \HH \to \CC$ gegeben durch
\[
\Bigl(f |_k \mymat{a}{b}{0}{d}\Bigr)' = n^{\frac k2} d^{-k} \frac ad f' \Bigl( \frac{az+b}{d} \Bigr) = n^{\frac k2} d^{-k-2} n f'\Bigl( \frac{az+b}{d} \Bigr)
\,.
\]
Setzt man $\frac{\Delta'}{\Delta} = 2\pi i \sum_{m=0}^\infty a(m) q^m$, so ergibt sich aus \autoref{Gl 2} unter formaler Anwendung der Hecke-Operatoren (siehe Beweis von \autoref{TnEndoMk}, ii)) für beliebige $m,n \in \NN$
\[
\sigma_1(n) a(m) = \sum_{d\vert(m,n)} d a\bigl( \frac{mn}{d^2} \bigr)
\,.
\]
Einsetzen von $m = 1$ liefert
\[
\sigma_1(n) a(1) = a(n)
\]
und garantiert damit, dass $\frac{\Delta'}{\Delta}$ von der Form
\[
\frac{\Delta'}{\Delta}(z) = 2\pi i \Bigl( a(0) + a(1) \sum_{n=1}^\infty \sigma_1(n) q^n \Bigr)
\]
ist. Multipliziert man nun beide Seiten mit $\Delta(z) = \sum_{m=1}^\infty \tau(m) q^m$ und beachtet dabei $\tau(1) = 1$ sowie $\tau(2) = -24$, ergibt sich durch Koeffizientenvergleich
\[
a(0) = 1 \qquad \text{und} \qquad a(1) = -24
\,,
\]
womit alles gezeigt ist.
\end{bewe}




