\section{Ergebnisse aus Funktionentheorie 2 (Errinnerung)}
\subsection{Fundamentalbereich}
Wie üblich sei
\[
	\HH = \Set{z \in \CC \mid \Im z > 0}
\]
die obere Halbebene und
\[
	\SL_2(\ZZ) = \Set{M \in M_2(\RR) \mid \det M = 1}
	\,.
\]

Dann operiert $\SL_2(\ZZ)$ auf $\HH$ durch
\[
	\abcd \circ z = \frac{az+b}{cz+d}
	\,,
\]
das heißt $E \circ z = z$ und $(M_1M_2) \circ z = M_1 \circ (M_2 \circ z)$.
Hierbei beachte man, dass
\[
	\Im \Bigl(\frac{az+b}{cz+d}\Bigr) = \frac{\Im z}{\abs{cz+d}^2}
	\,.
\]

$\Gamma(1) = \SL_2(\ZZ) \subset \SL_2(\RR)$ ist eine diskrete Untergruppe, spezielle Matrizen in $\Gamma(1)$ sind
\[
	T = \begin{pmatrix}
			1 & 1\\
			0 & 1
		\end{pmatrix}
	\qquad \text{und} \qquad
	S = \begin{pmatrix}
			0 & -1\\
			1 & 0
		\end{pmatrix}
\]
die Translation $T \circ z = z + 1$ und Stürzung $S \circ z = - \frac{1}{z}$.

Man interessiert sich für die Operation von diskreten Untergruppen $\Gamma \subset \SL_2(\ZZ)$ insbesondere $\Gamma = \Gamma(1)$.

\begin{defi}
	Eine Teilmenge $\F \subset \HH$ heißt \myemph{Fundamentalbereich} für die Operationen von $\Gamma \subset \SL_2(\RR)$ auf $\HH$, falls:
	\begin{enumerate}
		\item $\F$ ist offen,
		\item zu jedem $z \in \HH$ existiert ein $M \in \Gamma$ mit $M \circ z \in \closure\F$,
		\item Sind $z_1, z_2 \in \F$ und $z_2 = M \circ z_1$ mit $M \in \Gamma$, dann gilt $M = \pm E$ und somit $z_1 = z_2$.
	\end{enumerate}
\end{defi}

\begin{bsp}
	Die Menge $\F_1 := \Set{z=x+iy \mid \abs x < \frac{1}{2},\ \abs z > 1}$ ist ein Fundamentalbereich für die Operation von $\Gamma(1)$ auf $\HH$, dieser wird auch \myemph{Modulfigur} genannt.
	Siehe \autoref{fig:fundamentalbreichVolle}.
	
	\begin{figure}
	\begin{center}
		\includestandalone{images/chapter1/fundamentalbereich}
		\caption{Der Fundamentalbereich $\F_1$ der vollen Modulgruppe.}
		\label{fig:fundamentalbreichVolle}
	\end{center}
	\end{figure}
\end{bsp}

\begin{beme}
	Identifikationen in $\closure{\F_1}$ finden nur auf dem Rand statt. (Die Geraden $x = \pm \frac{1}{2}$ werden miteinander identifiziert unter $T$ bzw $T^{-1}$, Punkte auf den Kreisbögen rechts oder links von $i$ werden unter $S$ identifiziert.
\end{beme}

\begin{satz}
	Die Gruppe $\Gamma(1)$ wird erzeugt von $S$ und $T$.
\end{satz}

\subsection{Modulform}

\begin{defi}
	Eine Abbildung $f\colon \HH \ra \closure\CC = \CC \cup \{\infty\}$ heißt \myemph{Modulfunktion} vom Gewicht $k \in \ZZ$ für $\Gamma(1)$, falls gilt:
	\begin{enumerate}
		\item $f$ ist auf $\HH$ meromorph,
		\item $f(\frac{az+b}{cz+d}) = (cz+d)^k f(z)$ für alle $\abcd \in \Gamma(1)$,
		\item $f$ ist meromorph in $\infty$.
	\end{enumerate}
\end{defi}

\emph{Bedeutung von} (iii):
Wendet man (ii) an mit $M = T$, so erhält man $f(z+1) = f(z)$.
Sei $\mathcal R = \Set{q \in \CC \mid 0 < \abs q < 1}$.
Die Abbildung $z \mapsto q = e^{2\pi iz}$ bildet $\HH$ auf $\mathcal R$ ab und $F(q) := f(z)$ ist wohldefiniert und holomorph bis auf mögliche Polstellen, die sich prinzipiell gegen $q=0$ häufen könnten.
Bedingung (iii) fordert nun, dass $q=0$ eine unwesentliche isolierte Singularität\footnote{Das heißt es handelt sich um eine hebbare Singularität oder eine Polstelle.} von $F$ ist.
Nach Funktionentheorie 1 hat dann $F$ eine Laurententwicklung
\[
	F(q) = \sum_{n \geq n_0} a_n q^n
	\qquad \text{für }
	0 < \abs q < \abs{q_0}
\]
wobei $n_0 \in \ZZ$ fest.
Damit erhalten wir also
\[
	f(z) = \sum_{n \geq n_0} a_n e^{2\pi i nz}
	\qquad \text{für }
	 0 < y_0 < y
\]

\begin{defi}
	Ein solches $f$ heißt \myemph{Modulform} falls $f$ auf $\HH$ und in $\infty$ holomorph ist (letzteres bedeutet, dass $F$ in $q=0$ hebbar ist, also $f(z) = \sum_{n \geq 0} a_n e^{2\pi inz}$ für alle $z\in\HH$).
	Eine Modulform heißt Spitzenform, falls $a_0 = 0$.
\end{defi}

\begin{beme}
	Die Fourierkoeffizienten $a_n$ sind im Allgemeinen wichtige und interessante Größen (z.\,B. Darstellungsanzahlen von natürlichen ahlen durch quadratische Formen, etwa $r_4(n) = \#\Set{(x,y,z,w) \in \ZZ^4 \mid n = x^2+y^2+z^2+w^2}$ oder die Anzahl von Punkten auf elliptischen Kurven über $\FF_p$).
\end{beme}

\begin{defi}
	Sei $\colon \HH \ra \CC$, $k\in\ZZ$, $M = \abcd \in \SL_2(\RR)$.
	Man setzt
	\[
		(f|_kM)(z) := (cz+d)^{-k} f\Bigl(\frac{az+b}{cz+d}\Bigr)
	\]
	für $z\in\HH$, dies ist der \myemph{Peterssonscher Strichoperator}.
\end{defi}

Dann gilt $f|_kE = f$ und $f|_k(M_1M_2) = (f|_kM_1)|_kM_2$ für alle $M_1$, $M_2 \in \SL_2(\RR)$.
Es folgt:
\begin{enumerate}
	\item Es gilt $(f|_kM)(z) = (cz+d)^{-k} f(\frac{az+b}{cz+d}) = f(z)$ für alle $M = \abcd \in \Gamma(1)$ genau dann, wenn dies für $S$ und $T$ gilt, d.\,h. $f(-\frac{1}{z}) = z^k f(z)$ und $f(z+1) = f(z)$, da $S$ und $T$ $\SL_2(\ZZ)$ erzeugen.
	\item Eine Funktion $f\colon \HH \ra \CC$ ist genau dann eine Modulform vom Gewicht $k$, wenn $f$ eine Fourierentwicklung
	\[
		f(z) = \sum_{n \geq 0} a_n e^{2\pi inz}
		\qquad \text{für }
		z \in \HH
	\]
	hat und zusätzlich gilt
	\[
		f\Bigl(-\frac{1}{z}\Bigr) = z^k f(z)
	\]
\end{enumerate}

\subsection{Beispiele für Modulformen}

\subsubsection{Thetareihen}

\begin{defi}
	Sei $A \in M_m(\RR)$ symmetrisch und positiv definit.
	Dann heißt
	\[
		\theta_A(z) = \sum_{g \in \ZZ^m} e^{\pi i A[g]z}
		\qquad \text{für }
		z \in \HH
	\]
	eine \myemph{Thetareihe}, wobei $A[g] := g^t A g$ für $g \in \ZZ^m \cong M_{m,1}(\ZZ)$.
\end{defi}

\begin{satz-list}
	\item $\theta_A(z)$ ist gleichmäßig absolut konvergent auf $y \geq y_0 > 0$.
	Insbesondere ist $\theta_A(z)$ auf $\HH$ holomorph.
	\item Es gilt die Theta-Transformationsformel: $\theta_{A^{-1}} = \sqrt{\det A} \cdot (\frac{z}{i})^{\frac{m}{2}} \theta_A(z)$.
\end{satz-list}

\begin{satz}
	Sei $A \in M_m(\ZZ)$ symmetrisch, positiv definit, gerade\footnote{Das heißt für alle $\mu \in \Set{1, \ldots, m}$ gilt $a_{\mu\mu}$ ist gerade} und $\det A = 1$.
	Dann gilt $8|m$ und $\theta_A(z)$ ist eine Modulform vom Gewicht $\frac{m}{2}$ für $\Gamma(1)$.
	
	\emph{Beachte} $\theta_A(z) = 1 + \sum_{n \geq 1} r_A(n) q^n$ wobei $r_A(n)$ die Anzahl der Darstellungen von $n$ durch die ganzzahlige, positive definite quadratische Form $x \mapsto \frac{1}{2} x^t A x$ auf $\RR^m$ ist.
\end{satz}

\subsubsection{Eisensteinreihen}

\begin{defi}
	Sei $k \in \ZZ$, $k$ gerade und $k \geq 4$.
	Dann heißt
	\[
		G_k(z) = \sumprime_{m,n} \frac{1}{(mz+n)^k}
		\qquad \text{für }
		z \in \HH
	\]
	\myemph{Eisensteinreihe} vom Gewicht $k$.\footnote{$\displaystyle \sumprime_{m,n} := \sum_{\substack{(m,n) \in \ZZ^2 \\ (m,n) \not= (0,0)}}$}
\end{defi}

\begin{satz-list}
	\item $G_k(z)$ ist gleichmäßig absolut konvergent auf $D_\epsilon = \Set{ z=x+iy \mid y \geq \epsilon,\ x^2 \leq \frac{1}{\epsilon}}$, insbesondere also holomorph auf $\HH$.
	\item $G_k$ ist Modulform vom Gewicht $k$ für $\Gamma(1)$.
	\item Es gilt
	\[
		G_k(z) = 2\zeta(k) + \frac{2(2\pi i)^k}{(k-1)!} \sum_{n \geq 1} \sigma_{k-1}(n)q^n
	\]
	wobei $\zeta(k) = \sum_{n=1}^\infty \frac{1}{n^k}$ und $\sigma_{k-1}(n) = \sum_{d|n} d^{k-1}$.
	
	Setze $E_k := \frac{1}{2\zeta(k)} G_k$ die \myemph{normalisierte Eisensteinreihe}.
	Benutze nun
	\[
		\zeta(k) = \frac{(-1)^{\frac{k}{2}-1}2^{k-1} B_k}{k!} \pi^k
	\]
	für $k$ gerade und $k \geq 2$.
	Damit folgt
	\[
		E_k = 1 - \frac{2k}{B_k} \sum_{n \geq 1} \sigma_{k-1}(n)q^n
	\]
	wobei alle $B_k$ rationale Zahlen sind. Speziell gilt 
	\begin{align*}
		B_4 &= -\frac 1{30} &&\Ra& E_4 &= 1 + 240 \sum_{n \geq 1} \sigma_3 (n) q^n
		\,, \\
		B_6 &= \frac 1{42} &&\Ra& E_6 &= 1 - 504 \sum_{n \geq 1} \sigma_5 (n) q^n
		\,.
	\end{align*}
\end{satz-list}

\subsection{Valenzformel und Anwendungen}

\begin{satz}[Valenzformel]
	Sei $f$ eine Modulfunktion vom Gewicht $k \in \ZZ$, $f \not \equiv 0$. Dann gilt
	\[
	\ord_\infty f + \frac 12 \ord_i f + \frac 13 \ord_\rho f + \sum_{\substack{z \in \linksmodulo{\Gamma(1)}{\HH} \\ z \not \sim i, \rho}} \ord_z f = \frac {k}{12}
	\,.
	\]
	Hierbei ist $\rho = e^{\frac{2 \pi i}{3}}$ und
	\[
	\ord_\infty f := \ord_{q = 0} F(q)
	\]
	mit $F(q) = f(z)$ für $q = e^{2\pi i z}$.
\end{satz}

\begin{bewe}
	Zum Nachweis reduziert man auf den Fall, dass $f$ außer in $z = \rho, - \conj{\rho}, i$ keine Null- oder Polstellen auf $\partial \closure{F_1}$ hat und berechnet
	\[
	\frac{1}{2\pi i} \int_{\mathcal C} \frac{f'(z)}{f(z)} \opd z
	\,.
	\]
	Wobei die Kurve $\mathcal C$ wie in \autoref{fig:valenzformelweg} gewählt ist.
	
	\begin{figure}
		\begin{center}
			\includestandalone[scale=.7]{images/chapter1/valenzformelweg}
			\caption{Die Kurve $\mathcal C$ wobei $A$ und $E$ so gewählt sind, dass $\mathcal C$ alle Null- und Polstellen enthält.}
			\label{fig:valenzformelweg}
		\end{center}
	\end{figure}
\end{bewe}

\begin{defi}
	Sei
	\[
	\Delta (z) = \frac{1}{1728} \left( E_4^3(z) - E_6^2(z) \right)
	\]
	die \myemph{Diskriminantenfunktion}. Dann ist $\Delta$ eine Spitzenform vom Gewicht $k = 12$ mit $\Delta(z) \neq 0 \, \forall z \in \HH$ und $\ord_\infty \Delta = 1$, d.\,h. $\Delta = q + \ldots$
\end{defi}

\begin{beme}
	$\Delta$ ist in gewisser Weise die \myquote{erste} von 0 verschiedene Spitzenform und wurde von vielen Mathematikern studiert.
	
	\begin{bsp-list}
		\item Schreibe $\Delta(z) = \sum_{n \geq 1} \tau (n) q^n$, dann heißt $n \mapsto \tau (n)$ \myemph{Ramanujan-Funktion}. Es gilt: $\tau (n) \in \ZZ$ für alle $n \geq 1$. Ferner lässt sich zeigen, dass $\tau (n) \equiv \sigma_{11}(n) \mod 691$, mithilfe von $B_{12} = - \frac{691}{2730}$.
		\item Vermutung: $\tau (n) \neq 0$ für alle $n \geq 1$ (Lehner)
	\end{bsp-list}
	
\end{beme}


Sei $M_k$ der $\CC$-Vektorraum der Modulformen vom Gewicht $k \in \ZZ$ und $S_k \subset M_k$ der Unterraum der Spitzenformen.

\begin{beme}
	$M_k = \Set {0}$ für $k$ ungerade, da $f((-E) \circ z) = f(z) = (-1)^k f(z)$.
\end{beme}

\begin{satz}\label{M_k1}
	Sei $k \in \ZZ$ gerade. Dann gilt:
	\begin{enumerate}
		\item $M_k = \Set {0}$ für $k < 0$ und $M_2 = \Set {0}$.
		\item $M_0 = \CC$.
		\item $M_k = \CC E_k \oplus S_k$, falls $k \geq 4$.
		\item Die Abbildung $f \mapsto f \cdot \Delta$ gibt einen Isomorphismus von $M_{k-12}$ auf $S_k$.
		\item $\dim M_k < \infty$.
	\end{enumerate}
\end{satz}

\begin{satz}
	Sei $k \geq 0$ gerade. Dann gilt:
	\[
	\dim M_k = \begin{cases}
	\floor{\frac k{12}} & \text{ falls } k \equiv 2 \mod 12\\
	1 + \floor{\frac k{12}} & \text{ falls } k \not \equiv 2 \mod 12
	\end{cases}
	\]
\end{satz}

\begin{bsp-list}\label{M_k2}
	\item $M_4 = \CC E_4$.
	\item $M_6 = \CC E_6$.
	\item $M_8 = \CC E_8 = \CC E_4^2$.
	\item $M_{10} = \CC E_{10} = \CC E_4 E_6$.
	\item $M_{12} = \CC E_{12} \oplus \CC \Delta$.
	\item $M_{14} = \CC E_{14}$.
\end{bsp-list}

% Ab hier neu, aber gehört thematisch noch zu oben!

\begin{satz}
	Sei $k \geq 0$ gerade. Dann bilden $E_4^\alpha E_6^\beta$ mit $4\alpha + 6\beta = k$ eine Basis von $M_k$, insbesondere gilt also
	\[
	M_k = \bigoplus_{\substack{\alpha, \beta \geq 0\\ 4\alpha + 6\beta = k}} \CC E_4^\alpha E_6^\beta
	\].
\end{satz}

\begin{bewe}
	Wir zeigen zunächst induktiv, dass die Monome $M_k$ erzeugen. Für $k \leq 10$ ist dies nach Beispiel \ref{M_k2} klar. Sei also $k \geq 12$. Man bestimme eine beliebige Kombination $\alpha, \beta \geq 0$ mit $4 \alpha + 6 \beta = k$ und setze $g := E_4^\alpha E_6^\beta \in M_k$ mit konstantem Term gleich 1.
	
	Sei nun $f \in M_k$ beliebig mit konstantem Term $a_0$. Dann ist $f - a_0 \cdot g \in S_k$. Nach Satz \ref{M_k1}, iv) gilt daher $f - a_0 \cdot g = \Delta \cdot h$ mit $h \in M_{k-12}$. Nach Induktionsvoraussetzung ist $h$ eine Linearkombination von Monomen $E_4^\gamma E_6^\delta$ mit $4 \gamma + 6 \delta = k - 12$. Aber $\Delta = \frac{1}{1728} (E_4^3 - E_6^2)$ und daher ist $f - a_0 \cdot g$ Linearkombination von Monomen $E_4^{\gamma + 3}E_6^{\delta}$ und $E_4^{\gamma}E_6^{\delta + 2}$. Wegen
	\[
	4(\gamma + 3) + 6 \delta = k - 12 + 12 = k
	\]
	\[
	4 \gamma + 6 (\delta + 2) = k - 12 + 12 = k
	\]
	ist also auch $f$ als Linearkombination von Monomen der behaupteten Form schreibbar. Somit erzeugen die Monome tatsächlich $M_k$.
	
	Noch zu zeigen ist, dass die Monome über $\CC$ linear unabhängig sind. Beweis durch Widerspruch: \emph{Angenommen}, es existiere eine nicht-triviale lineare Relation
	\[
	\sum_{\substack{\alpha, \beta \geq 0\\ 4 \alpha + 6 \beta = k}} \lambda_{\alpha, \beta} E_4^\alpha E_6^\beta = 0
	\,.
	\]
	\emph{Fall 1:} Sei $k \equiv 0 \mod 4$. Dann sind alle $\beta$ gerade, also schreibe jeweils $\beta = 2 \beta'$ mit $\beta' \geq 0$. Es folgt $\alpha = \frac k4 - 3 \beta'$ und somit
	\[
	E_4^\alpha E_6^\beta = E_4^{\frac k4 - 3\beta'}E_6^{2\beta'} = E_4^{\frac k4} \left( \frac {E_6^2}{E_4^3} \right)^{\beta'}
	\,.
	\]
	Da $E_4^{\frac k4}$ nicht die Nullfunktion ist, ergibt sich eine nicht-triviale Polynom-Relation für $\frac{E_6^2}{E_4^3}$, d.\,h. die meromorphe Funktion $\frac{E_6^2}{E_4^3}$ ist Nullstelle eines nicht-trivialen Polynoms über $\CC$. Da $\CC$ algebraisch abgeschlossen ist (jedes nicht-konstante Polynom über $\CC$ zerfällt vollständig über $\CC$ in Linearfaktoren), ist $\frac{E_6^2}{E_4^3}$ somit konstant.
	
	Wir zeigen $\frac{E_6^2}{E_4^3} \equiv 0$ mit einem \emph{Trick}: Es gilt $E_6 (- \frac 1z) = z^6 E_6(z)$, denn $E_6 \in M_6$. Auswerten in $z = i = - \frac 1i$ liefert $E_6 (i) = 0$. Ferner gilt
	\[
	E_4(z) = 1 + 240 \sum_{n \geq 1} \sigma_3 (n) e^{2\pi i n z} \quad \Ra \quad E_4 (i) = 1 + 240 \sum_{n \geq 1} \sigma_3 (n) e^{-2 \pi n}
	\,.
	\] 
	Da alle Summanden positiv sind, folgt $E_4(i) \neq 0$ und somit $\frac{E_6^2(i)}{E_4^3(i)} = 0$. Dies impliziert jedoch da $\frac{E_6^2}{E_4^3}$ konstant ist bereits $E_6 \equiv 0$. \blitz
	
	\emph{Fall 2:} Sei $k \equiv 2 \mod 4$, dann sind alle $\beta$ ungerade. Analoges Vorgehen zum ersten Fall liefert ebenfalls einen Widerspruch.
	
	Somit sind die Monome über $\CC$ linear unabhängig.
\end{bewe}

\begin{beme}
	Der Satz impliziert additive Faltungsformeln für die multiplikativen Funktionen $\sigma_{k-1} (n)$ (weiterhin $k \in \ZZ$, $k \geq 4$ gerade). \myquote{Multiplikativ} bedeutet hier
	\[
	\ggt (m,n) = 1 \Ra \sigma_{k-1}(m \cdot n) = \sigma_{k-1}(m)\cdot \sigma_{k-1}(n)
	\,.
	\]
\end{beme}

\begin{bsp}
	$E_8 = E_4^2$, ferner $E_4 = 1 + 240 \sum_{n \geq 1} \sigma_3(n) q^n$, also $\sigma_7 (n) = \sigma_3 (n) + 120 \sum_{m=1}^{n-1} \sigma_3 (n-m) \sigma_3 (m)$.
	
	Allgemeiner kann man $E_k$ ausdrücken als Linearkombination von Monomen der Form $E_4^\alpha E_6^\beta$ und erhält hieraus Formeln für $\sigma_{k-1}(n)$.
\end{bsp}