\section[Die Modulinvariante \texorpdfstring{$j$}{j}]{Die Modulinvariante {\boldmath $j$}}

\begin{defi}
Sei $j := \frac{E_4^3}{\Delta}$.
\end{defi}

\begin{satz-list}
\item $j$ ist holomorph auf $\HH$ und hat einen einfachen Pol in $\infty$.
\item $j$ ist eine Modulfunktion vom Gewicht $0$.
\item $j$ liefert eine Bijektion $\linksmodulo{\Gamma(1)}{\HH} \cong \CC$.
\end{satz-list}

\begin{satz}
Sei $f\colon \HH \to \closure{\CC}$ eine meromorphe Funktion. Dann sind folgende Aussagen äquivalent:
\begin{enumerate}
\item $f$ ist eine Modulfunktion vom Gewicht 0.
\item $f$ ist Quotient zweier Modulformen gleichen Gewichts.
\item $f$ ist eine rationale Funktion in $j$.
\end{enumerate}
\end{satz}