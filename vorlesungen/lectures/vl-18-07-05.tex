Eine für uns sehr wichtige Darstellung ist die Integraldarstellung von Euler:

\begin{satz}
Es gilt für alle $s \in \CC$ mit $\Re (s) > 0$:
\[
	\Gamma(s) = \int_0^\infty e^{-x} x^{s-1} \opd x
	\,.
\]
\end{satz}

Diese Formel ist aus FT 2 bekannt; zum Beweis nutzt man den Satz von Wielandt und partielle Integration zur Verifizierung der Funktionalgleichung.

Die enorme Bedeutung der Gammafunktion in der Zahlentheorie wird in ihrem Zusammenspiel mit Dirichletreihen deutlich.

\begin{satz}[Mellin-Transformation]
Es sei $F(s) = \sum_{n=1}^\infty \frac {a(n)}{n^s}$ eine Dirichletreihe, welche irgendwo konvergiert. Dann gilt für die zugehörige Potenzreihe $P(z) = \sum_{n=1}^\infty a(n) z^n$ und für alle $s \in \CC$ mit $\Re (s) > \max \Set {0, \sigma_a(F)}$:
\[
	F(s) = \frac 1{\Gamma(s)} \int_0^\infty P(e^{-x}) \underbrace{x^{s-1}}_{\text{Mellin-Kern}} \opd x
	\,.
\]
\end{satz}

\begin{bewe}
Wir sehen über die Substitution $x = ny$ mit $n \in \NN$:
\[
	\Gamma(s) = n^s \int_0^\infty e^{-ny} y^{s-1} \opd y
	\,.
\]
Daraus folgt für alle $s \in \CC$ mit $\Re (s) > \max \Set {0, \sigma_a(F)}$:
\[
	F(s) = \frac 1{\Gamma(s)} \sum_{n=1}^\infty a(n) \int_0^\infty e^{-ny} y^{s-1} \opd y
	\,.
\]
Da 
\[
	\sum_{n=1}^\infty \int_0^\infty \abs{a(n) e^{-ny} y^{s-1} } \opd y = \sum_{n=1}^\infty \abs{a(n)} \Gamma(\sigma) n^{-\sigma} \overset {\sigma > \max \Set {0, \sigma_a(F)}}< \infty
	\,,
\]
können wir nach dem Satz von Lebesgue Summe und Integral vertauschen und erhalten die Behauptung.
\end{bewe}

\begin{bsp-list}
\item Für $F(s) = \zeta(s)$ erhalten wir für $s \in \CC$ mit $\Re (s) > 1$:
\[
	F(s) = \zeta(s) = \frac 1{\Gamma(s)} \int_0^\infty \frac {x^{s-1}}{e^x - 1} \opd x
	\,.
\]
\item Ist $G(s) = 1 - \frac 1{2^s} + \frac 1{3^s} - \frac 1{4^s} \pm \ldots = (1 - 2^{1-s}) \zeta(s)$, so gilt sogar für alle $s \in \CC$ mit $\Re (s) > 0$:
\[
	G(s) = \frac 1{\Gamma(s)} \int_0^\infty \frac {x^{s-1}}{e^x + 1} \opd x
	\,.
\]
\end{bsp-list}

Zum Schluss beweisen wir noch eine nützliche Darstellung von $\log \Gamma(s+1)$ in Form der expliziten Taylor-Entwicklung im Bereich $\abs{s} < 1$: 

\begin{satz}
Es gilt für alle $s \in U_1(0)$:
\[
	\log \Gamma(s+1) = - \gamma s + \sum_{n=2}^\infty (-1)^n \frac {\zeta(n)}n s^n
	\,.
\]
\end{satz}

\begin{bewe}
Es gilt
\[
	\Gamma(s+1) = s \Gamma(s) = \lim_{N \to \infty} \frac {N^s}{\left( 1 + \frac s1 \right) \left( 1 + \frac s2 \right) \cdots \left( 1 + \frac s{N-1} \right)}
\]
und somit folgt durch Logarithmieren:
\begin{align*}
	\log \Gamma(s+1) 
	&= \lim_{N \to \infty} \left[ s \log N - \sum_{n=1}^{N-1} \log(1 + \frac sn) \right] \\
	&= \lim_{N \to \infty} \left[ s \log N - \sum_{n=1}^{N-1} \left( \frac sn - \frac {s^2}{2n^2} + \frac {s^3}{3n^3} \mp \ldots \right) \right] \\
	&= \lim_{N \to \infty} \left[ s \left( \log N - \left( 1 + \frac 12 + \frac 13 + \ldots + \frac 1{N-1} \right) \right) \right. \\
	&\qquad\qquad + \frac {s^2}2 \left( 1 + \frac 1{2^2} + \frac 1{3^2} + \ldots + \frac 1{(N-1)^2} \right) \\
	&\qquad\qquad - \frac {s^3}3 \left( 1 + \frac 1{2^3} + \frac 1{3^3} + \ldots + \frac 1{(N-1)^3} \right) \\
	&\qquad\qquad \left. \pm \ldots \right] \\
	&= -s\gamma + \frac {\zeta(2)}2 s^2 - \frac {\zeta(3)}3 s^3 \pm \ldots
\end{align*}
Da die Folgen $a_r(N) = \sum_{n=1}^N \frac 1{n^r}$ für $r \geq 2$ gleichmäßig gegen die Werte $\zeta(r)$ konvergieren, ist Vertauschung von Limes und Summation im letzten Schritt erlaubt.
\end{bewe}



\subsection{Die Riemannsche Zetafunktion}

Die einfachste und wichtigste Dirichletreihe ist die Riemannsche Zetafunktion
\[
	\zeta(s) = \sum_{n=1}^\infty \frac 1{n^s}, \quad \Re(s) > 1
	\,.
\]
Wir haben bereits die Darstellungen 
\[
	\zeta(s) = \prod_{p \in \PP} \frac 1{1 - p^{-s}}, \quad \Re(s) > 1
\]
und
\[
	\zeta(s) = \frac 1{\Gamma(s)} \int_0^\infty \frac {x^{s-1}}{e^x - 1} \opd x, \quad \Re(s) > 1
\]
kennengelernt. Die wichtigsten bisher bewiesenen Eigenschaften der Zetafunktion sind im folgenden Satz zusammengefasst:

\begin{satz}
Die auf $\Set {z \in \CC \mid \Re(s) > 1}$ durch $\zeta(s) := \sum_{n=1}^\infty \frac 1{n^s}$ definierte Funktion $\zeta$ besitzt eine meromorphe Fortsetzung in die komplexe Zahlenebene $\CC$ mit einem einfachen Pol an der Stelle $s = 1$ mit Residuum $1$. Dies ist die einzige Polstelle von $\zeta (s)$. Die Werte der Zetafunktion bei nichtpositiven ganzen Zahlen sind rational, genauer:
\begin{align*}
	\zeta(0) &= - \frac 12 \\
	\zeta(-2n) &= 0 && \forall \, n \in \NN \\
	\zeta(1-2n) &= - \frac {B_{2n}}{2n} && \forall \, n \in \NN
	\,,
\end{align*}
wobei die rationalen Zahlen $B_2 = \frac 16$, $B_4 = - \frac 1{30}$, \ldots die durch
\[
	\frac t{e^t - 1} = \sum_{k=0}^\infty \frac {B_k}{k!} t^k, \quad \abs{t} < 2\pi
\]
definierten \myemph{Bernoulli-Zahlen} sind. Die Werte der Zetafunktion an positiven geraden Stellen sind durch
\[
	\zeta(2n) = \frac {(-1)^{n-1} 2^{2n-1} B_{2n}}{(2n)!} \pi^{2n}, \quad n \in \NN
\]
gegeben.
\end{satz}

\begin{bewe}
Wir gehen von der bereits bewiesenen Integraldarstellung 
\[
	\zeta(s) = \frac 1{\Gamma(s)} \int_0^\infty \frac {x^{s-1}}{e^x - 1} \opd x, \quad \Re(s) > 1
\]
aus. Wir entwickeln 
\[
	\frac t{e^t - 1} = \frac t{t + \frac{t^2}{2!} + \frac {t^3}{3!} + \ldots} = 1 - \frac t2 + \frac {t^2}{12} - \frac {t^4}{720} + \ldots
\]
und definieren $B_n$ als $n!$ mal den Koeffizienten von $t^n$ auf der rechten Seite. Aus
\[
	\frac t{e^t - 1} - \frac {-t}{e^{-t} - 1} = -t
\]
folgt, dass abgesehen von $B_1 = - \frac 12$ alle $B_n$ mit $n$ ungerade verschwinden. Sei jetzt $n > 0$ fest und
\[
	f_n(t) = \sum_{k=0}^n (-1)^k \frac {B_k}{k!} t^k = 1 + \frac t2 + \frac {B_2}{2!}t^2 + \ldots + \frac {B_n}{n!}t^n + \ldots
\]
(beachte, dass $(-1)^n B_n = B_n$ für $n > 1$). Dann ist für $\sigma > 1$:
\begin{align*}
	\Gamma(s) \zeta(s) 
	&= \int_0^\infty \frac {te^t}{e^t - 1} e^{-t} t^{s-2} \opd t \\
	&= \int_0^\infty \left( \frac {te^t}{e^t - 1} - f_n(t) \right) e^{-t} t^{s-2} \opd t + \int_0^\infty f_n(t) e^{-t} t^{s-2} \opd t \\
	&= I_1(s) + I_2(s)
	\,,
\end{align*}
wobei wir mit $I_1(s)$ und $I_2(s)$ die beiden Integrale bezeichnen. Die Funktion ${te^t}{e^t - 1}$ ist lokal um $t = 0$ holomorph und hat dort die Taylorentwicklung
\[
	{te^t}{e^t - 1} = {-te^t}{e^{-t} - 1} = \sum_{k=0}^\infty \frac {(-1)^k}{k!} B_k t^k
\]
und somit 
\[
	{te^t}{e^t - 1} - f_n(t) = \mathcal O (t^{n+1})
\]
für $t \to 0$. Es folgt daraus, dass das Integral $I_1(s)$ konvergiert für alle $s$ mit $\sigma > -n$ (da der Integrand für $t \to 0$ in $\mathcal O (t^{n+\sigma-1})$ ist und für $t \to \infty$ exponentiell abklingt), also stellt $I_1(s)$ in den Bereichen $\sigma > -n$ eine holomorphe Funktion dar. 

Das zweite Integral ist nur für $\sigma > 1$ konvergent, berechnet sich aber elementar aber zu
\begin{align*}
	I_2(s) 
	&= \int_0^\infty f_n(t) e^{-t} t^{s-2} \opd t \\
	&= \int_0^\infty \left( 1 + \frac t2 + \frac {B_2}{2!}t^2 + \ldots + \frac {B_n}{n!}t^n + \ldots \right) e^{-t} t^{s-2} \opd t \\
	&= \Gamma(s-1) + \frac 12 \Gamma(s) + \sum_{k=2}^n \frac {B_k}{k!} \Gamma(s+k-1)
	\,.
\end{align*}
Das ist 
\end{bewe}

Das ist eine in der ganzen komplexen Ebene meromorphe Funktion und, da $n$ beliebig gewählt werden kann, folgt, dass $\zeta (s)$ eine in ganz $\CC$ meromorphe Fortsetzung besitzt.

Setzen wir dies für $I_1(s) + I_2(s)$ ein und benutzen die Funktionalgleichung von $\Gamma(s)$, erhalten wir insgesamt
\[
	\zeta(s) = \frac 1{s-1} + \frac 12 + \sum_{k=2}^n \frac{B_k}{k!} s(s+1)(s+2)\ldots(s+k-2) + \frac 1{\Gamma(s)}I_1(s)
	\,,
\]
wobei $I_1(s)$ für $\sigma > -n$ holomorph ist. Da zudem $\Gamma(s)$ nullstellenfrei, also $\frac 1{\Gamma(s)}$ ganz ist, zeigt diese Formel, dass $\zeta(s) - \frac 1{s-1}$ in $\sigma > -n$ holomorph ist und da $n$ beliebig war, folgt der zweite Teil der Behauptung.

Sei jetzt $s \in (-n,0]$ eine ganze Zahl, dann ist $\frac 1{\Gamma(s)} I_1(s)$ wegen des Pols von $\Gamma(s)$ an dieser Stelle gleich Null, und somit ist
\[
	\zeta(s) = \frac 1{s-1} + \frac 12 + \frac s{12} - \frac {s(s+1)(s+2)}{720} + \frac {s(s+1)(s+2)(s+3)(s+4)}{30240} - \ldots + \frac {B_n}{n!}s(s+1)\ldots(s+n-2)
	\,.
\]
Dies zeigt, dass 
\begin{align*}
	\zeta(0) &= \frac 1{-1} + \frac 12 = - \frac 12 \\
	\zeta(-1) &= \frac 1{-2} + \frac 12 - \frac 1{12} = -\frac 1{12} \\
	\zeta(-2) &= \frac 1{-3} + \frac 12 - \frac 16 = 0 \\
	\zeta(-3) &= \frac 1{-4} + \frac 12 - \frac 14 + \frac 1{120} = \frac 1{120}
\end{align*}










