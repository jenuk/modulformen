\subsection{Anwendungen}

\begin{beme}
Es gilt $P_0 = E_k$, wie man durch Vergleich mit \ref{Ek_per_Gamma(1)infty} leicht einsieht.
\end{beme}

\begin{satz-list}\label{<f,Pn>}
\item Die Reihe $P_n$ konvergiert auf Kompakta in $\HH$ gleichmäßig absolut, stellt also dort eine holomorphe Funktion dar. Es gilt $P_n \in S_k$ für $n \geq 1$.
\item Es gilt 
\[
	\scalarprd {f}{P_n} = \frac{(k-2)!}{(4\pi n)^{k-1}} a_f(n)
	\,.
\]
für alle $f \in S_k$ mit $f = \sum_{m\geq 1} a_f(m) q^m$.
\end{satz-list}

\begin{bewe-list}
\item Wegen $\frac{az+b}{cz+d} \in \HH$ ist
\[
	\abs{e^{2\pi i n \frac{az+b}{cz+d}}} \leq 1
\]
und daher 
\[
	\sum_{\substack{(c,d)\in\ZZ^2 \\ \ggt(c,d) = 1 \\ ad-bc = 1}} \abs{cz+d}^{-k} \cdot \abs{e^{2\pi i n \frac{az+b}{cz+d}}} \leq \sum_{\substack{(c,d)\in\ZZ^2 \\ \ggt(c,d) = 1 \\ ad-bc = 1}} \abs{cz+d}^{-k}
	\,,
\]
sodass die Reihe der Absolutbeträge nach \ref{Ek_per_Gamma(1)infty} durch die Eisensteinreihe von Gewicht $k$ majorisiert wird. Letztere konvergiert nach FT~2 auf Kompakta in $\HH$ gleichmäßig absolut.

Zeige noch $P_n \in S_k$ für $n \geq 1$. Schreibe zunächst
\[
	P_n(z) = \frac 12 \sum_{M \in \linksmodulo{\Gamma(1)_\infty}{\Gamma(1)}} (e^n |_k M)(z)
	\,,
\]
mit $e^n(z) := e^{2 \pi inz}$ und beachte, dass $e^n |_k M = e^n$ für $M \in \Gamma(1)_\infty$. Hierbei ist wie in \ref{Ek_per_Gamma(1)infty} 
\[
	\Gamma(1)_\infty := \Set{M \in \SL_2(\ZZ) \mid M = \mymat ab0d}
	\,.
\]	
Für $P_n \in S_k$ müssen wir zeigen, dass $P_n |_k M = P_n$ für alle $M \in \SL_2(\ZZ)$ und zudem in $z = i \infty$ verschwindet. Wie im Fall der Eisensteinreihen ist hierfür zu zeigen, dass 
\[
	\lim_{z \to i\infty} P_n(z) = 0
	\,,
\]
also
\[
	\lim_{\nu \to \infty} P_n(z_\nu) = 0
\]
für jede Folge von $z_\nu \in \HH$ mit $z_\nu \to i\infty$. Wegen gleichmäßiger Konvergenz gilt
\[
	\lim_{\nu \to \infty} P_n(z_\nu) = \frac 12 \sum_{\substack{(c,d)\in\ZZ^2 \\ \ggt(c,d) = 1 \\ ad-bc = 1}} \lim_{\nu \to \infty} (cz_\nu + d)^{-k} e^{2\pi in \frac{az_\nu + b}{cz_\nu + d}}
\]
und alle Grenzwerte unter der Summe sind 0. In der Tat ist der Exponentialterm wegen $\frac{az_\nu + b}{cz_\nu + d} \in \HH$ beschränkt und für $c \neq 0$ strebt $(cz_\nu + d)^{-k}$ gegen 0. Andererseits ist für $c = 0$ der vordere Term gleich $d^{-k}$ und somit beschränkt, während
\[
	\tfrac{az_\nu + b}{d} \to i\infty \quad \Ra \quad e^{2\pi in \frac{az_\nu + b}{d}} \to 0
	\,.
\]
Damit ist alles gezeigt.

\item Unter Benutzung der Darstellung
\[
	P_n(z) = \frac 12 \sum_{M \in \linksmodulo{\Gamma(1)_\infty}{\Gamma(1)}} (e^n |_k M)(z)
\]
zeigt man mit dem gleichen \glqq{}Konvolutionstrick\grqq{} wie im Beweis von Satz \ref{CharEk}, dass
\[
	\scalarprd {f}{P_n} = \int_0^\infty \int_{-\frac 12}^{\frac 12} f(z) e^{\conj{2\pi inz}} y^{k-2} \opd x \opd y
	\,.
\]
Man stelle sich hierzu vor, dass $\HH$ als disjunkte Vereinigung von Bildern des exakten Fundamentalbereichs unter Linksmultiplikation mit $M \in \Gamma(1)$ entsteht. Teilt man nun $\Gamma(1)_\infty$ heraus, also alle Translationen, so verbleibt noch der Streifen $\abs{x} < \frac 12, y > 0$. 

Es gilt weiter für beliebiges $f \in S_k$ mit Darstellung $f(z) = \sum_{m \geq 1} a(m)q^m$, wie üblich $q = \exp(2\pi iz)$ und $z = x + iy$, dass
\begin{align*}
	\scalarprd {f}{P_n} &= \int_0^\infty \int_{-\frac 12}^{\frac 12} \sum_{m \geq 1} a(m) e^{2 \pi imx}e^{-2\pi my}e^{-2\pi inx}e^{-2\pi ny}y^{k-2} \opd x \opd y\\
	&= \int_0^\infty \int_{-\frac 12}^{\frac 12} \sum_{m \geq 1} a(m) e^{2 \pi i(m-n)x}e^{-2\pi (n+m)y}y^{k-2} \opd x \opd y\,.
\end{align*}
Wegen 
\[
	\int_{- \frac 12}^{\frac 12} e^{2 \pi irx} \opd x = \delta_{r,0} := \begin{cases}1, & r = 0\\0, & r \neq 0\end{cases} \quad \text{(Kronecker-Delta)}
\]
für beliebiges $r \in \ZZ$ folgt
\begin{align*}
	\scalarprd {f}{P_n} &= a(n) \int_0^\infty e^{-4\pi ny} y^{k-2} \opd y \\
	&= a(n) \frac{1}{(4\pi n)^{k-1}} \underbrace{\int_0^\infty e^{-y} y^{k-2} \opd y}_{= \Gamma(k-1)} \\
	&= a(n) \frac{(k-2)!}{(4\pi n)^{k-1}}
	\,.
\end{align*}
\end{bewe-list}

\begin{koro}
Die Poincare-Reihen $\Set{ P_n \mid n \in \NN}$ erzeugen den Raum $S_k$.
\end{koro}

\begin{bewe}
Angenommen die $P_n$ erzeugen nicht ganz $S_k$, dann existiert ein $f \in S_k$ mit $\scalarprd {f}{P_n} = 0$ für alle $n \in \NN$. Mit \autoref{<f,Pn>}, ii) folgt hieraus aber $a(n) = 0$ für alle $n \in \NN$ und damit $f \equiv 0$.
\end{bewe}

\begin{satz}
Die Reihe $P_n$ hat die Fourier-Entwicklung
\[
	P_n(z) = \sum_{m \geq 1} g_n(m) q^m
\]
mit
\[
	g_n(m) := \delta_{m,n} + 2\pi \cdot (-1)^{\frac k2} \cdot \left(\frac mn\right)^{\frac {k-1}2} \cdot \sum_{c \geq 1} \left[ \frac 1c \cdot K(m,n,c) \cdot J_{k-1}\left(\frac{4\pi \sqrt{mn}}c\right) \right]
	\,.
\]
Hierbei ist die Kloosterman-Summe $K$ definiert als
\[
	K(m,n,c) := \sum_{\substack{d (\operatorname{mod} c) \\ (c,d)=1}} e^{2\pi i \frac{md + n\bar{d}}{c}}
	\,,
\]
wobei $\bar d \in \ZZ$ mit $\bar d d \equiv 1 \mod c$ ist, und die Besselfunktion $J_{k-1}$ definiert als
\[
	J_{k-1}(x) := \left(\frac x2\right)^{k-1} \sum_{\ell \geq 0} \frac{(-\frac 14 x^2)^\ell}{\ell! (k-1+\ell)!}
	\,.
\]
\end{satz}

\begin{bewe}
Nach Definition ist 
\[
	P_n(z) = \frac 12 \sum_{\substack{(c,d)\in\ZZ^2 \\ \ggt(c,d) = 1 \\ ad-bc = 1}} (cz+d)^{-k} e^{2\pi in \frac{az+b}{cz+d}}
	\,.
\]
Ist $c = 0$, so folgt aus $\ggt(c,d) = 1$ bereits $d = a = \pm 1$ und unabhängig von $b \in \ZZ$ ergibt sich zweimal der Term
\[
	\frac 12 (\pm 1)^{-k} e^{2\pi in \frac{\pm z + b}{\pm 1}} = \frac 12 e^{2\pi in z} e^{\pm 2\pi in b} = \frac 12 e^{2 \pi inz}
	\,,
\]
zusammengenommen also $e^{2 \pi inz}$. Die übrigen Terme ergeben den Beitrag 
\[
	\sum_{\substack{c \geq 1, d \in \ZZ \\ \ggt(c,d) = 1 \\ ad-bc = 1}} (cz+d)^{-k} e^{2 \pi i n \frac{az+b}{cz+d}} = \sum_{c \geq 1} \sum_{\substack{d' (\operatorname{mod} c) \\ \ggt(c,d') = 1 \\ ad'-b'c = 1}} \sum_{\nu \in \ZZ} (c(z + \nu) +d')^{-k} e^{2\pi in \frac{a(z+\nu)+b'}{c(z+\nu)+d'}}
	\,.
\]
Die rechte Seite entsteht aus der linken, indem man für festes $c \geq 1$ und ein festes Vertretersystem $d' (\operatorname{mod} c)$ jedes $d \in \ZZ$ in der Form $d = d' + c\nu$ mit $v \in \ZZ$ und $d'$ im vorgegebenen Vertretersystem schreibt. Schreibt man zudem mit geeignetem $b' \in \ZZ$ auch $b = b' + a\nu$, so wird die Bedingung $ad - bc = 1$ zu
\[
	1 = ad - bc = a(d' + c\nu) - (b' + a\nu)c = ad' - b'c
\]
und die obige Darstellung folgt durch Ausklammern von $c$ und $a$. Im Folgenden schreiben wir wieder $d$ und $b$ statt $d'$ und $b'$.

\begin{lemm}
Sei $A = \mymat abcd \in \SL_2(\RR)$ mit $c > 0$. Sei $\gamma > 0$ beliebig (insbesondere nicht unbedingt ganzzahlig). Dann gilt
\[
	\sum_{\nu \in \ZZ} (c(z + \nu) +d)^{-k} e^{2\pi i\gamma \frac{a(z+\nu)+b}{c(z+\nu)+d}} 
	= 
	\tfrac{2\pi (-1)^{\frac k2}}c \sum_{m \geq 1} \left(\!\frac m\gamma\!\right)^{\!\!\frac{k-1}2} \!\! J_{k-1}\left(\!\tfrac{4\pi\sqrt{m\gamma}}c\right) e^{\frac{2\pi i}c (\gamma a + md)} e^{2\pi imz}
	\,.
\]
\end{lemm}

\begin{bewe}
Es genügt, diese Aussage nur für den Fall $A = \mymat 0{-1}10$ zu zeigen, d.h.
\[
	\sum_{\nu \in \ZZ} (z + \nu)^{-k} e^{-2\pi i\gamma \frac{1}{z+\nu}} = 2\pi (-1)^{\frac k2} \sum_{m \geq 1} \left(\!\frac m\gamma\!\right)^{\!\!\frac{k-1}2} \!\! J_{k-1}(4\pi\sqrt{m\gamma}) e^{2\pi imz}
	\,.
\]
In der Tat: Ersetzt man in dieser Gleichung $z$ durch $z + \frac dc$ und $\gamma$ durch $\frac{\gamma}{c^2}$ und multipliziert dann mit $c^{-k} e^{2\pi i \gamma \frac ac}$, so wird die linke Seite zu
\begin{align*}
	c^{-k} e^{2\pi i\gamma \frac ac} \sum_{\nu \in \ZZ} (z + \tfrac dc + \nu)^{-k} e^{-2\pi i \frac{\gamma}{c^2} \frac{1}{z + \frac dc + \nu}} 
	&= \sum_{\nu \in \ZZ} (cz + d + c\nu)^{-k} e^{2\pi i\gamma \frac ac - 2\pi i \frac{\gamma}{c} \frac{1}{cz + d + c\nu}} \\
	&= \sum_{\nu \in \ZZ} (c(z + \nu) + d)^{-k} e^{\frac{2\pi i\gamma}c \left( a - \frac{1}{c(z + \nu) + d} \right)} \\[-12pt]
	&= \sum_{\nu \in \ZZ} (c(z + \nu) + d)^{-k} e^{\frac{2\pi i\gamma}c \frac{ac(z + \nu) + \overbrace{\scriptscriptstyle ad - 1}^{= bc}}{c(z + \nu) + d}} \\
	&= \sum_{\nu \in \ZZ} (c(z + \nu) + d)^{-k} e^{2\pi i\gamma \frac{a(z + \nu) + b}{c(z + \nu) + d}} \\
\end{align*}
sowie die rechte Seite zu
\begin{multline*}
	c^{-k} e^{2\pi i \gamma \frac ac} 2\pi (-1)^{\frac k2} \sum_{m\geq 1} \left(\!\frac {mc^2}{\gamma}\right)^{\!\!\frac {k-1}2} \!\! J_{k-1} \left(\! 4\pi \sqrt{\frac{m\gamma}{c^2}}\right) e^{2\pi i m(z + \frac dc)} \\
\begin{split}
	= c^{-k+2\frac{k-1}2} 2\pi (-1)^{\frac k2} \sum_{m\geq 1} \left(\!\frac {m}{\gamma}\!\right)^{\!\!\frac {k-1}2} \!\! J_{k-1} \left(\! \tfrac{4\pi \sqrt{m\gamma}}c\right) e^{2\pi i \gamma \frac ac + 2\pi i mz + 2\pi im\frac dc} \\
	= \frac{2\pi (-1)^{\frac k2}}c \sum_{m\geq 1} \left(\!\frac {m}{\gamma}\!\right)^{\!\!\frac {k-1}2} \!\! J_{k-1} \left(\! \tfrac{4\pi \sqrt{m\gamma}}c\right) e^{\frac{2\pi i}c \left(\gamma a + md\right)} e^{2\pi i mz}
	\,.
\end{split}
\end{multline*}

%\begin{align*}
%	c^{-k} e^{2\pi i \gamma \frac ac} 2\pi (-1)^{\frac k2} &\sum_{m\geq 1} \left(\!\frac {mc^2}{\gamma}\right)^{\!\!\frac {k-1}2} \!\! J_{k-1} \left(\! 4\pi \sqrt{\frac{m\gamma}{c^2}}\right) e^{2\pi i m(z + \frac dc)} \\
%	= c^{-k+2\frac{k-1}2} 2\pi (-1)^{\frac k2} &\sum_{m\geq 1} \left(\!\frac {m}{\gamma}\!\right)^{\!\!\frac {k-1}2} \!\! J_{k-1} \left(\! \tfrac{4\pi \sqrt{m\gamma}}c\right) e^{2\pi i \gamma \frac ac + 2\pi i mz + 2\pi im\frac dc} \\
%	= \frac{2\pi (-1)^{\frac k2}}c &\sum_{m\geq 1} \left(\!\frac {m}{\gamma}\!\right)^{\!\!\frac {k-1}2} \!\! J_{k-1} \left(\! \tfrac{4\pi \sqrt{m\gamma}}c\right) e^{\frac{2\pi i}c \left(\gamma a + md\right)} e^{2\pi i mz}
%	\,.
%\end{align*}

Fortsetzung folgt (Beweis des Lemmas für $A = S$).

\end{bewe}

Fortsetzung folgt (Beweis des Satzes).

\end{bewe}






