\begin{satz-list}\label{satz:j_eigenschaften}
	\item $j$ ist holomorph auf $\HH$ und hat einen einfachen Pol in $\infty$.
	\item $j$ ist eine Modulfunktion vom Gewicht $0$.
	\item $j$ liefert eine Bijektion $\linksmodulo{\Gamma(1)}{\HH} \cong \CC$.
\end{satz-list}

\begin{bewe-list}
	\item  Da $\Delta(z) \not= 0$ für alle $z\in\HH$, ist $j(z)$ holomorph auf $\HH$.
	Ferner gilt
	\[
		\ord_\infty j = \ord_\infty E_4^3 - \ord_\infty \Delta = 0 - 1 = -1
		\,.
	\]
	\item Da $E_4^3$, $\Delta \in M_{12}$ folgt die Aussage.
	\item Sei $\lambda \in \CC$. Dann ist zu zeigen, dass die Modulfunktion $j_\lambda := j - \lambda$ vom Gewicht Null eine modulo $\SL_2(\ZZ)$ eindeutig bestimmte Nullstelle hat.
	Man wendet auf $j_\lambda$ die Valenzformel an!
	Es gilt $\ord_z j_\lambda \geq 0$ für alle $z\in\HH$ und $\ord_\infty j_\lambda = -1$.
	Da $k = 0$ folgt mit der Valenzformel
	\[
		-1 + n + \frac{n'}{2} + \frac{n''}{3} = 0
	\]
	mit $n$, $n'$, $n'' \in \NN_0$.
	Also
	\begin{equation}\label{eq:basicj_valenzformel}
		n + \frac{n'}{2} + \frac{n''}{3} = 1
	\end{equation}
	Man prüft nach: die einzigen Lösungen $(n,n',n'') \in \NN_0^3$ von \eqref{eq:basicj_valenzformel} sind $(1,0,0)$, $(0,2,0)$ und $(0,0,3)$.
	Dies impliziert die Behauptung.
\end{bewe-list}

\begin{satz}\label{satz:charakterisierung_modulfunktion_0}
	Sei $f\colon \HH \to \closure{\CC}$ eine meromorphe Funktion. Dann sind folgende Aussagen äquivalent:
	\begin{enumerate}
		\item $f$ ist eine Modulfunktion vom Gewicht 0.
		\item $f$ ist Quotient zweier Modulformen gleichen Gewichts.
		\item $f$ ist eine rationale Funktion in $j$.
	\end{enumerate}
\end{satz}

\begin{bewe-list}
	\item[(iii) $\Rightarrow$ (ii)] Sei $f = \frac{P(j)}{Q(j)}$ wobei $P(X) = a_0 + a_1X + \ldots + a_mX^m$ mit $a_\nu \in \CC$, $a_m \not= 0$ und $Q(X) = b_0 + b_1X + \ldots + b_nX^n$ mit $b_\nu \in \CC$, $b_n \not= 0$ mit $Q \not\equiv 0$, insbesondere also auch $Q(j) \not\equiv 0$.
	Wegen $j = \frac{E_4^3}{\Delta}$ folgt
	\begin{align*}
		f
		&= \frac{a_0 + a_1\frac{E_4^3}{\Delta} + \ldots + a_m\bigl(\frac{E_4^3}{\Delta}\bigr)^m}{b_0 + b_1\frac{E_4^3}{\Delta} + \ldots + b_n\bigl(\frac{E_4^3}{\Delta}\bigr)^n} \\
		&= \frac{(a_0\Delta^m + a_1E_4^3\Delta^{m-1} + \ldots + a_m(E_4^3)^m)\cdot \Delta^n}{(b_0\Delta^n + b_1E_4^3\Delta^{n-1} + \ldots + b_n(E_4^3)^n) \cdot \Delta^m}
		\,.
	\end{align*}
	Hier sind Zähler und Nenner Modulformen vom Gewicht $12(m+n)$.
	Also folgt die Behauptung.
	
	\item[(ii) $\Rightarrow$ (i)] klar
	
	\item[(i) $\Rightarrow$ (iii)] Sei $f$ eine Modulfunktion vom Gewicht Null und $f \not\equiv 0$.
	Seien $z_1, \ldots z_r$ die modulo $\Gamma(1)$ verschiedenen Polstellen von $f$ und $m_1, \ldots m_r$ deren Ordnungen.
	Sei
	\[
		P(z)
		:= \prod_{\nu = 1}^r \bigl(j(z) - j(z_\nu)\bigr)^{m_\nu}
		\,.
	\]
	Dann gilt
	\[
		\ord_{z_\nu} P
		= \ord_{z_\nu} \bigl(j(z) - j(z_\nu)\bigr)^{m_\nu}
		= m_\nu \ord_{z_\nu} \bigl(j(z) - j(z_\nu)\bigr)
		\geq m_\nu
		\,.
	\]
	Dann ist $P(z)f(z)$ eine Modulfunktion vom Gewicht Null und holomorph auf $\HH$.
	Da $P(z)$ ein Polynom in $j$ ist, genügt es die Behauptung für $P(z)f(z)$ zu zeigen.
	Insbesondere kann man voraussetzen, dass $f$ holomorph auf $\HH$ ist.
	Da $\ord_\infty \Delta = 1$, gibt es $n\in\NN_0$ so dass $g := \Delta^nf$ in unendlich holomorph ist.
	Dann ist $f = \frac{g}{\Delta^n}$ und $g$ ist eine Modulform vom Gewicht $12n$.
	Nach \autoref{satz:basis_modulformen} ist $g$ eine Linearkombination von Monomen $E_4^\alpha E_6^\beta$ mit $4\alpha + 6\beta = 12n$.
	Es genügt somit die Behauptung für $\frac{E_4^\alpha E_6^\beta}{\Delta^n}$ zu zeigen.
	Insbesondere gilt $3|\alpha$ und $2|\beta$, schreibe $\alpha = 3p$ und $\beta = 2q$.
	Dann gilt
	\[
		\frac{E_4^\alpha E_6^\beta}{\Delta^n}
		= \frac{(E_4^3)^p (E_6^2)^q}{\Delta^{p+q}}
		= j^p (j-1728)^q
		\,,
	\]
	denn $j-1728 = j - \frac{E_4^3 - E_6^2}{\Delta} = \frac{E_4^3}{\Delta} - \frac{E_4^3-E_6^2}{\Delta} = \frac{E_6^2}{\Delta}$.
\end{bewe-list}

\begin{beme-list}
	\item Der Quotient $\linksmodulo{\Gamma(1)}{\HH}$ besitzt in natürlicher Weise die Struktur einer Riemannschen Fläche isomorph zu $S^2 \setminus\Set{\text{Punkt}}$ indem man die Ränder in $\closure{\F_1}$ identifiziert.
	Fügt man den Punkt $\infty$ hinzu, so erhält an $\closure{\linksmodulo{\Gamma(1)}{\HH}} := \linksmodulo{\Gamma(1)}{\HH} \cup \Set{\infty} \cong S^2$ (die Sphäre in $\RR^3$).
	\autoref{satz:j_eigenschaften} (iii) besagt dann, dass $j$ ein Isomorphismus von $\closure{\linksmodulo{\Gamma(1)}{\HH}} \cong S^2 \cong \mathds P^1(\CC) = \CC \cup \infty$ ist.
	\autoref{satz:charakterisierung_modulfunktion_0} entspricht dann der Tatsache, dass die einzigen meromorphen Funktionen auf $S^2$ die rationalen Funktionen sind.
	
	\item Man kann zeigen (schwer!)
	\[
		\Delta(z) = q \prod_{n\geq1} (1-q^n)^{24}
		\,.
	\]
	Damit folgt
	\begin{align*}
		j
		&= \frac{E_4^3}{\Delta}
		= \frac{1}{q} \biggl(1 + 240\sum_{n\geq1} \sigma_3(n)q^n\biggr)^3 \frac{1}{\prod_{n\geq1} (1 - q^n)^{24}} \\
		&= \frac{1}{q} \biggl(1 + 240\sum_{n\geq1} \sigma_3(n)q^n\biggr)^3 \prod_{n\geq1} \Bigl(\sum_{m\geq0} q^{mn}\Bigr)^{24} \\
		&= \frac{1}{q} + 744 + \sum_{n \geq 1} c(n)q^n \qquad \text{mit } c(n) \in \NN
		\,.
	\end{align*}
	Also hat die $j$-Funktion eine Fourierentwicklung in $q$, wobei die Koeffizienten positive ganzen Zahlen sind.
	
	\item Man zeigt leicht: $\frac{1}{\prod_{n\geq1} (1-q^n)} = 1 + \sum_{n\geq1} p(n)q^n$ wobei $p(n)$ die Anzahl der Partionen von $n$ ist, d.\,h. die Anzahl der Zerlegungen von $n$ als Summe positiver, ganzer Zahlen (Beispielsweise $p(4) = 5$, denn $4 = 3 + 1 = 2 + 2 = 2 + 1 + 1 = 1 + 1 + 1 + 1$).
	Man sagt: die erzeugende Reihe von $p(n)$ wird durch $\frac{1}{\prod_{m\geq1} (1-q^n)}$ gegeben.
	
	\emph{Beachte} $1 + \sum_{n\geq1} p(n)q^n = \frac{e^{\pi i\frac{z}{12}}}{\eta(z)}$ wobei $\eta(z) = e^{\pi i\frac{z}{12}} \prod_{n\geq1} (1-q^n)$ die sogenannte \myemph{Dedekindische $\eta$-Funktion} ist.
	Beachte $\eta^{24} = \Delta$.
	$\eta$ sollte also eine Modulform vom Gewicht $\frac{1}{2}$ sein.
	Mit Hilfe der Theorie der Modulformen kann man zeigen $p(n) \sim \frac{1}{4\sqrt 3 n} \cdot e^{\pi \sqrt{\frac{3}{2}n}}$ für $n\to\infty$ (hier $a(n) \sim b(n)$ genau dann, wenn $\lim_{n\to\infty} \frac{a(n)}{b(n)} = 1$).
\end{beme-list}