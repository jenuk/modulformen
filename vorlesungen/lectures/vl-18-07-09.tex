\begin{bewe} % Fortsetzung
Es ist klar, dass man dieses Verfahren für beliebig große $n$ fortsetzen könnte, um $\zeta(-n)$ zu berechnen. Jedoch kann man auch eine geschlossene Form entwickeln: Aus dem erarbeiteten Ausdruck 
\[
	\zeta(s)
	= \frac 1{s-1} + \frac 12 + \sum_{r=2}^n \frac{B_r}{r!} s(s+1)(s+2)\ldots(s+r-2)
\]
erhält man für $s = -k$ und beliebiges $n > k$ (z.B. $n = k+1$) den Ausdruck
\begin{align*}
	\zeta(-k)
	&= \frac 1{-k-1} + \frac 12 + \sum_{r=2}^n \frac {B_r}{r!} (-k)(-k+1) \ldots (-k+r-2) \\
	&= - \frac 1{k+1} + \frac 12 + \sum_{r=2}^{k+1} (-1)^{r-1} \frac {B_r}{r!} \frac {k!}{(k+1-r)!} \\
	&= - \frac 1{k+1} \sum_{r=0}^{k+1} \binom {k+1}r B_r
	\,.
\end{align*}
Die Bernoulli-Zahlen erfüllen nun für beliebiges $n \in \NN$ die Beziehung
\[
	\sum_{r=0}^n \binom nr B_r = (-1)^n B_n
	\,,
\]
was per Koeffizientenvergleich und mit $k = n-r$ aus
\begin{align*}
	\sum_{n=0}^\infty \left( \sum_{r=0}^n \binom nr B_r \right) \frac {t^n}{n!}
	&= \sum_{r=0}^\infty \sum_{k=0}^\infty \frac {B_r t^{r+k}}{r!k!} \\
	&= \underbrace{\left( \sum_{r=0}^\infty \frac {B_r}{r!} t^r \right)}_{= \frac t{e^t - 1}} \underbrace{\left( \sum_{k=0}^\infty \frac {t^k}{k!} \right)}_{= e^t} \\
	&= \frac {-t}{e^{-t} - 1}
	= \sum_{k=0}^\infty \frac {B_n}{n!} (-t)^n
	= \sum_{k=0}^\infty (-1)^n B_n \frac {t^n}{n!}
\end{align*}
folgt. Damit ergibt sich wie behauptet
\[
	\zeta(-k) = - \frac 1{k+1} \sum_{r=0}^{k+1} \binom {k+1}r B_r = - \frac {B_{k+1}}{k+1}
	\,.
\]

Zuletzt verbleibt noch, die Behauptung für die Werte $\zeta(2n)$ mit $n \in \NN$ zu beweisen. Unter Benutzung des Eulerschen Ergänzungssatzes
\[
	\Gamma(1-s) \Gamma(s) = \frac \pi{\sin \pi s}
	\,,
\]
erhalten wir
\begin{align*}
	\sum_{n=1}^\infty (-1)^{n-1} 2^{2n-1} \pi^{2n} \frac {B_{2n}}{(2n)!} s^{2n}
	&= - \frac 12 \sum_{n=1}^\infty \frac {B_{2n}}{(2n)!} (2\pi is)^{2n} \\
	&= - \frac 12 \left[ \sum_{n=0}^\infty \frac {B_{2n}}{(2n)!} (2\pi is)^{2n} - 1 \right] \\
	&= - \frac 12 \left[ \sum_{n=0}^\infty \frac {B_n}{n!} (2\pi is)^n - 1 - 2 B_1 \pi i s \right] \\
	&= - \frac 12 \left[ \frac {2\pi is}{e^{2\pi is} - 1} - 1 + \pi is \right] \\
	&= \frac 12 - \frac {\pi is}2 \cdot \left( \frac 2{e^{2\pi is} - 1} + 1 \right) \\
	&= \frac 12 - \frac {\pi is}2 \cdot \left( \frac {2 + e^{2\pi is} - 1}{e^{2\pi is} - 1} \right) \\
	&= \frac 12 - \frac {\pi is}2 \cdot \frac {e^{\pi is} + e^{-\pi is}}{e^{\pi is} - e^{-\pi is}} \\
	&= \frac 12 \left( 1 - \frac {\pi s}{\tan \pi s} \right) \\
	&= \frac s2 \left( \frac 1s - \frac \pi{\tan \pi s} \right) && \Big| \; \text{nachrechnen!} \\
	&= \frac s2 \frac {\opd}{\opd s} \Log \frac {\pi s}{\sin \pi s} && \Big| \; \text{Ergänzungssatz} \\
	&= \frac s2 \frac {\opd}{\opd s} \Log \big( \Gamma(1+s) \Gamma(1-s) \big) && \Big| \; \text{\myref{LogGamma(s+1)}} \\
	&= \frac s2 \frac {\opd}{\opd s} \left[ \sum_{n=2}^\infty \left( (-1)^n \frac {\zeta(n)}n + \frac {\zeta(n)}n \right) s^n \right] \\
	&= \frac s2 \frac {\opd}{\opd s} \left[ \sum_{n=1}^\infty 2\frac {\zeta(2n)}{2n} s^{2n} \right] \\
	&= \sum_{n=1}^\infty \zeta(2n) s^{2n}
	\,.
\end{align*}
Hieraus folgt über Koeffizientenvergleich wie behauptet für $n \in \NN$ beliebig
\[
	\zeta(2n) = \frac {(-1)^{n-1} 2^{2n-1} B_{2n}}{(2n)!} \pi^{2n} 
	\,.
\]
\end{bewe}

Die Tatsache, dass die Werte von $\zeta(2n)$ und $\zeta(1-2n)$ dieselben Bernoulli-Zahlen enthalten, lässt erahnen, dass es eine Beziehung zwischen beiden Werten geben könnte. Dies in der Tat der Fall: Setzen wir 
\[
	\xi(s) := \pi^{- \frac s2} \cdot \Gamma \! \left( \frac s2 \right) \cdot \zeta(s)
	\,,
\]
so gilt für alle $s \in \CC \setminus \Set {0,1}$ die Gleichheit
\begin{equation}\label{eq:Xi-Beziehung}
	\xi(1-s) = \xi(s)
	\,.
\end{equation}
Diese Relation wurde zuerst von Euler vermutet und schließlich von Riemann bewiesen.

Für $\sigma > 1$ ist die rechte Seite der Gleichung \eqref{eq:Xi-Beziehung} von $0$ verschieden, was sich mit der Darstellung von $\zeta$ als Eulerprodukt leicht einsehen lässt. Es folgt dann aus \eqref{eq:Xi-Beziehung}, dass für $\sigma < 0$ nur die in \myref{Satz 1} bereits ermittelten \glqq{}trivialen Nullstellen\grqq{} $s = -2n$ mit $n \in \NN$ als Nullstellen von $\zeta$ infrage kommen. Zudem kann man zeigen, dass $\zeta$ auf den Geraden $\Re(s) = 0$ und $\Re(s) = 1$ keine Nullstellen besitzt. Die einzigen \glqq{}nicht-trivialen\grqq{} Nullstellen von $\zeta$ liegen somit im sogenannten \glqq{}kritischen Streifen\grqq{} $\Set {s \in \CC \mid 0 \leq \sigma := \Re(s) \leq 1}$. Die \glqq{}ersten\grqq{} hiervon haben die Form
\begin{align*}
	\tfrac 12 &\pm 14,134725\ldots i \,, \\
	\tfrac 12 &\pm 21,022040\ldots i \,, \\
	\tfrac 12 &\pm 25,010852\ldots i \,.
\end{align*}
Es ist bekannt, dass $\zeta$ unendlich viele nicht-triviale Nullstellen besitzt. Die bis heute ungelöste \myemph{Riemann-Vermutung} besagt, dass all diese Nullstellen den Realteil $\frac 12$ haben.

\subsection{Heckesche L-Reihen}

Einer Modulform $f = \sum_{n=0}^\infty a(n) q^n \in M_k$ ordnet man die L-Reihe
\[
	L(f,s) := \sum_{n=1}^\infty a(n)n^{-s}
\]
zu. Nach Hecke impliziert das Transformationsverhalten von $f$ \glqq{}gute\grqq{} analytische Eigenschaften für $L(f,s)$, wie zum Beispiel die Existenz einer meromorphen Fortsetzung nach $\CC$ oder die Gültigkeit einer Funktionalgleichung. Der Übergang von $f$ zu $L(f,s)$ erfolgt mittels Mellin-Transformation.

Konvention: Sei im Folgenden $k \geq 4$ stets gerade. 

\begin{defi}
Sei $f = \sum_{n=0}^\infty a(n) q^n \in M_k$. Dann heißt die Reihe 
\[
	L(f,s) := \sum_{n=1}^\infty a(n)n^{-s}
\]
die \myemph{Heckesche L-Reihe} zu $f$. 
\end{defi}

\begin{satz}
Sei $f = \sum_{n=0}^\infty a(n) q^n \in M_k$. Dann gilt:
\begin{enumerate}
\item $a(n) = \mathcal O(n^{k-1})$.
\item Ist $f \in S_k$, so gilt sogar $a(n) = \mathcal O(n^{\frac k2})$.
\end{enumerate}
\end{satz}

\begin{bewe}
Wegen $M_k = \CC E_k \oplus S_k$ und
\[
	E_k \propto G_k = 2 \zeta(k) + \frac {2(2\pi i)^k}{(k-1)!} \sum_{n=1}^\infty \sigma_{k-1}(n) q^n
\]
folgt die Aussage (i) mit (ii) und
\[
	\sigma_{k-1}(n) := \sum_{d|n} d^{k-1} = n^{k-1} \sum_{d|n} \left( \frac dn \right)^{k-1} \leq n^{k-1} \underbrace{\sum_{l=1}^\infty \frac 1{l^{k-1}}}_{< \infty} = \mathcal O(n^{k-1})
	\,.
\]
Wir müssen also nur noch (ii) zeigen. Sei dazu $f \in S_k$. Nach Definition ist
\[
	a(n) = \int_{ci}^{ci + 1} f(z) e^{-2\pi inz} \opd z
\]
mit $c \in \RR_{>0}$ beliebig. Man schreibe $f(z) = y^{- \frac k2} y^{\frac k2} f(z)$. Wie schon früher gezeigt, ist $g(z) := y^{\frac k2} f(z)$ auf ganz $\HH$ beschränkt. Es folgt somit
\[
	a(n) = \int_{ci}^{ci + 1} y^{-\frac k2} g(z) e^{-2\pi inz} \opd z = \int_0^1 c^{- \frac k2} g(t+ic) e^{2\pi nc} e^{-2 int} \opd t
\]
und damit $\abs{a(n)} \leq c^{- \frac k2} e^{2\pi nc} M$, wobei $M > 0$ nicht mehr von $n$ abhängt. Man wähle $c = \frac 1n$ und folgere $\abs{a(n)} = \mathcal O(n^{\frac k2})$. Damit ist alles gezeigt.
\end{bewe}

\begin{koro}
Sei $f \in M_k$. Dann gilt $\sigma_a (L(f,s)) \leq k$. Ist zudem $f \in S_k$, so gilt sogar $\sigma_a (L(f,s)) \leq \frac k2 + 1$. Die Funktion $s \mapsto L(f,s)$ ist in der Halbebene $\Re(s) > \sigma_c (L(f,s))$ holomorph.
\end{koro}

\begin{satz}[Hecke]
Sei $f = \sum_{n=0}^\infty a(n) q^n \in M_k$. Setzt man für $\Re(s) > k$
\[
	L^*(f,s) := (2\pi)^{-s} \Gamma(s) L(f,s)
	\,,
\]
dann gilt: Die Funktion 
\[
	s \mapsto L^*(f,s) - \frac {a(0)}s + \frac {(-1)^{\frac k2} a(0)}{k-s}
\]
hat eine holomorphe Fortsetzung auf ganz $\CC$, ist beschränkt in jedem Vertikalstreifen $\Set {s \in \CC \mid \nu_1 \leq \Re(s) \leq \nu_2}$ und erfüllt die Funktionalgleichung
\[
	L^*(f,k-s) = (-1)^{\frac k2} L^*(f,s)
	\,.
\]
Ist $f \in S_k$, so ist $a(0) = 0$ und daher sogar $s \mapsto L^*(f,s)$ selbst bereits eine ganze Funktion.
\end{satz}

\begin{bewe}
Nach \myref{Mellin-Trafo} (Mellin-Transformation) ist
\[
	L(f,s) 
	= \frac 1{\Gamma(s)} \int_0^\infty \sum_{n=1}^\infty a(n) (e^{-x})^n x^{s-1} \opd x
	= \frac 1{\Gamma(s)} \int_0^\infty \Bigg( \underbrace{\sum_{n=0}^\infty a(n) e^{-nx}}_{= f(ix)} - a(0) \Bigg) x^{s-1} \opd x
	\,,
\]
sodass nach Substitution $x \mapsto y := 2\pi x$ für alle $s \in \CC$ mit $\Re(s) > k$ gilt:
\begin{align*}
	L^*(f,s) 
	&= (2\pi)^{-s} \Gamma(s) L(f,s) \\
	&= \int_0^\infty \left( f(iy) - a(0) \right) y^{s-1} \opd y \\
	&= \underbrace{\int_0^1 \left( f(iy) - a(0) \right) y^{s-1} \opd y}_{=: I_1(s)} + \underbrace{\int_1^\infty \left( f(iy) - a(0) \right) y^{s-1} \opd y}_{=: I_2(s)}
	\,.
\end{align*}
Wie man schnell sieht, ist $L^*(f,s)$ für $\Re(s) > k$ holomorph. Darüber hinaus sieht man schnell ein, dass $I_2(s)$ für alle $s \in \CC$ konvergent (also eine ganze Funktion) und außerdem auf jedem Vertikalstreifen beschränkt ist. Es verbleibt somit nur noch das Studium von
\[
	I_1(s) = - a(0) \int_0^1 y^{s-1} \opd y + \int_0^1 f(iy) y^{s-1} \opd y
	\,.
\]
Nun gilt für den ersten Summanden
\[
	\int_0^1 y^{s-1} \opd y = \left[ \frac {y^s}s \right]_0^1 = \frac 1s
\]
und für den zweiten Summanden nach Substitution $y \mapsto \inv y$ mit $\opd (\inv y) = -y^{-2} \opd y$:
\begin{align*}
	\int_0^1 f(iy) y^{s-1} \opd y
	&= \int_\infty^1 f(i \inv y) y^{-s+1} \opd (\inv y) \\
	&= \int_1^\infty f \big( S \circ (iy) \big) y^{-s-1} \opd y \qquad \qquad \qquad \qquad \Big| \; f \in M_k \\
	&= \int_1^\infty (iy)^k f(iy) y^{-s-1} \opd y \\
	&= (-1)^{\frac k2} \int_1^\infty f(iy) y^{k-s-1} \opd y \\
	&= (-1)^{\frac k2} \left( \int_1^\infty \left( f(iy) - a(0) \right) y^{k-s-1} \opd y + a(0) \int_1^\infty y^{k-s-1} \opd y \right) \\
	&= (-1)^{\frac k2} \left( I_2(k-s) - \frac {a(0)}{k-s} \right)
	\,.
\end{align*}

Insgesamt erhalten wir also
\begin{align}
	\label{L*-SummeGanz}
	\notag{}
	L^*(f,s) + \frac {a(0)}s + \frac {(-1)^{\frac k2} a(0)}{k-s}
	&= I_1(s) + I_2(s) + \frac {a(0)}s + \frac {(-1)^{\frac k2} a(0)}{k-s} \\
	&= I_2(s) + (-1)^{\frac k2} I_2(k-s)
	\,.	
\end{align}
Da $I_2(s)$ ganz und auf jedem Vertikalstreifen beschränkt ist, müssen wir nur noch die Funktionalgleichung nachrechnen. Hierzu ersetzen wir in \eqref{L*-SummeGanz} das $s$ durch $k-s$ und beobachten ($k$ gerade):
\[
	L^*(f,k-s) + \frac {a(0)}{k-s} + \frac {(-1)^{\frac k2} a(0)}s
	&\overset{\eqref{L*-SummeGanz}}= I_2(k-s) + (-1)^{\frac k2} I_2(s) \\
	&= (-1)^{\frac k2} \left( (-1)^{\frac k2} I_2(k-s) + I_2(s) \right) \\
	&\overset{\eqref{L*-SummeGanz}}= (-1)^{\frac k2} \left( L^*(f,s) + \frac {a(0)}s + \frac {(-1)^{\frac k2} a(0)}{k-s} \right) \\
	&= (-1)^{\frac k2} L^*(f,s) + \frac {(-1)^{\frac k2} a(0)}s + \frac {a(0)}{k-s}
	\,,
\]
was nach Subtraktion von $\frac {a(0)}{k-s} + \frac {(-1)^{\frac k2} a(0)}s$ genau die Behauptung ergibt.
\end{bewe}
