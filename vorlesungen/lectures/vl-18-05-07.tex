Fortsetzung Satz 2 nach Beweis von i) und ii):

Im dritten Schritt zeigt man (Übungsaufgabe) durch Induktion nach $s \in \NN$, dass 
\[
T(p^\nu) T(p^s) = \sum_{\alpha = 0}^{\min \Set {\nu, s}} (p^\alpha)^{k-1} T(p^{\nu + s - 2\alpha})
\,,
\]
was sich mit Teilern der Form $d = p^\alpha$ umschreiben lässt zu
\[
T(p^\nu) T(p^s) = \sum_{d \vert (p^\nu, p^s)} d^{k-1} T(\frac{p^{\nu + s}}{d^2})
\,.
\]
Im abschließenden vierten Schritt behandelt man nun den allgemeinen Fall mithilfe von Induktion über die verschiedenen Primteiler von $m$. Für beliebiges $n \in N$ und einen solchen Primteiler $p$ von $m$ sei $m = p^\nu m', n = p^s n'$ mit $p \not \vert m', p \not \vert n'$. Dann lässt sich $T(m)T(n)$ nach i) umschreiben zu
\[
T(m)T(n) = T(m'p^\nu) T(n'p^s) = T(m')T(p^r)T(n')T(p^s) = T(m') T(n') T(p^r) T(p^s)
\,.
\]
Verfährt man nun analog mit $T(m')T(n')$, so sind $m', n'$ nach endlich vielen Iterationen teilerfremd und man kann wieder i) anwenden:
\[
T(m)T(n) = \sum_{d \vert (m',n')} d^{k-1} T\left(\frac {m'n'}{d^2}\right) \sum_{t \vert (p^r, p^s)} t^{k-1} T\left(\frac{p^rp^s}{t^2}\right)
\,,
\]
was sich nach erneuter Anwendung von i) vereinfacht zu
\[
T(m)T(n) = \sum_{\substack{d \vert (m',n')\\t \vert (p^r, p^s)}} (dt)^{k-1} T\left(\frac {p^rm'p^sn'}{(dt)^2}\right)
\]
und mit $D = dt$ schließlich zu 
\[
T(m)T(n) = \sum_{D \vert (m,n)} D^{k-1} T\left(\frac {mn}{D^2}\right)
\,.
\]

\section{Folgerungen}

\begin{satz}
Die Hecke-Operatoren $T(n)$ für $n \in \NN$ erzeugen eine kommutative $\CC$-Algebra von Endomorphismen von $M_k$, welche $S_k$ stabil lässt. Die Algebra wird sogar bereits von den Hecke-Operatoren $T(p)$ für $p$ prim erzeugt.
\end{satz}

\begin{bewe}
Die Kommutativität folgt direkt aus \myref{Satz 2}. Wir zeigen noch, dass für beliebiges $n \in \NN$ der Hecke-Operator $T(n)$ durch Hecke-Operatoren der Form $T(p)$ mit $p$ prim darstellbar ist. Sei dazu $n = \prod_{i=1}^{r} p_i^{\alpha_i}$ die Primzahlzerlegung von $n$, dann ist nach \myref{Satz 2}, i)
\[
T(n) = \prod_{i=1}^{r} T(p_i^{\alpha_i})
\,.
\]
Ferner gilt nach \myref{Satz 2}, ii)
\[
T(p) T(p^\nu) = T(p^{\nu+1}) + p^{k-1} T(p^{k-1})
\,,
\]
also lässt sich beispielsweise durch Wahl von $\nu = 1$ und Umstellen der Gleichung der Hecke-Operator $T(p^2)$ als Funktion von $T(p^1) = T(p)$ und $T(p^0) = T(1) = \id_{M_k}$ ausdrücken. Induktiv gilt dies für alle Hecke-Operatoren der Form $T(p_i^{\alpha_i})$, sodass sich $T(n)$ bereits als Funktion der $T(p_i)$ darstellen lässt. Damit erzeugen die Hecke-Operatoren $T(p)$ mit $p$ prim bereits die gesamte Algebra.
\end{bewe}

\begin{defi}
Sei $f \in M_k$ mit $k > 0$. Dann heißt $f$ \myemph{Hecke-Eigenform}, falls gilt
\begin{enumerate}
\item $f \not \equiv 0$,
\item $f | T(n) = \lambda(n) f$ für alle $n \in \NN$, wobei $\lambda(n) \in \CC$.
\end{enumerate}
\end{defi}

\begin{satz}
Sei $f = \sum_{m \geq 0} a(m) q^m \in M_k$ eine Hecke-Eigenform mit $f | T(n) = \lambda(n) f$ für alle $n \in \NN$, dann gilt
\begin{enumerate}
\item $a(n) = \lambda(n) \cdot a(1)$ für alle $n \in \NN$,
\item $a(1) \neq 0$, 
\item \[
\lambda(m) \lambda(n) = \sum_{d \vert (m,n)} d^{k-1} \lambda \left( \frac {mn}{d^2} \right)
\]
für alle $m,n \in \NN$. Speziell ist für $(m, n) = 1$
\[
\lambda(m) \lambda(n) = \lambda(mn)
\]
sowie für $\nu \geq 1$ und $p$ prim
\[
\lambda(p) \lambda(p^\nu) = \lambda(p^{\nu+1}) + p^{k-1}\lambda(p^{\nu-1})
\,.
\]
\end{enumerate}
\end{satz}

\begin{bewe-list}
\item Nach \myref{Satz 1}, ii), §2 gilt
\[
\lambda(n) f = f | T(n) = \sum_{m \geq 0} \Bigl( \sum_{d \vert (m,n)} d^{k-1} a\bigl( \frac {mn}{d^2} \bigr) \Bigr) q^m
\,.
\]
Koeffizientenvergleich bei $q^1$ liefert sofort $\lambda (n) \cdot a(1) = a(n)$.
\item Angenommen, $a(1) = 0$. Dann ist nach i) auch $a(n) = 0$ für alle $n \in \NN$. Somit ist $f = a(0)$ konstant und daher in $M_0$. Da für eine Hecke-Eigenform $f \in M_k$ nach Definition $k > 0$ gefordert wird, folgt aus $f \in M_0 \cap M_k$ bereits $f \equiv 0$, was im Widerspruch zur Definition der Hecke-Eigenformen steht.
\item Folgt aus \myref{Satz 2, § 2} und wegen $f \not \equiv 0$. Genauer gilt
\[
f | T(m) T(n) = \sum_{d \vert (m,n)} d^{k-1} f | T \left( \frac{mn}{d^2} \right) \Ra \lambda(m) \lambda(n) = \sum_{d \vert (m,n)} d^{k-1} \lambda\left(\frac {mn}{d^2}\right)
\,.
\]
\end{bewe-list}

\begin{defi}
Man nennt $f = \sum_{m \geq 0} a(m) q^m$ eine normalisierte Hecke-Eigenform, falls $a(1) = 1$.
\end{defi}

\begin{beme}
Durch Division durch $a(1) \neq 0$ lässt sich jede Hecke-Eigenform normalisieren. Beachte jedoch, dass zum Beispiel die \glqq{}normalisierten Eisensteinreihen\grqq{} $E_k$ zwar Hecke-Eigenformen, aber keine normalisierten Hecke-Eigenformen sind. Die beiden Normalisierungsbegriffe unterscheiden sich also.
\end{beme}

\emph{Frage:} Gibt es immer Hecke-Eigenformen? Gibt es vielleicht sogar eine Basis von Hecke-Eigenformen?

\begin{beme-list}
\item Man zeigt \glqq{}leicht\grqq{}, dass die Eisensteinreihe \[
E_k = 1 - \frac{2k}{B_k} \sum_{n \geq 1} \sigma_{k-1} (n) q^n \quad \text{ mit } \sigma_{k-1} (n) = \sum_{d \vert n} d^{k-1}
\]
eine Hecke-Eigenform ist mit $E_k | T(n) = \sigma_{k-1}(n) E_k$ für alle $n \geq 1$.

In der Tat: Der konstante Term von $E_k | T(n)$ ist gleich $\sigma_{k-1}(n)$, siehe \myref{Satz 1, § 2}. Die höheren Terme ergeben sich nach demselben Satz als 
\[
- \frac{2k}{B_k} \sum_{d \vert (m,n)} d^{k-1} \sigma_{k-1} \left(\frac{mn}{d^2}\right) = - \frac{2k}{B_k} \sigma_{k-1}(m) \sigma_{k-1}(n)
\]
wegen
\[
\sum_{d \vert (m,n)} d^\alpha \sigma_\alpha \left(\frac{mn}{d^2}\right) = \sigma_\alpha(m) \sigma_\alpha(n)
\,.
\]
für beliebiges $\alpha \in \NN$. Diese Identität lässt sich leicht induktiv zeigen (Übungsaufgabe), besitzt jedoch nur für $\alpha = k - 1$ im Kontext der Modulformen eine sinnvolle Interpretation.
\item Es ist $S_{12} = \CC \Delta$, wobei $\Delta = \frac {1}{1728} (E_4^3 - E_6^2) = \sum_{n \geq 1} \tau(n) q^n$ mit $\tau (n) \in \ZZ$ und $\tau(1) = 1$. Daher ist $\Delta$ eine normalisierte Hecke-Eigenform in $S_{12}$. Insbesondere ist
\begin{align*}
\tau(m) \tau(n) &= \sum_{d \vert (m,n)} d^{11} \tau \left( \frac {mn}{d^2} \right)\,,\\
\tau(m) \tau(n) &= \tau(mn) &&\text{ für } (m,n) = 1,\\
\tau(p) \tau(p^\nu) &= \tau(p^{\nu+1}) + p^{11} \tau(p^{\nu-1}) &&\text{ für } p \text{ prim.}
\end{align*}
\item Man kann $S_k$ mit einem Skalarprodukt versehen, derart dass die $T(n)$ hermitesch bezüglich dieses Skalarproduktes sind. Dann folgt aus der Linearen Algebra bereits, dass die $T(n)$ \emph{simultan} diagonalisierbar sind. Dies garantiert die Existenz einer Basis von Hecke-Eigenformen.
\end{beme-list}

\chapter{Das Petersson'sche Skalarprodukt}
\section{Invariantes Maß und Skalarprodukt}

\emph{Ziel:} Definition eines \glqq{}natürlichen\grqq{} Skalarprodukts auf $S_k$. Hierzu benötigt man zunächst ein $\Gamma(1)$-invariantes Maß auf $\RR^2$.

\begin{defi}
Für $z = x + iy \in \HH$ setze man
\[
\opd \omega(z) := \frac {\opd x \opd y}{y^2}
\]
Dann gilt für alle $M \in \SL_2(\RR)$
\[
\opd \omega(M \circ z) = \opd \omega(z) 
\]
d.h. $\opd \omega(z)$ ist $\SL_2(\RR)$-invariant.
\end{defi}

\begin{bewe}
Es gilt $\opd \omega(z) = \frac {i}{2y^2} \opd z \conj{\opd z}$, denn
\begin{align*}
\opd z \conj{\opd z} &= (\opd x + i \opd y)(\opd x - i \opd y)\\
&= \opd x \opd x - i \opd x \opd y + i \opd y \opd x + \opd y \opd y\\
&= 0 - i \opd x \opd y - i \opd x \opd y + 0\\
&= -2i \opd x \opd y\,.
\end{align*}
Sei nun $M = \abcd \in \SL_2(\RR)$, dann gilt unter Verwendung von
\[
\frac {\opd (M \circ z)}{\opd z} = \frac {\opd \frac{az + b}{cz + d}}{\opd z} = \frac {a(cz+d) - (az+b)d}{(cz+d)^2} = \frac{1}{(cz+d)^2}
\]
die Behauptung nach
\begin{align*}
\opd \omega(M \circ z) &= \frac {i}{2 (\Im \left(M \circ z)^2\right)} \opd (M \circ z) \conj{\opd (M \circ z)}\\
&= \frac {i}{2 \frac{y^2}{\abs{cz+d}^4}} \frac{\opd z}{(cz+d)^2} \conj{\frac {\opd z} {(cz+d)^2}}\\
&= \frac {i}{2 \frac{y^2}{\abs{cz+d}^4}} \frac{\opd z}{(cz+d)^2} \frac {\conj{\opd z}} {\conj{(cz+d)^2}}\\
&= \frac{i \abs{cz+d}^4}{2 y^2} \cdot \frac {1}{\abs{cz+d}^4} \cdot \opd z \conj{\opd z}\\
&= \frac {i}{2y^2} \opd z \conj{\opd z}\\
&= \opd \omega(z)\,.
\end{align*}
\end{bewe}

\emph{Ansatz:} $f, g \in S_k$, setze:
\[
<f,g> := \int_{\closure{\mathcal F}} y^k f(z) \conj{g(z)} \opd \omega
\]
wobei $\mathcal F$ ein Fundamentalbereich ist.













