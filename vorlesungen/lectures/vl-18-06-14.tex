Sei $f_1$, \ldots, $f_r$ eine Basis von normierten, simultanen, orthogonalen Eigenformen für die $T(m)$, d.\,h.

\begin{align*}
	f_i &= \sum_{n=1}^\infty a_n^i q^n \\
	a_1^i &= 1 \\
	T(m)f_i &= a_m^if_i \\
	\scalarprd {f_i}{f_j} = 0 &\Rla i \not= j
	\,.
\end{align*}

%\begin{erin}
%	$\scalarprd {f_i}{f_j} = 0$ genau dann, wenn $i \not= j$.
%\end{erin}

Für $m \geq 1$ definieren wir
\[
	h_m(z, z') = \sum_{ad-bc=m} (czz' + dz' + az + b)^{-k}
	\,,
\]
dabei erstreckt sich die Summe über alle ganzzahligen Matrizen $\abcd$ mit Determinante $m$.
Offenbar gilt ebenso
\[
	h_m(z,z') = \sum_{ad-bc=m} (cz+d)^{-k} \Bigl(z' + \frac{az+b}{cz+d}\Bigr)^{-k}
	= \sum_{M \in \mathcal{M}(m)} (z'+z)^{-k}|_{k,z} M
\]
Man zeigt schnell wegen $k \geq 4$, dass die Reihe auf kompakten Teilmengen $K \subset \HH \times \HH$ absolut und gleichmäßig konvergiert und dort in beiden Variablen holomorphe Funktionen darstellt.
Da $\mathcal{M}(m) \xto{\sim} \mathcal{M}(m)$, $M \mapsto ML$ mit $L \in \SL_2(\ZZ)$,
folgt $h(z,z')|_{k,z} L = h(z, z')$.

Da weiter
\begin{align*}
	\lim_{z \to i\infty} \sum_{M \in \M(m)} (z+z')^{-k}|_{k,z} M
	= \sum_{M \in \M(m)} \lim_{z \to i\infty} (z'+z)^{-k}|_{k,z} M
	= 0\,,
\end{align*}
folgt $h_m(-, z') \in S_k$ und $h_m(z, -) \in S_k$ aus Symmetriegründen.


\begin{satz}
	Sei
	\begin{equation}%\tag{2}
		C_k = \frac{(-1)^{\frac{k}{2}} \pi}{2^{(k-3)}(k-1)}\,.
	\end{equation}
	Dann gilt
	\begin{enumerate}
		\item Die Funktion $C_k^{-1} m^{k-1} h_m(z,z')$ ist ein Kern für den Operator $T(m)\colon S_k \ra S_k$, das heißt:
		\begin{equation}\label{eq:faltung_hm}%\tag{3}
			(f * h_m)(z') = C_km^{-k+1} (T(m)f)(z')
		\end{equation}
		\item Es gilt die Identität 
		\begin{equation}%\tag{4}
			C_k^{-1} m^{k-1} h_m(z,z') = \sum_{i=1}^r a_m^i \frac{f_i(z) \cdot f_i(z')}{\scalarprd{f_i}{f_i}}
		\end{equation}
		\item Die Spur $\Tr T(m)$ ist gegeben durch
		\begin{equation}%\tag{5}
			\Tr T(m) = C_k^{-1} m^{k-1} \int_\F h_m(z, -\conj z) y^{k-2} \opd x \opd y
		\end{equation}
	\end{enumerate}
\end{satz}

\begin{bewe}
	Sei zunächst $m=1$.
	Falls $\gamma = \abcd \in \SL_2(\ZZ)$, dann gilt
	\[
		(c\conj z + d)^{-k} f(z) y^k = f(\gamma z) \cdot \Im (\gamma z)^k
	\]
	Aus der Definition von $h_m$ erhalten wir demnach
	\[
		f(z)\conj{h_1(z, z')} y^k
		= \sum_{\gamma \in \Gamma(1)} \bigl(\conj{z'} + \gamma \conj z\bigr)^{-k} f(\gamma z) \Im(\gamma z)^k
	\]
	und demnach
	\begin{equation}\label{eq:faltung_h1}%\tag{6}
	\begin{split}
		(f * h_1)(z') &= \int_\F \sum_{\gamma \in \Gamma(1)} (-z' + \gamma \conj z)^{-k} f(\gamma z) \Im(\gamma z)^k \frac{\opd x \opd y}{y^2} \\
		&= \sum_{\gamma \in \Gamma} \int_{\partial \F} (-z' + \conj z)^{-k} f(z) y^{k-2} \opd x \opd y \\
		&= 2 \int_0^\infty \int_{-\infty}^\infty (x-iy - z')^{-k} f(x+iy) y^{k-2} \opd x \opd y
		\end{split}
	\end{equation}
	Nach Cauchy's Formel (und da $f$ Spitzenform) gilt
	\[
		\int_{-\infty}^\infty (x-iy - z')^{-k} f(x+iy) \opd x
		= \frac{2\pi i}{(k-1)!} f^{(k-1)} (2iy + z')
	\]
	
	Daraus folgt, dass die rechte Seite von \eqref{eq:faltung_h1} wie folgt umgeformt werden kann
	\begin{equation*}
	\begin{split}
		(f * h_1)(z') &= \frac{4\pi i}{(k-1)!} \int_0^\infty y^{k-2} f^{(k-1)} (2iy + z') \opd y \\
		&= \frac{4\pi i}{(k-1)!} \int_0^\infty \frac{1}{(2i)^{k-2}} \frac{\opd^{k-2}}{\opd t^{k-2}} f'(2ity + z') \Big|_{t=1} \opd y \\
		&= \frac{4\pi i}{(k-1)!} \frac{1}{(2i)^{k-2}} \frac{\opd^{k-2}}{\opd t^{k-2}} \int_0^\infty f'(2i ty + z') \opd y \Big|_{t=1} \\
		&= \frac{4\pi i}{(k-1)!} \frac{1}{(2i)^{k-2}} \frac{\opd^{k-2}}{\opd t^{k-2}} \Bigl( 0 - \frac{f(z')}{2it}\Bigr) \Big|_{t=1} \\
		&= C_k f(z')
	\end{split}
	\end{equation*}
	Das beweist \eqref{eq:faltung_hm} im Fall $m=1$.
	Für den allgemeinen Fall beachte
	\[
	\begin{split}
		T(m)h_1(z,z')
		&= T(m) \sum_{\gamma\in\Gamma(1)} (z'+z)^{-k} |_{k,z} \gamma \\
		&= m^{k-1} \sum_{\substack{M \in \linksmodulo{\Gamma(1)}{\M(m)}\\ \gamma \in \Gamma(1)}} (z'+z)^{-k} |_{k,z} \gamma M \\
		&= m^{k-1} \sum_{R \in \M(m)} (z'+z)^{-k}|_{k,z} R \\
		&= m^{k-1} h_m
	\end{split}
	\]
	Damit folgt (i).
	
	Für (ii) beachte, dass wir $h_m$ schreiben können mit $z_1$, \ldots, $z_r$ paarweise verschieden und $z_\nu \not\equiv i, \rho \mod \Gamma(1)$ als
	\[
		h_m(z,z') = \sum_{i,j=1}^r c_{ij}f_i(z)f_j(z)
	\]
	Denn wir können $h_m(z, z') = h_1(z) f_1(z') + \ldots + h_r(z) f_r(z')$, da $f_1$, \ldots, $f_r$ eine Basis der Spitzenformen sind, mit Funktionen $h_j\colon \HH \ra \HH$.
	Diese sind auch Spitzenformen, denn die Matrix in
	\[
		\begin{pmatrix}
			h_1(z,z_1) \\
			\vdots \\
			h_m(z,z_r)
		\end{pmatrix}
		= \begin{pmatrix}
			f_1(z_1) & \ldots & f_r(z_1) \\
			\vdots & \ddots & \vdots \\
			f_1(z_r) & \ldots & f_r(z_r)
		\end{pmatrix}
		\cdot 
		\begin{pmatrix}
			h_1(z) \\
			\vdots \\
			h_r(z)
		\end{pmatrix}
	\]
	ist invertierbar, sonst würden $\alpha_1$, \ldots, $\alpha_r \in \CC$ existieren, so dass $\alpha_1 f_1 + \ldots + \alpha_r f_r = 0$ ist und dies steht im Widerspruch dazu, dass die $f_j$ eine Basis bilden.
	Somit sind die $h_j$ Linearkombination von Spitzenformen und somit selbst Spitzenformen und als Linearkombination von den $f_j$ darstellbar.
	
	Wende nun \eqref{eq:faltungskern} auf die Funktion $f = f_\mu$ mit $1 \leq \mu \leq r$, an:
	\[\begin{split}
		(f_\mu * h_m)(z')
		&= \int_\F f_\mu(z) \sum_{i,j}^r \conj{c_{ij} f_i(z) f_j(-\conj{z'})} y^{k-2} \opd x \opd y \\
		&= \sum_{i,j}^r \conj{c_{ij}} f_j(z') \int_\F f_\mu(z) \conj{f_i(z)} \opd x \opd y \\
		&= \sum_{j=1}^r \conj{c_{\mu j}} f_j(z') \scalarprd{f_\mu}{f_\mu}
		\stackrel{\text{(i)}}{=} C_km^{1-k} a_m^\mu f_\mu(z')
	\end{split}
	\]
	Da $f_1$, \ldots $f_r$ eine Basis, folgt
	\[
		c_{\mu j} =
		\begin{cases}
			0 & \text{falls } j \not= \mu \\
			C_k m^{1-k} \scalarprd{f_\mu}{f_\mu}^{-1} & \text{falls } j = \mu
		\end{cases}
	\]
	Damit folgt (ii).
	
	Für (iii) beachte
	\[\begin{split}
		C_k^{-1} m^{k-1} \int_\F h_m(z, -\conj{z}) y^{k-2}\opd x \opd y
		&= \int_\F \sum_{i=1}^r a_m^i \frac{f_i(z)f_i(-\conj z)}{\scalarprd{f_i}{f_i}} y^{k-2} \opd x \opd y \\
		&= \sum_{i=1}^r a_mî \int_F \frac{f_i(z) \conj{f_i(z)}}{\scalarprd{f_i}{f_i}} y ^{k-2} \opd x \opd y \\
		&= \sum_{i=1}^r a_m^i = \Tr T(m)
	\end{split}
	\]
\end{bewe}

Die zweite arithmetische Darstellung liefert eine explizite Beschreibung der Spur.
Dafür müssen wir etwas ausholen.

\begin{defi}
	Ein Polynom $q \in \ZZ[X, Y]$ mit $q(X,Y) = aX^2 + bXY + cY^2$ heißt ganze, binäre \myemph{quadratische Form}.
	Diese ist induziert von der Matrix
	\[
		Q = \mymat*{a}{\frac{b}{2}}{\frac{b}{2}}{c}
	\]
	via $q(x,y) = (x, y) \cdot Q \cdot (\begin{smallmatrix} x \\ y \end{smallmatrix})$.
	
	$q$ heißt positiv definit, falls $q(x,y) > 0$ für alle $(x,y) \in \ZZ^2\setminus\{(0,0)\}$.
	
	Wir bezeichnen $D = b^2 - 4ac$ als die \myemph[quadratische Form!Diskriminante einer quadratischen Form]{Diskriminante} von $q$.
	
	Zwei quadratische Formen $q$ und $q'$ heißen äquivalent, falls es eine Matrix $U \in \SL_2(\ZZ)$ gibt mit $Q' = U^t Q U$, man kann zeigen, dass $Q$ eine Klasseninvariante ist.
	Die Rückrichtung ist im Allgemeinen falsch.
	
	Definiere eine Abbildung
	\[
		H\colon \ZZ \ra \QQ
	\]
	durch
	\begin{enumerate}
		\item $H(n) = 0$ für $n > 0$,
		\item $H(0) = - \frac{1}{12}$,
		\item $H(n)$ ist für $n > 0$ die Zahl der Äquivalenzklassen positiv definiter binärer ganzer quadratischen Formen mit Diskriminante $D = b^2-4ac = -n < 0$, wobei Klassen mit Repräsentanten der Form $d\cdot (X^2 + Y^2)$ respektive $e \cdot (X^2 + XY + Y)^2$ mit Vielfachheit $\frac{1}{2}$ beziehungsweise $\frac{1}{3}$ gezählt werden sollen.
	\end{enumerate}
	Man kann zeigen, dass $H(n)$ wohldefiniert ist.
	Definiere zudem Polynome via
	\[
		(1-tx+Nx^2)^{-1} = \sum_{k=0}^\infty P_{k+2}(t,N) x^k
	\]
\end{defi}

Mit diesen Werkzeugen gilt nun
\begin{theorem}[Spurformel, Eichler-Selberg]
	Sei $k \geq 4$ gerade und $m > 0$ beliebig. Dann gilt
	\[
		\Tr T(m) = -\frac{1}{2} \sum_{t=-\infty}^\infty P_k(t,m) H(4m-t^2) - \frac{1}{2} \sum_{d|m} \min\Bigl(d, \frac{m}{d}\Bigr)^{k-1}\,.
	\]
\end{theorem}

\begin{bewe}
	Wir übergehen den langen Beweis und verweisen auf Sergo Lang, Introduction to modular forms.
\end{bewe}