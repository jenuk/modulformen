\chapter{Heckeoperatoren}
\section{Vorbemerkung, Motivation}

\begin{defi}
Definiere die Gruppe
\[
\GL_2^+(\RR) = \Set{ \abcd \in M_2(\RR) \mid ad - bc > 0 }
\,,
\]
welche $\SL_2(\RR)$ als Untergruppe enthält.
\end{defi}

\begin{defi-list}
\item Seien $z \in \HH$ und 
\[
M = \abcd \in \GL_2^+(\RR)
\,,
\]
dann setze
\[
M \circ z := \frac{az+b}{cz+d}
\,.
\]
\item Für $k \in \ZZ$, $M \in \GL_2^+(\RR)$ und $f \colon \HH \to \CC$ setze
\[
(f |_k M)(z) := (ad - bc)^{\frac{k}{2}} (cz + d)^{-k} f \left( M \circ z \right)
\,.
\]
\end{defi-list}

Diese Definitionen verallgemeinern die früheren Definitionen für $\SL_2(\RR)$ (siehe ...). Beachte, dass weiterhin für alle $\lambda \in \RR_+$ gilt:
\[
f |_k 
\begin{pmatrix}
\lambda & 0\\
0 & \lambda
\end{pmatrix}
= f
\,.
\]

\begin{lemm-list} \label{lemm}
\item Die Abbildung $(M, z) \mapsto M \circ z$ definiert eine Operation von $\GL_2^+(\RR)$ auf $\HH$.
\item Man hat $f |_k M_1 M_2 = (f |_k M_1) |_k M_2$.
\end{lemm-list}

\begin{bewe-list}
\item Rechne nach und beachte hierbei, dass $\Im \Bigl( \frac{az+b}{cz+d} \Bigr) = (ad-bc) \frac{\Im z}{\abs{cz+d}^2}$.
\item Für $M = \abcd \in \GL_2^+(\RR)$ setze $j(M, z) := cz+d$. Dann gilt für beliebige Matrizen $M_1, M_2 \in \GL_2^+(\RR)$, dass
\[
j(M_1 M_2, z) = j(M_1, M_2 \circ z) \cdot j(M_2, z)
\,,
\]
woraus wegen $(cz+d)^{-k} = j(M, z)^{-k}$ die Behauptung folgt.
\end{bewe-list}

\emph{Ziel:} Definition gewisser linearer Operatoren $T \colon M_k \to M_k$ auf den Vektorräumen $M_k$ (Modulformen vom Gewicht $k \in \ZZ$) durch geeignete Mittelbildung.

\emph{Idee:} Sei $\mathcal{M} \subset \GL_2^+(\RR)$ eine Teilmenge mit folgenden Eigenschaften (mit $\cdot$ die gewöhnliche Matrizenmultiplikation): \begin{enumerate}
\item $\Gamma (1) \cdot \mathcal M \subset \mathcal M$
\item $\mathcal M \cdot \Gamma(1) \subset \mathcal M$
\item $\mathcal M$ zerfällt in endlich viele disjunkte Linksnebenklassen, d.h.
\[
\mathcal M = \dot{\bigcup_{M \in \linksmodulo{\Gamma(1)}{\mathcal M}}} \Gamma(1) \cdot M
\,,
\]
wobei die Vereinigung disjunkt und endlich ist. 
\end{enumerate}
Für eine Modulform $f \in M_k$ setze dann
\[
f | T_{\mathcal M} := \sum_{M \in \linksmodulo{\Gamma(1)}{\mathcal M}} f|_k M
\,.
\]
Dann ist $f | T_{\mathcal M}$ wohldefiniert, denn jede Rechtsnebenklasse $\Gamma (1) \cdot M \in \linksmodulo{\Gamma(1)}{\mathcal M}$ besteht aus Vertretern der Form $N_{\mathcal M} M$ und es gilt
\[
f |_k N_{\mathcal M} M = (f |_k N_{\mathcal M}) |_k M = f |_k M
\,
\]
wegen Lemma \ref{lemm}, ii) und 
\[
f \in M_k \Ra \bigl( \forall N \in \Gamma(1) : f |_k N = f \bigr)
\,.
\]

\emph{Ferner:} Sei eine Matrix $M_0 \in \Gamma(1)$ gegeben. Dann ist 
\[
(f | T_{\mathcal M}) |_k M_0 = \sum_{M \in \linksmodulo{\Gamma(1)}{\mathcal M}} f |_k M M_0 = \sum_{M \in \linksmodulo{\Gamma(1)}{\mathcal M}} f |_k M
\,,
\]
denn mit $M$ durchläuft auch $MM_0$ ein Vertretersystem der Rechtsnebenklassen. (Begründung: Sind zwei Matrizen $M_1, M_2 \in \mathcal M$ nicht äquivalent unter Linksmultiplikation, so gilt dies trivialerweise auch für $M_1M_0, M_2M_0$. Auch ist
\[
\mathcal M M_0 = \Bigl( \dot{\bigcup_{M \in \linksmodulo{\Gamma(1)}{\mathcal M}}} \Gamma(1) \cdot M \Bigr) M_0 = \dot{\bigcup_{M \in \linksmodulo{\Gamma(1)}{\mathcal M}}} \Gamma(1) \cdot M M_0 = \mathcal M
\,,
\]
denn nach Voraussetzung gilt sowohl $\mathcal M M_0 \subset \mathcal M$ als auch $\mathcal M = \mathcal M \inv{M_0}M_0 \subset \mathcal M M_0$.)

\emph{Folgerung:} $f | T_{\mathcal M}$ hat das Transformationsverhalten einer Modulform vom Gewicht $k$.

\section{Die Heckeoperatoren $T(n)$}

\begin{defi}
Sei $n \in \NN$. Setze 
\[
\mathcal M(n) := \Set{ \abcd \in M_2(\ZZ) \mid ad - bc = n }
\,.
\]
\end{defi}

\emph{Beobachtung:} $\mathcal M(n)$ ist invariant unter Links- und Rechtsmultiplikation von $\Gamma (1)$. 

\begin{lemm}
\[
\mathcal M (n) = \dot{\bigcup_{\substack{ad = n\\ d > 0\\
b \mod d}}} \Gamma(1) \begin{pmatrix}a&b\\0&d\end{pmatrix}
\,,
\]
wobei die Vereinigung über alle Matrizen $\begin{pmatrix}a&b\\0&d\end{pmatrix}$ geht, derart dass $a,b,d \in \ZZ$, $ad = n$, $d > 0$ und $b$ ein volles Restsystem modulo $d$ durchläuft (also z.B. $b \in \Set{1, 2, \ldots, d}$).
\end{lemm}

\begin{bewe}
RHS $\subset$ LHS klar. Zeige noch LHS $\subset$ RHS. Sei dazu $M = \abcd \in \mathcal M (n)$. Da $ad-bc = n > 0$, können nicht $a$ und $c$ gleichzeitig Null sein. Deswegen existiert $t := \ggt (a,c) \in \NN$. Also sind $-\frac ct$ und $\frac at$ teilerfremd und es existieren $\alpha, \beta \in \ZZ$ mit
\[
\begin{pmatrix}
\alpha & \beta\\
-\frac ct & \frac at
\end{pmatrix} \in \Gamma (1)
\,.
\]
Dann
\[
\begin{pmatrix}
\alpha & \beta\\
-\frac ct & \frac at
\end{pmatrix} \abcd
= \begin{pmatrix}
* & *\\ 0 & *
\end{pmatrix}
\,.
\]
Man kann also voraussetzen, dass $c = 0$. Wegen $\det M = n$ gilt dann $ad = n$. Multipliziert man gegebenenfalls mit $-1 E$, so kann man annehmen, dass $d > 0$. Schließlich multipliziere mit 
\[
\begin{pmatrix}
1 & \nu\\
0 & 1
\end{pmatrix} \in \Gamma (1), \nu \in \ZZ \Ra \begin{pmatrix}
1 & \nu\\
0 & 1
\end{pmatrix}
\begin{pmatrix}
a & b + \nu d\\
0 & d
\end{pmatrix}
\,.
\]
Durch geeignete Wahl von $\nu \in \ZZ$ kann man erreichen, dass $b + \nu d$ in einem vorgegebenen Restsystem modulo $d$ liegt.

Noch zu zeigen ist, dass die Vereinigung disjunkt ist. Angenommen, zwei Matrizen
\[
M \begin{pmatrix}
a&b\\0&d
\end{pmatrix} = \begin{pmatrix}
a'&b'\\0&d'
\end{pmatrix}
\]
für $M \in \Gamma(1)$, $ad = n = a'd'$, $d, d' > 0$, $b \mod d, b' \mod d'$ vollständige Vertretersysteme. Dann folgt, dass die (2,1)-Komponente von $M$ Null ist, $M \in \SL_2(\ZZ)$ also die Gestalt
\[
\begin{pmatrix}
\pm 1 & \nu\\
0 & \pm 1
\end{pmatrix}
\]
mit $\nu \in \ZZ$ hat. Dann ist 
\[
\begin{pmatrix}
a' & b'\\ 0 & d'
\end{pmatrix} = \begin{pmatrix}
\pm 1 & \nu\\
0 & \pm 1
\end{pmatrix} \begin{pmatrix}
a & b\\0 & d
\end{pmatrix} = \begin{pmatrix}
\pm a & \pm b + \nu d \\ 0 & \pm d
\end{pmatrix}
\,.
\]
Es folgt $d' = \pm d$, da $d, d' > 0$ also ist das Vorzeichen $+$ und $d' = d$, $b' = b + \nu d$. Somit unterscheiden sich $b, b'$ nur um ein Vielfaches von $d = d'$ und es gilt $b' = b$ un daher
\[
\begin{pmatrix}
a' & b' \\ 0 & d'
\end{pmatrix} = \begin{pmatrix}
a & b \\ 0 & d
\end{pmatrix}
\]
\end{bewe}

\begin{defi}
Sei $n \in \NN$. Man setze dann für $f \in M_k$
\[
f | T(n) := \sum_{M \in \linksmodulo{\Gamma(1)}{\mathcal M (n)}} f |_k M
\,.
\]
\end{defi}

\begin{satz-list}
\item Durch $T(n)$ wird eine lineare Abbildung $M_k \to M_k$ definiert. Diese lässt $S_k$ invariant (d.h. Spitzenformen werden auf Spitzenformen geschickt). Man nennt $T(n)$ den $n$-ten Hecke-Operator.
\item Ist $f = \sum_{m \geq 0} a(m) q^m \in M_k$, so gilt
\[
f | T(n) = \sum_{m \geq 0} \left( \sum_{d | (m,n)} d^{k-1} a\left(\frac{mn}{d^2}\right) \right) q^m
\]
\end{satz-list}

\emph{Beachte:} Der konstante Term von $f | T(n)$ ist gleich
\[
\left( \sum_{d|n} \right) a(0) = \sigma_{k-1}(n) a(0)
\]

\begin{bsp}
Sei $n = p$ prim. Dann ist
\begin{align*}
f | T(p) &= \sum_{m \geq 0} \left( \sum_{d | (m,p)} d^{k-1} a\left(\frac{mp}{d^2}\right) \right) q^m\\
&= \sum_{m \geq 0} \left( 1^{k-1} a\left(\frac{mp}{1}\right) \right) + \left( p^{k-1} a\left(\frac{mp}{p^2}\right) \right) q^m \\
&= \sum_{m \geq 0} \left( a\left(mp\right) \right) + \left( p^{k-1} a\left(\frac{m}{p}\right) \right) q^m
\,,
\end{align*}
wobei $a\left(\frac mp\right) := 0$, falls $p \not | m$. Denn:
\[
\sum_{d | (m,p)} d^{k-1} a \left( \frac {mn}{p^2} \right) = a \left(mp\right) + \begin{cases} 0 (p \not | m) \\ p^{k-1} a \left(\frac mp \right) (p | m) \end{cases}
\]
\end{bsp}

\begin{bewe-list}
\item Nach den Überlegungen in §1 wissen wir, dass $f | T(n)$ das Transformationsgesetz einer Modulform vom Gewicht $k$ hat. Auch ist $f | T(n)$ holomorph auf $\HH$. Dass $f | T(n)$ holomorph in $\infty$ ist, folgt aus ii). Dass $T(n)$ den Raum $S_k$ invariant lässt, folgt aus ii).
\end{bewe-list}

\begin{satz-list}
\item $T(m) T(n) = \sum_{d | (m,n)} d^{k-1} a \left( \frac {mn}{d^2} \right)$
\end{satz-list}

Speziell gilt (vergleiche mit Ramanujan-$\tau$-Funktion: \begin{enumerate}
\item $T(m) T(n) = T(mn)$, falls $(m,n) = 1$.
\item $T(p) T(p^n) = T(p^{n+1}) + p^{k+1} T(p^{\nu-1})$ für $p$ prim und $\nu \geq 1$.
\end{enumerate}
