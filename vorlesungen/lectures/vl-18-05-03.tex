\begin{satz-list}
	\item Durch $T(n)$ wird eine lineare Abbildung $M_k \to M_k$ definiert. Diese lässt $S_k$ invariant (gemeint ist: Spitzenformen werden auf Spitzenformen geschickt). Man nennt $T(n)$ den $n$-ten \myemph{Hecke-Operator}.
	\item Ist $f = \sum_{m \geq 0} a(m) q^m \in M_k$, so gilt
	\[
	f | T(n) = n^{\frac k2 - 1} \sum_{m \geq 0} \left( \sum_{d | (m,n)} d^{k-1} a\left(\frac{mn}{d^2}\right) \right) q^m
	\,.
	\]
\end{satz-list}

\emph{Beachte:} Der konstante Term von $f | T(n)$ ist gleich
\[
n^{\frac k2 - 1} \sum_{d|n} d^{k-1} a(0) = n^{\frac k2 - 1} \sigma_{k-1}(n) a(0)
\]

\begin{bsp}
Sei $n = p$ prim. Dann ist
\begin{align*}
f | T(p) &= \sum_{m \geq 0} \left( \sum_{d | (m,p)} d^{k-1} a\left(\frac{mp}{d^2}\right) \right) q^m\\
&= \sum_{m \geq 0} \left( 1^{k-1} a\left(\frac{mp}{1}\right) \right) + \left( p^{k-1} a\left(\frac{mp}{p^2}\right) \right) q^m \\
&= \sum_{m \geq 0} \left( a\left(mp\right) \right) + \left( p^{k-1} a\left(\frac{m}{p}\right) \right) q^m
\,,
\end{align*}
wobei $a\left(\frac mp\right) := 0$, falls $p \not | m$. Denn:
\[
\sum_{d | (m,p)} d^{k-1} a \left( \frac {mn}{p^2} \right) = a \left(mp\right) + \begin{cases} 0 (p \not | m) \\ p^{k-1} a \left(\frac mp \right) (p | m) \end{cases}
\]
\end{bsp}

\begin{bewe-list}
\item Nach den Überlegungen in §1 wissen wir, dass $f | T(n)$ das Transformationsgesetz einer Modulform vom Gewicht $k$ hat. Auch ist $f | T(n)$ holomorph auf $\HH$. Dass $f | T(n)$ holomorph in $\infty$ ist, folgt aus ii). Dass $T(n)$ den Raum $S_k$ invariant lässt, folgt aus ii).

\item Benutze \autoref{lemma:Mn_schoen}, damit folgt
\begin{align*}
	f|T(n)
	&= n^{\frac{k}{2}-1} \sum_{\substack{ad=n\\ d>0 \\ b\bmod d}} f|_k \mymat ab0d \\
	&= n^{\frac{k}{2}-1} \sum_{\substack{ad=n\\ d>0 \\ b\bmod d}} n^{\frac{k}{2}} d^{-k} f\Bigl(\frac{az+b}{d}\Bigr) \\
	&= n^{k-1} \sum_{\substack{m>0\\ ad=n,\ d>0\\ b\bmod d}} d^{-k} a(m) e^{2\pi im \frac{az+b}{d}} \\
	&= n^{k-1} \sum_{\substack{m>0\\ d|n,\ d>0}} d^{-k} a(m) e^{2\pi im \frac{n}{d^2}z} \biggl( \sum_{b\bmod d} e^{2\pi im \frac{b}{d}}\biggr)
	\,.
\end{align*}

% einschub
Es gilt
\[
	\sum_{b \bmod d} e^{2\pi im \frac{b}{d}}
	= \begin{cases}
		0 & \text{falls } d\nmid m \\
		d & \text{falls } d| m
	  \end{cases}
\]

Allgemein $1+q+\ldots q^{N-1} = \frac{q^n-1}{q-1} = 0$, falls $q\not=1$ und $q^N = 1$, wende dies an mit $q=e^{2\pi i \frac{m}{d}}$, $N=d$.
% einschub ende

Damit erhalten wir
\begin{align*}
	f|T(n)
	&= n^{k-1} \sum_{\substack{m \geq 0 \\ m \equiv 0 \bmod d \\ d|n,\ d > 0}} d^{-k+1} a(m) e^{2\pi i\frac{mn}{d^2}z} &&(m\mapsto md) \\
	&= \sum_{\substack{m \geq 0\\ d|n,\ d>0}} \Bigl(\frac{n}{d}\Bigr)^{k-1} a(md) e^{2\pi i \frac{mn}{d}z} &&(d\mapsto\frac{n}{d}) \\
	&= \sum_{\substack{m \geq 0 \\ d|n,\ d>0}} d^{k-1} a\Bigl(\frac{mn}{d}\Bigr) e^{2\pi imdz} && (md\mapsto m) \\
	&= \sum_{\substack{m\geq 0 \\ m\equiv 0 \bmod d \\ d|n,\ d>0}} d^{k-1} a\Bigl(\frac{mn}{d^2}\Bigr) e^{2\pi imz} \\
	&= \sum_{m\geq 0} \Biggl(\sum_{d|m, d|n} d^{k-1} a\biggl(\frac{mn}{d^2}\biggr)\Biggr) q^m
\end{align*}

Man beachte: alle Vertauschungen sind gerechtfertigt wegen absoluter Konvergenz.
\end{bewe-list}

\begin{satz}
Für alle $m$, $n \in \NN$ gilt
\[
	T(m) T(n) = \sum_{d | (m,n)} d^{k-1} a \left( \frac {mn}{d^2} \right)
\]
Speziell gilt (vergleiche mit Ramanujan-$\tau$-Funktion):
\begin{enumerate}
	\item $T(n)T(m) = T(mn)$ falls $\ggt(m,n) = 1$
	\item $T(p) T(p^\nu) = T(p^{\nu+1}) + p^{k-1} T(p^{\nu-1})$ für $p$ prim und $\nu \geq 1$.
\end{enumerate}

Beachte dass (ii) äquivalent ist zur Identität
\[
	\frac{1}{1-T(p)X+p^{k-1}X^2} = \sum_{\nu \geq 0} T(p^\nu) X^\nu
\]
\end{satz}

\begin{bewe}
	in mehreren Schritten:
	1. Schritt: Beweis von (i):
	Seien $m$, $n$ teilerfremd.
	Benutze \autoref{lemma:Mn_schoen}, dann gilt
	\begin{align*}
		f|T(m)T(n)
		&= (mn)^{\frac{k}{2}-1} \sum_{\substack{ad=m \\ d>0,\ b\bmod d}} \sum_{\substack{a'd' = n\\ d'>0,\ b' \bmod d'}} f|_k \mymat ab0d \mymat{a'}{b'}{0}{d'} \\
		&= (mn)^{\frac{k}{2}-1} \sum_{\substack{ad=m \\ d>0,\ b\bmod d}} \sum_{\substack{a'd' = n\\ d'>0,\ b' \bmod d'}} f|_k \mymat{aa'}{ab'+bd'}{0}{dd'}
		\,.
	\end{align*}
	Durchläuft $d$ alle positiven Teiler von $m$ und $d'$ alle positiven Teiler von $n$, so durchläuft $D := dd'$ alle positiven Teiler von $mn$, denn $\ggt(m, n) = 1$.
	Setzt man $A := aa'$, so gilt dann $AD = mn$.
	Ferner gilt: Durchläuft $b$ ein volles Restsystem $\mod d$ und $b'$ ein solches $\mod d'$, so durchläuft $B = ab+bd'$ ein volles Restsystem $\mod dd'$.
	In der Tat genügt es zu zeigen, dass diese Zahlen inkongruent $\bmod dd'$ sind (denn dann sind dies genau $dd'$ paarweise inkongruente Zahlen).
	
	Angenommen
	\[
		ab_1' + b_1d' \equiv ab_2' +b_2d' \mod dd'
	\]
	dann gilt
	\[
		a(b_1' - b_2') \equiv d(b_2 - b_1) \mod dd'
	\]
	Dies impliziert $a(b_1'- b_2') \equiv 0 \mod d'$. Aber $\ggt(a, d') = 1$, denn $a|m$ und $d'|n$ und $\ggt(m, n) = 1$ nach Voraussetzung.
	Also folgt $b_1' \equiv b_2' \mod d'$, also $b_1' = b_2'$.
	Es folgt jetzt $b_2 \equiv b_1 \mod d$, also $b_2 = b_1$.
	Also folgt die Behauptung.
	Und damit
	\[
		f|T(m)T(n)
		= (mn)^{\frac{k}{2}-1} \sum_{\substack{AD = mn\\ D > 0,\ B \bmod D}} f|_k \mymat AB0D
		= f|_kT(mn)
	\]
	
	
	2. Schritt: Beweis von (ii):
	Es gilt nach \autoref{lemma:Mn_schoen}: \[f|T(p) = p^{\frac{k}{2}-1} \Bigl(f|_k \mymat p001\Bigr) + \sum_{\mu \bmod p} f|_k \mymat 1\mu0p\]
	und
	\[
		f|T(p^\nu) = (p^\nu)^{\frac{k}{2}-1} \sum_{\substack{0 \leq \beta \leq \nu \\ b \bmod p^\beta}} f|_k \mymat{p^{\nu-\beta}}b0{p^\beta}
		\,.
	\]
	
	Dann
	\begin{align*}
		f|T(p)T(p^\nu)
		&= (p^{\nu+1})^{\frac{k}{2}-1} \biggl( \sum_{\substack{0 \leq \beta \leq \nu \\ b \bmod p^\beta}} f|_k \mymat p001 \mymat {p^{\nu-\beta}}b0{p^\beta} + \sum_{\substack{0 \leq \beta \leq \nu \\ b \bmod p^\beta}} f|_k \mymat 1\mu0p \mymat {p^{\nu-\beta}}b0{p^\beta} \biggr) \\
		&= (p^{\nu+1})^{\frac{k}{2}-1} ( \sum_{\substack{0 \leq \beta \leq \nu\\b \bmod p^\beta}} f|_k \mymat {p^{\nu +1-\beta}}{pb}{0}{p^\beta} + \sum_{\substack{0 \leq \beta \leq \nu \\ b \bmod p^\beta, \mu(p)}} f|_k \begin{smallmatrix}
		p^{nu-\beta} & b+\mu p^\beta \\ 0 & p^{\beta+1}
		\end{smallmatrix}) \tag{*}
	\end{align*}
	Betrache 2. Summe in \myref{(*)}:
	Durchläuft $b$ ein Restsystem modulo $p^\beta$ und $\mu$ ein Restsystem modulo $p$, so durchläuft $b+\mu p^\beta$ ein solches modulo $p^{\beta+1}$ (denn insgesamt $p^{\beta+1}$ Zahlen, paarweise inkongruent modulo $p^{\beta+1}$).
	Man sieht daher, dass die 2. Summe gleich
	\[
		f|T(p^{\nu+1}) - (p^{\nu+1})^{\frac{k}{2}} f|_k \begin{smallmatrix} p^{\nu+1} & 0 \\ 0 & 1 \end{smallmatrix} \,.
	\]
	
	Betrachte 1. Summe in \myref{(*)}. Diese ist gleich
	\[
		(p^{\nu + 1})^{\frac{k}{2}-1} (f|_k\begin{smallmatrix} p^{\nu+1} & 0 \\ 0 & 1 \end{smallmatrix} + \sum_{1 \leq \beta \leq \nu, b(p^\beta)} f|_k \begin{smallmatrix} p^{\nu+1-\beta} & pb \\ 0 & p^\beta \end{smallmatrix} =
	\]
	Man erhält also
	\[
		f|_k T(p)T(p')
		= F|T(p^{\nu+1}) + (p^{\nu+1})^{\frac{k}{2}-1} \underbrace{\sum_{1 \leq \beta \leq \nu, b(p^\beta)} f|_k \begin{smallmatrix} p & 0 \\ 0 & p \end{smallmatrix} |_k \begin{smallmatrix} p^{\nu-\beta} & b \\ 0 & p^{\beta-1} \end{smallmatrix}
		}_{=: R}
	\]
	In $R$ ersetze $\beta$ durch $\beta + 1$, erhalte
	\[
		R = \sum_{0 \leq \beta \leq \nu-1, b(p^{\beta+1})} f|_k \begin{smallmatrix} p^{\nu-1-\beta} & b \\ 0 & p^{\beta+1} \end{smallmatrix}
		% Man setze $b = \tilde b + \mu p^\beta$ wobei $\mu$ modulo $p$ und $\tilde b$ modulo $p^\beta$ läuft
		= \sum_{ 0 \leq \beta \leq \nu-1, \tilde b(p^\beta)} f|_k \begin{smallmatrix} 1 & \mu \\ 0 & 1 \end{smallmatrix}|_k \begin{smallmatrix} p^{\nu-1-\beta} & \tilde b \\ 0 & p^\beta \end{smallmatrix}
	\]
	da $f$ Periode 1 hat, erhält man $(p^{\nu+1})^{\frac{k}{2}-1}R$
	\[
	 	(p^{\nu+1})^{\frac{k}{2}-1} p \sum_{0 \leq \beta \leq \nu-1, \tilde b(p^\beta)} f|_k \begin{smallmatrix} p^{\nu-1-\beta} & \tilde b \\ 0 & p^\beta \end{smallmatrix}
	 	= p^{k-1} T(p^{\nu-1})
	\]
\end{bewe}
