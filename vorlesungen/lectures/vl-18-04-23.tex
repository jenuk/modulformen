% Fortsetzung Eisensteinreihen
wobei alle $B_k$ rationale Zahlen sind. Speziell gilt 
\begin{align*}
	B_4 &= -\frac 1{30} &&\Ra& E_4 &= 1 + 240 \sum_{n \geq 1} \sigma_3 (n) q^n
	\,, \\
	B_6 &= \frac 1{42} &&\Ra& E_6 &= 1 - 504 \sum_{n \geq 1} \sigma_5 (n) q^n
	\,.
\end{align*}

\subsection{Valenzformel und Anwendungen}

\begin{satz}[Valenzformel]
Sei $f$ eine Modulfunktion vom Gewicht $k \in \ZZ$, $f \not \equiv 0$. Dann gilt
\[
	\ord_\infty f + \frac 12 \ord_i f + \frac 13 \ord_\rho f + \sum_{\substack{z \in \linksmodulo{\Gamma(1)}{\HH} \\ z \not \sim i, \rho}} \ord_z f = \frac {k}{12}
	\,.
\]
Hierbei ist $\rho = e^{\frac{2 \pi i}{3}}$ und
\[
	\ord_\infty f := \ord_{q = 0} F(q)
\]
mit $F(q) = f(z)$ für $q = e^{2\pi i z}$.
\end{satz}

\begin{bewe}
Zum Nachweis reduziert man auf den Fall, dass $f$ außer in $z = \rho, - \conj{\rho}, i$ keine Null- oder Polstellen auf $\partial \closure{F_1}$ hat und berechnet
\[
	\frac{1}{2\pi i} \int_{\mathcal C} \frac{f'(z)}{f(z)} \opd z
	\,.
\]
Wobei die Kurve $\mathcal C$ wie in \autoref{fig:valenzformelweg} gewählt ist.

\begin{figure}
\begin{center}
	\includestandalone[scale=.7]{images/chapter1/valenzformelweg}
	\caption{Die Kurve $\mathcal C$ wobei $A$ und $E$ so gewählt sind, dass $\mathcal C$ alle Null- und Polstellen enthält.}
	\label{fig:valenzformelweg}
\end{center}
\end{figure}
\end{bewe}

\begin{defi}
Sei
\[
	\Delta (z) = \frac{1}{1728} \left( E_4^3(z) - E_6^2(z) \right)
\]
die \myemph{Diskriminantenfunktion}. Dann ist $\Delta$ eine Spitzenform vom Gewicht $k = 12$ mit $\Delta(z) \neq 0 \, \forall z \in \HH$ und $\ord_\infty \Delta = 1$, d.\,h. $\Delta = q + \ldots$
\end{defi}

\begin{beme}
$\Delta$ ist in gewisser Weise die \myquote{erste} von 0 verschiedene Spitzenform und wurde von vielen Mathematikern studiert.

\begin{bsp-list}
\item Schreibe $\Delta(z) = \sum_{n \geq 1} \tau (n) q^n$, dann heißt $n \mapsto \tau (n)$ \myemph{Ramanujan-Funktion}. Es gilt: $\tau (n) \in \ZZ$ für alle $n \geq 1$. Ferner lässt sich zeigen, dass $\tau (n) \equiv \sigma_{11}(n) \mod 691$, mithilfe von $B_{12} = - \frac{691}{2730}$.
\item Vermutung: $\tau (n) \neq 0$ für alle $n \geq 1$ (Lehner)
\end{bsp-list}

\end{beme}


Sei $M_k$ der $\CC$-Vektorraum der Modulformen vom Gewicht $k \in \ZZ$ und $S_k \subset M_k$ der Unterraum der Spitzenformen.

\begin{beme}
$M_k = \Set {0}$ für $k$ ungerade, da $f((-E) \circ z) = f(z) = (-1)^k f(z)$.
\end{beme}

\begin{satz}\label{M_k1}
Sei $k \in \ZZ$ gerade. Dann gilt:
\begin{enumerate}
	\item $M_k = \Set {0}$ für $k < 0$ und $M_2 = \Set {0}$.
	\item $M_0 = \CC$.
	\item $M_k = \CC E_k \oplus S_k$, falls $k \geq 4$.
	\item Die Abbildung $f \mapsto f \cdot \Delta$ gibt einen Isomorphismus von $M_{k-12}$ auf $S_k$.
	\item $\dim M_k < \infty$.
\end{enumerate}
\end{satz}

\begin{satz}
Sei $k \geq 0$ gerade. Dann gilt:
\[
\dim M_k = \begin{cases}
\floor{\frac k{12}} & \text{ falls } k \equiv 2 \mod 12\\
1 + \floor{\frac k{12}} & \text{ falls } k \not \equiv 2 \mod 12
\end{cases}
\]
\end{satz}

\begin{bsp-list}\label{M_k2}
\item $M_4 = \CC E_4$.
\item $M_6 = \CC E_6$.
\item $M_8 = \CC E_8 = \CC E_4^2$.
\item $M_{10} = \CC E_{10} = \CC E_4 E_6$.
\item $M_{12} = \CC E_{12} \oplus \CC \Delta$.
\item $M_{14} = \CC E_{14}$.
\end{bsp-list}

% Ab hier neu, aber gehört thematisch noch zu oben!

\begin{satz}
Sei $k \geq 0$ gerade. Dann bilden $E_4^\alpha E_6^\beta$ mit $4\alpha + 6\beta = k$ eine Basis von $M_k$, insbesondere gilt also
\[
M_k = \bigoplus_{\substack{\alpha, \beta \geq 0\\ 4\alpha + 6\beta = k}} \CC E_4^\alpha E_6^\beta
\].
\end{satz}

\begin{bewe}
Wir zeigen zunächst induktiv, dass die Monome $M_k$ erzeugen. Für $k \leq 10$ ist dies nach Beispiel \ref{M_k2} klar. Sei also $k \geq 12$. Man bestimme eine beliebige Kombination $\alpha, \beta \geq 0$ mit $4 \alpha + 6 \beta = k$ und setze $g := E_4^\alpha E_6^\beta \in M_k$ mit konstantem Term gleich 1.

Sei nun $f \in M_k$ beliebig mit konstantem Term $a_0$. Dann ist $f - a_0 \cdot g \in S_k$. Nach Satz \ref{M_k1}, iv) gilt daher $f - a_0 \cdot g = \Delta \cdot h$ mit $h \in M_{k-12}$. Nach Induktionsvoraussetzung ist $h$ eine Linearkombination von Monomen $E_4^\gamma E_6^\delta$ mit $4 \gamma + 6 \delta = k - 12$. Aber $\Delta = \frac{1}{1728} (E_4^3 - E_6^2)$ und daher ist $f - a_0 \cdot g$ Linearkombination von Monomen $E_4^{\gamma + 3}E_6^{\delta}$ und $E_4^{\gamma}E_6^{\delta + 2}$. Wegen
\[
4(\gamma + 3) + 6 \delta = k - 12 + 12 = k
\]
\[
4 \gamma + 6 (\delta + 2) = k - 12 + 12 = k
\]
ist also auch $f$ als Linearkombination von Monomen der behaupteten Form schreibbar. Somit erzeugen die Monome tatsächlich $M_k$.

Noch zu zeigen ist, dass die Monome über $\CC$ linear unabhängig sind. Beweis durch Widerspruch: \emph{Angenommen}, es existiere eine nicht-triviale lineare Relation
\[
\sum_{\substack{\alpha, \beta \geq 0\\ 4 \alpha + 6 \beta = k}} \lambda_{\alpha, \beta} E_4^\alpha E_6^\beta = 0
\,.
\]
\emph{Fall 1:} Sei $k \equiv 0 \mod 4$. Dann sind alle $\beta$ gerade, also schreibe jeweils $\beta = 2 \beta'$ mit $\beta' \geq 0$. Es folgt $\alpha = \frac k4 - 3 \beta'$ und somit
\[
E_4^\alpha E_6^\beta = E_4^{\frac k4 - 3\beta'}E_6^{2\beta'} = E_4^{\frac k4} \left( \frac {E_6^2}{E_4^3} \right)^{\beta'}
\,.
\]
Da $E_4^{\frac k4}$ nicht die Nullfunktion ist, ergibt sich eine nicht-triviale Polynom-Relation für $\frac{E_6^2}{E_4^3}$, d.\,h. die meromorphe Funktion $\frac{E_6^2}{E_4^3}$ ist Nullstelle eines nicht-trivialen Polynoms über $\CC$. Da $\CC$ algebraisch abgeschlossen ist (jedes nicht-konstante Polynom über $\CC$ zerfällt vollständig über $\CC$ in Linearfaktoren), ist $\frac{E_6^2}{E_4^3}$ somit konstant.

Wir zeigen $\frac{E_6^2}{E_4^3} \equiv 0$ mit einem \emph{Trick}: Es gilt $E_6 (- \frac 1z) = z^6 E_6(z)$, denn $E_6 \in M_6$. Auswerten in $z = i = - \frac 1i$ liefert $E_6 (i) = 0$. Ferner gilt
\[
E_4(z) = 1 + 240 \sum_{n \geq 1} \sigma_3 (n) e^{2\pi i n z} \quad \Ra \quad E_4 (i) = 1 + 240 \sum_{n \geq 1} \sigma_3 (n) e^{-2 \pi n}
\,.
\] 
Da alle Summanden positiv sind, folgt $E_4(i) \neq 0$ und somit $\frac{E_6^2(i)}{E_4^3(i)} = 0$. Dies impliziert jedoch da $\frac{E_6^2}{E_4^3}$ konstant ist bereits $E_6 \equiv 0$. \blitz

\emph{Fall 2:} Sei $k \equiv 2 \mod 4$, dann sind alle $\beta$ ungerade. Analoges Vorgehen zum ersten Fall liefert ebenfalls einen Widerspruch.

Somit sind die Monome über $\CC$ linear unabhängig.
\end{bewe}

\begin{beme}
Der Satz impliziert additive Faltungsformeln für die multiplikativen Funktionen $\sigma_{k-1} (n)$ (weiterhin $k \in \ZZ$, $k \geq 4$ gerade). \myquote{Multiplikativ} bedeutet hier
\[
\ggt (m,n) = 1 \Ra \sigma_{k-1}(m \cdot n) = \sigma_{k-1}(m)\cdot \sigma_{k-1}(n)
\,.
\]
\end{beme}

\begin{bsp}
$E_8 = E_4^2$, ferner $E_4 = 1 + 240 \sum_{n \geq 1} \sigma_3(n) q^n$, also $\sigma_7 (n) = \sigma_3 (n) + 120 \sum_{m=1}^{n-1} \sigma_3 (n-m) \sigma_3 (m)$.

Allgemeiner kann man $E_k$ ausdrücken als Linearkombination von Monomen der Form $E_4^\alpha E_6^\beta$ und erhält hieraus Formeln für $\sigma_{k-1}(n)$.
\end{bsp}

\section{Die Modulinvariante $j$}

\begin{defi}
Sei $j := \frac{E_4^3}{\Delta}$.
\end{defi}

\begin{satz-list}
\item $j$ ist holomorph auf $\HH$ und hat einen einfachen Pol in $\infty$.
\item $j$ ist eine Modulfunktion vom Gewicht $0$.
\item $j$ liefert eine Bijektion $\linksmodulo{\Gamma(1)}{\HH} \cong \CC$.
\end{satz-list}

\begin{satz}
Sei $f\colon \HH \to \closure{\CC}$ eine meromorphe Funktion. Dann sind folgende Aussagen äquivalent:
\begin{enumerate}
\item $f$ ist eine Modulfunktion vom Gewicht 0.
\item $f$ ist Quotient zweier Modulformen gleichen Gewichts.
\item $f$ ist eine rationale Funktion in $j$.
\end{enumerate}
\end{satz}

