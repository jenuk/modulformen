\begin{bewe}
Wir zeigen nun die behaupteten Formeln für den Mittelwert der Integrale $\frac 12 \left( I(m,t) + I(m,-t) \right)$ abhängig von $D := t^2 - 4m$ und unterscheiden wie \myref{in der Behauptung} vier Fälle:

\emph{Fall 1:} $D < 0$. In diesem Fall ist $\Gamma_g$ endlich (genau genommen von Ordnung $1 \leq \abs{\Gamma_g} \leq 3$, Beweis entfällt aus Zeitgründen). Für eine quadratische Form $g(u,v) = \alpha u^2 + \beta uv + \gamma v^2$ mit Diskriminante $D_g = D = t^2 - 4m$ erhalten wir demnach:
\begin{align*}
	\int_{\F_g} R_g (z,t) \frac{\opd x \opd y}{y^2} 
	&= \frac {1}{\abs{\Gamma_g}} \int_\HH R_g (z,t) \frac{\opd x \opd y}{y^2} \\
	&= \frac {1}{\abs{\Gamma_g}} \int_\HH \frac{y^{k-2}}{\left( \alpha(x^2 + y^2) + \beta x + \gamma - ity \right)^k} \opd x \opd y
	\,.
\end{align*}
Fasse nun die obere Halbebene $\HH$ als Teilmenge des $\RR^2$ auf und nutze den Diffeomorphismus
\[
	\Phi: \HH \to \HH, (x, y) \mapsto \left( \frac {2x-\beta}{2\alpha}, \frac y\alpha \right)
\]
mit Jacobi-Matrix
\[
	D\Phi = \mymat* {\inv \alpha}00{\inv \alpha} \quad \Ra \quad \det D\Phi = \frac 1{\alpha^2}
\]
zur Substitution per Transformationssatz: 
\begin{align*}
	\int_{\F_g} R_g (z,t) \frac{\opd x \opd y}{y^2}
	&= \frac {1}{\abs{\Gamma_g}} \int_\HH \frac {1}{\alpha^2} \frac{\left(\frac {y}{\alpha} \right)^{k-2}}{\left( \alpha \left( \left( \frac {2x-\beta}{2\alpha} \right)^2 + \left( \frac y\alpha \right)^2 \right) + \beta \left( \frac {2x-\beta}{2\alpha} \right) + \gamma - it \frac y\alpha \right)^k} \opd x \opd y \\
	&= \frac {1}{\abs{\Gamma_g}} \int_\HH \frac{y^{k-2}}{\left( (x^2 + y^2) - \frac {\beta^2}4 + \gamma \alpha - ity \right)^k} \opd x \opd y \\
	&= \frac {1}{\abs{\Gamma_g}} \int_\HH \frac{y^{k-2}}{\left( \abs{z}^2 - \frac 14 D_g - ity \right)^k} \opd x \opd y
	\,.
\end{align*}
Im letzten Schritt geht hierbei ein, dass $D_g = \beta^2 - 4 \alpha \gamma$, also $- \frac {\beta^2}4 + \alpha \gamma = - \frac 14 D_g$. Das Integral hängt somit nicht von den konkreten Parametern $\alpha, \beta, \gamma$ der Form $g$ ab, sondern nur von ihrer Diskriminante $D_g$. Da wir nur Formen $g$ mit Diskriminante $D_g = D = t^2 - 4m$ betrachten, können wir mit
\[
	I(D, t) := \int_\HH \frac{y^{k-2}}{\left( \abs{z}^2 - \frac 14 D - ity \right)^k} \opd x \opd y
\]
schreiben:
\[
	\sum_{\substack{g, D_g = D \\ g (\operatorname{mod} \Gamma(1))}} \int_{\F_g} R_g(z,t) \frac{\opd x \opd y}{y^2} = \sum_{\substack{g, D_g = D \\ g (\operatorname{mod} \Gamma(1))}} \frac{1}{\abs{\Gamma_g}} I(D,t) = 2 H(-D) I(D,t)
\]
Die letzte Umformung liefert einen derart einfachen Term, da die Definition von $H$ wegen
\[
	\abs{\Gamma_g} = \begin{cases}
	2 &\text{, falls } g \sim e(x^2 + y^2) \text{ für ein } e \in \ZZ \setminus \Set {0} \\
	3 &\text{, falls } g \sim e(x^2 + xy + y^2) \text{ für ein } e \in \ZZ \setminus \Set {0}	\\
	1 &\text{, sonst }
	\end{cases}
\]
bereits den Vorfaktor $\frac 1{\abs{\Gamma_g}}$ beinhaltet. Der Faktor $2$ rührt von der Tatsache her, dass in der Definition von $H$ nur Klassen positiv definiter quadratischer Formen gezählt werden. Die obige Summe berücksichtigt jedoch Klassen aller Formen $g$ mit geeigneter Diskriminante $D_g = D$. Da alle diese Formen wegen $D_g = D < 0$ definit (also entweder positiv oder negativ definit) sind, besteht die Summe aus genau doppelt so vielen Summanden wie durch $H$ angegeben (zu jeder positiv definiten Form erhält man eine negativ definite Form gleicher Diskriminante durch Multiplikation mit $-1$ und umgekehrt).

Nutzt man nun die für beliebiges $A \in \RR_{>0}$ gültige Formel
\[
	\int_{-\infty}^\infty (x^2 + A)^{-k} \opd x = \frac \pi{(k-1)!} \cdot \frac 12 \cdot \frac 32 \cdot \frac 52 \cdots (k-\frac 32) \cdot A^{\frac 12 -k}
	\,,
\]
erhält man schließlich
\begin{align*}
	I(D, t) &= \int_\HH \frac{y^{k-2}}{\left( \abs{z}^2 - \frac 14 D - ity \right)^k} \opd x \opd y \\
	&= \int_0^\infty y^{k-2} \int_{-\infty}^\infty (x^2 + y^2 - ity - \frac 14 D)^{-k} \opd x \opd y && \Big| \text{ siehe oben} \\
	&= \frac \pi{(k-1)!} \cdot \frac 12 \cdot \frac 32 \cdot \frac 52 \cdots (k-\frac 32) \int_0^\infty (y^2 - ity - \frac 14 D)^{\frac 12 - k} y^{k-2} \opd y && \Big| \text{ Leibniz} \\
	&= \frac{\pi}{(k-1)!} \cdot \frac 12 \cdot \frac 1{i^{k-2}} \left( \! \frac {\opd}{\opd t} \right)^{\!\! k-2} \int_0^\infty (y^2 - ity - \frac 14D)^{-\frac 32} \opd y \\
	&= \frac{\pi i^{k-2}}{2(k-1)!} \left( \! \frac {\opd}{\opd t} \right)^{\!\! k-2} \left[ \frac{4}{t^2 - D} \cdot \frac{y - \frac 12 it}{\sqrt{y^2 - ity - \frac 14 D}} \right]_0^\infty && \Big| \text{ nachrechnen!} \\
	&= \frac{\pi i^{k-2}}{2(k-1)!} \left( \! \frac {\opd}{\opd t} \right)^{\!\! k-2} \left( \frac 4{\sqrt{\abs{D}}} \cdot  \frac 1{\sqrt{\abs{D}} - it} \right) \\
	&= \frac{2\pi}{k-1} \cdot \frac 1{\sqrt{\abs{D}}} \cdot \frac {i^{k-2}}{(k-2)!} \left( \! \frac {\opd}{\opd t} \right)^{\!\! k-2} \left( \frac 1{\sqrt{\abs{D}} - it} \right) \\
	&= \frac {2\pi}{k-1} \cdot \frac 1{\sqrt{\abs{D}}} \cdot \frac 1{\left( \sqrt{\abs{D}} - it \right)^{k-1}}
	\,.
\end{align*}
Zusammengefasst gilt also (unter Beachtung von $D = t^2 - 4m < 0$), dass
\begin{align*}
	I(m, t) &= C_k^{-1} m^{k-1} \int_\F \sum_{g, D_g = D} R_g(z,t) \frac {\opd x \opd y}{y^2} \\
	& = C_k^{-1} m^{k-1} \sum_{\substack{g, D_g = D \\ g (\operatorname{mod} \Gamma(1))}} \int_{\F_g} R_g (z,t) \frac{\opd x \opd y}{y^2} \\
	& = C_k^{-1} m^{k-1} 2 H(-D) I(D,t) \\
	& = C_k^{-1} m^{k-1} 2 H (4m-t^2) \frac {2\pi}{k-1} \frac 1{\sqrt{4m-t^2}} \frac 1{\left( \sqrt{4m-t^2} - it \right)^{k-1}}
	\,,
\end{align*}
was sich mit $\rho := \frac 12 ( t + i \sqrt{4m-t^2} )$ nach kurzer Rechnung vereinfachen lässt zu:
\[
	I(m, t) = \frac{\conj{\rho}^{k-1}}{\rho - \conj{\rho}} H(4m-t^2)
	\,.
\]

Beachtet man nun, dass
\[
	\rho + \conj{\rho} = \frac 12 \left( t + i \sqrt{4m-t^2} \right) + \frac 12 \left( t - i \sqrt{4m-t^2} \right) = t
\]
und
\[
	\rho \conj{\rho} = \abs{\rho}^2 = \frac 14 \abs{t^2 + 4m - t^2} = m
	\,,
\]
so kann man für die per
\[
	(1 - tx + Nx^2)^{-1} = \sum_{k=0}^\infty P_{k+2}(t,N)x^k = P_2(t,N) + P_3(t,N)x + P_4(t,N)x^2 + \ldots
\]
definierten Polynome $P_k(t, N)$ mithilfe von Partialbruchzerlegung und der geometrischen Reihe die Beziehung
\[
	P_k(t, m) = \frac{\rho^{k-1} - \conj{\rho}^{k-1}}{\rho - \conj{\rho}}
\]
zeigen.

Daraus folgt in der Tat wie behauptet
\[
	\frac 12 \left( I(m,t) + I(m,-t) \right) = - \frac 12 \left( \frac{\rho^{k-1}}{\rho - \conj{\rho}} + \frac{- \conj{\rho}^{k-1}}{\rho - \conj{\rho}} \right) H(4m-t^2) = - \frac 12 P_k(t,m) H(4m-t^2)
	\,.
\]

\emph{Fall 2:} $D = 0$. Wir benutzen hierfür die oben hergeleitete Formel (5)
\[
	\int_\F \sum_{g, D_g = D} R_g(z,t) \frac{\opd x \opd y}{y^2} = \int_\F R_{g_0} (z,t) \frac{\opd x \opd y}{y^2} + \int_{\F_\infty} \sum_{\substack{r \in \ZZ \\ r \neq 0}} R_{g_r}(z,t) \frac{\opd x \opd y}{y^2}
	\,.
\]

Wegen $g_0(u,v) \equiv 0$ sind alle Parameter $\alpha, \beta, \gamma$ der quadratischen Form $g_0$ gleich 0, sodass sich $R_{g_0}$ und damit der erste Summand leicht berechnen lassen:
\begin{align*}
	\int_\F R_{g_0} (z,t) \frac{\opd x \opd y}{y^2} 
	&= \int_\F \left( \frac y{-ity} \right)^{\!\! k} \frac{\opd x \opd y}{y^2} 
	= \int_\F \left( \frac it \right)^{\!\! k} \frac{\opd x \opd y}{y^2} \\
	&= \left( \frac it \right)^{\!\! k} \int_\F \frac{\opd x \opd y}{y^2}
	= \frac {(-1)^{\frac k2}}{t^k} \cdot \frac \pi 3 
	= (-1)^{\frac k2} \frac \pi{3t^k}
	\,.
\end{align*}

Für den zweiten Term finden wir mit $g_r(u,v) = rv^2$ und der Partialbruchzerlegung des Kotangens:
\begin{align*}
	\int_{\F_\infty} \sum_{\substack{r \in \ZZ \\ r \neq 0}} R_{g_r}(z,t) \frac{\opd x \opd y}{y^2}
	&= \int_0^\infty \int_0^1 y^{k-2} \sum_{\substack{r \in \ZZ \\ r \neq 0}} (r-ity)^{-k} \opd x \opd y \\
	&= \frac{i^{k-2}}{(k-2)!} \left( \! \frac {\opd}{\opd t} \right)^{\!\! k-2} \int_0^\infty \frac 1{t^2y^2} - \frac {\pi^2}{\sinh^2(\pi ty)} \opd y \\
	&= \frac{i^{k-2}}{(k-1)!} \left( \! \frac {\opd}{\opd t} \right)^{\!\! k-2} \frac \pi {\abs{t}} \\ 
	&= (-1)^{\frac {k-2}{2}} \frac{\pi}{k-1} \abs{t}^{-k+1}
	\,.
\end{align*}

Für $t = \pm 2\sqrt m$ (da $D := t^2 - 4m = 0$) bekommen wir damit nach kurzer Rechnung
\begin{align*}
	I(m,t) 
	&= C_k^{-1} m^{k-1} \int_\F \sum_{g, D_g = 0} R_g(z,t) \frac {\opd x \opd y}{y^{k-2}} \\
	&= C_k^{-1} m^{k-1} \int_\F R_{g_0} (z,t) \frac{\opd x \opd y}{y^2} + C_k^{-1} m^{k-1} \int_{\F_\infty} \sum_{\substack{r \in \ZZ \\ r \neq 0}} R_{g_r}(z,t) \frac{\opd x \opd y}{y^2} \\
	&= C_k^{-1} m^{k-1} (-1)^{\frac k2} \frac \pi{3t^k} + C_k^{-1} m^{k-1} (-1)^{\frac {k-2}{2}} \frac{\pi}{k-1} \abs{t}^{-k+1} \\
	&= \frac {k-1}{24} m^{\frac {k-2}2} - \frac 14 m^{\frac {k-1}2}
\end{align*}

\emph{Fall 3:} $D = u^2$ mit $u \in \NN$. Wie im Fall $D < 0$ gibt es hier nur eine endliche Anzahl von Klassen mit Diskriminante $D$ und $\Gamma_g$ ist eine endliche Gruppe. Es folgt damit 
\[
	\int_\F \sum_{g, D_g = D} R_g(z,t) \frac {\opd x \opd y}{y^2} = H_D \cdot I(D, t)
\]
mit
\[
	H_D := \sum_{\substack{g, D_g = D \\ g (\operatorname{mod} \Gamma(1))}} \frac 1{\abs{\Gamma_g}}
\]
und (vergleiche Fall 1)
\[
	I(D, t) := \int_\HH \frac{y^{k-2}}{\left( \abs{z}^2 - \frac 14 D - ity \right)^k} \opd x \opd y
	\,.
\]
Man kann nun zeigen, dass in Fall 3 alle $\Gamma_g$ trivial sind (d.h. $\abs{\Gamma_g} = 1$) und es darüber hinaus genau $u$ Klassen quadratischer Formen $g$ mit Diskriminante $D_g = u^2$ gibt. Hieraus folgt $H_D = u$. 

Analog zu Fall 1 können wir zudem wieder das Integral umformen zu
\[
	I(D, t) = \frac{\pi i^{k-2}}{2(k-1)!} \left( \frac {\opd}{\opd t} \right)^{k-2} \left[ \frac{4}{t^2 - D} \frac{y - \frac 12 it}{\sqrt{y^2 - ity - \frac 14 D}} \right]_0^\infty
	\,,
\]
wobei der ausgewertete Term rechts diesmal unter Beachtung von $D = t^2 - 4m > 0$ zu 
\[
	\frac {-4}{\sqrt{D}} \frac 1{\sqrt{D} + \abs{t}}
\]
wird. Es folgt wegen $\sqrt D = \sqrt {u^2} = u$, dass
\[
	I(D, t) = (-1)^{\frac {k-2}2} \frac{2\pi}{k-1} \cdot \frac 1{\sqrt{D}} \cdot \frac 1{(\sqrt{D} + \abs{t})^{k-1}} = (-1)^{\frac {k-2}2} \frac{2\pi}{k-1} \cdot \frac 1u \cdot \frac 1{(u + \abs{t})^{k-1}}
	\,,
\]
und somit nach kurzer Rechnung
\begin{align*}
	I(m,t)
	&= C_k^{-1} m^{k-1} \int_\F \sum_{g, D_g = D} R_g(z,t) \frac {\opd x \opd y}{y^2} \\
	&= C_k^{-1} m^{k-1} H_D \cdot I(D, t) \\ 
	&= C_k^{-1} m^{k-1} (-1)^{\frac {k-2}2} \frac{2\pi}{k-1} \cdot \frac 1{(u + \abs{t})^{k-1}} \\
	&= - \frac 12 \left( \frac{\abs{t} - u}2 \right)^{k-1}
	\,.
\end{align*}

\end{bewe}