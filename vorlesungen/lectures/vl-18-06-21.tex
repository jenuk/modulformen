\begin{theorem}[Spurformel, Eichler-Selberg]
	Sei $k \geq 4$ gerade und $m > 0$ beliebig. Dann gilt
	\[
		\Tr T(m) = -\frac{1}{2} \sum_{t=-\infty}^\infty P_k(t,m) H(4m-t^2) - \frac{1}{2} \sum_{d|m} \min\Bigl(d, \frac{m}{d}\Bigr)^{k-1}\,.
	\]
\end{theorem}

\begin{bewe}
Wir unterscheiden nun folgende Fälle:

\emph{Fall 1:} $D < 0$. In diesem Fall ist $\Gamma_g$ endlich (genau genommen von Ordnung $1, 2, 3$, Beweis entfällt aus Zeitgründen). Für eine quadratische Form
\[
	g(u,v) = \alpha u^2 + \beta uv + \gamma v^2
\]
mit Diskriminante $D$ erhalten wir demnach:
\[
	\int_{\F_g} R_g (z,t) \frac{\opd x \opd y}{y^2} = \frac {1}{\abs{\Gamma_g}} \int_\HH R_g (z,t) \frac{\opd x \opd y}{y^2} = \frac {1}{\abs{\Gamma_g}} \int_\HH \frac{y^{k-2}}{\left( \alpha(x^2 + y^2) + \beta x + \gamma - ity \right)^k} \opd x \opd y
\]
Mache Substitution auf folgende Weise: Nutze den Diffeomorphismus
\[
	\Phi: \HH \to \HH, (x, y) \mapsto \left( \frac {2x-\beta}{2\alpha}, \frac y\alpha \right)
\]
($\HH$ hier aufgefasst als Teilmenge des $\RR^2$). Erhalte
\[
	D\Phi = \mymat {\frac 1\alpha}00{\frac 1\alpha}
\]
und mit dem Transformationssatz
\begin{align*}
	&= \frac {1}{\abs{\Gamma_g}} \int_\HH \frac {1}{\alpha^2} \frac{\left(\frac {y}{\alpha} \right)^{k-2}}{\left( \alpha(\left( \frac {2x-\beta}{2\alpha} \right)^2 + \left( \frac y\alpha \right)^2) + \beta \left( \frac {2x-\beta}{2\alpha} \right) + \gamma - it \frac y\alpha \right)^k} \opd x \opd y \\
	&= \frac {1}{\abs{\Gamma_g}} \int_\HH \frac {1}{\alpha^2} \frac{y^{k-2}}{\left( (x^2 + y^2) - \frac {\beta^2}4 + \gamma \alpha - ity \right)^k} \opd x \opd y
\end{align*}
Da $m = \frac 14 (t^2 - \beta^2) + \gamma\alpha$ folgt $\frac 14 D = \frac {\beta^2}4 - \gamma \alpha$ und damit
\[
	= \frac {1}{\abs{\Gamma_g}} \int_\HH \frac {1}{\alpha^2} \frac{y^{k-2}}{\left( \abs{z}^2 - \frac 14 D - ity \right)^k} \opd x \opd y
\]
Bezeichne nun $I(D, t)$ den Wert des Integrals. Es gilt daher
\[
	\sum_{g, D_g = D \mod \Gamma(1)} \int_{\F_g} R_g(z,t) \frac{\opd x \opd y}{y^2} = \sum \frac{1}{\abs{\Gamma_g}} I(D,t) = 2 H(-D) I(D,t)
\]
Die letzte Gleichheit rührt von der Tatsache, dass
\[
	\abs{\Gamma_g} = \begin{cases}
	2, falls g \sim e(x^2 + y^2), e \in \ZZ \setminus \Set {0} \\
	3, falls g \sim e(x^2 + xy + y^2), e \in \ZZ \setminus \Set {0}	\\
	1, sonst
	\end{cases}
\]
Die 2 rührt von der Tatsache her, dass wir in der Definition von $H$ nur positiv definite quadratische Formen betrachtet haben, die Summe jedoch alle Formen beachtet. Beachte aber, dass es in diesem Fall nur positiv bzw. negativ definitite Formen gibt (wegen $b^2y^2 - 4acy^2 < 0$). Zusammen mit der Formel
\[
	\int_{-\infty}^\infty (x^2 + A)^{-k} \opd x = \frac \pi{(k-1)!} \frac 12 \frac 32 \frac 52 \cdots (k-\frac 32) A^{\frac 12 -k}, \quad A > 0
\]
erhalten wir schließlich
\begin{align*}
	I &= \int_0^\infty y^{k-2} \int_{-\infty}^\infty (x^2 + y^2 - ity - \frac 14 D)^{-k} \opd x \opd y \\
	&= \frac \pi{(k-1)!} \frac 12 \frac 32 \frac 52 \cdots (k-\frac 32) \int_0^\infty (y^2 - ity - \frac 14 D)^{\frac 12 - k} y^{k-2} \opd y \\
	&= \frac{\pi i^{k-2}}{2(k-1)!} \left( \frac {\opd}{\opd t} \right)^{k-2} \int_0^\infty (y^2 - ity - \frac 14D)^{-\frac 32} \opd y \\
	&= \frac{\pi i^{k-2}}{2(k-1)!} \left( \frac {\opd}{\opd t} \right)^{k-2} \left[ \frac{4}{t^2 - D} \frac{y - \frac 12 it}{\sqrt{y^2 - ity - \frac 14 D}} \right]_0^\infty \\
	&= \frac{\pi i^{k-2}}{2(k-1)!} \left( \frac {\opd}{\opd t} \right)^{k-2} \left( \frac 4{\sqrt{\abs{D}}} \frac 1{\sqrt{\abs{D}} - it} \right) \\
	&= \frac {2\pi}{k-1} \frac 1{\sqrt{\abs{D}}} \frac 1{\left( \sqrt{\abs{D}} - it \right)^{k-1}}
\end{align*}
Damit gilt
\begin{align*}
	I(m, t) &= C_k^{-1} m^{k-1} \int_\F \sum_{g, D_g = t^2-4m} R_g(z,t) \frac {\opd x \opd y}{y^2} \\
	&= C_k^{-1} m^{k-1} 2 H (4m-t^2) \frac {2\pi}{k-1} \frac 1{\sqrt{4m-t^2}} \frac 1{\left( \sqrt{4m-t^2} - it \right)^{k-1}} \\
	&= \frac{\conj{\rho}^{k-1}}{\rho - \conj{\rho}} H(4m-t^2)
\end{align*}
mit $\rho = \frac 12 \left( t + i \sqrt{4m-t^2} \right)$
Über Partialbruchzerlegung und über die geometrische Reihe kann man zeigen: Ist
\[
	(1 - tx + Nx^2)^{-1} = \sum_{k=0}^\infty P_{k+2}(t,N)x^k = P_2(t,N) + P_3(t,N)x + P_4(t,N)x^2 + \ldots
\]
dann gilt
\[
	P_k(t,N) = \frac{\rho^{k-1} - \conj{\rho}^{k-1}}{\rho - \conj{\rho}}
\]
mit $\rho + \conj{\rho} = t$ und $\rho \conj{\rho} = N$ (für beliebige $\rho$?). Mit $\rho = \frac 12 \left( t + i \sqrt{4m-t^2} \right)$ gilt dann
\[
	\rho \conj{\rho} = \abs{\rho}^2 = \frac 14 \abs{t^2 + 4m - t^2} = m
\]
und 
\[
	\rho + \conj{\rho} = \frac 12 \left( t + i \sqrt{4m-t^2} \right) + \frac 12 \left( t - i \sqrt{4m-t^2} \right) = t
\]
Daraus folgt
\[
	\frac 12 \left( I(m,t) + I(m,-t) \right) = - \frac 12 \frac{\rho^{k-1} - \conj{\rho}^{k-1}}{\rho - \conj{\rho}} H(4m-t^2) = - \frac 12 P_k(t,m) H(4m-t^2)
\]

\emph{2. Fall} $D = 0$. Wir benutzen hierfür die Formel
\[
	\int_\F \sum_{g, D_g = 0} R_g(z,t) \frac{\opd x \opd y}{y^2} = \int_\F R_{g_0}(z,t) \frac{\opd x \opd y}{y^2} + \int_{\F_\infty} \sum_{r \neq 0} R_{g_r}(z,t) \frac{\opd x \opd y}{y^2}
\]
Der erste Term ist wegen $g_0(u,v) = 0$ und damit 
\[
	R_{g_0}(z,t) = \left( \frac y{-ity} \right)^k 
\]
gleich
\[
	= \int_\F \left( \frac it \right)^k \frac{\opd x \opd y}{y^2} = (-1)^{\frac k2} \frac \pi{3t^2}
\]
da 
\[
	\int_\F \frac{\opd x \opd y}{y^2} = \frac \pi 3
\]
Für den zweiten Term finden wir mit PBZ des $\cot$
\begin{align*}
	&= \int_0^\infty \int_0^1 y^{k-2} \sum_{r \neq 0} (r-ity)^{-k} \opd x \opd y = \frac{i^{k-2}}{(k-2)!} \left( \frac {\opd}{\opd t} \right)^{k-2} \int_0^\infty \frac 1{t^2y^2} - \frac {\pi^2}{\sinh^2(\pi ty)} \opd y
	&= \frac{i^{k-2}}{(k-1)!} \left( \frac {\opd}{\opd t} \right)^{k-2} \frac \pi {\abs{t}} = (-1)^{\frac {k-2}{2}} \frac{\pi}{k-1} \abs{t}^{-k+1}
\end{align*}

Für $t = \pm 2\sqrt m$ bekommen wir damit
\[
	I(m,t) = C_k^{-1} m^{k-1} \int_\F \sum_{g, D_g = 0} R_g(z,t) \frac {\opd x \opd y}{y^{k-2}} = \frac {k-1}{24} m^{\frac {k-2}2} - \frac 14 m^{\frac {k-1}2}
\]

\emph{Fall 3} $D = u^2$ mit $u \in \NN$. Wie im Fall $D < 0$ gibt es hier nur eine endliche Anzahl von Klassen mit Diskriminante $D$ und $\Gamma_g$ ist eine endliche Gruppe. Es folgt damit 
\[
	\int_\F \sum_{g, D_g = D} R_g(z,t) \frac {\opd x \opd y}{y^2} = H \cdot I
\]
wobei
\[
	H = \sum_{g, D_g = D \mod \Gamma(1)} \frac 1{\abs{\Gamma_g}}
\]
und
\[
	I = \int_\HH \frac {y^k}{\left( \abs{z}^2 - ity - \frac 14 D \right)^k} \frac {\opd x \opd y}{y^2}
\]
Die $\Gamma_g$ sind in diesem Fall, wie man zeigen kann, trivial. Da es $u$ Klassen quadratischer Formen mit Diskriminante $u^2$ gibt (kann man zeigen), folgt $H = u$. Wie im Fall $D < 0$ erhalten wir zudem
\[
	I = \frac{\pi i^{k-2}}{2(k-1)!} \left( \frac {\opd}{\opd t} \right)^{k-2} \left[ \frac{4}{t^2 - D} \frac{y - \frac 12 it}{\sqrt{y^2 - ity - \frac 14 D}} \right]_0^\infty
\]
nur für $D > 0$ ist der Ausdruck in Klammern gleich 
\[
	\frac {-4}{\sqrt{D}} \frac 1{\sqrt{D} + \abs{t}}
\]
Daher haben wir
\[
	HI = (-1)^{\frac {k-2}2} \frac{2\pi}{k-1} \frac 1{(u + \abs{t})^{k-1}}
\]
also
\[
	I(m,t) = C_k^{-1} m^{k-1} HI = - \frac 12 \left( \frac{\abs{t} - u}2 \right)^{k-1}
\]


\end{bewe}












